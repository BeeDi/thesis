\documentclass[a4paper,11pt,open=right,cleardoublepage=empty,bibliography=totoc,BCOR=0mm,DIV=12,parskip=half,french]{scrbook}

%%%%%%%%%%%%%%%%%%%%%%%%%%%%%%%%%%%%%%%%%%%%%%%%%%%%%%%%%%%%%%%%%%%%%%%%%%%%%%%%%%%%%%%%%%%%%%%%%%%
%%%%%%%%%%%%%%%%%%%%%%%%%%%%%%%%%%%%%%%%%%%%%%%%%%%%%%%%%%%%%%%%%%%%%%%%%%%%%%%%%%%%%%%%%%%%%%%%%%%
%%%%%%%%%%%%%%%%%%%%%%%%%%%%%%%%%%%%%%%%%%%%%%%%%%%%%%%%%%%%%%%%%%%%%%%%%%%%%%%%%%%%%%%%%%%%%%%%%%%
%%%%%%%%%%%%%%%%%%%%%%%%%%%%%%%%%%%%%%%%%%%%%%%%%%%%%%%%%%%%%%%%%%%%%%%%%%%%%%%%%%%%%%%%%%%%%%%%%%%
%%%%%%%%%%%%%%%%%%%%%%%%%%%%%%%%%%%%%%%%%%%%%%%%%%%%%%%%%%%%%%%%%%%%%%%%%%%%%%%%%%%%%%%%%%%%%%%%%%%

%%%%% Packages vitaux (XeTeX, KOMA-Script…)
\XeTeXdefaultencoding utf-8
\usepackage{xunicode}
\usepackage{fontspec}
\usepackage{xltxtra}
\usepackage{scrhack}
\usepackage{typearea}
\usepackage{scrpage2}
\renewcommand*\chapterpagestyle{scrheadings}
\renewcommand*{\captionformat}{~: }

%%%%% Hacks
\interfootnotelinepenalty=10000

%%%%% Français
\usepackage[frenchb]{babel}
\frenchspacing
\usepackage[babel]{csquotes}

%%%%% Tableaux
\usepackage{array}
\usepackage{booktabs}
\usepackage{ctable}
\usepackage{hhline}
\usepackage{longtable}
\usepackage{tabularx}
\usepackage{supertabular}

%%%%% Divers
\usepackage{epigraph}
\usepackage{fancybox}
\usepackage{framed}
\usepackage[stable]{footmisc}
\usepackage{framed}
\usepackage{graphicx}
\usepackage{mdwlist}
\usepackage{setspace}
\usepackage{url}
\usepackage[xetex,table,x11names,svgnames,svgnames*]{xcolor}
\usepackage{verbatim}
\usepackage{ifthen}
\usepackage{xifthen}

\usepackage{fancyvrb}
\usepackage{moreverb}

%%%%% Mise en page
\usepackage{appendix}
\usepackage{chngpage}
\usepackage{enumitem}
\usepackage{multirow}
\usepackage{shorttoc}
\usepackage{titlesec}
\usepackage{titletoc}


%%%%%%% Références
% \usepackage{hyperref}

% CHNPAGE : Permet d'utiliser les commandes \spacing, doublespace (double interligne), singlespace (simple interligne) et onehalfspace (un interligne et demi).
% ENUMITEM : Pour paramétrer l'espacement dans les listes (partopsep itemsep topsep).
% SHORTTOC : En français, la table des matières n'apparait qu'à la fin. Au début doit se trouver le sommaire. Or, il n'y a qu'une seule commande en LaTeX qui fait apparaître la table des matières, c'est \tableofcontents. Afin de satisfaire aux normes françaises, le fichier de style shorttoc.sty (créé par J.P. Drucbert et amélioré par DarkPetitTroll), peut être utilisé. Il ne faut pas oublier de placer ce fichier de style dans le même répertoire que le fichier source ou dans /texmf/tex/latex/! Voir cette page : http://petitroll.free.fr/index.php/2006/09/29/48-ajouter-un-sommaire-dans-un-document-latex

%%%%% Bibliographie
\usepackage[bibstyle=verbose-trad1,citestyle=authoryear-ibid,ibidtracker=true,maxnames=1,minnames=1,uniquename=allfull,backref=true,backrefstyle=two]{biblatex}
%Original: bibstyle=verbose-trad1,citestyle=authoryear-ibid,maxnames=2,ibidtracker=true
%Mine:  bibstyle=verbose-trad1,citestyle=authoryear-ibid,ibidtracker=true,maxnames=1,minnames=1,uniquename=allfull,backref=true,backrefstyle=two

%%%%% Symboles
\usepackage{dingbat}
\usepackage{latexsym}
\usepackage{manfnt}
\usepackage{marvosym}
\usepackage{phaistos}
\usepackage{pifont}
\usepackage{textcomp}
\usepackage{wasysym}

%%%%%%%%%%% À METTRE APRÈS LE \begin{document}

%\deffootnote[1em]{0em}{0em}{\thefootnotemark{}.~}

%%%%%%%%%%%%%%%%%%%%%%%%%%%%%%%%%%%%%%%%%%%%%%%%%%%%%%%%%%%%%%%%%%%%%%%%%%%%%%%%%%%%%%%%%%%%%%%%%%%
%%%%% BALISAGE INLINE
%%%%%%%%%%%%%%%%%%%%%%%%%%%%%%%%%%%%%%%%%%%%%%%%%%%%%%%%%%%%%%%%%%%%%%%%%%%%%%%%%%%%%%%%%%%%%%%%%%%

%%%%% Contractions
\newcommand{\e}[1]{\emph{#1}}
\newcommand{\g}[1]{\textbf{#1}}
\newcommand{\pc}[1]{\textsc{#1}}
\newcommand{\ita}[1]{\textit{#1}}

%%%%% Combinaisons
\newcommand{\eg}[1]{\textbf{\emph{#1}}}
\newcommand{\gse}[1]{\textbf{\textsf{#1}}}
\newcommand{\gpc}[1]{\textbf{\textsc{#1}}}
\newcommand{\gri}[1]{\textbf{\textit{#1}}}

%%%%% Usages
\newcommand{\me}[1]{\textit{#1}}
\newcommand{\patronyme}[1]{\textsc{#1}}

%%%%% Items exergue
\newcommand{\tii}[1]{\textbf{#1}~---~}
\newcommand{\poi}[1]{\textbf{#1}.~}

%%%%% Citation En Ligne
\newcommand{\ciel}[1]{\textit{\gui{#1}}}

%%%%% Un bout de code
\newcommand{\cd}[1]{\textcolor{black!70}{\texttt{#1}}}


%%%%% Guillemets français
\newcommand{\gui}[1]{«\,#1\,»}

%%%%% Référence bibliographique entre crochets
\newcommand{\citcroc}[1]{[\cite{#1}]}

%%%%% Référence bibliographique en PetiteCapitale et ()
\newcommand{\parcite}[1]{(\pc{\cite{#1}})}

%%%%% Séparateur
\newcommand{\sep}{\begin{center}{\wtof\addfontfeature{Scale=6}\symbol{57355}}\end{center}\vspace{-2em}}

%%%%%%%%%%%%%%%%%%%%%%%%%%%%%%%%%%%%%%%%%%%%%%%%%%%%%%%%%%%%%%%%%%%%%%%%%%%%%%%%%%%%%%%%%%%%%%%%%%%
%%%%% ENVIRONNEMENTS, FIGURES ET AUTRES ZONES SPÉCIALES
%%%%%%%%%%%%%%%%%%%%%%%%%%%%%%%%%%%%%%%%%%%%%%%%%%%%%%%%%%%%%%%%%%%%%%%%%%%%%%%%%%%%%%%%%%%%%%%%%%%

%%%%% Centerpage (utile pour une page de remerciements)
\newenvironment{vcenterpage}
{\newpage\vspace*{\fill}}
{\vspace*{\fill}\par\pagebreak}

%%%%% J'en ai marre des tableaux
\newcommand{\lquatre}[4]{#1&#2&#3&#4\\\hline}

%%%%% Épigraphes
\newcommand{\epigraphii}[2]{\epigraph{\textit{#1}}{\textsc{#2}}\vspace{2cm}}
\newcommand{\epigraphsec}[2]{\epigraph{\textit{#1}}{\textsc{#2}}}

%%%%% Items des listes
\newcommand{\setlabelitems}[4]{%
	\renewcommand{\labelitemi}{#1}%
	\renewcommand{\labelitemii}{#2}%
	\renewcommand{\labelitemiii}{#3}%
	\renewcommand{\labelitemiv}{#4}%
}

%%%%% Liste simple Amleth
\newenvironment{liste}%
{\begin{list}{\labelitemi}{
	\setlength{\itemindent}{0em}%
	\setlength{\listparindent}{0em}%
	\setlength{\rightmargin}{0em}%
	\setlength{\leftmargin}{2cm}%
	\setlength{\labelwidth}{20pt}%
	\setlength{\labelsep}{5pt}%
	\setlength{\itemsep}{0cm}%
	\setlength{\parsep}{0em}%
	\setlength{\topsep}{0em}%
	\setlength{\partopsep}{0em}%
}}
{\end{list}}

%%%%% Liste énumérotée Amleth
\newenvironment{listenum}%
{\begin{enumerate}[leftmargin=2cm,noitemsep,nolistsep]{

}}
{\end{enumerate}}

%[itemindent=0em,listparindent=0em,rightmargin=0em,leftmargin=2cm,itemsep=0em,topsep=0em,partopsep=0em]

%%%%% Liste non indentée
\newenvironment{listeni}%
{\begin{list}{\labelitemi}{
	\setlength{\itemindent}{0em}%
	\setlength{\listparindent}{0em}%d
	\setlength{\rightmargin}{0em}%
	\setlength{\leftmargin}{12pt}%
	\setlength{\labelwidth}{20pt}%
	\setlength{\labelsep}{5pt}%
	\setlength{\itemsep}{0cm}%
	\setlength{\parsep}{0em}%
	\setlength{\topsep}{0em}%
	\setlength{\partopsep}{0em}%
}}
{\end{list}}

%%%%% Citation longue
\newsavebox{\auteurcitation}
\newsavebox{\boitecitation}
\newboolean{auteurcitationpresent}
\newenvironment{ciloac}[1]%
{%
	%\vspace{-1em}
	\savebox{\auteurcitation}{\small#1}%
	\begin{lrbox}{\boitecitation}%
		\begin{minipage}{.8\linewidth}%
			\small\itshape«\,\ignorespaces%
}%
{%
			\unskip\,»\par\mbox{}\hfill\usebox{\auteurcitation}%
		\end{minipage}%
	\end{lrbox}%
	\begin{center}%
		\usebox{\boitecitation}%
	\end{center}%
	%\vspace{-1em}
}

%%%%% Citation courte
\newenvironment{cico}{\begin{quote}\begin{small}«\,}{\,»\end{small}\end{quote}}

%%%%% Figure
\newcommand{\figue}[4]
{
	\begin{figure}[!ht]
		\centering
		\includegraphics[width=#2\linewidth]{#1}
		\caption{\small #3}
		\label{#4}
	\end{figure}
}

%%%%% Encadré coloré
\newenvironment{cadrecol}[1]
{%
	\setlength\fboxrule{1pt}%
	\def\FrameCommand{\fboxsep=\FrameSep\fcolorbox{black}{#1}}%
	\MakeFramed{\FrameRestore}%
}{%
	\endMakeFramed
}

%%%%% Changer ponctuellement les marges
\newenvironment{changemargin}[2]{\begin{list}{}{%
\setlength{\topsep}{0pt}%
\setlength{\leftmargin}{0pt}%
\setlength{\rightmargin}{0pt}%
\setlength{\listparindent}{\parindent}%
\setlength{\itemindent}{\parindent}%
\setlength{\parsep}{0pt plus 1pt}%
\addtolength{\leftmargin}{#1}%
\addtolength{\rightmargin}{#2}%
}\item }{\end{list}}

%%%%%%%%%%%%%%%%%%%%%%%%%%%%%%%%%%%%%%%%%%%%%%%%%%%%%%%%%%%%%%%%%%%%%%%%%%%%%%%%%%%%%%%%%%%%%%%%%%%
%%%%% CARACTÈRES ACTIFS
%%%%%%%%%%%%%%%%%%%%%%%%%%%%%%%%%%%%%%%%%%%%%%%%%%%%%%%%%%%%%%%%%%%%%%%%%%%%%%%%%%%%%%%%%%%%%%%%%%%

\lccode"2019="2019
\catcode`\'\active
%\def'{\@ifnextchar_\parfollowedbyunderscore\parnotfollowedbyunderscore}
\def'{\kern0.1em\string'}

\lccode"171="171
\catcode`\«\active
\def«{\string«~}

\lccode"187="187
\catcode`\»\active
\def»{~\string»}






%%%%%%%%%%%%%%%%%%%%%%%%%%%%%%%%%%%%%%%%%%%%%%%%%%%%%%%%%%%%%%%%%%%%%%%%%%%%%%%%%%%%%%%%%%%%%%%%%%%%%%%%%%%%%%%%%%%%%%%%%
%%%%%%%%%%% CESURE et HYPHENATION
%%%%%%%%%%%%%%%%%%%%%%%%%%%%%%%%%%%%%%%%%%%%%%%%%%%%%%%%%%%%%%%%%%%%%%%%%%%%%%%%%%%%%%%%%%%%%%%%%%%%%%%%%%%%%%%%%%%%%%%%%


\hyphenation{audio-visuel inter-com-pré-hen-sion infor-ma-ti-sation appa-reil exploi-ta-tion impli-ca-tion ingé-nie-rie métho-de enregi-stre-ment hypo-thèse inter-pré-ta-tion igno-ran-ce}

%%%%%%%%%%%%%%%%%%%%%%%%%%%%%%%%%%%%%%%%%%%%%%%%%%%%%%%%%%%%%%%%%%%%%%%%%%%%%%%%%%%%%%%%%%%%%%%%%%%%%%%%%%%%%%%%%%%%%%%%%
%%%%%%%%%%% FONTEX XETEX
%%%%%%%%%%%%%%%%%%%%%%%%%%%%%%%%%%%%%%%%%%%%%%%%%%%%%%%%%%%%%%%%%%%%%%%%%%%%%%%%%%%%%%%%%%%%%%%%%%%%%%%%%%%%%%%%%%%%%%%%%

\newfontfamily\fichierexternef[Mapping=tex-text,Scale=1,RawFeature=+onum;+swsh;+dlig;+clig;+hlig]{Brioso Pro}

%%%%%%%%%%%%%%%%%%%%%%%%%%%%%%%%%%%%%%%%%%%%%%%%%%%%%%%%%%%%%%%%%%%%%%%%%%%%%%%%%%%%%%%%%%%%%%%%%%%%%%%%%%%%%%%%%%%%%%%%%
%%%%%%%%%%% SYMBOLES
%%%%%%%%%%%%%%%%%%%%%%%%%%%%%%%%%%%%%%%%%%%%%%%%%%%%%%%%%%%%%%%%%%%%%%%%%%%%%%%%%%%%%%%%%%%%%%%%%%%%%%%%%%%%%%%%%%%%%%%%%

%%%%% Déclarations
\newfontfamily\htof[Mapping=tex-text]{HoeflerText Ornaments}
\newfontfamily\wtof[Mapping=tex-text]{Adobe Wood Type Ornaments Std}
\newfontfamily\wsdf[Mapping=tex-text]{Wiesbaden Swing LT Std Dingbats}
\newfontface\WHIIIf[Mapping=tex-text]{WFRP3Symbols}
\newfontface\sangraelf[Mapping=tex-text]{Sangrael™}

%%%%% Accès
\newcommand{\wto}[2]{{\wtof\addfontfeature{Scale=#2}\symbol{#1}}}
\newcommand{\hto}[2]{{\htof\addfontfeature{Scale=#2}\symbol{#1}}}
\newcommand{\wsd}[2]{{\wsdf\addfontfeature{Scale=#2}\symbol{#1}}}

%%%%% Symboles utilisés
\newcommand{\petitbourdon}[1]{\wsd{57354}{#1}}
\newcommand{\sangraelducks}[1]{\sangraelf\addfontfeature{Scale=#1}\symbol{66}}
\newcommand{\sangraeltower}[1]{\sangraelf\addfontfeature{Scale=#1}\symbol{75}}
\newcommand{\sangraelbridge}[1]{\sangraelf\addfontfeature{Scale=#1}\symbol{77}}

%%%%%%%%%%%%%%%%%%%%%%%%%%%%%%%%%%%%%%%%%%%%%%%%%%%%%%%%%%%%%%%%%%%%%%%%%%%%%%%%%%%%%%%%%%%%%%%%%%%%%%%%%%%%%%%%%%%%%%%%%
%%%%%%%%%%% MAQUEREAUX
%%%%%%%%%%%%%%%%%%%%%%%%%%%%%%%%%%%%%%%%%%%%%%%%%%%%%%%%%%%%%%%%%%%%%%%%%%%%%%%%%%%%%%%%%%%%%%%%%%%%%%%%%%%%%%%%%%%%%%%%%

\newcommand{\att}[1]{\textbf{\textsc{#1}}}

%%%%% Texte désignant un fichier externe
\newcommand{\fichierexterne}[1]{\centering\fichierexternef{#1}}

%%%%% Figure encadrée
\definecolor{btfond}{rgb}{0.75,0.75,0.75}
\newenvironment{btdedahs}
{%
	\setlength\fboxrule{1pt}%
	\def\FrameCommand{\fboxsep=\FrameSep\fcolorbox{black}{Silver}}%
	\MakeFramed{\FrameRestore}%
}{%
	\endMakeFramed
}
\newenvironment{ff}[2]
{%
	\begin{center}%
	\begin{minipage}{#1}%
	\begin{center}%
	\begin{btdedahs}%
	{\addfontfeature{Scale=1.11}\textit{\textbf{#2}}}\vspace{1em}\par%
}{%
	\end{btdedahs}%
	\end{center}%
	\end{minipage}%
	\end{center}%
}	

%%%%% Exemple
\newenvironment{exemplejdr}
{%
	\begin{center}%
	\begin{minipage}{11cm}%
	\textsc{\textbf{Exemple.}~}%
	\itshape%
}{%
	\end{minipage}%
	\end{center}%
}

%%%%% Description-Chiffrage-Acquisition-Perte-Augmentation-Développement
\newcommand{\DCAPADpuce}{\wto{57395}{1}~}
\newcommand{\DCAPAD}[6]{%
	\par\textsc{\textbf{\DCAPADpuce Définition.~}}#1%
	\par\textsc{\textbf{\DCAPADpuce Chiffrage.~}}#2%
	\par\textsc{\textbf{\DCAPADpuce Acquisition.~}}#3%
	\par\textsc{\textbf{\DCAPADpuce Perte.~}}#4%
	\par\textsc{\textbf{\DCAPADpuce Augmentation.~}}#5%
	\par\textsc{\textbf{\DCAPADpuce Diminution.~}}#6%
	\par
}

%%%%% Donnée du système de jeu (nom de compétence, etc.)
\newcommand{\jd}[1]{\textit{#1}}

%%%%% Point + par
\newcommand{\pointp}[2]{\par\textsc{\textbf{#1}~}#2\par}

%%%%% Point
\newcommand{\point}[1]{\par\textsc{\textbf{#1}~}}

%%%%% Point de règle tiré d'un autre jeu
\definecolor{jfond}{rgb}{1,0.8,0.8}
\newenvironment{jdedahs}
{%
	\setlength\fboxrule{1pt}%
	\def\FrameCommand{\fboxsep=\FrameSep\fcolorbox{black}{DarkGoldenrod1}}%
	\MakeFramed{\FrameRestore}%
}{%
	\endMakeFramed
}
\newenvironment{jeu}[1]
{%
	\begin{jdedahs}%
	\textbf{\textsc{#1}}%
	\small%
}{%
	\end{jdedahs}%
}


%%%%%%%%%%%%%%%%%%%%%%%%%%%%%%%%%%%%%%%%%%%%%%%%%%%%%%%%%%%%%%%%%%%%%%%%%%%%%%%%%%%%%%%%%%%%%%%%%%%
%%%%%%%%%%%%%%%%%%%%%%%%%%%%%%%%%%%%%%%%%%%%%%%%%%%%%%%%%%%%%%%%%%%%%%%%%%%%%%%%%%%%%%%%%%%%%%%%%%%
%%%%%%%%%%%%%%%%%%%%%%%%%%%%%%%%%%%%%%%%%%%%%%%%%%%%%%%%%%%%%%%%%%%%%%%%%%%%%%%%%%%%%%%%%%%%%%%%%%%
%%%%%%%%%%%%%%%%%%%%%%%%%%%%%%%%%%%%%%%%%%%%%%%%%%%%%%%%%%%%%%%%%%%%%%%%%%%%%%%%%%%%%%%%%%%%%%%%%%%
%%%%%%%%%%%%%%%%%%%%%%%%%%%%%%%%%%%%%%%%%%%%%%%%%%%%%%%%%%%%%%%%%%%%%%%%%%%%%%%%%%%%%%%%%%%%%%%%%%%


\parindent=0em
\setlength{\bibitemsep}{0.5em}
\raggedbottom

%%%%% scrpage2
\pagestyle{scrheadings}
\clearscrheadings
\clearscrplain
\clearscrheadfoot		
\setheadsepline{.2pt}
	\lehead{\headingsf\pagemark}
	\rehead{\headingsf\scshape Chapitre~\thechapter}
	\lohead{\headingsf\headmark}
	\rohead{\pagemark}

%%%%%%%%%%%%%%%%%%%%%%%%%%%%%%%%%%%%%%%%%%%%%%%%%%%%%%%%%%%%%%%%%%%%%%%%%%%%%%%%%%%%%%%%%%%%%%%%%%%
%%%%% TABLE DES MATIERES ET NUMÉROTATIONS
%%%%%%%%%%%%%%%%%%%%%%%%%%%%%%%%%%%%%%%%%%%%%%%%%%%%%%%%%%%%%%%%%%%%%%%%%%%%%%%%%%%%%%%%%%%%%%%%%%%
\setcounter{tocdepth}{3}
\setcounter{secnumdepth}{3}

\usepackage{remreset}
\makeatletter
\@removefromreset{footnote}{chapter}
\makeatother

%\titlecontents{chapter}[0pt]{\addvspace{5pt}}{\bfseries Chapitre\thecontentslabel~---~}{}{\titlerule*[8pt]{.}\contentspage}
%\titlecontents{section}[10pt]{\addvspace{2pt}}{Section\thecontentslabel.\ }{}{\titlerule*[8pt]{.}\contentspage}
%\titlecontents{subsection}[20pt]{}{\S\ \thecontentslabel.\ }{}{\titlerule*[8pt]{.}\contentspage}
%\titlecontents{subsubsection}[30pt]{}{\thecontentslabel.\ }{}{\titlerule*[8pt]{.}\contentspage}
%\titlecontents{paragraph}[40pt]{}{\thecontentslabel.\ }{}{\titlerule*[8pt]{.}\contentspage}
%\titlecontents{subparagraph}[50pt]{}{\thecontentslabel.\ }{}{\titlerule*[8pt]{.}\contentspage}

%%%%%%%%%%%%%%%%%%%%%%%%%%%%%%%%%%%%%%%%%%%%%%%%%%%%%%%%%%%%%%%%%%%%%%%%%%%%%%%%%%%%%%%%%%%%%%%%%%%
%%%%% SURCHARGES
%%%%%%%%%%%%%%%%%%%%%%%%%%%%%%%%%%%%%%%%%%%%%%%%%%%%%%%%%%%%%%%%%%%%%%%%%%%%%%%%%%%%%%%%%%%%%%%%%%%

%%%%% Liste d'items avec description
\renewenvironment{description}[1]{%
	\begin{basedescript}{%
		\desclabelstyle{\multilinelabel}%
		\renewcommand{\makelabel}[1]{\bfseries\scshape##1}%
		\setlength{\labelwidth}{#1}%
		\setlength{\labelsep}{5pt}%
		\setlength{\leftmargin}{\labelwidth+\labelsep}%
	}%
}{%
	\end{basedescript}%
}

%%%%%%%%%%%%%%%%%%%%%%%%%%%%%%%%%%%%%%%%%%%%%%%%%%%%%%%%%%%%%%%%%%%%%%%%%%%%%%%%%%%%%%%%%%%%%%%%%%%
%%%%% MISE EN PAGE ET TYPOGRAPHIE
%%%%%%%%%%%%%%%%%%%%%%%%%%%%%%%%%%%%%%%%%%%%%%%%%%%%%%%%%%%%%%%%%%%%%%%%%%%%%%%%%%%%%%%%%%%%%%%%%%%

%%%%% Police principale
\setmainfont[Mapping=tex-text,Scale=1,Ligatures=Common]{Adobe Jenson Pro}
\setsansfont[Mapping=tex-text]{Hypatia Sans Pro}
\setmonofont[Scale=0.77]{Monaco}

%%%%% Polices des titres
\newcommand\salicornefonte{Hypatia Sans Pro}
\newfontfamily\secf[Mapping=tex-text,Color=000000FF,LetterSpace=-0.0,Scale=1.5]{\salicornefonte}%BB0000FF
\newfontfamily\subsecf[Mapping=tex-text,Color=000000FF,LetterSpace=-0.0,Scale=1.3]{\salicornefonte}%DD0000FF
\newfontfamily\subsubsecf[Mapping=tex-text,Color=000000FF,LetterSpace=-0.0,Scale=1.1]{\salicornefonte}%FF0000FF
\newfontfamily\parf[Mapping=tex-text,Color=000000FF,LetterSpace=-0.0,Scale=1]{\salicornefonte}%BB0000FF

%%%%% Trucs stranges
\newfontfamily\grec[Mapping=tex-text,Color=000000FF,LetterSpace=-0.0,Scale=1]{Cambria}
\newfontfamily\awto[Mapping=tex-text]{Adobe Wood Type Ornaments Std}
\newfontfamily\cjkfont[Script=CJK]{AR PL UKai CN}

%%%%% Polices des entêtes de chapitres
\newcommand\fontechapitres{Adobe Garamond Pro}
\newfontfamily\headingsf[Mapping=tex-text,Color=000000FF,LetterSpace=-0.0,Scale=1,Ligatures=Common]{Adobe Garamond Pro}
\newfontfamily\chapf[Mapping=tex-text,Color=000000FF,LetterSpace=-0.0,Scale=2.5]{\fontechapitres}
\newfontfamily\chaptertitlenamef[Mapping=tex-text,Color=AAAAAAFF,LetterSpace=-0.0,Scale=2.5]{\fontechapitres}
\newfontfamily\thechapterf[Mapping=tex-text,Color=777777FF,LetterSpace=-0.0,Scale=6]{\fontechapitres}

%%%%% titlesec :: titles
\titleformat*{\section}{\bfseries\scshape\secf}
\titleformat*{\subsection}{\bfseries\subsecf}
\titleformat*{\subsubsection}{\bfseries\subsubsecf}
\titleformat*{\paragraph}{\bfseries\parf}

%%%%% titles :: chapter
\renewcommand{\thechapter}{\Roman{chapter}}
\titleformat{\chapter}[display]
	{\setlength{\baselineskip}{2\baselineskip}\chapf\scshape\filleft}
	{\chaptertitlenamef\MakeUppercase{\chaptertitlename}~\thechapterf\thechapter}
	{1em}
	{}
	{}
\titlespacing{\chapter}{0pt}{*1}{*12}[0pt] % Avec les épigraphes
%\titlespacing{\chapter}{0pt}{*1}{*24}[0pt] % Sans les épigraphes

%%%%% Police des tableaux
\let\oldlongtable=\longtable
\def\longtable{\small\oldlongtable}

%%%%%%%%%%%%%%%%%%%%%%%%%%%%%%%%%%%%%%%%%%%%%%%%%%%%%%%%%%%%%%%%%%%%%%%%%%%%%%%%%%%%%%%%%%%%%%%%%%%
%%%%%%%%%%%%%%%%%%%%%%%%%%%%%%%%%%%%%%%%%%%%%%%%%%%%%%%%%%%%%%%%%%%%%%%%%%%%%%%%%%%%%%%%%%%%%%%%%%%
%%%%%%%%%%%%%%%%%%%%%%%%%%%%%%%%%%%%%%%%%%%%%%%%%%%%%%%%%%%%%%%%%%%%%%%%%%%%%%%%%%%%%%%%%%%%%%%%%%%
%%%%%%%%%%%%%%%%%%%%%%%%%%%%%%%%%%%%%%%%%%%%%%%%%%%%%%%%%%%%%%%%%%%%%%%%%%%%%%%%%%%%%%%%%%%%%%%%%%%
%%%%%%%%%%%%%%%%%%%%%%%%%%%%%%%%%%%%%%%%%%%%%%%%%%%%%%%%%%%%%%%%%%%%%%%%%%%%%%%%%%%%%%%%%%%%%%%%%%%
\title{\scshape{Description Sémantique de Documents Audiovisuels Structurés}}
\author{Benjamin \textsc{Diemert}}
\date{}
\bibliography{dsdav}

\begin{document}

\deffootnote[2em]{0em}{0em}{\thefootnotemark{}.~}

%%%%% Items des listes
\setlabelitems{\wto{57364}{1}}{\wto{57364}{1}}{\wto{57364}{1}}{\wto{57364}{1}} 

\KOMAoptions{twoside=no}

% \pagestyle{empty}

% \input{PERRON/GARDE}

% \newpage

% ~

% \cleardoublepage

% \pagestyle{empty}

% \input{PERRON/REMERCIEMENTS}

% \cleardoublepage

% \KOMAoptions{twoside=yes}

\frontmatter

\pagestyle{scrheadings}

\shorttableofcontents{Sommaire}{2}

\cleardoublepage


%%%%%%%%%%%%%%%%%%%%%%%%%%%%%%%%%%%%%%%%%%%%%%%%%%%%%%%%%%%%%%%%%%%%%%%%%%%%%%%%%%%%%%%%%%%%%%%%%%%
%%%%%%%%%%%%%%%%%%%%%%%%%%%%%%%%%%%%%%%%%%%%%%%%%%%%%%%%%%%%%%%%%%%%%%%%%%%%%%%%%%%%%%%%%%%%%%%%%%%
\mainmatter

% \pagestyle{empty}

% ~

% \bigskip

% \vspace{11em}

% \bigskip

% \epigraphii{La totalité est la non vérité.}{Adorno, \ita{Minima Moralia}}

% \cleardoublepage
\addcontentsline{toc}{part}{État de l'Art}
\pagestyle{empty}

~
\bigskip

\vspace{11em}

\bigskip

\epigraphii{Se demander si un ordinateur peut penser ... est aussi intéressant que de se demander si un sous-marin peut nager.}{ Edsger Wybe Dijkstra, \ita{The threats to computing science}}

\pagestyle{scrheadings}

%%%%%%%%%%%%%%%%%%%%%%%%%%%%%%%%%%%%%%%%%%%%%%%%%%%%%%%%%%%%%%%%%%%%%%%%%%%%%%%%%%%%%%%%%%%%%%%%%%%
%%%%%%%%%%%%%%%%%%%%%%%%%%%%%%%%%%%%%%%%%%%%%%%%%%%%%%%%%%%%%%%%%%%%%%%%%%%%%%%%%%%%%%%%%%%%%%%%%%%
\part*{Exposition}
\addcontentsline{toc}{part}{Exposition}

%%%%%%%%%%%%%%%%%%%%%%%%%%%%%%%%%%%%%%%%%%%%%%%%%%%%%%%%%%%%%%%%%%%%%%%%%%%%%%%%%%%%%%%%%%%%%%%%%%%
%%%%%%%%%%%%%%%%%%%%%%%%%%%%%%%%%%%%%%%%%%%%%%%%%%%%%%%%%%%%%%%%%%%%%%%%%%%%%%%%%%%%%%%%%%%%%%%%%%%
% \section*{Préambule (n)}
% \addcontentsline{toc}{section}{Préambule}



%%%%%%%%%%%%%%%%%%%%%%%%%%%%%%%%%%%%%%%%%%%%%%%%%%%%%%%%%%%%%%%%%%%%%%%%%%%%%%%%%%%%%%%%%%%%%%%%%%%
%%%%%%%%%%%%%%%%%%%%%%%%%%%%%%%%%%%%%%%%%%%%%%%%%%%%%%%%%%%%%%%%%%%%%%%%%%%%%%%%%%%%%%%%%%%%%%%%%%%
\chapter{Introduction}\label{chap:intro}
% \epigraphii{On peut s'étonner que les actes spontanés par lesquels l'homme a mis en forme sa vie, se sédimentent au dehors et y mènent l'existence anonyme des choses. La civilisation à laquelle je participe existe pour moi avec évidence dans les ustensiles qu'elle se donne.}{Merleau-Ponty\\\hfill\ita{Phénoménologie de la perception}}

%%%%%%%%%%%%%%%%%%%%%%%%%%%%%%%%%%%%%%%%%%%%%%%%%%%%%%%%%%%%%%%%%%%%%%%%%%%%%%%%%%%%%%%%%%%%%%%%%%%
% \section{Contexte}\label{chap:contexte}
Le travail de thèse dont nous rendons compte dans ce mémoire s'est déroulé dans le cadre du projet MediaMap\footnote{Voir \url{http://www.mediamapproject.org/}} soutenu par le cluster européen Eureka Celtic et la Direction Générale de la Compétitivité, de l'Industrie et
des Services (DGCIS) du ministère de l'Economie, des finances et de l'industrie.
La participation à ce projet de recherche et développement nous a imprégnés de connaissances sur la production audiovisuelle.
%, les besoins qui émergent des tendances actuelles et les problèmes rencontrés pour les combler. 



%Nous nous intéresserons ensuite (\g{Chapitre \ref{chap:problo}}) aux 


%%%%%%%%%%%%%%%%%%%%%%%%%%%%%%%%%%%%%%%%%%%%%%%%%%%%%%%%%%%%%%%%%%%%%%%%%%%%%%%%%%%%%%%%%%%%%%%%%%%
\section{L'impact du numérique sur l'audiovisuel}\label{sec:motiv}
\e{
La révolution numérique initiée depuis une trentaine d'années en conjonction avec la révolution électronique et informatique, s'est progressivement imposée à tous types d'information et de contenus. 
%jusqu'à devenir prépondérante et indispensable dans un monde informatisé.
Ces mouvements d'informatisation des pratiques et de numérisation de l'information impactent les organisations et les métiers en redistribuant les tâches entre humains et machines. 
Dans le cadre de cette thèse, nous nous pencherons sur le cas de l'audiovisuel et de la numérisation de ce contenus et des informations associées.}

%L'audiovisuel représente à plus d'un titre un objet singulier.
%l'exploitation des contenus numériques est en passe de s'étendre à toutes les étapes du cycle de vie d'un objet audiovisuel. % ? pourquoi exploitation ?
Après une informatisation des étapes de postproduction (logiciels de montage, effet spéciaux etc.) puis des équipements de captation (caméra, micro etc.) et de la distribution du côté des diffuseurs, nous avons connu une véritable explosion d'appareils destinés au grand public (lecteur multimédia portable, appareil photo, téléphone portable, dictaphone etc.).


Plusieurs grands chantiers s'ouvrent désormais dans ce mouvement d'informatisation des étapes de la chaîne de production :
\begin{liste}
	\item L'archivage des contenus de manière à les faire rentrer dans l'histoire malgré la dégradation inexorable des supports. 
	Il ne s'agit pas simplement de leur permettre de survivre jusqu'à la prochaine génération d'appareils électroniques, mais également de garantir sa \ciel{lisibilité technique et culturelle}, (\cite[p.~12]{Bachimont2000}).

	\item La préproduction des contenus qui est presque inexistante et qui permettrait de récolter des informations sur les contenus avant même leur fabrication. 
	L'enjeu plus général est d'initier l'indexation des contenus au moment du \e{Scripting} et de la continuer tout au long de la chaîne, chaque étape pouvant rajouter des informations supplémentaires ou bien réévaluer les anciennes.
\end{liste}

À côté de la numérisation et de l'informatisation, il ne faudrait pas oublier le développement considérable des réseaux de télécommunications qui a grandement favorisé les échanges de contenus de manière illégale ou légale, de pairs à pairs (\e{Peer to Peer}), entre diffuseur et spectateurs (\e{Business to Consumers}) ou entre acteurs professionnels de la chaîne (\e{Business to Business}). 
Ainsi, numérisation et informatisation sont dorénavant implictement associées aux facilités de transfert de ces réseaux. 

Sans tenter de tirer toutes les conséquences de ces révolutions technologiques (électronique et informatisation, numérisation, mise en réseaux) sur les usages des spectateurs et des acteurs de la chaîne de production audiovisuelle, on retiendra trois constats important qui charpentent notre réflexion : 
\begin{liste}
	\item \eg{il existe un nombre croissant de contenus audiovisuels en circulation.}
	L'offre et la demande augmentent de même que les capacités de transfert des réseaux et la généralisation des appareils électroniques de captation et de visionnage.

	\item \eg{les pratiques en lien avec les contenus audiovisuels se développent et s'individualisent.}
	La consommation de ces contenus se joue de plus en plus à un niveau individuel depuis l'apparition d'appareils personnels de communication et de visionnage. 
	De même, la production de contenus est facilité par l'informatisation et les progrès des appareils électroniques de captation. 
	La création de contenus n'est donc plus l'apanage de grandes équipes spécialisées et lourdement équipées.
	
	\item \eg{la numérisation complique le maintien de l'unicité des contenus audiovisuels}. 
	%Le principe même du numérique (\ciel{ça a été manipulé} [Bachimont2010]) repose sur la représentation de l'information et le calcul. 
	L'environnement numérique, par rapport à l'analogique, est plus propice à la copie, le transfert, la fragmentation et la manipulation des contenus ramenés invariablement à une donnée muette quelle que soit sa nature (texte, son, image animée ou non etc.) ou sa signification.
	Ainsi coupé des sens et du sens, les contenus semblent perdre leur identité dans le \gui{monomédia} numérique (\cite[p.~13]{Bachimont2000}) dans lequel on a peine à les gérer pour ce qu'ils représentent. 	

	%plus manipulables au sens où chaque visionnage construit une forme perceptive du contenu 

	%est une reconstruction de leur forme perceptible par des appareils de . Il y a donc par définition une certaine instabilité 
	%mobile, versatile, altérable, mouvant, variable, instable
\end{liste} 

%Le développement de l'électronique a ainsi favorisé une  le nombre de contenus audiovisuels, à favoriser leur circulation
%On constate ainsi une augmentation des sources de contenus ainsi qu'une intensification de leur circulation qui ont des conséquences sur les usages des spectateurs et les acteurs de la chaîne de production.
Nous allons maintenant préciser cet argumentaire et présenter les enjeux posés par la numérisation, l'informatisation et le développement de l'électronique.




%%%%%%%%%%%%%%%%%%%%%%%%%%%%%%%%%%%%%%%%%%%%%%%
\subsection{La numérisation des contenus audiovisuels}\label{sec:num}
Le contenu audiovisuel est avant tout un objet temporel, c'est-à-dire un flux d'images et de sons qui s'écoule en un temps donné. 
Un des enjeux liés à l'audiovisuel constitue alors de trouver une technologie d'enregistrement et de manipulation de ces flux.

\paragraph{L'enregistrement analogique du flux}
Les techniques d'enregistrement analogique reposent sur la conversion continue du flux original en un signal analogue dont les variations s'effectuent sur une échelle physique différente, par exemple une fréquence sonore transformée en une tension électrique. 
L'enregistrement s'effectue sur différents types de supports, cassettes magnétiques, films argentiques etc. 
La caractéristique commune de ces supports est que l'accès au contenu enregistré ne peut se faire que de manière linéaire ou séquentielle, c'est-à-dire qu'il faut avancer ou rembobiner la bande jusqu'au point de départ avant de pouvoir commencer la lecture. 
Ainsi, chaque opération de manipulation de ces contenus compte toujours un temps pas forcément négligeable consacré à l'alignement avec le point de départ désiré. 
Pour bien s'en rendre compte, il suffit de se souvenir du temps qu'il fallait pour rembobiner la cassette de votre film préféré, puis du temps passé en avance rapide pour passer les inévitables publicités précédant le film.

\paragraph{Numérisation et délinéarisation de l'accés}
Avec le numérique, cette expérience autrefois familière a complètement disparu et se voit remplacée par un accès immédiat à n'importe quel moment du contenu audio-visuel. 
La numérisation des contenus consiste à discrétiser le flux en un ensemble de valeurs que l'on convertit ensuite en flux binaire. Ce flux est ensuite enregistré sur des supports de mémoire magnétiques (disquettes 3'1/2, disques durs etc.) optiques (CD, DVD etc.) ou électronique (mémoire vive dite RAM, mémoire flash etc.). 
Seules ces dernières garantissent un accès arbitraire ou délinéarisé aux données stockées en mémoire (n'importe quelle donnée, à n'importe quel moment). 
Les supports magnétiques et optiques proposent un accés séquentiel comme l'analogique, mais plus rapide et surtout que l'on peut coupler avec les mémoires électroniques afin de se rapprocher de leurs performances.

\paragraph{Fragmentation}
Par ailleurs, au-delà des avantages liées aux temps d'accès au contenu, le numérique facilite également sa fragmentation et sa manipulation.
Contrairement à l'analogique, le numérique permet de représenter de manière arbitraire tout type d'information puis d'effectuer des calculs sur ces représentations. 
Ainsi, on peut associer au contenu audiovisuel, d'autres contenus de différentes natures pour les enrichir et faciliter son exploitation ultérieure.

\paragraph{Discrétisation}
La discrétisation du flux audiovisuel quant-à-elle remet en cause la temporalité du contenu et permet de ce fait une fragmentation plus aisée et plus fine. 
Lorsque l'analogique effectue une transformation continue et analogue, le numérique définit une fréquence d'échantillonnage et quantifie les valeurs de la source sur une échelle arbitraire finie. 
C'est donc véritablement la fréquence d'échantillonnage qui constitue la première unité de réprésentation de l'information au-dessus du bit.
C'est donc en décidant du nombre de pixels pour représenter une image, ou aux nombres de valeurs par secondes prises pour représenter un son que l'on décide d'une première échelle de fragmentation. 
% Premier niveau de manip technique, ensuite il y a des USI
Toute autre fragmentation à l'échelle supérieure est potentiellement (re)constructible par calcul, pour autant qu'on possède une méthode opérationnalisable. 
%De même que chaque élément de cette fragmentation, chaque unité, devient adressable et donc manipulable par calcul. 
%\cite{Bachimont2000} parle ainsi d'utm, usi ...
% COUPURE DU SENS ?

% \g{== Révision à faire, introduire les UTM et USI ==}
\paragraph{Numériser, c'est informatiser le métier}
Parmi toutes les fragmentations possibles, il convient alors de déterminer leur pertinence par rapport aux types de calculs que l'on souhaite réaliser à chaque étape de la chaîne de production. 
Il s'agit ici de contrôler les calculs à effectuer en suivant les règles du métier qui en disposera.
Chaque étape ayant ses objectifs propres, les calculs effectués varient et s'opèrent à différents niveaux de fragmentation. 
Or la numérisation des contenus et de l'information s'effectue toujours en vue de leur manipulation par des programmes informatiques.
Ainsi, la numérisation entraîne toujours une informatisation qui impacte le métier dans son organisation et ses pratiques parce qu'il transforme les possibilités techniques qui le concerne.
% L'enjeu est donc double, d'une part identifier les niveaux de fragmentation pertinents pour chaque étape de la chaîne de production, et d'autre part de se donner les moyens de reconstruire la cohérence . 

% numériser => (a) fragmenter + (b) mettre en réseau => (a) besoin de conserver la cohérence de l'ensemble + (b) besoin d'autonomiser pour une future situation d'usage





%%%%%%%%%%%%%%%%%%%%%%%%%%%%%%%%%%%%%%%%%%%%%%%
\subsection{Le développement de l'électronique et des réseaux de télécommunications}\label{sec:electro}
L’explosion des appareils multimédia et des possibilités de transférer des contenus par les réseaux de télécommunication a promu de nouvelles pratiques de consommation et d'échanges des contenus tant chez les professionnels de l'audiovisuel que dans le grand public.
% 
Du côté des professionnels, on voit ainsi l’émergence de systèmes de production qui utilisent le réseau pour faire transiter les contenus entre les systèmes d'information de leurs différents départements. 
Ces systèmes reprennent les principes d'architecture multi-tiers utilisés sur le Web avec les contenus représentés par des fichiers ou des flux binaires, d'où leur appelation de \e{file-based production system}.
% 
Ce genre de système favorise également l’échange de fichiers entre organisations, puisque l’architecture permet d'exposer les données stockées de la même manière sur un réseau interne (intranet) ou externe (extranet, Internet).
% 
Du côté du grand public, les appareils portables acquièrent de plus en plus de connectivité avec leur environnement et les réseaux. 
De simples lecteurs à brancher en USB, les appareils sont passés au stade communicant avec la 3G, le Wifi ou le Bluetooth. 
La fonction d'échange se banalise et inversement, les appareils communicant comme les téléphones deviennent eux-même des stations multimédia à part entière, musique, photo, vidéo, courriel, sms etc.\\


Un autre facteur à noter, est que ces appareils portables sont personnels, c’est-à-dire qu’ils sont majoritairement utilisés par un seul individu contrairement à l’usage du téléviseur qui était et reste encore largement un objet collectif. 
Une autre distinction fondamentale se situe dans le fait que ce sont des appareils informatiques qui fonctionnent entièrement dans le numérique. 
Les possibilités d’interaction en font non plus de simples terminaux de lecture, mais potentiellement de véritables instruments de création et de communication. En effet, l'insertion de données et la capture de contenus (photos, sons, textos etc.) est prévu et le couplage avec des plates-formes de publication (réseaux sociaux, dépôts de contenus, CMS etc.) est de mieux en mieux réalisé.

De ce fait, en plus des capacités toujours plus importante de consommation de contenus, ces appareils favorisent eux aussi leur circulation ou la circulation d’information annexes. 
Le partage d’opinions s’est considérablement développé avec la vague d'applications Web dites sociales qui permettaient aux utilisateurs de créer du contenu sans avoir à maîtriser les arcanes techniques du Web. 
Cet ajout d’opinion, même s’il est parfois réduit au minimum à une marque d'appréciation, implique ainsi l’utilisateur dans le processus de diffusion du contenu en le portant à l'attention d'autres utilisateurs (ses contacts dans le cas des réseaux sociaux, ses lecteurs dans le cas d'un weblog ou autres CMS, un ensemble d'utilisateurs anonymes dans le cas de services sociaux tels que Delicious\footnote{Delicious : \url{http://delicious.com/} est une plate-forme de sauvegarde, d'indexation par mot-clé et de partage de marques-pages. Pour l'utilisateur il s'agit soit de sauvegarder et d'indexer ses marques-pages, soit de découvrir les marques-pages correspondant à tel ou tel mot(s)-clé(s) déjà sauvegardées par la communauté. Les grandes tendances d'indexation sont ainsi accessibles à tous, tout en permettant à chacun de développer son système d'indexation personnelle -- adaptation française du mot anglais \e{folksonomy}. Il est également possible de partager directement ses trouvailles avec d'autres utilisateurs par un système d'abonnement et de notification.} ou Digg \footnote{Digg : \url{http://digg.com/} est un site de marque-page social qui fonctione sur le principe du vote. Un utilisateur peut proposer une page qui est alors soumise aux votes des autres utilisateurs. Suivant le succès de la page, celle-ci sera mise en avant sur la page principale de Digg, ou bien mise de côté avec le reste des pages moins populaires, et finira par être supprimée.}).\\


Les appareils numériques multimédia possèdent également des capteurs de plus en plus performant et de moins en moins coûteux qui permettent au grand public de découvrir de nouvelles activités de création (photographie et retouche d’image, tournage et montage vidéo, prise de son et mixage audio etc.).
Cet abaissement du coût d’entrée dans la production a favorisé l'émergence d’une production amateur hétéroclite qui va du passant prenant une photo d’un évènement se déroulant devant ses yeux jusqu’à l’amateur qui pratique par amour mais avec l'exigence d’un professionnel. 
Cette production amateur rentre alors en concurrence avec la production professionnelle, voire la remplace dans certains cas (lorsqu’un passant est le seul témoin d’un évènement inattendu par exemple). 
La concurrence est d'autant plus forte depuis l’apparition de plates-formes de partages qui facilitent la distribution des contenus. 
De manière générale, l’opposition classique entre producteurs et consommateurs se brouille et les professionnels cherchent de plus en plus à mettre à contribution les amateurs dans leurs processus. 
On se dirige ainsi vers un modèle où professionnels et amateurs contribuent à divers degrés et divers moments au cycle de vie des contenus.


Avec l'informatisation des appareils multimédia s’est introduit la possibilité de personnaliser la communication avec l'utilisateur, d'y ajouter de l'interactivité et de connecter les utilisateurs entre eux. 
Ces nouvelles possibilités transforment les attendus et les pratiques du grand public. 
De ce fait, cela impacte les rapports avec les professionnels qui tendent à vouloir intégrer les contributions externes à leurs propres productions. 
Ainsi, il ne s’agit plus simplement de produire des contenus qui s'adressent indistinctement aux masses, mais de trouver des moyens de personnaliser son offre, de recommander des contenus, de faciliter la récupération de contenus, de diversifier les occasions de consommer ou de contribuer.








%%%%%%%%%%%%%%%%%%%%%%%%%%%%%%%%%%%%%%%%%%%%%%%
\section{Le projet MediaMap}\label{sec:mm}
% %%%%%%%%%%%%%%%%%%%%%%%%%
\paragraph{Objectifs}
Le projet MediaMap vise à développer des modèles et des applications pour promouvoir la production audiovisuelle collaborative articulant contenus professionnels et amateurs. 
En particulier, l'ambition est d'intégrer les amateurs et leurs contenus dans la chaîne de production professionnelle en améliorant d'une part la qualité technique et éditoriale des contenus fabriqués, et d'autre part en facilitant la collaboration et l'intercompréhension entre les différents acteurs de la chaîne.

La piste de travail retenue a été de construire des ontologies capables de représenter et décrire les contenus au fur et à mesure de leur processus de production. 
Ces informations serviraient de base de connaissances pour de nouvelles applications s'intégrant dès le début à la chaîne de production audiovisuelle, c'est-à-dire dès la conception du contenu. 
Les partenaires du projet ont ainsi développé des applications d'organisation du processus de production, de description du contenu, d'assistance au tournage ainsi qu'un moteur de recherche utilisant ces ontologies comme modèle d'information de référence.

%%%%%%%%%%%%%%%%%%%%%%%%%
\paragraph{Composition}
Le projet MediaMap a rassemblé une dizaine d'entreprises ainsi que deux équipes de recherche de l'Université de Technologie de Compiègne :
\begin{liste}
	\item l'équipe de recherche \pc{Information Connaissance Interaction} (ICI) chargée de la partie modélisation qui a abouti à la construction d'ontologies.

	\item l'équipe de recherche \pc{Automatique, Systèmes Embarqués, Robotique} (ASER) chargée de la conception d 'algorithmes d'analyse d'images et de vidéos.
\end{liste}


Parmi les entreprises du consortium, on compte les deux grandes chaînes de télévision publiques belges ainsi que de nombreuses PME belge ou française qui apportent leurs expertises dans différents domaines :
\begin{liste}
	\item \pc{BelgaVox} qui gère un des plus grands stocks d'archives audiovisuelles belges et produit des documentaires.

	\item \pc{Exalead} qui est un éditeur de solution de recherche pour les entreprises.

	\item \pc{Kane Consulting} qui propose des analyses du marché et des usages aux acteurs de la production audiovisuelle.

	\item \pc{Memnon} qui est spécialisé dans la numérisation, la documentation et l'archivage de contenus audio et vidéo.

	\item \pc{Perfect Memory} qui s'est créé pendant le projet afin d'accompagner les solutions du projet sur le marché grand public et prospecte également le marché professionnel.

	\item \pc{Skema} qui développe des applications de production de contenu audiovisuel amateur pour mobiles et caméras.

	\item \pc{Solution 2.0} qui est une agence de conception et de réalisation de plate-forme Web.

	\item la \pc{Radio-Télévision Belge de la communauté Française} (RTBF).

	\item \pc{Vitec Multimédia} qui développe et manufacture du matériel vidéo numérique.

	\item la \pc{Vlaamse Radio- en Televisieomroep} (VRT).
\end{liste}









%%%%%%%%%%%%%%%%%%%%%%%%%%%%%%%%%%%%%%%%%%%%%%%
\section*{Organisation du mémoire}\label{sec:plan}
\addcontentsline{toc}{section}{Organisation du mémoire}

\paragraph{Exposition}
\e{
La première partie de ce mémoire a pour objectif de présenter les problèmes qui se posent à la production audiovisuelle depuis son avancée vers la numérisation. 
Elle nous permet également de préciser la manière dont nous posons le problème, à la fois en terme métiers et avec nos lunettes de scientifiques.}

Le chapitre \g{\ref{chap:intro}. Introduction} nous sert à rappeler le contexte technologique général qui s'impose au monde de l'audiovisuel. 
En effet, la production audiovisuelle se dirige progressivement vers une numérisation et une mise en réseau de ses produits ainsi qu'une informatisation de ses pratiques qui n'est pas sans conséquences. 

Le chapitre \g{\ref{chap:problo}. Problématisation} nous permet de préciser comment ces tendances impactent le monde de l'audiovisuel.
Nous nous appuyerons sur l'étude du fonctionnement de la chaîne de production audiovisuelle classique (\ref{sec:prod}) pour dresser un bilan des attentes des professionnels vis-à-vis du numérique (\ref{sec:besoins}).
% En particulier, 
Nous précisons alors la manière dont nous posons le problème sur le plan métier, ce qui nous amènera à définir le problème scientifique. 
Sur le plan métier, le défi posé par le numérique consiste à passer d'une vision à l'échelle du document à une vision à l'échelle du fragment. 
En effet, le numérique favorise la fragmentation et la circulation des contenus qu'il s'agit alors de rendre autonome pour en permettre l'exploitation (\ref{sec:pmetiers}). 
Sur le plan scientifique, nous posons le problème en terme de modélisation des objets audiovisuels et des connaissances associées afin de construire une compréhension commune et dynamique pour tous les acteurs impliqués dans la production audiovisuelle (\ref{sec:scien}).
% Nous concluons en expliquant comment nous mobilisons diverses disciplines scientifiques pour construire une réponse aux problèmes posés (\ref{sec:posd}).
% proposent d'une part des outils et des méthodes de modélisation, et d'autre part proposent des modélisations de l'audiovisuel (\ref{sec:posd}).


\paragraph{État de l'art}
\e{
La deuxième partie de ce mémoire vise à étudier des outils, méthodes et langages de modélisation. Elle permet également d'étudier les modélisations existantes des objets audiovisuels, mais égalements des connaissances métiers qui y sont associées afin de faciliter leur exploitation.}

Le chapitre \g{\ref{chap:omod}. Outils de modélisation} commence par clarifier les besoins de modélisations à partir d'un scénario de production collaborative impliquant des acteurs professionnels et amateurs (\ref{sec:cdcf}).
Ces besoins ne se situent pas seulement au niveau de la modélisation conceptuelle, mais également sur le plan des jargons utilisées pour présenter ces concepts à des contributeurs de la production.
Ainsi, nous examinons les définitions des concepts de \gui{systèmes d'organisation de connaissances} (SOC) et en particulier les relations qu'entretiennent \gui{terminologie} et d'\gui{ontologie} (\ref{sec:defs}).
Nous étudions ensuite les langages de structuration et de représentation des connaissances qui permettent de modéliser ces deux types de SOC (\ref{sec:mods}).

Le chapitre \g{\ref{chap:mav}. Modélisations de l'audiovisuel} a pour objectif de mettre en rapport les représentations des professionnels de la production avec diverses communautés scientifiques, en vue de clarifier la définition d'un objet audiovisuel (\ref{sec:dav}) et de sa réutilisation (\ref{sec:gest}).
Cette étude s'appuie sur la poursuite du scénario d'usage du chapitre précédent, et met en exergue la nécessité de fragmenter la modélisation des objets audiovisuels pour favoriser leur réutilisation (\ref{sec:cdc-av}).
Nous étudions ensuite les solutions utilisées dans l'industrie pour gérer la circulation des programmes (\ref{sec:wrapper}) et les décrire (\ref{sec:desc}).
Cette dernière partie analyse les méthodes de fabrication de ces description ainsi que leur nature, puis les modélisations développées.
Une attention particulière est portée à de MPEG-7 (\ref{sec:mpeg7}), qui tient lieu de référence à de nombreux travaux de formalisation sous la forme d'ontologie (\ref{sec:mpeg7etc}).
Nous présentons également des approches de description plus proche de la perspective de la production audiovisuelle (\ref{sec:insitu}).
% à revoir


\paragraph{Contribution}
\e{La troisième partie de ce mémoire présente notre contribution conceptuelle et informatique aux problèmes de modélisations que nous avons soulevés.
Nous détaillons nos choix de représentation pour opérationnaliser notre contribution en une ontologie informatique.}

Le chapitre \g{\ref{chap:mod}. Approche et modélisation} revient sur les langages et les modélisations étudiés dans le chapitre précédent et introduit les principes de notre approche (\ref{sec:principes}).
Notre positionnement au sein de la chaîne de production audiovisuelle, nous permet de modéliser le déroulement de la chaîne, la définition d'une structure documentaire première, puis les fragments audiovisuels construits ainsi que les connaissances qui s'y rapportent.
Nous détaillons la modélisation conceptuelle en parties, chacune correspondant à un besoin fonctionnel, en expliquant leur mise en relation par des exemples (\ref{sec:concept}).

Le chapitre \g{\ref{chap:op}. Mise en oeuvre} présente la représentation informatique de notre conceptualisation.
Nous argumentons d'abord nos choix de langage et montrons comment nous les utilisons (\ref{sec:ln}). 
Nous détaillons ensuite la structuration de notre ontologie et son articulation avec des thésaurus et des bases de faits (\ref{sec:op}).

\paragraph{Discussion}
\e{La dernière partie de ce mémoire montre comment notre contribution est utilisé dans le cadre du projet MediaMap et ouvre la discussion sur ce travail de thèse.}

Le chapitre \g{\ref{chap:app}. Applications et expérimentations} introduit les diverses applications qui ont été développé (\ref{sec:app}) par nos partenaires et les expérimentations qu'elles ont permis de mener (\ref{sec:xp}).
Il s'agit d'éclairer l'appropriation de notre travail dans le cadre de scénarios de production audiovisuelle collaborative.
En particulier, nous expliquons quelle partie de l'ontologie est mobilisée par les applications pour construire ou bien intégrer des connaissances sur la production, ses contributeurs et ses produits.

La \g{Conclusion} remet en perspective notre contribution et les applications développées par rapport aux problèmes métiers et scientifiques posés.
Nous ouvrons également la discussion sur la poursuite de nos recherches et l'avenir des applications du projet MediaMap.

\chapter{Problématisation}\label{chap:problo}
\minitoc
\section{La production audiovisuelle}\label{sec:metier}
% faire le point sur ce que le numérique pourrait apporté à la production audiovisuelle
\e{
L'objectif de cette première section est de donner des éléments de compréhension du métier de la production audiovisuelle.
Dans un premier temps, nous rappelons comment s'organise classiquement la fabrication des objets audiovisuels (\ref{sec:prod}).
Ensuite, nous détaillons les notions et des mots utilisées par les professionnels pour parler de l'objet audiovisuel à construire (\ref{sec:docvoc}).
Enfin, à partir de ces éléments nous précisons les besoins que rencontrent ces professionnels avec l'émergence du numérique et de la mise en réseau (\ref{sec:besoins}). 
}


%%%%%%%%%%%%%%%%%%%%%%%%%%%%%%%%%%%%%%%%%%%%%%%
\subsection{Déroulement de la chaîne de production}\label{sec:prod}
% \addcontentsline{toc}{subsection}{La production audiovisuelle}
La création de documents audiovisuels est une entreprise collective qui suit généralement ce qu'on appelle la chaîne de production audiovisuelle. 
Organisée de manière linéaire, cette chaîne peut se décomposer en 4 grandes étapes -- voir Figure \ref{img:intro:chaine}.

\begin{liste} 
	\item \g{Préproduction} : Cette première étape consiste à construire une ébauche du futur document audiovisuel de manière à prévoir les moyens à engager pour le réaliser.

	\item \g{Production} : Cette étape vise à tourner plusieurs prises pour chaque partie du document et à commencer à faire le tri entre elles.

	\item \g{Postproduction} : L'objectif est d'assembler les prises et de les retoucher de manière à former un document cohérent et adapté à une audience et un mode de distribution.

	\item \g{Exploitation} : Une fois le document achevé, on valorise sa construction par une distribution auprès d'une audience ainsi qu'un archivage qui permettra de le réutiliser ultérieurement.
\end{liste}

\begin{figure}[ht!]
\centering
\includegraphics[width=\textwidth]{images/Workflow-Thesis-v0.png}
\caption{La chaîne de production audiovisuelle classique}
\label{img:intro:chaine}
\end{figure}


%%%%%%%%%%%%%%%%%%%%%%%%%
\subsubsection*{Préproduction}\label{sec:preprod}
L'objectif de cette étape est de construire une ébauche du futur document audiovisuel, de manière à prévoir les moyens à engager pour le réaliser. 
On distingue deux phases de préparation, l'écriture ou \e{Scripting} et le \e{Planning} ou planification.

À partir d'une idée, l'écriture du document se déroule en plusieurs étapes où l'on fixe progressivement le message à faire passer ainsi que sa forme audiovisuelle. 
Une fiction par exemple s'écrit à partir d'un résumé de l'histoire, puis on développe les scènes, les personnages, les lieux, les dialogues, etc. jusqu'à arriver à la manière dont cette histoire sera racontée à l'écran. 
On fixe ainsi le type de plan à filmer, les mouvements de caméra, le type de lumière, etc. 
Dans certaines grosses productions, on aura même recours au dessin pour aider à la représentation visuelle.


Toutes ces informations sur le contenu et la forme du futur document audiovisuel servent à estimer le temps nécessaire et les moyens à engager. 
L'estimation du coût d'une production est un élément essentiel pour la production audiovisuelle. 
En effet, dès le départ le producteur doit estimer la rentabilité du futur document afin de voir quels moyens il peut engager. 
Il s'agit bel et bien d'investissement, parfois très lourd, surtout en comparaison avec d'autres industries culturelles – musique, littérature, radio, presse – à l'exception récente du jeu vidéo. 
Contrainte budgétaire et forme esthétique sont donc en négociation dans cette première étape.


%%%%%%%%%%%%%%%%%%%%%%%%%
\subsubsection*{Production}
Une fois l'ébauche et les moyens déterminés, la production vise à réaliser chaque partie du document une à une puis à les assembler en un montage cohérent.

La phase de \e{Fabrication} consiste à capter du réel que l'on a mis en scène. La captation d'un évènement est réalisée grâce à des appareils d'enregistrement (caméra, microphones, etc.). 
La mise en scène du réel se construit à partir d'un ensemble de techniques, d'équipements et d'accessoires (lumière, costume, décors, maquillage, etc.) qui permettent d'obtenir l'image et le son souhaités. 
La technique est ainsi mobilisée dans un objectif esthétique. 
Dans le cas d'une fiction, la fabrication du contenu se fait en fonction du planning de mobilisation des personnes et des équipements plutôt que suivant l'ordre chronologique de l'histoire. 
On regroupe ainsi le tournage des scènes dans tel lieu ou avec tel équipement afin de réduire les coûts. Au final, la fabrication
produit des séquences vidéo et audio qu'il s'agit ensuite de redécouper pour mieux les assembler.

La phase de \e{Derushing} consiste à examiner les séquences réalisées pendant la fabrication et à les trier en vue de faciliter le montage. 
Par exemple, le tournage d'une fiction produit des séquences vidéo comprenant plusieurs prises d'une même scène et souvent des séquences captées par des caméras ayant des angles de prise de vue différents. 
Il faut donc redécouper les séquences en prises puis regrouper les prises d'une même scène. 
Le montage se fera d'autant plus facilement qu'on saura également identifier la qualité et les avantages de chaque prise d'une même scène.


%%%%%%%%%%%%%%%%%%%%%%%%%
\subsubsection*{Postproduction}
Lorsque le contenu audiovisuel est fabriqué et trié, il reste à le structurer en un document et à le conditionner en fonction de sa future exploitation.

La phase de \e{Montage} consiste à agencer des petites séquences de vidéo et d'audio pour construire la structure du document audiovisuel.
L'agencement des plans, leur durée et la transition entre ces plans constituent les ressorts esthétiques propres à l'audiovisuel. 
Ils participent à la transmission du message en ceci qu'ils servent de raccord entre les plans, comme le souligne le réalisateur \pc{Sergueï Mikhaïlovitch Eisenstein}\footnote{L'origine de cette fameuse citation est assez obscur, on la retrouve dans de nombreux documents, dont cet article --\cite{Montage et Réalisme}-- datant des années 60 et extrait de la revue québecoise \gui{Séquence : La revue du cinéma}.} :

\begin{cico}
Le montage est l 'art d'exprimer ou de signifier par le rapport de deux plans juxtaposés de telle sorte que cette juxtaposition fasse naître l'idée ou exprime quelque chose qui n'est contenu dans aucun des deux plans pris séparément. L'ensemble est supérieur à la somme des parties.
\end{cico}

Après le montage, une phase de \e{Finition} est nécessaire pour intégrer d'autres ressources au document en fonction de sa distribution. 
Pour une diffusion antenne d'un reportage, on ajoute le logo de la chaîne de télévision, un jingle, le nom des intervenants ou des titres, etc. 
Une diffusion sur DVD nécessitera l'ajout des conditions
légales d'usages, l'intégration d'un menu de navigation, etc. 
La distribution détermine également un format d'encapsulation (.avi, .mkv, .ogg etc.) et un encodage du contenu audiovisuel (MPEG-2, H.264, MPEG-4, theora vorbis etc.). 
Le résultat de cette phase de post-production est de fournir des documents prêts à l'usage et dans certains cas plusieurs variantes pour chacun des modes de distribution envisagés.


%%%%%%%%%%%%%%%%%%%%%%%%%
\subsubsection*{Exploitation}
Une fois le document achevé, on valorise sa construction par une distribution auprès d'une audience ainsi qu'un archivage qui permettra de le réutiliser ultérieurement.

La phase de \e{Distribution} consiste à rendre le document matériellement accessible à une audience. 
Il s'agit d'un transfert qui peut faire l'objet d'une transaction commerciale ou s'appuyer sur d'autres types de modèles économiques (publicité entre autres). 
La nature du transfert varie et porte à la fois sur les modalités d'accès au contenu et les droits d'usages.

La phase d'\e{Archivage} consiste le plus souvent à stocker le contenu diffusé afin de pouvoir le réutiliser tel quel plus tard, soit en le rediffusant, soit en le vendant à un autre diffuseur. 
C'est aussi généralement la phase où l'on construit, récupère et attache des descriptions du contenu au document audiovisuel. 
En effet, l'archivage n'a de sens que s'il permet de retrouver, voire redécouvrir, les documents archivés.
\ciel{Au plan économique, un film est un bien informationnel, d'expérience, caractérisé par une très forte densité d'informations. Son exploitation s'organise autour de versions différentes, distribuées sur des marchés distincts par des acteurs spécialisés.} (\cite{Blanc2006}).%Gilles Le Blanc - Innovations numériques, distribution et différenciation  : le cas de la projection numérique dans le cinéma.




%<TODO
%TODO>
\subsection{Documents et vocabulaire de la production (n)}\label{sec:docvoc}
\e{
Dans cette section, nous présentons des définitions utilisées dans le milieu professionnel et tirées du \gui{Dictionnaire technique du Cinéma (\cite{Pinel2008})} afin de présenter les principaux documents et le vocabulaire utilisé pendant la phase de préproduction. 
Il s'agit ainsi de mettre en exergue la manière dont se construit une description textuelle de l'objet audiovisuel en devenir ainsi que le vocabulaire utilisé pour faire cette description.}
% Des éléments qui nous serviront à mieux cerner les problèmes qui se posent à la production audiovisuelle.

\paragraph{La notion de plan}
La notion de plan peut se présenter de diverses manières suivant le point de vue adopté. 
Du point de vue technique, il s'agit d'une série d'images (ou photogrammes) qui sont enregistré par un appareil de capatation (une caméra) au cours d'une même prise de vue : \ciel{série de photogrammes enregistrés au cours d'une même prise} (\cite{Pinel2008}).
Il s'agit donc de l'enregistrement qui est effectué entre le moment où l'on presse sur le bouton pour lancer et celui où l'on arrête l'enregistrement.
Cette définition, certes robuste, ne permet pas pour autant de caractériser les plans ou de les comparer entre eux. 
Ainsi, on s'intéresse au point de vue de l'écriture filmique et de la réalisation qui considère non seulement l'action d'enregistrement, mais également la manière dont il est effectué : 

\ciel{
Fragment de temps et d'espace enregistré d'un seul tenant, selon un point de vue déterminé, et donnant à la projection le sentiment de la continuité d'une même \e{image en mouvement}.} (\cite{Pinel2008})

Cette définition complète la précédente en considérant le rapport au sujet (le point de vue) et la temporalité du plan et son rapport à un ensemble d'autres plans (la continuité). 
Elle permet d'envisager les plans comme des éléments de base que l'on assemblera ensuite pour construire un objet audiovisuel : 

\ciel{
La notion de plan est apparue [\dots] lorsqu'on a abandonné le point de vue unique du tableau pour envisager le sujet sous différents angles et à différentes distances et lorsque, par la grâce du montage, on a mis en relation ces plans entre eux.
[\dots]
Si le photogramme représente l'unité technique de la prise de vues, la scène et la séquence les unités narratives de l'oeuvre cinématographique, le plan est la cellule fondamentale de l'écriture du film, de sa préparation jusqu'à la copie standard.} (\cite{Pinel2008})

Dans cette citation, on voit aussi émerger l'idée qu'il existe différents niveaux d'analyse dans l'objet audiovisuel : 
\begin{liste}
	\item un \g{niveau technique} avec le photogramme mais également le pixel dans le numérique.
	\item un \g{niveau narratif} avec la scène (unité de temps et de lieu dans l'histoire) et la séquence (suite de scènes constituant une action dramatique autonome ou distincte).
	\item \g{un niveau dont le plan est l'unité de base qui sert tout au long de la chaîne de production}. 
	On parle d'unité, car il s'agit du résultat de base d'une prise de vue, c'est-à-dire de la fabrication (production) de l'objet audiovisuel. 
	La pré-production, étape de préparation du tournage, utilise donc naturellement cette unité. 
	De même, le montage consiste à organiser ces plans pour former un ensemble cohérent, quitte à les ajuster (raccourcir, allonger, modification du cadrage etc.). 
	Ainsi, il semble que cette unité servent non seulement d'unité de travail de référence pour la chaîne de production\footnote{La production sonore d'un objet audiovisuel ne s'organise pas forcément de la même manière que celle de la production de l'image. Néanmoins, par la force des choses, la construction de l'image prime bien souvent sur celle du son et son unité de base sert donc de référence même pour la production sonore.}, mais également du premier niveau signifiant propre à l'audiovisuel (l'image seul pouvant être rattaché à la photographie).
\end{liste}

Ces distinctions nous permettent de définir le plan selon les caractéristiques suivantes : 
\begin{liste}
	\item \g{l'échelle relative du cadre par rapport au(x) sujet(s)} (personnages, objets etc.).
	C'est ce qui permet de définir un ensemble de \gui{valeurs de plan}, gros plan, plan américain etc. que nous définirons par la suite.
	
	\item \g{l'angle de la prise de vue} (plongée, contre-plongée, cadre incliné etc.)

	\item \g{le mouvement de la caméra} et d'autres paramètres de son objectif (panoramique, travelling, rotation, zoom, focale, focus etc.)

	\item \g{l'articulation des plans entre eux}, d'une part en terme de durée, mais aussi en terme de transition et d'impression de continuité entre les plans. 
	Par exemple, il existe des règles de cadrage et de montage pour aider les spectateurs à situer les personnes sur un plateau\footnote{La règle des 180° oblige ainsi à maintenir les mêmes relations est-à-gauche/droite-de entre les personnes, de manière à ce que le spectateur puisse se souvenir des positions des interlocuteurs sur un plateau. En inversant ces relations topographiques, on donne l'impression au spectateurs que les personnes ont échangé leurs places alors que c'est juste la caméra qui a changé de point de vue. Il s'agit donc d'une règle très importante pour assurer la continuité et la compréhension du spectateur.}.
\end{liste}


\paragraph{Valeurs de plan utilisées dans un script}
La valeur de plan est un des éléments le plus utilisé pour distinguer les plans entre eux, notamment au moment de l'écriture. 
Plutôt que de préciser les paramètres optiques de la caméra (considérés comme des détails très difficile à préciser à l'avance), le réalisateur préfère parler d'un type de plan pour donner une idée générale de l'image à obtenir. 
Le Tableau \ref{tab:vplans} présente les principales valeurs de plans utilisées par les professionnels, avec leur abréviations et leur(s) dénomination(s) anglaise(s) tandis que la Figure \ref{img:intro:plans} en propose une illustration.

\begin{table}[ht!]
   \begin{center}
		\begin{tabularx}{\textwidth}{|p{100pt}|X|p{100pt}|}
		   \hline
	\pc{Dénomination française} & \pc{Défintion} & \pc{Dénomination anglaise} \\ \hline\hline
 	\g{très gros plan (t.g.p.)} & plan cadrant une partie du visage, un détail du corps (un oeil, une bouche, un doigt etc.) ou le détail d'un objet. & extreme close-up, e.c.u. ; big close-up, b.c.u.\\ \hline

 	\g{gros plan (g.p.)} & plan isolant un visage, généralement cadré à la hauteur du noeud de cravate, ou un autre détail du corps (plan de détail ; insert), voire \e{tout ou partie d'un petit objet}. & close-up, c.u.\\ \hline

	\g{plan rapproché} & plan cadrant le(s) personnage(s) au niveau de la taille (plan rapproché taille, p.r.t.) ou de la poitrine (plan rapproché poitrine, p.r.p.). & medium close-up, m.c.u.\\ \hline
	
	\g{plan ceinture} & plan coupant les personnages au niveau de la ceinture & belt shot\\ \hline 

	\g{plan américain} (p.a.) & plan coupant les personnages à mi-cuisse & american shot ; medium close-shot, m.c.s.\\ \hline
	
	\g{plan moyen (p.m.) ou plan en pied} & plan présentant le(s) personnage(s) en pied. Il existe également des variations de ce plan qui sont nommées \e{serré} (aussi nommé plan américain large) ou \e{large} et qui font varier légèrement le cadrage. & medium shot, middle-shot, mid-shot, m.s. ; full shot, f.s.\\ \hline
	

	\g{plan de demi-ensemble (p.d.e., 1/2e.)} & plan mettant en place les personnages dans leur milieu en cadrant une bonne partie du décor & medium-long shot, m.l.s.\\ \hline

	\g{plan d'ensemble (p.e.)} & plan cadrant l'ensemble du décor construit & long shot, l.s.\\ \hline
	
	\g{plan de grand ensemble (p.g.e.)} & plan cadrant l'ensemble du décor construit de grande envergure. & very long shot, v.l.s.\\ \hline

	\g{plan général (p.g.)} & plan couvrant un vaste ensemble qui situe le décor construit dans son cadre : le décor dans le décor. & master shot ; extreme long shot, e.l.s\\ \hline 
		\end{tabularx}
		\caption{Valeurs de plans : du plus précis au plus général \label{tab:vplans}}
   \end{center}
\end{table}

% \begin{liste}
% 	\item \g{très gros plan} (t.g.p.) : \ciel{plan cadrant une partie du visage, un détail du corps (un oeil, une bouche, un doigt etc.) ou le détail d'un objet}. [extreme close-up, e.c.u. ; big close-up, b.c.u.]

% 	\item \g{gros plan} (g.p.) : \ciel{plan isolant un visage, généralement cadré à la hauteur du noeud de cravate, ou un autre détail du corps} (plan de détail ; insert), voire \ciel{tout ou partie d'un petit objet}. [close-up, c.u.]

% 	\item \g{plan rapproché} : \ciel{plan cadrant le(s) personnage(s) au niveau de la taille (plan rapproché taille, p.r.t.) ou de la poitrine (plan rapproché poitrine, p.r.p.).} [medium close-up, m.c.u.]
	
% 	\item \g{plan ceinture} : plan coupant les personnages au niveau de la ceinture [belt shot] 

% 	\item \g{plan américain} (p.a.) : \ciel{plan coupant les personnages à mi-cuisse} [american shot ; medium close-shot, m.c.s.]
	
% 	\item \g{plan moyen} (p.m.) ou \g{plan en pied} : \ciel{plan présentant le(s) personnage(s) en pied.} [medium shot, middle-shot, mid-shot, m.s. ; full shot, f.s.]
% 	Il existe également des variations de ce plan qui sont nommées \ciel{serré} (aussi nommé plan américain large) ou \ciel{large} et qui font varier légèrement le cadrage.

% 	\item \g{plan de demi-ensemble} (p.d.e., 1/2e.): \ciel{plan mettant en place les personnages dans leur milieu en cadrant une bonne partie du décor}. [medium-long shot, m.l.s.]

% 	\item \g{plan d'ensemble} (p.e.) : \ciel{plan cadrant l'ensemble du décor construit}. [long shot, l.s.]
	
% 	\item \g{plan de grand ensemble} (p.g.e.) : \ciel{plan cadrant l'ensemble du décor construit de grande envergure}. [very long shot, v.l.s.]

% 	\item \g{plan général} (p.g.) : \ciel{plan couvrant un vaste ensemble qui situe le décor construit dans son cadre : le décor dans le décor}. [master shot ; extreme long shot, e.l.s] 
% \end{liste}
%TODO:source

\begin{figure}[ht!]
\centering
\includegraphics[width=0.4\textwidth]{./images/ValeurPlan-v1.png}
\caption{Différentes valeurs de plan pour le cadrage d'un personnage à l'écran}
\label{img:intro:plans}
\end{figure}


\paragraph{Quelques documents de (pré)production}
%TODO:description + source
La pré-production se repose sur différents types de documents qui permettent de faire émerger progressivement la structure narrative ou documentaire, le découpage en plans et tous les détails de réalisation nécessaire à une bonne préparation du tournage. 
On notera que chacun de ces documents constitue un jalon dans la préparation du projet et que le script, résultat final de cette écriture, constitue une sorte de cahier des charges de l'objet audiovisuel à fabriquer.
\begin{liste}
	\item \g{sujet} : \ciel{matière première du film enrichie et développée lors de la préparation puis mise en forme au cours de la réalisation et du montage.} 
	
	\item \g{synopsis} : \ciel{exposé sommaire en quelques lignes, voire en quelques pages, du contenu dramatique ou documentaire d'un film}. 
	À noter que ce document est également utilisé plus tard dans la chaîne de production, notamment pour être transmis aux journalistes ou aux archivistes.
	De plus, il constitue la première mise en forme narrative du contenu du film, à la différence du sujet qui ne se constitue que de quelques idées directrices. 

	\item en cas d'adaptation d'une oeuvre littéraire en un objet audiovisuel, on développe un \g{traitement} : 
	\ciel{Travail littéraire préparatoire effectué à partir d'une oeuvre pré-existante ou d'une oeuvre originale pour assurer sa transmutation en termes cinématographiques.}

	\item lorsqu'on développe un objet audiovisuel original, à défaut de traitement on peut parler de \g{scénario} :
	\ciel{description de l'action d'un film épousant la forme \e{littéraire} du récit, rendant compte des articulations narratives et comportant une ébauche des dialogues, quelquefois la description plus précise de certaines scènes-clefs.}

	\item lorsque le besoin de préciser encore la construction de la narration, les auteurs peuvent construire une \g{continuité (dialoguée)} : \ciel{étape de la préparation écrite du film qui permet d'enrichir le traitement en développant chronologiquement les fragments d'action, en mettant au point le détail de chaque scène et en précisant le dialogue.}

	\item \g{plan de tournage} : \ciel{ultime travail de préparation effectué par le réalisateur avant le tournage. Il consiste à fragmenter la continuité en unités cinématographiques de temps et d'espace : les plans.}

	\item \g{script} : \ciel{dernière mouture du scénario, guide complet du tournage}.La Figure \ref{img:intro:script} présente un exemple de script extrait d'un film récent écrit et réalisé par Quentin Tarantino.
\end{liste}

\begin{figure}[ht!]
\centering
\includegraphics[width=0.7\textwidth]{images/ScriptExample-v1.png}
\caption{Extrait du script de Kill Bill écrit et réalisé par Quentin Tarantino}
\label{img:intro:script} 
\end{figure}

% des réécritures de documents et des mises à jours qui pourraient être réalisées par des machines, des informations qui pourraient être transmises automatiquement à travers un réseau numérique d'information


% un vocabulaire bien défini qui fait l'objet de nombreux dictionnaires, donc prêt à être formalisé


% des équipes qui sont réduites lors de tournage en extérieur




%%%%%%%%%%%%%%%%%%%%%%%%%%%%%%%%%%%%%%%%%%%%%%%%%%%%%%%%%%%%%%%%%%%%%%%%%%%%%%%%%%%%%%%%%%%%%%%%%%%
\subsection{Besoins Métiers (m)}\label{sec:besoins}
\e{
Face à une numérisation qui fragmente les contenus, une mise en réseau qui facilite la fabrication amateur, intensifie la circulation de ces fragments (\ref{sec:motiv}), la production audiovisuelle rencontre de nouveaux défis qui remettent en jeu son organisation et sa manière de se représenter le monde. 
Ainsi d'une part la fragmentation ne doit pas mettre en péril la cohérence de l'ensemble et d'autre part, la circulation des contenus ne doit pas compromettre l'exploitation future du tout ou des parties. 
L'ouverture d'une chaîne de production à des acteurs tiers implique de clarifier les attendus de chacun. 
Lorsqu'il s'agit de faire fabriquer ou de récupérer du contenu, il devient nécessaire pour le client de décrire la commande de contenu au fournisseur, de même que le fournisseur doit décrire à son client le contenu livré pour faciliter son exploitation. 
Ainsi, on souhaite adjoindre aux contenus des descriptions qui permettent de faciliter leur recherche et leur manipulation.
}


Pour les professionnels de la production audiovisuelle, le défi porte à la fois sur l'organisation de leur chaîne de production et sur la gestion de leurs produits :
\begin{liste}
	\item[(1)] \g{comment transformer la chaîne de production afin de l'ajuster à la diversification des formes de fabrication et de distribution des contenus, mais aussi aux changements dans les pratiques de consommation des audiences ?}

	\item[(2)] \g{comment passer d'une gestion de fichiers à une gestion de contenus audiovisuels considérés comme des objets numériques fragmentés dont il s'agit de garantir l'autonomie dès leur conception et jusque dans leurs différents cadres d'exploitation ?}
\end{liste}

% Problèmes métiers ? Tant que ça ne devient pas une solution, mais des tendances à prendre en compte 

On peut ensuite détailler ces défis en objectifs plus précis :
\begin{liste}
	\item[(1a)] \e{accorder contribution amateur et production professionnelle pour fabriquer ou valoriser du contenu.}

	L'amélioration croissante des capteurs des appareils multimédia ajoutée aux capacités de communication offrent au grand public de plus en plus de manières de participer aux processus de fabrication ou de diffusion des contenus.
	Les possibilités accrues de participation au processus médiatique (participation à l'émission, envoi de contenu, propagation via ses contacts, commentaires etc.) valorisent le spectateur, le contenu et la plate-forme de diffusion (comme d'une certaine manière peut le faire le bouche à oreille).

	De manière générale, l'intégration de contenus externes dans une chaîne de production professionnelle ne s'envisage  qu'à partir d'un certain niveau de qualité du contenu livré.  
	Paradoxalement, dans certains cas les signes d'une production amateur (tremblements, caméra à l'épaule etc.) peuvent être revendiquées comme des marque de style qui suggère une collaboration avec le public ou une proximité avec une réalité éloignée des images diffusées par les médias.
	Ainsi, les professionnels souhaitent encadrer plus ou moins fortement la production amateur par des indications, recommendations, obligations.\\


	\item[(1b)] \e{créer de nouvelles étapes dans la chaîne visant à réutiliser les contenus existants et les adpater à de nouveaux modes de consommation.}

	L'augmentation de l'offre de contenus accessibles aux spectateurs (chaînes, enregistrements, balladodiffusion, vidéo à la demande etc.) se traduit par une mise en concurrence accrue des contenus diffusés par les professionnels.
	Le contrôle de l'offre n'étant plus atteignable, il faut adopter de nouvelles stratégies de valorisation des contenus produits ou diffusés pour maintenir leur visibilité et leur rentabilité. 
	Une autre approche consiste à fournir un service de recommandation aux spectateurs et ainsi rentrer dans une démarche de fidélisation. 

	Par ailleurs, l'augmentation des terminaux de lecture multimédia et leur portabilité offrent de plus en plus d'occasions aux spectateurs de consommer des contenus. 
	Par exemple, les situations de mobilités peuvent impliquer des capacités de transfert diminués, un écran plus petit, des temps de disponibilités plus courts etc.
	Il semble alors que la production doivent évoluer pour fournir de nouveaux formats ou des formes retravaillées de contenus existants.

	Dans tous les cas, cela implique de se consacrer à des tâches d'éditorialisation des contenus pour répondre aux exigences et aux attentes de ces nouveaux modes de consommation.\\
% \end{liste}


% \begin{liste}
	\item[(2a)] \e{gérer l'intégration de contenu externes, les variations d'un même contenu pour les rattacher à un même objet numérique.} 
	% représentation

	L'utilisation de contenus provenant de sources externes de même que la production de multiples variations d'un même contenu augmente le nombre de ressources à gérer. 
	De plus, les relations entre ces différentes ressources nécessitent d'être clarifiées et explicitées dans le système de gestion. 
	% chaîne éditoriale ? 
	
	Il ne s'agit plus simplement de gérer des fichiers mais un ensemble de fichiers et de données qui constituent un ensemble cohérent et fragmenté que l'on nomme un objet numérique. 
	Cet objet doit intégrer à la fois les diverses sources qui le composent mais aussi des variations correspondants aux exploitations visées, des descriptions et tout ce qui permet de garantir son autonomie. 
	Il doit également s'agir d'un objet \e{métier} car son statut, son organisation, sa sémantique correspondent à la vision d'un métier, à la manière dont il pense le monde. 

	% Ainsi, on ne souhaite plus gérer des fichiers mais des objets numériques qui doivent acquérir un statut, une sémantique correspondant à la manière dont les métiers de la production les considèrent.\\
	% qui possède une valeur et une sémantique propre à un contexte d'usage. 	


	\item[(2b)] \e{associer des descriptions aux contenus pour faciliter leur exploitation dans un environnement numérique.}
	% description

	L'augmentation des contenus audiovisuels en circulation, la diversification de leurs modes d'exploitation compliquent la gestion des contenus.  
	Afin de favoriser la réutilisation de ces contenus, il faut pouvoir leur attacher des informations pertinentes pour les professionnels qui les manipulent. 
	La description du contenu peut varier suivant les besoins de chaque métier impliqué dans la chaîne de production. 
	Les opérations n'étant pas les mêmes, les descriptions de ces opérations varient donc également et sont nécessaires pour faciliter la réutilisation du contenu. 
	%de manière à faciliter modalités d'exploitation envisagées  

	Lorsque la réutilisation et production s'entremêlent, il est également nécessaire de construire les descriptions en même temps que le contenu. 
	De cette manière on récupère ou on réévalue l'information à mesure de l'avancée dans la chaîne. 

	Cela nécessite d'informatiser l'étape de pré-production de la chaîne et de modéliser les informations utilisées par les professionnels. 
	%commencer plus tôt, avoir plusieurs niveaux/types de description, raccrocher les bons éléments à la représentation de l'objet numérique
	
	%et embarquent des descriptions explicitant leurs modalités d'exploitation.

\end{liste}

\e{
En guise de synthèse, nous pouvons dire qu'il s'agit de constituer des objets audiovisuels autonomes dans les chaînes de production audiovisuelle. 
Nous précisons ce caractère autonome, car ces objets seront porteurs de leur propre description et associés à des connaissances sur l'organisation de la chaîne de production dans lesquels ils évoluent.
Ainsi, ces objets audiovisuels pourront être (ré)introduits à n'importe quelle étape d'une chaîne de production et fourniront aux contributeurs concernés des informations propres à faciliter leur (ré)utilisation.
}
	% \item \eg{valoriser et éditorialiser les contenus existants pour les rendres plus visibles, plus attrayants auprès des audiences ciblées.}
	% L'augmentation de l'offre de contenus accessibles aux spectateurs (chaînes, enregistrements, balladodiffusion, vidéo à la demande etc.) se traduit par une mise en concurrence accrue des contenus diffusés par les professionnels.

	% Le contrôle de l'offre n'étant plus atteignable, il faut adopter de nouvelles stratégies de valorisation des contenus produits ou diffusés pour maintenir leur visibilité et leur rentabilité. 
	% Une autre approche consiste à fournir un service de recommandation aux spectateurs et ainsi rentrer dans une démarche de fidélisation. 
	% Cela implique de se consacrer à des tâches d'éditorialisation des contenus pour des audiences plus ciblées.
	
	% \item \eg{produire de nouvelles formes de contenus ou adapter les rmes existantes pour satisfaires aux nouveaux modes de distribution/consommation.}
	% L'augmentation des terminaux de lecture multimédia et leur portabilité offrent de plus en plus d'occasions aux spectateurs de consommer des contenus. Par exemple, les situations de mobilités peuvent impliquer des capacités de transfert diminués, un écran plus petit, des temps de disponibilités plus courts etc.
	% Il semble alors s'ouvrir une place pour de nouveaux formats ou des formes retravaillés de contenus existants. 
	% Il s'agit de faire de la production multi-support et d'adapter les contenus en fonction des conditions de distribution et de l'audience visé (réutilisation).
	
	% \item \eg{articuler la contribution amateur avec la chaîne de production professionnelle.}
	% L'amélioration croissante des capteurs des appareils multimédia ajoutée aux capacités de communication offrent au grand public de plus en plus de manières de participer aux processus de fabrication ou de diffusion des contenus.
	% Les possibilités accrues de participation au processus médiatique (participation à l'émission, envoi de contenu, propagation via ses contacts, commentaires etc.) valorisent le spectateur, le contenu et la plate-forme de diffusion.

	% Cependant, il faut être capable d'intégrer ces contributions externes au sein de la production professionnelle en les encadrant plus ou moins fortement, par des indications, recommendations ou des contraintes.

	% \item \eg{gérer les objets audiovisuels dès le début et tout au long de leur cycle de vie.}
	% %
	
	% Une circulation plus importante des contenus implique de trouver un moyen de gérer non plus des fichiers mais des objets numériques qui unifient plusieurs variations d'un même contenu et embarquent des descriptions explicitant leurs modalités d'exploitation.



% numériser => (a) fragmenter + (b) mettre en réseau => (a) besoin de conserver la cohérence de l'ensemble + (b) besoin d'autonomiser pour une future situation d'usage








%%%%%%%%%%%%%%%%%%%%%%%%%%%%%%%%%%%%%%%%%%%%%%%%%%%%%%%%%%%%%%%%%%%%%%%%%%%%%%%%%%%%%%%%%%%%%%%%%%%
\newpage
\section{Problèmes}\label{sec:prob}



%%%%%%%%%%%%%%%%%%%%%%%%%%%%%%%%%%%%%%%%%%%%%%%%%%%%%%%%%%%%%%%%%%%%%%%%%%%%%%%%%%%%%%%%%%%%%%%%%%%
\subsection{Problèmes métiers (m)}\label{sec:pmetiers}
% Nos champs d'applications sont : 
% Prenant acte des besoins de la production audiovisuelle nous distinguons deux problèmes métiers (1) identifier le(s) niveau(x) de fragmentation et (2) le(s) type(s) de description susceptibles de favoriser la fabrication mixte, la circulation et la réutilisation des contenus.
% Ces problèmes remettent en cause à la fois la représentation classique des contenus et leur description. 
% de manière à faciliter (1) la production mixte amateur-professionnel et (2) leur réutilisation dans de nouveaux contextes d'exploitation.
% l'informatisation du début de la chaîne de production, préproduction 
% l'inscription formelle de l'écriture audiovisuelle ?

\e{
L'objectif central qui se pose à la production audiovisuelle est de constituer des objets audiovisuels autonomes et donc (ré)utilisable à n'importe quel étape de la chaîne. 
Or, dans la chaîne de production classique les programmes n'émergent qu'à la fin de la chaîne et sont gérés d'une pièce. 
Il n'y a pas forcément de place pour les éléments de contenu intermédiaires, et ce sont justement à la modélisation de ces fragments que l'on s'attaque. 
Ces fragments doivent devenir des éléments documentaires qui possèdent leur unité propre de même que les objets finis que sont les programmes. 
Une des difficultés réside dans cette articulation entre des fragments et le tout ou l'ensemble que constitue les programmes. 
Par ailleurs, il existe des problèmes sous-jacents à cette fragmentation documentaire : 
}
\begin{liste}
	\item \e{identifier quels niveaux de fragments peuvent prétendre à ce genre de transformation}.
	Si le numérique permet de fragmenter à l'envie, il faut cependant prendre en compte les pratiques du métier pour identifier les niveaux de fragmentation pertinents ou déjà utilisé dans le métier mais non modélisé.
	Par exemple, la prise de vue est le résultat d'une activité de tournage, pour autant elle ne constitue pas un objet éditorial comme peut l'être une interview.
	De plus, il s'agit également de déterminer comment manipuler ces fragments comme des objets à part entière sans compromettre l'articulation de l'ensemble. 
	Par exemple, une interview peut s'intégrer dans un journal télévisé ou bien un reportage dans des versions plus ou moins courtes. 
	Pour autant, il s'agit du fruit d'une même activité, simplement le montage, et donc le résultat, est différent suivant le programme dans lequel l'interview s'insère.\\
	
	\item \e{identifier quelles informations et connaissances doivent être rattacher à ces fragments pour les rendre autonome}.
	D'une manière similaire à la démarche pour les objets entiers, certaines connaissances, notamment relatives au contexte de production, doivent être attachées au fragment pour garantir sa réutilisation et sa cohérence. 
	En reprenant l'exemple de la prise de vue, on peut lui associer le bout de script qui a prescrit ce qu'elle devait montrer. 
	Si l'on pousse encore cette logique, il faut également incorporer le document qui définit le programme pour lequel on a tourné cette prise de vue, la personne qui l'a effectué, les équipements utilisés etc. 
	Ainsi, on étend la modélisation de l'objet audiovisuel à son contexte de production et tout ce qui renforce les possibilités de recherche, de manipulation, de gestion et de transformation de ces objets. 
	De plus, s'ajoute à cela la question de la collecte de ces informations. 
	En effet, la saisie de ces informations au sein d'un système d'information et leur utilisation par les acteurs de la chaîne n'est pas une simple formalité.
	Ce problème pousse également dans le sens d'une modélisation plus contextuelle, de manière à proposer un environnement de travail adapté et utilisable aux contributeurs de la chaîne.\\
	
	% \item \e{}.
\end{liste}


% Notre problème se situe dans le croisement de la représentation des contenus et la représentation des activités humaines qui construisent, manipulent, éditent, transforment, publient et documentent ces contenus. 
% Cet angle de recherche nous amène donc à considérer non pas le contenu audiovisuel dans son ensemble, une fois terminé et validé, mais la construction de tout ces fragments qui le composent. 
% Il nous faut aussi considérer, non pas seulement le contenu audiovisuel, mais aussi d'autres informations, d'autres documents qui constituent son contexte de production, dans un sens très général. 
% Chaque prise de vue constitue donc un objet à représenter en tant que tel, tout autant que le bout de script qui a prescrit ce que cette prise de vue devait montrer. 
% Si l'on pousse encore cette logique, on peut alors représenter de même le document qui définit le programme pour lequel on a tourné cette prise de vue, la personne qui l'a effectué, les équipements utilisés etc. 
% Ainsi, on étend la représentation du contenu à son contexte de production entendu comme toutes les informations qui renforceront les possibilités de recherche, de manipulation, de gestion et de transformation de ces contenus. 

%%%%%%%%%%%%%%%%%%%%%%%%%%%%%%%%%%%%%%%%%%%%%%%
\subsection{Problèmes scientifiques (m)}\label{sec:scien}
Au fur et à mesure que la circulation des contenus s'intensifie, il y a un besoin grandissant de faciliter l'échange d'information tant à la fois sur le plan informatique, que sur le plan humain. 
De plus, l'ouverture de la chaîne de production à de nouveaux contributeurs (amateurs et professionnels) ne fait qu'accentuer la disparité des connaissances et des systèmes utilisés. 
Afin de construire une compréhension commune à tous les contributeurs au cycle de vie, on fabrique un modèle conceptuel capable d'intégrer et de mettre en relation leurs connaissances. 
Il s'agit là d'un apport par rapport à la situation existante où le consensus n'existait pas, ou alors de manière éphémère, locale au sein d'une équipe.
L'objectif est de fluidifier les échanges d'information et de contenus en formalisant les connaissances utilisées pour :
\begin{liste}
% modéliser tous les objets de la chaîne de production audiovisuelle
	\item[(A)] \g{modéliser les objets construits au fil de la chaîne de production audiovisuelle}.
	\item[(B)] \g{modéliser les connaissances sur ces objets} (descriptions, contexte de production, contribution au cycle de vie).
	% \item[(B)] décrire les objets audiovisuels
	% \item[(C)] représenter la contribution de chacun des acteurs au cycle de vie des objets audiovisuels
\end{liste}

\e{
Ainsi, notre problème de recherche général s'articule autour de la modélisation des connaissances et des informations que les contributeurs construisent, utilisent, échangent au cours du cycle de vie des objets audiovisuels. 
Cette modélisation constitue une première étape dans la mise en place d'un système d'information servant à mieux gérer les objets audiovisuels et médier la communication entre systèmes informatiques tout autant qu'entre contributeurs humains.
Après avoir numérisé les contenus audiovisuels, on souhaite transformer chaque élément les composant en objet documentaire et documenter leur cycle de vie.\\}



\g{(A)} Le problème est d'organiser la gestion des objets audiovisuels en proposant une modélisation capable de faciliter leur identification, leur manipulation et leur réutilisation tout au long de leur cycle de vie. 
En particulier,	l'objet audiovisuel professionnel est produit de manière collective, chaque contributeur apportant un élément à l'ensemble. 
Ces contributions doivent donc pouvoir être identifiées comme appartenant à un ensemble, de même que chaque élément doit pouvoir être considéré pour soi afin d'être intégré dans un autre ensemble (réutilisation).

Pour cela on adopte une représentation des différents niveaux d'abstraction des objets audiovisuels numériques de façon à rétablir les liens entre les différentes versions ou copies d'un même contenu, quelque soit la nature des variations entre elles (encodage, format d'encapsulation, montage, finition, langue etc.).
La distinction entre différents niveaux de modélisation (technique, esthétique, éditorial etc.) doit permettre de construire une représentation dynamique de l'objet audiovisuel qui suit l'avancement du processus de production.\\
% La distinction entre différents niveaux de modélisation (technique, esthétique, éditorial etc.) doit permettre de construire une représentation de l'objet audiovisuel au fur et à mesure de l'avancement du processus de production.\\


\g{(B)} Le problème est d'attacher plusieurs types de connaissances aux objets audiovisuels de manière à les rendre autonomes dans leur circulation et leur réutilisation. 
Afin de faciliter l'échange d'information et la réutilisation des objets audiovisuels entre différents contextes, il faut modéliser des connaissances sur ces objets qui sont parfois déjà existantes mais non formalisées, ou bien qu'il faut rendre compréhensibles.
En effet, l'échange d'information dans la production audiovisuelle est primordial et s'effectue entre métiers et organisations différents, voire avec des amateurs. 
En particulier, on souhaite s'appuyer sur le vocabulaire de l'écriture filmique utilisé dans des documents de préproduction pour spécifier les résultats attendus de la production.
L'information contenue dans ces documents est importante mais repose sur des conventions plus ou moins tacites qu'il faut expliciter pour les professionnels, expliquer pour les amateurs.
La formalisation des ces éléments devra donc pouvoir être lu et modifié tout au long du cycle de vie des objets audiovisuels par tout types de contributeurs.

% échange d'info, adaptation par l'explicitation du vocabulaire et la contribution au cycle de vie
La formalisation de l'écriture filmique permettra d'adapter la présentation de l'information en fonction des connaissances, de l'implication du contributeur dans la chaîne (rôle, tâche, niveau de compétences etc.), de son référentiel professionnel ou linguistique. 
Il s'agit alors d'établir des correspondances entre les connaissances connues par le lecteur d'une information et celles utilisées par la personne qui l'a exprimé.
Ainsi, d'une part on explicite l'expression de l'information, ce qui permet l'adpatation, et facilite son interprétation ultérieure.
Le processus prend tout son sens lorsqu'il s'agit de traduire un concept de la réalisation audiovisuelle pour guider un amateur dans son tournage.
% Par exemple, l'action écrite par l'auteur et transformé en scène par le réalisateur, doit ensuite être tourné un caméraman, des acteurs etc. 


% réutilisation, attachement des connaissances pertinentes pour manipuler ou exploiter chaque fragment ou l'objet en entier
De plus, les descriptions utilisées, en plus de permettre de spécifier le résultat attendu de la production, doivent permettre de faciliter la recherche, la manipulation et la réutilisation des objets audiovisuels.
Pour cela, il faut articuler ces connaissances aux objets et aux fragments qui les composent. 
Le vocabulaire de l'écriture filmique sera également précieux, puisqu'il nous donne une unité de base, le plan, ainsi que ces caractéristiques qui permettent de le distinguer des autres. 
La recherche dans des dépôts de contenus se ferra ainsi de manière similaire à la commande de contenu à d'autres contributeurs, qu'ils soient professionnels ou amateurs.
./



% Il s'agit donc de définir un modèle de description susceptible d'être utilisé par les contributeurs professionnels ou amateurs, à toutes les étapes de la chaîne de production. 


% Afin de décrire les contenus, on souhaite s'appuyer sur le vocabulaire de l'écriture audiovisuelle utilisé dans les différentes étapes de la chaîne et notamment dès la préproduction. 
% Cette écriture repose sur un vocabulaire des techniques de réalisation audiovisuelle (prise de vue, transition, composition de l'image etc.) renvoyant à des effets largement connus dans le milieu de l'audiovisuel et chez les cinéphiles. 
% L'écriture est utilisée avant la fabrication du contenu pour la spécifier, puis pendant la fabrication pour enregistrer les différences. 
% Une formalisation de ce vocabulaire permettrait de construire une description textuelle d'un contenu à partir d'une description objective de la réalisation (réglages des appareils, position des acteurs etc.).\\



% \g{(C)} Le problème est de représenter et faciliter l'échange d'information entre des contributeurs hétérogènes dans leurs connaissances et leur implication dans la chaîne de production.
% Une première diffculté réside dans l'articulation entre les connaissances des contributeurs qui expriment l'information et ceux qui l'interprèteront.
% La seconde diffculté consiste dans l'articulation des représentations du cycle de vie, de l'objet audiovisuel et des descriptions qui leurs sont associées. 
% En effet, chaque contributeur peut participer à la constitution d'informations associées au contenu en cours de sa production.
% Ces informations varient en fonction de l'implication du contributeur dans la chaîne (rôle, tâche, niveau de compétences etc.). 
% De plus, les informations construites à un moment sont susceptibles d'être utilisées plus tard dans la chaîne, par un contributeur ne partageant pas forcément les mêmes connaissances ou le même référentiel professionnel ou linguistique.

% On cherche alors à réaliser une adaptation de la forme d'expression de ces informations afin de faciliter le déroulement du processus de production. 
% Dans un premier temps, on explicite les connaissances utilisées par un premier utilisateur pour exprimer une information. 
% Ensuite, on établit une correspondance avec les connaissances connues d'un autre utilisateur et on adapte au besoin la forme de d'expression de cette information pour faciliter son interprétation. 
% L'adpatation qui en résulte prend tout son sens lorsqu'il s'agit de traduire un concept de la réalisation audiovisuelle pour guider un amateur dans son tournage.


%%%%%%%%%%%%%%%%%%%%%%%%%%%%%%%%%%%%%%%%%%%%%%%
\section{Positionnement Disciplinaire (n,i)}\label{sec:posd}
[Ingénierie des connaissances ; Media Asset Management ; Gestion électronique de Documents ; Ingénierie documentaire]
% ingénierie des connaissances (représentation des connaissances) ingénierie des inscriptions numériques de connaissances, dont les documents
% ingénierie documentaire (modélisation des documents propres à la production audiovisuelle)
% indexation et gestion des connaissances (description des contenus audiovisuels)
% la ged s'occupe de la gestion de documents, nous proposons de gérer des fragments de documents, de gérer leur construction en plusieurs étapes, par plusieurs acteurs et dans le cadre de différentes missions.

% Pour définir cet ensemble d'informations qui forment le contexte de production, nous nous sommes appuyés sur les partenaires du projet MediaMap. 



% des réécritures de documents et des mises à jours qui pourraient être réalisées par des machines, des informations qui pourraient être transmises automatiquement à travers un réseau numérique d'information
% un vocabulaire bien défini qui fait l'objet de nombreux dictionnaires, donc prêt à être formalisé
% des équipes qui sont réduites lors de tournage en extérieur

% à mettre dans le pos. disciplinaire, comment on aborde les problèmes posées
% [Nous avons ainsi dégagé plusieurs perspectives métiers qui nous ont servi de guide pour identifier les échanges d'informations les plus importants ainsi que le vocabulaire utilisé pour les exprimer. 
% Chacune de ces perspective possède un objectif propre et des spécificités, cependant il apparaît qu'un langage commun est utilisé par tous les acteurs de la production. 
% En se concentrant sur la description d'un contenu existant ou à venir, ce langage permet à ces acteurs de communiquer entre eux. Le réalisateur qui spécifie un attendu dans son script, les caméraman qui réalisent le cadrage, les opérateurs lumières etc. 
% Tous utilisent ce langage pour imaginer le résultat à produire et en déduire les gestes à opérer. 
% Les usages n'étant jamais complètement figé, chaque organisation développe ses propres idiomatismes de langage. 
% Dans ce cas, la collaboration entre organisations impliquent de pouvoir réaliser des ajustements dans l'expression de la description du contenu. 
% De même, la collaboration avec des contributeurs amateurs soulève un problème de compréhension de ce langage (et donc de l'attendu) mais aussi de connaissances des gestes à opérer (pour produire le résultat attendu).
% Ainsi, à mesure que la circulation des contenus s'intensifie, que les besoins de collaboration augmentent, naît un besoin grandissant d'explicitation des échanges d'information afin de dégager une vue d'ensemble de la chaîne de production, de ses acteurs, de leurs interactions, de leurs produits. 

% Notre proposition consiste à modéliser ces éléments et à en informatiser l'accès de manière à fluidifier les échanges de contenus et faciliter la compréhension des informations afférentes.] 


% \cleardoublepage





%%%%%%%%%%%%%%%%%%%%%%%%%%%%%%%%%%%%%%%%%%%%%%%%%%%%%%%%%%%%%%%%%%%%%%%%%%%%%%%%%%%%%%%%%%%%%%%%%%%
%%%%%%%%%%%%%%%%%%%%%%%%%%%%%%%%%%%%%%%%%%%%%%%%%%%%%%%%%%%%%%%%%%%%%%%%%%%%%%%%%%%%%%%%%%%%%%%%%%%
\part*{État de l'Art}
%%%%%%%%%%%%%%%%%%%%%%%%%%%%%%%%%%%%%%%%%%%%%%%%%%%%%%%%%%%%%%%%%%%%%%%%%%%%%%%%%%%%%%%%%%%%%%%%%%%
%%%%%%%%%%%%%%%%%%%%%%%%%%%%%%%%%%%%%%%%%%%%%%%%%%%%%%%%%%%%%%%%%%%%
%%%%%%%%%%%%%%%%%%%%%%%%%%%%%%%%%%%%%%%%%%%%%%%%%%%%%%%%%%%%%%%%%%%%
%%%%%%%%%%%%%%%%%%%%%%%%%%%%%%%%%%%%%%%%%%%%%%%%%%%%%%%%%%%%%%%%%%%%
%%%%%%%%%%%%%%%%%%%%%%%%%%%%%%%%%%%%%%%%%%%%%%%%%%%%%%%%%%%%%%%%%%%%
\chapter{Outils de modélisation}\label{chap:omod}
\minitoc

% pourquoi on parle de SOC ? Parce qu'on souhaite représenter des objets audiovisuels, leur production, leur description et que ceci ne peut se faire sans une représentation du vocabulaire utilisé dans la production audiovisuelle. 

Dans le cadre de la production audiovisuelle collaborative, qui implique à la fois des amateurs et des professionnels, un des enjeux que nous avons noté (\ref{sec:scien}) est de rendre plus compréhensible l'échange d'informations entre contributeurs. 
D'une part, nous avons des practiciens professionnels utilisant un ou plusieurs vocabulaires métiers suffisamment définis pour que l'on puisse en faire des dictionnaires (\cite{Journot2008}; \cite{Pinel2008}). 
Ces personnes font usage de la langue d'une manière précise, régie par des conventions et portée par une conception de la production audiovisuelle et de ses objets, que l'on suppose stabilisée, au moins localement (au sein d'un même pays, d'une même école de pensée, organisation, équipe etc.).
Toutefois, on s'attend à des variations dans les usages de la langue, au même titre que l'on suppose qu'il existe des variations entre pratiques des gens du métiers.
Cependant, il semble qu'il s'agit seulement de variations et qu'il soit possible alors d'identifier les éléments communs et distincts.

D'autre part, nous avons les amateurs, qui n'ont pas le support de ces conventions travaillées au quotidien.
Ils sont, au mieux, des intermittents éclairés qui ont saisi le sens d'éléments de langage propres aux métiers (par des lectures, des rencontres, des formations etc.). 
Il faut donc supposer qu'il y a tout à expliquer à ces amateurs, plutôt que de parier sur leur compréhension innée des métiers de la production audiovisuelle.
En particulier s'il s'agit de demander du contenu à des amateurs sous la forme d'un script de tournage, il faudra alors trouver un moyen d'expliciter cette commande et d'expliquer ou d'assister sa réalisation. 

Ainsi, l'écart entre collaborateurs (qu'ils soient tous professionnels, ou un mélange d'amateurs et de professionnels) peut se situer sur différents niveaux : 
\begin{liste} 
	\item au niveau du vocabulaire métier, comme les mots utilisés dans le script pour désigner tel ou tel type de plan etc. 
	Dans le cas d'amateurs, il faut supposer que ces mots sont inconnus ou méconnus ; dans le cas de professionnels, on préfèrera expliciter le vocabulaire afin d'éviter toute confusion. 
	
	\item au niveau des connaissances et de la manière de conceptualiser le métier, le cycle de production et ses objets. 
	Là encore, un amateur ne connaît pas ou très peu les détails classiques des méthodes de production professionnelle. 
	De plus, la production audiovisuelle étant organisée en projets distincts, les méthodes peuvent fortement varier entre la production d'un documentaire et d'une émission de variétés. 
	Chaque genre, chaque équipe aura donc ses propres objets, ses propres méthodes qu'il faut alors expliciter aux autres professionnels pour s'assurer de leur collaboration. 

	\item sur le plan pratique, il faut également remarquer que les compétences peuvent également varier fortement entre métiers, suivant les genres de production audiovisuelle. 
	Ainsi, en plus d'expliciter les échanges d'informations, il serait également souhaitable de proposer une assistance aux collaborateurs amateurs ou professionnels pour s'assurer que le résultat produit correspond bien à l'attente initiale. 	
\end{liste}


Nous présentons dans une première section un exemple de commande de tournage qui illustre ces différents écarts en impliquant des communautés d'amateurs et de professionnels (\ref{sec:cdcf}). 
Ce scénario d'usage nous permet de préciser les besoins en modélisation exprimés précédemment (\ref{sec:prob}).
Notamment, il apparaît nécessaire de représenter à la fois la ou les conceptualisation(s), le(s) vocabulaire(s) utilisé(s) ainsi que les résultats attendus par les acteurs de la chaîne de production audiovisuelle. 
Ainsi, nous nous intéresserons à l'utilisation de divers Systèmes d'Organisations de Connaissances (SOC, \cite{Zacklad2010}) pour mettre en place un partage d'information normalisée. 
Après un retour sur les définitions principales que nous utiliserons, nous examinerons les langages, modèles et normes existants qui permettent de représenter des SOC (\ref{sec:defs}).
La distinction entre terminologie et ontologie nous permet de détailler le fonctionnement d'une méthode de construction d'ontologie différentielle (\ref{sec:construction}).
Enfin, nous présenterons les langages permettant de représenter ces SOCs (\ref{sec:mods}). 

%Dans une seconde partie, nous examinerons les modèles de l'audiovisuel existants.




%%%%%%%%%%%%%%%%%%%%%%%%%%%%%%%%%%%%%%%%%%%%%%%%%%%%%%%%%%%%%%%%%%%%
%%%%%%%%%%%%%%%%%%%%%%%%%%%%%%%%%%%%%%%%%%%%%%%%%%%%%%%%%%%%%%%%%%%%
\section{Cahier des charges fonctionnel (n)}\label{sec:cdcf}


%%%%%%%%%%%%%%%%%%%%%%%%%%%%%%%%%%%%%%%%%%%%%%%%%%%%%%%%%%%%%%%%%%%%
\subsection{Scénario de commande de tournage}\label{sec:scenar}
Considérons comme cas d'étude une commande de tournage en vue de réaliser des reportages sur des évènements culturels de type concert ou opéra. 
Il met en jeu trois communautés en collaboration :

\begin{itemize}
	\item la RTBF (Radio Télévision Belge Francophone) établit des commandes de contenu dans un jargon métier propre. Son objectif est d'externaliser dès que possible la réalisation de la commande. Cela implique une compréhension commune sur le contenu à réaliser qui passe par un accord sur la manière de décrire la commande. 
	
	\item le contenu commandé est tourné soit par la VRT (Radio-Télévision Flamande) qui utilise un jargon différent de la RTBF, soit des amateurs qui ne connaissent pas les concepts de la réalisation audiovisuelle. Dans le premier cas, la conceptualisation est commune, seuls les termes changent. Dans le second cas, il s'agit d'expliquer et d'illustrer les concepts utilisés.\\
\end{itemize}

Le développement d'une application d'assistant de tournage pour guider les amateurs paraît souhaitable pour amener le contenu filmé à un niveau de qualité exploitable. 
Il faut cependant faire la distinction entre les propositions de dépôt spontané de contenu (comme le pratique une chaîne d'information telle que BFM\footnote{La rubrique témoins BFM permet à un utilisateur de déposer des photos ou vidéos sur le site. 
Après modération, le contenu est diffusé et peut même faire l'objet d'une vente. Voir http$:$//temoins.bfmtv.com/}) et les appels à contribution où le professionnel passe commande auprès d'amateurs en détaillant ses exigences. 

Dans ce dernier cadre, on souhaite fournir un plan de tournage au caméraman afin de guider sa prise de vue. 
Le plan de tournage est construit à partir de recommandations rédigées par un réalisateur (position, cadrage, lumière, etc.) utilisant un vocabulaire métier. 
L'originalité de l'application est d'adapter l'information présentée au caméraman suivant ses capacités (amateur, professionnel) ou son employeur si c'est un professionnel travaillant dans une tierce organisation. 
On suppose ainsi que malgré quelques variations dans le vocabulaire utilisé, les professionnels de l'audiovisuel utilisent les mêmes concepts pour décrire le contenu. 
Par exemple, la notion de cadrage fait appel à des concepts de valeurs de plan indiquant la portion visible d'un personnage à l'écran (voir Figure \ref{img:intro:script}, page \pageref{img:intro:script}).
Un \textit{plan américain} indique ainsi que le personnage principal est cadré de la tête jusqu'au dessus des genoux. 
Le terme est utilisé en Europe en rappel à son emploi caractéristique dans les films américains des années 1910-1940, notamment dans les westerns où il permettait de montrer l'ensemble du pistolet à la ceinture des personnages\footnote{Roger Boussinot, l'Encyclopédie du Cinéma, Bordas.}. 
Ce cadrage est aussi appelé \textit{plan 3/4} et en anglais 3/4 shot, medium-long shot ou american shot pour traduire l'expression popularisée en Europe. 
Si le terme utilisé varie suivant le lieu et la littérature de référence, la définition de ce type de cadrage est sans équivoque. 

%=============
% \begin{figure}[htb]
% \centering
% \includegraphics[width=0.5\textwidth]{./images/ValeurPlan-v1.png}
% \caption{Différentes valeurs de plan pour le cadrage d'un personnage à l'écran}
% \label{fig:cadrage}
% \end{figure}
%=============

Les amateurs quand à eux ignorent ces concepts et n'ont pas été initiés à ces pratiques. Ils ont donc besoin d'explications et d'illustrations pour comprendre les recommandations du réalisateur. 
Dans le cas du cadrage, une illustration graphique est d'autant plus pertinente. 
L'enjeu se situe donc dans la collaboration entre un prescripteur et un opérateur qui doivent s'accorder sur le contenu à produire malgré la différence de vocabulaire. 
% Un exemple des différences de présentation entre amateur et professionnel est illustré figure \ref{fig:prescription}.


% %%=============
% \begin{figure}[htb]
% \centering
% \includegraphics[width=0.3\textwidth]{./images/ShootingRecommandation-v1.png}
% \caption{Exemple de prescription de tournage à destination de professionnels (en haut) ou d'amateurs (en bas)}
% \label{fig:eda:prescription}
% \end{figure}
% %%=============


%%%%%%%%%%%%%%%%%%%%%%%%%%%%%%%%%%%%%%%%%%%%%%%%%%%%%%%%%%%%%%%%%%%%
\subsection{Besoins en modélisation}\label{sec:bm}
La mise en place d'une telle application nécessite de représenter le vocabulaire de la réalisation audiovisuelle dans toutes ses variations possibles et de le documenter suffisamment afin de le rendre compréhensible pour des novices. 
Cet objectif nous amène à considérer la construction d'une ressource termino-ontologique. L'ontologie permet de représenter les concepts partagés par les professionels de la réalisation audiovisuelle et la terminologie permet de capturer les différentes formes d'expression associées à ces concepts. 

La spécificité de notre problématique est de considérer la collaboration de communautés hétérogènes par leur degré de compréhension des concepts ou leur utilisation de la terminologie. 
Ceci nous amène à envisager la terminologie comme un moyen d'associer à des éléments ontologiques (concept, relation, instances) une chaîne lexicale ou des ressources média. 
Chaque chaîne ou ressource s'adresse en particulier à une communauté dont les membres partagent une capacité d'interprétation commune. 
Il n'existe donc plus une terminologie de référence par langue, mais des terminologies pour chaque communauté d'utilisateurs. 
On remarquera que notre acception de la terminologie sert bien à normaliser les pratiques linguistiques entre les membres d'une même organisation. 
En plus de cela, elle permet de fixer la manière de s'adresser à d'autres communautés.

Par ailleurs, les types de réalisations sont divers et nécessitent des concepts spécifiques pour être décrits. 
Une fiction se structure en séquences et en scènes alors que les documentaires ou magazines d'information se composent de sujets. 
La variabilité des types de contenu à filmer implique donc de pouvoir étendre le fond conceptuel initial pour représenter de nouveaux usages. 
De la même manière, la collaboration avec de nouveaux partenaires nécessite de pouvoir ajouter de nouvelles terminologies au fond conceptuel existant. 
Ontologie et terminologie doivent se gérer de manière indépendante. A partir de ces besoins, nous définissons maintenant les exigences en terme de modélisation. 

Nos besoins en modélisation peuvent être exprimés par les assertions suivantes:
\begin{enumerate}
	\item[(\e{A1}]\e{: multi-jargon}) la variabilité des pratiques linguistiques des organisations et des communautés implique d'associer plusieurs termes à un même concept. 
	Il n'y a pas de choix des termes préférés par une communauté mais une \e{correspondance} entre les termes d'une ou plusieurs communautés, quels que soient la langue, le jargon et le code d'écriture utilisé.
	
	\item[(\e{A2}]\e{: documentation}) la variabilité de compréhension des communautés implique d'associer des explications (chaîne lexicale) ou des illustrations (ressource média) aux concepts afin d'en enrichir la documentation. 
	
	\item[(\e{A3}]\e{: gestion, évolution}) la variabilité des cas de collaboration implique de pouvoir étendre la conceptualisation initiale ou la terminologie pour s'adapter à de nouvelles pratiques ou de nouvelles communautés. 
	Cela implique une gestion et une évolution indépendante de l'ontologie et de la terminologie. 
\end{enumerate}


Dans le cas d'une demande de cadrage en plan américain, la demande est d'abord exprimée dans le jargon de la RTBF puis traduite dans le jargon de la VRT (plan américain pour la RTBF, plan 3/4 pour la VRT) [\g{A1 : multi-jargon}]. 
Ensuite, pour les amateurs, la terminologie est enrichie par des illustrations [\g{A2 : documentation}]. Enfin, un nouveau concept de cadrage est ajouté (plan américain large ou plan moyen serré) [\g{A3 : gestion, évolution}] en vue d'une nouvelle coopération avec la VRT. En plus de cela, le problème de la langue (français et flamand) s'ajoute à la question des jargons métiers. 



\section{Les Systèmes d'Organisation de Connaissances (m)}\label{chap:defs}
%Distinguer entre terme et concept ; dictionnaire, thésaurus, ontologies etc.

\e{
L'objectif de cette section est de clarifier ce qui appartient au domaine de la  linguistique et ce qui relève du domaine conceptuel, en vue d'identifier les notions qui nous serviront à spécifier une solution au cahier des charges dressés dans la section précédente. 
La confusion qui nous intéresse concerne principalement la définition des ontologies par rapport à d'autres SOC tels ques les thésaurus, notamment du fait qu'on utilise parfois les mêmes langages pour les représenter.}

La définition des SOC proposée par \cite{Zacklad2010}, étend celle de \cite{Hodge2000} à \ciel{l'ensemble des formes d'écritures codifiées participant à la description documentaire primaire ou secondaire d'une situation}. 
L'ensemble défini par \citeauthor{Hodge2000} comprend ainsi tout type de :
\begin{liste}
	\item \e{liste de termes} (fichiers d'autorités, glossaires, dictionnaires, répertoires géographiques)
	\item \e{schème de classification/catégorisation} (vedettes-matières, taxonomie)
	\item \e{schème qui se structure par le types de relations qui unit ses membres} (thésaurus, réseaux sémantiques, ontologies). 
\end{liste}

À cela \citeauthor{Zacklad2010} souhaite ajouter des modes de description du contenu émergents ou plus faiblement codifiés comme par exemple les folksonomies. 
Les SOC qui nous intéressent en particulier sont les schèmes structurés par types de relations. 
% pourquoi ? 


\subsection{Thésaurus, terminologie, ontologie}\label{sec:tto}
Dans cette section, nous nous reposerons majoritairement sur les définitions de \cite{bachimont:icc}. Concernant le thésaurus, l'auteur écrit :

\g{Thésaurus :} 
\ciel{
Une organisation de libellés linguistiques selon des relations d'hyperonymie et d'hyponimie. 
Les libellés sont également reliés par des relations dites d'association, qui sont de nature quelconque. 
Même si en pratique les libellés d'un thésaurus correspondent souvent à des termes du domaine, ce n'est pas nécessairement systématique.}

Cette définition situe clairement les thésaurus comme faisant partie du cadre de la linguistique. 
Il s'agit d'un ensemble de mots structurés et reliés suivant leur \e{signification}, c'est à dire leur sens normé ou commun à plusieurs contextes d'usage particuliers (à l'inverse du sens, qui lui varie suivant les usages, \cite{Roche2005}). 
\citeauthor{bachimont:icc} finit sa définition en comparant les mots issus d'un thésaurus aux termes. La distinction se joue à deux niveaux, la stabilité d'écriture du terme (niveau linguistique) et le fait qu'un terme renvoit à un concept (niveau conceptuel) : 

\g{Terme :} 
\ciel{
Une unité linguistique dont le signifié est un concept, c'est-à-dire un signifié normé. 
Le terme se manifeste linguistiquement par une stabilité et régularité de sa forme signifiante.
En particulier, un terme possède des contextes d'occurrence réguliers, obéissant à des canevas morpho-syntaxiques typiques. 
La détection de ces canevas est à la base des outils de détection des termes en corpus. 
Un terme peut posséder des variantes terminologiques.
Dans une optique normative, on détermine une forme préférée.}

Ainsi, plus qu'un repérage des mots (signifiant) utilisés dans un domaine donné, la terminologie s'attache à identifier les signifiés correspondants. 
Au-delà des débats sur les méthodes utilisées pour constituer les couples signifiant-signifié\footnote{L'approche \e{sémasiologique} (initiée par \cite{Bourigault1994}) s'appuie sur l'analyse linguistique d'un corpus de textes pour repérer les couplages signifiant-signifié ainsi que l'organisation conceptuelle sous-jacente. L'identification de ces \e{désignations} est ensuite validée par des experts du domaine. 
Dans une optique différente, l'approche \e{onomasiologique} prend comme appui la modélisation conceptuelle pour nommer ensuite les concepts. On parle alors de \e{dénominations} dont l'objectif est de refléter sans ambiguïté la structure conceptuelle dont elles sont issues.
Une critique faite à la première approche par \cite{Roche2006} est que les relations identifiées entre désignations sont purement linguistiques (hyper/hyponymie, méronymie etc.) et ne se rattachent pas à une structure conceptuelle. Ainsi, l'analyse de texte n'est qu'une première étape dans la constitution d'une terminologie, elle permet d'identifier les usages des mots, mais pas de les raccorder à des concepts. De même, la constitution du corpus va également grandement influer sur les résultats de l'analyse et pose alors un problème de réutilisabilité.}, on cherche à repérer la modélisation conceptuelle sous-jacente d'un domaine, de manière à pouvoir adosser chaque terme à un concept.
À noter que l'inverse n'est pas forcément valable, car il existe des concepts qui n'ont pas d'appelation usuelle et que l'on doit alors désigner par une phrase. 
Une autre conséquence de cette définition est que plusieurs mots peuvent être adossés au même concept.
Il devient alors important de pouvoir expliciter cette équivalence et éventuellement de spécifier un signifié préféré pour le terme.

Cette définition du terme préfigure ainsi la relation qu'entretient la terminologie avec l'ontologie pour \cite{bachimont:icc} : 

\g{Terminologie :} 
\ciel{un recensement et une organisation d'unités linguistiques à l'usage stabilisé et attesté, dont le signifié correspond à un concept du domaine.
La terminologie est l'organisation des termes du domaine.
La terminologie est la face linguistique de l'ontologie, qui en est le côté conceptuel. 
Il n'y a pas une stricte correspondance cependant entre ontologie et terminologie : si tout terme doit correspondre à un concept de l'ontologie, tout concept n'a pas forcément d'usage linguistique régulier attesté.}

De son côté, \cite[\S 2.4]{Roche2005} parle de manière similaire \ciel{[d']un système de termes reflétant une modélisation conceptuelle, [...] plus généralement dénommé \e{système notionnel} [qui] trouve sa raison d'être dans la façon dont nous appréhendons les objets du monde.}
\citeauthor{Roche2005} précise que si les systèmes notionnels ne relèvent pas de la linguistique, ils ne dépassent pas forcément le cadre d'une langue, sauf \ciel{pour des communautés de pratique dont les langues d'usage partagent la même conceptualisation du monde.}.
La distinction est ainsi faite entre les mots d'usage (qui peuvent être polysémiques) et les termes dont on spécifié une forme préférée (signifiant) et qu'on adosse à une signification (signifié normé).  

Concernant les particularismes qui peuvent exister dans chaque communauté, \citeauthor{Roche2005} propose de s'éloigner d'une vision purement normalisatrice. 
Ainsi, sur le plan linguistique, il est possible de rattacher les différents mots d'usages et de préciser leur contextes d'utilisation.
Sur le plan conceptuel, l'auteur propose de constituer des \gui{terminologies régionales} que l'on cherchera ensuite à mettre en correspondance. 




\paragraph{Ontologie}
Concernant les ontologies, nous nous limiterons aux définitions proposés dans le cadre de l'ingénierie des connaissances (IC). Les travaux de \cite{Charlet2002} nous rappelle qu'il existe de multiples définitions : 

Pour \cite{Gruber1993} : \ciel{Une ontologie est une spécification explicite d'une conceptualisation.}

 La définition proposée par \cite{Uschold1996} nous permet de préciser de quoi se compose une conceptualisation et en quoi une ontologie la spécifie : 

\ciel{Une ontologie implique ou comprend une certaine vue du monde par rapport à un domaine donné. Cette vue est souvent conçue comme
un ensemble de concepts -- e.g. entités, attributs, processus --, leurs définitions
et leurs interrelations. On appelle cela une conceptualisation. [...]
Une ontologie peut prendre différentes formes mais elle inclura nécessairement
un vocabulaire de termes et une spécification de leur signification. [...]
Une ontologie est une spécification rendant partiellement compte d’une conceptualisation.}

\citeauthor{Charlet2002} en conclut qu'une ontologie est une conceptualisation, c'est-à-dire un ensemble de concepts et de relations dont on cherche à normer la signification. 
Pour faire de la conceptualisation un objet informatique, il faut spécifier une théorie logique dotée d'un vocabulaire (les concepts et les relations), à la manière des travaux de \cite{Guarino1995}.

% Roche puis Babache
Pour \citeauthor{Roche2005}, une ontologie est équivalente au système notionnel des terminologies, d'où la relation forte établie par les chercheurs en IC : 

\ciel{définie pour un objectif donné et un domaine particulier, une ontologie est pour l'ingénierie des connaissances une représentation d'une modélisation d'un domaine partagée par une communauté d'acteurs. Objet informatique défini à l'aide d'un formalisme de représentation, elle se compose principalement d'un ensemble de concepts définis en compréhension, de relations et de propriétés logiques.} (\cite{Roche2005})

\citeauthor{Bachimont2000a} insiste sur le fait qu'on utilise une sémantique donnée (différentielle, référentielle, psychologique, distributionnelle, conceptuelle etc., \cite{bachimont:hdr}) pour établir la signification des concepts de l'ontologie. Chaque sémantique propose un point de vue particulier qui permet de faire correspondre une signification propre à chaque unité d'expression : 

\ciel{une ontologie est la signature fonctionnelle et relationnelle, munie de sa sémantique, d'un langage formel de représentation et manipulation des connaissances.} (\cite{Bachimont2000a})

Les ontologies se construisent ainsi en s'adossant à des théories, et ce sont ces théories qui fixent des principes pour déterminer la signification des unités linguistiques qu'elles emploient et chargent d'un sens bien précis (sémantique). 

\paragraph{Sémantiques et ontologies}
Pour bien cerner les conséquences de cette définition, voici quelques sémantiques décrites par \cite{bachimont:hdr} qui se distinguent dans leur manière d'expliciter la signification d'une unité d'expression : 
\begin{liste}
	\item \e{sémantique différentielle} : la signification d'une unité consiste en l'identité et la différence par rapport aux autres unités linguistiques de la langue. On reste donc dans le cadre de la linguistique. La différenciation des  unités peut se faire par différentes méthodes ; par observation empirique d'un corpus de texte (\e{sémantique distributionnelle}) ; ou bien suivant une modélisation de la signification des concepts du domaine établie par des experts par exemple (\e{sémantique conceptuelle}). 

	\item \e{sémantique référentielle} : la signification d'une unité est l'objet auquel elle fait référence, dans un univers extralinguistique. Ici, on s'attache à une théorie propre à cet univers et qui explicite la définition des objets.
	
	\item \e{sémantique psychologique} : la signification d'une unité est la représentation mentale que l'on s'en fait. Là encore, il s'agit de suivre une théorie, mais dans le champ de la psychologie. 
\end{liste}
Cette liste de sémantiques permet de comprendre la grande variabilité des ontologies qu'il est possible de construire. 


\paragraph{Classification d'ontologie}
Il peut également être utile de définir des propriétés pour distinguer des "genres" d'ontologies, non pas en fonction de la sémantique utilisé, mais en fonction de l'usage que l'on souhaite en faire. 
\cite{Oberle2006} propose une classification qui repose sur trois propriétés : 
\begin{liste}
	\item l'\g{objectif} de l'ontologie (purpose) où l'on distingue entre deux objectifs, servir de référence ou bien être utilisé dans un cas d'application :
	\begin{liste}
		\item l'\e{ontologie de référence} vise à établir un consensus entre des agents (humains, machines) d'une même communauté, ou bien à servir d'explication et de langages communs avec des agents de communautés différentes. 
		\item l'\e{ontologie d'application} qui se limite à un cas d'application et suit des contraintes et simplifications propres.
	\end{liste}
	La différence réside dans l'arbitrage entre l'expressivité de la représentation et sa décidabilité (\cite{Borgo2002}). 
	Typiquement, une référence n'est consulté qu'occasionnellement et se doit d'être la plus exhaustive possible alors qu'une ontologie d'application doit servir à faire des raisonnements à l'exécution. 

	\item l'\g{expressivité} de l'ontologie (expressiveness) où l'on considère un engagement plus ou moins fort sur le formalisme de représentation :
	\begin{liste}
		\item l'\e{ontologie légère} (lightweight) qui peut se limiter à une hiérarchie de concepts bien connus dans une communauté avec quelques relations. 
		L'apport se situe alors dans la structuration des connaissances qui clarifie leur signification plus qu'il ne les établit.
		\item l'\e{ontologie lourde} (heavyweight) qui vise à exclure toute ambiguïté terminologique et conceptuelle.
		Pour cela, la formalisation se veut beaucoup plus contrainte et détaillée afin de forcer une interprétation. 
	\end{liste}

	\item la \g{spécificité} de l'ontologie qui peut se limiter à une domaine ou s'étendre à un ensemble de domaines voire plusieurs champs disciplinaires : 
	\begin{liste}
		\item l'\e{ontologie générique} (generic, upper/top level) contient des concepts utilisés dans de nombreux champs disciplinaires (évènements, processus etc.)
		\item l'\e{ontologie noyau} (core) comporte des concepts qui se situent à la croisée de plusieurs domaines. 
		La distinction avec une ontologie générique se fait car elle comporte des concepts utilisable quelque soit le domaine.
		De même, on distingue les ontologies noyaux des ontologies de domaine car les premières comportent des éléments réutilisables dans des plusieurs domaines proches. 
		\item l'\e{ontologie de domaine} contient des concepts propre à un domaine, et bien souvent des éléments plus génériques extraits de domaine différents.
		Cependant, l'ontologie de domaine présente généralement des éléments plus spécifiques, propres au domaine concerné. 
	\end{liste}
\end{liste}
\subsection{Méthodes de construction d'ontologie}\label{sec:construction}
Nous avons défini dans la section précédente (\ref{sec:tto}) ce qu'est une ontologie, présenté des exemples de sémantiques et montré comment il était possible de classifier ces ontologies suivant leur usage. 
Nous abordons maintenant les méthodes de construction d'ontologies.


\paragraph{Méthode d'\cite{Uschold1996}}
\citeauthor{Uschold1996} ont défini une méthode de construction à partir de leur expérience de développement d'ontologies en entreprises. 
Elle se décompose en quatre étapes :
\begin{listenum}
	\item Une phase de \g{conception} qui vise à identifier le domaine concerné, le but et la portée de l'ontologie.
	\item Une phase de \g{construction} qui se décompose en trois étapes ; définir les concepts clés et les relations entre ces concepts ; expliciter la représentation de la conceptualisation dans un langage formel ; intégrer des connaissances d'autres ontologies.
	\item Une phase d'\g{évaluation} de l'ontologie construite.
	\item Une phase de \g{documentation} qui doit expliciter les décisions effectuées aux étapes précédentes afin de faciliter la réutilisation de l'ontologie.
\end{listenum}


\paragraph{Methontology}
La méthode proposée par le Laboratoire d'Intelligence Artificielle (LAI) de l'université Polytechnique de Madrid a pour particularité d'intégrer le développement de l'ontologie à une méthodologie de gestion de projet \parcite{Fernandez1997, Blazquez1998}. La méthodologie distingue trois types d'activités se déroulant en parallèle, et dont les deux premières servent à soutenir la construction de l'ontologie :
\begin{liste}
	\item Les activités de \g{gestion de projet}, notamment la planification avant-projet puis le \e{contrôle de la qualité} des résultats produits. 
	
	\item Les activités de \g{support} qui concernent l'\e{acquisition} des connaissances du domaine ; l'\e{intégration} de connaissances d'autres ontologies ; la \e{documentation} de l'ontologie et de sa production ; la \e{gestion de version} des résultats produits ; l'\e{évaluation} technique de la construction de l'ontologie ainsi que de sa documentation. 

	\item Les activités de \g{développement technique} qui permettent de construire l'ontologie par étape. 
	Dans un premier temps, la \e{spécification} définit l'objectif de l'ontologie, les applications et les utilisateurs concernés ; puis la \e{conceptualisation} structure les connaissances du domaine, qui sont ensuite formalisées (étape de \e{formalisation}) et enfin représentées dans un langage informatique (étape d'\e{implémentation}). La séquence se poursuit par une étape de \e{maintenance}.
\end{liste}


% Ontospec (Kassel)
% Guarino et Welty
% voir livre BB pour la justification du manque de sémantique

\paragraph{Archonte}
La méthode \pc{Archonte} (\pc{arch}itecture for \pc{ont}ological \pc{e}laborating), proposée par \cite{Bachimont2000a}, met en avant l'importance cruciale donnée au choix d'une sémantique dans la construction d'une ontologie.
La méthode repose sur trois étapes successives qui aboutissent à une ontologie computationnelle, exprimée dans un langage opérationnel de représentation des connaissances.
% , et à partir duquel on peut effectuer des inférences. 
% Précisons maintenant les étapes de cette méthode ainsi que les résultats obtenus :
% \begin{liste}
Le point de départ de la méthodologie est constitué d'expressions linguistiques (signifiant) issues du domaine considéré.
L'intérêt de disposer d'un tel ensemble de traces linguistiques est qu'elles servent à exprimer des concepts (signifiés) ou des connaissances sur le monde. 
Ainsi, on se retrouve avec un corpus de candidats-termes dont la signification peut être source d'ambiguïtés et dont on cherche à clarifier l'interprétation.\\

\g{[1.]} La première étape de cette méthode (aussi appelée \e{normalisation sémantique}) consiste à établir un \g{engagement sémantique}  qui précise la manière de mener l'interprétation des candidats-termes et de construire une première structure de connaissances. 
Pour cela, on fixe d'abord un contexte de référence, la tâche ou le problème qui a poussé à l'élaboration de l'ontologie, qui permet de cadrer l'interprétation des candidats-termes.

Ensuite, pour préciser l'interprétation on s'appuie sur la sémantique différentielle afin d'expliciter les différences et les similarités entre une notion et son voisinage direct (notion parente, notions soeurs) : 

	\ciel{
	La méthodologie que nous proposons ici repose sur l'organisation générale des unités en un réseau d'identités et de différences.
	Ce sont les propriétés structurelles de ce réseau qui permettent de contraindre l'interprétation des unités définies dans le réseau : la position d'une unité dans le réseau prescrit comment la comprendre et lui prescrit une signification qui pourra dès lors lui être associée, quel que soit le contexte où elle se rencontre.} (\cite[p.139]{bachimont:icc})

Cette caractérisation des notions par leur voisinage repose sur quatre relations à expliciter : 
\begin{liste}
	\item la \gui{communauté avec le parent} (\ciel{similarity with parent}) : pourquoi la notion hérite des proprités de son parent.
	\item la \gui{différence avec le parent} (\ciel{difference with parent}) : en quoi la notion est différente de son parent.
	\item la \gui{différence avec les soeurs} (\ciel{difference with siblings}) : en quoi un notion est différente de ses notions soeurs.
	\item la \gui{communauté avec les soeurs} (\ciel{similarity with siblings}) : quelle est la propriété que partagent les notion soeurs -- dont on distingue plusieurs valeurs exclusives, une par soeur.\\ 		
\end{liste}
% \end{liste}

Cette première étape aboutit à la construction d'un \g{arbre ontologique différentiel} qui structure un ensemble de notions de manière hiérarchique et non ambiguië par rapport à un contexte de référence.
Les candidats-termes sont structurés par des prescriptions interprétatives et deviennent ainsi des primitives de modélisation.\\

% \begin{liste}
\g{[2.]} L'\g{engagement ontologique} consiste à munir l'ontologie différentielle d'une sémantique formelle extensionnelle.
Rappelons que cette sémantique définit les concepts par leur extension, c'est-à-dire tous les individus qu'ils désignent parmi un ensemble de référence. 
Il s'agit donc de relier des primitives dotées d'une signification linguistique normalisée à des concepts désignant un ensemble de référents (ou individus).
Pour cela, il faut adjoindre à l'ontologie différentielle un modèle référentiel : 

	\ciel{
	l'ontologie référentielle obéit aux contraintes sémantiques de l'ontologie différentielle : [s]a structure arborescente se retrouve dans l'ontologie référentielle et lui donne son squelette.
	Chaque relation de spécialisation sémantique au niveau différentiel se traduit par une spécialisation d'extension au niveau référentiel.} (\cite[p.148]{bachimont:icc})

Ce changement de sémantique permet d'enrichir l'ontologie de nouveaux concepts et d'en modifier la structuration. 
En effet, on peut désormais avoir recours à des opérations ensemblistes (réunion, intersection, complémentaire) qui composent le sens des concepts et permettent ainsi de définir de nouveaux concepts.
L'ajout de ces \gui{concepts définis} modifie également la structure de l'ontologie, qui passe d'une arborescence à une structure en treillis, c'est-à-dire admettant l'héritage multiple. 
Par exemple, une primitive différentielle de \cd{mandat politique} spécialisée en concepts de \cd{député} et de \cd{maire} ne permet pas de représenter de double mandat.
Par contre, une définition extensionnelle permet de définir le concept de \cd{député-maire} simplement par l'intersection des extensions de ces concepts\footnote{Pour plus de détails sur cet exemple, se reporter à l'exemple donné par \cite[p.149]{bachimont:icc}}.
% \end{liste}

À l'issue de cette étape on obtient donc une \g{ontologie référentielle}, c'est-à-dire un treillis de concepts définis par une sémantique référentielle.\\

% \begin{liste}
\g{[3.]} L'\g{engagement computationnel} vise à doter les concepts de l'ontologie référentielle d'une signification en termes d'opérations informatiques.
Pour cela, il faut d'abord choisir un langage opérationnel de représentation des connaissances qui détermine l'expressivité et les opérations de calculs à disposition pour élaborer une version informatique de l'ontologie.
Nous présentons quelques uns de ces langages dans la section \ref{sec:onto-mc}.
La transposition dans un langage a des conséquences au niveau de l'expressivité et de la décidabilité du modèle.
% \end{liste}

Nous obtenons une \g{ontologie computationelle} qui est une version de l'ontologie référentielle exploitable informatiquement.





\subsection{La validation en Ingénierie des Connaissances}\label{sec:valid-ic}
En suivant l'analyse proposée par \cite{Bachimont2004}, il en découle que  l'Ingénierie des Connaissances (IC) \ciel{exprime les connaissances d'un domaine dans un langage de modélisation et l'opérationnalise en un système}. 
En d'autres termes, la modélisation de l'IC porte sur les concepts utilisés par les membres de ce domaine pour penser et établir des connaissances sur le monde, mais pas directement sur le monde.
Ainsi, les modèles de l'IC n'ont pas pour vocation à \ciel{prédire quoi que ce soit sur le monde ni sur la connaissance}, mais plutôt d'\ciel{instrumenter le travail intellectuel, l'exercice de la pensée, le travail de la connaissance}. 
Dans cette perspective, \citeauthor{Bachimont2004} nous propose de théoriser l'IC comme \ciel{une ingénierie des inscriptions numériques des connaissances qui vise à instrumenter le travail cognitif associé à ces inscriptions}. 
        
Les inscriptions possèdent une double dimension ; \e{matérielle} (et donc manipulable par des techniques de calcul logique) ; \e{sémiotique} (et donc interprétable selon des conventions propres à une situation d'usage).  
En d'autres termes, le système d'IC permet d'agir de manière prédictible sur les inscriptions de connaissances, ces actions produisant de nouvelles inscriptions qui donnent matière à penser à l'utilisateur. 
        
Il y a donc plusieurs éléments à valider en IC, les calculs qui seront faits sur les inscriptions (on teste le comportement du système informatique) puis l'interprétation de ces inscriptions (on évalue le gain apporté par le système et les inscriptions qu'il fournit à l'utilisateur selon une situation d'usage). 
La modélisation prise en charge par l'IC ne porte donc ni sur le monde, ni sur l'activité cognitive et ne peut être validée uniquement par le formalisme de ses inscriptions. 
        
Les inscriptions de connaissances doivent être considérées sous deux angles : d'un point de vue \e{nomographique} (on formalise la manipulation symbolique des inscriptions pour prévoir/définir le comportement du système) et \e{idiographique} (on décrit le sens des manipulations symboliques et des inscriptions produites par rapport aux normes, conventions, concepts du domaine).



% Au final, on construit un objet informatique en suivant une méthode de construction qui nous guide pour spécifier le comportement de l'ontologie. 

% RTO





\subsection*{Discussion sur la méthodologie suivie}
\addcontentsline{toc}{subsection}{Discussion}
% pourquoi on choisit celle de BB, et pourquoi on ne l'utilise pas globalement, mais seulement localement sur des patrons d'utilisation précis.



Notre étude des méthodes de construction d'ontologie ne vise pas à être exhaustive, pour cela nous renvoyons à \cite{Gomez-Perez2004}.
Il faut cependant remarquer que nous avons écarté les approches d'acquisition des connaissances à partir d'un corpus de texte, comme la méthode \g{Terminae} développée par \cite{Aussenac-Gilles2003}.
Dans cette approche, l'analyse linguistique d'un corpus permet de repérer des candidats-termes (\pc{Syntex}), d'effectuer des regroupements de contexte (\pc{Upery}) et d'identifier des relations (\pc{Yakwa}) afin d'accompagner la modélisation conceptuelle.
Or, dans notre contexte de travail, les documents professionnels qui décrivent en détails la production audiovisuelle sont rares (peu d'organisation ont les ressources de les produire) et constituent des ressources de valeur auquelles il est difficile d'avoir accès. 
De plus, la notion de document audiovisuel est largement absente des ouvrages généralistes, ce qui fonde précisément l'intérêt de notre travail de recherche.
\citeauthor{Uschold1996} proposent un point de vue global sur le processus de construction d'ontologie, qui reste un point de référence dans le domaine.
L'approche de \pc{Methontology} clarifie cependant le déroulement de la construction et s'efforce de l'intégrer dans une vision de gestion de projet assez exhaustive. 
Si cette vision est intéressante, il n'est pas forcément possible de l'appliquer telle quelle en pratique, du fait de contraintes spécifiques d'un projet.
Par ailleurs, \pc{Archonte} détaille une méthode de formalisation logique qui permet de positionner la conceptualisation sur le plan sémantique. 
De ce fait, elle précise la manière de formaliser une conceptualisation et peut s'intégrer de manière complémentaire aux autres méthodes exposées.

Dans notre cas, nous avons suivi une méthodologie \e{ad-hoc}, nécessaire pour suivre les contraintes d'un projet de recherche et développement impliquant de multiples partenaires.
Les aspects de gestion de projet (tels que décrits dans \pc{Methontology}) étaient donc largement assujetis à l'avancement du projet MediaMap.
De même, la spécification et l'acquisition des connaissances et la conceptualisation reposent principalement sur un dialogue avec des experts du domaine (dans le cas du projet MediaMap ce fût principalement les membres de la RTBF et de la VRT) dans le cadre de réunions générales et de séminaires plus focalisés.
Les autres aspects de la construction de l'ontologie ont été réalisés par les membres de l'équipe de recherche ICI, en s'inspirant des séquences de développement technique proposées par \pc{Methontology}.

Sur le plan sémantique, l'étendue de notre modélisation nous pousse à introduire des concepts de haut-niveau afin d'articuler diverses connaissances se rapportant à la production audiovisuelle (l'organisation du processus, ses contributeurs, ses résultats, et leur description par un vocabulaire professionnel et compréhensible par des amateurs).
Ces ontologies génériques de haut-niveau proposent des fondements théoriques important et des patrons de conception détaillés (au sens de \cite{Isaac2005}) pour modéliser certaines situations. 
On pense par exemple, au patron \ciel{Description \& Situations} de l'ontologie DOLCE \parcite{Gangemi2005}. 
Pour autant, la modélisation ne doit pas perdre de vue notre perspective applicative, nécessaire à l'adoption et la compréhension du modèle par des utilisateurs du domaine.
Ainsi, \citeauthor{Isaac2005} proposent d'adapter la structure de ces patrons aux besoins descriptifs de l'application.
Cela consiste à simplifier la structure des patrons en créant des \e{raccourcis relationnels}, quitte à faire disparaître les subtilités de représentation.
L'enjeu d'une telle approche est de faire co-exister la forme simplifiée (adaptée aux usages du domaine et aux besoins expressifs de l'application) et la forme de haut-niveau qui la rend réutilisable dans d'autres domaines.
De notre point de vue, notre modélisation se construit à partir de tels patrons de conception, qu'il s'agira ensuite d'articuler par des relations. 
Nous avons appliqué la méthodologie \pc{Archonte} pour formaliser la sémantique de ces patrons, qui sont ensuite regroupés pour former une seule ontologie.
En effet, nous considérons que le point important est de justifier la modélisation de ces parties et de leur mise en relation, plutôt que de la structure de l'ensemble.






% est-ce qu'on a adapté des patrons de conception générique aux notions du domaine, pour simplifier la modélisation (modélisation réduite) ? VOIR \cite{Isaac2005}









\section{Langages de représentations}\label{s:mods}

% \subsection{Langages de balisages }
\subsection{eXtended Markup Language (?)}
The eXtended Markup Language (XML) aims to give a hierarchical structure to  text in a machine-, yet human-readable way. It is widely used to store or exchange information as it also supports Unicode.
XML is formally defined as a Standard Generalized Markup Language's subset (SGML) designed to improve parser efficiency. Work on XML began in 1996 and it became a W3C Recommendation in early 1998.

\paragraph{Mark-up}
This is achieved by adding mark-up elements that are easily noticed as they begin with '<' and end with a '>'. Mark-up elements are used to enclose unicode text, and give thus a mean to identify them and possibly to process it. As its name indicates, it is said extensible because we can define our own mark-up elements and writes a line like that:

\begin{Verbatim}[fontsize=\small,formatcom=\color{black!70}]
opening mark-up element 			enclosing mark-up element
<structural_element>Some unicode text inside</structural_element>
\end{Verbatim}

Attributes can be defined for each mark-up element. 
For instance, the xml:lang attributes indicates the natural language used to write the enclosed text. The « 1812 Overture » full title can be written like that:
\begin{Verbatim}[fontsize=\small,formatcom=\color{black!70}]
<title xml:lang='ru'>Торжественная увертюра 1812 года, Toržestvennaja uvertjura 1812 goda</title>
<title xml:lang='fr'>Ouverture Solennelle, L'Année 1812, Op. 49</title>
\end{Verbatim}

\paragraph{Syntax}
XML does not only enclose text with mark-up elements. It also enables to imbricate mark-up elements in such a way that the elements conforms to a tree structure. 

\begin{Verbatim}[fontsize=\small,formatcom=\color{black!70}]
<element>
	<sub-element>Example of text</sub-element>
</element>
\end{Verbatim}

Other syntaxic rules have been defined to enables conforming parser to process XML file. Any file conspuing to these rules is said to be well-formed.

\paragraph{Schema}
Furthermore, if XML provides us with a syntax we also have the ability to makes purpose-specific XML-based mark-up languages – i.e. define constraints on structure, mark-up elements or even datatyping definition. Indeed, several schema languages exists and are used to encode documents or serialize text data according to a particular schema. XML files complying with a schema – i.e. conforming to the constraints defined in the schema – are said to be valid.

\paragraph{Namespace}
When creating schemas, ambiguity problems usually arise and namespace declaration can take care of that. Indeed, it provides an abstract container for XML elements and attributes and gives to their name a scope. As each namespace is identified by an URI, the ambiguity between identically named elements or attributes from differents namespace can be resolved. 

Therefore, I can declare my own « title » element and simultaniously use the « DC Terms » property title. We show here a complete example with xml heading:
\begin{Verbatim}[fontsize=\small,formatcom=\color{black!70}]
<?xml version="1.0" encoding="UTF-8"?>
<ex:musical_opus xmlns:dc="http://purl.org/dc/elements/1.1/"
    			 xmlns:ex="http://example.org">
	<dc:title>1812 Overture</dc:title>
    <ex:title>Festival Overture, The Year 1812</ex:title>
</ex:musical_opus>
\end{Verbatim}

In this example, dc:title and ex:title denotes two different elements for which we can give informally different meaning – ex:title indicating the complete name, dc:title a short version. 

\subsubsection*{XML Schema}
This W3C recommendation was published in 2001 and is one of several xml schema languages\footnote{We can cite, the old and very simple \gui{Document Type Defintion}(DTD) as well as the major rival of XML Schema, namely \gui{Relax NG}.}. It is often called XSD in reference to its files suffix – '.xsd'.

XSD can define imbrication, quantification and naming rules for xml elements and attributes – in order to enable vocabulary and content model validation. XSD also supports namespace so parts of other schemas can be included or imported. 

\paragraph{Datatypes}
But one of the main and most criticized characteristic of XSD is DataType validation. It can be applied to elements or attributes to constraint their content.  DataType definition must use XSD primitive or derived datatypes – see the scheme for a detailled hierarchy. This dependence upon specific datatypes is the source of many criticism. 

Derived datatypes can be built by restriction – of the permitted values set –, list – declaration of values –, or union – between several types. 
As an complete example, we define a XSD schema describing « MusicalOpus » as a list of XML elements named:
\begin{liste}
	\item Title: title of the musical opus
	\item Extent: length or duration – as a string
	\item Composer: name of the composer
	\item composed: date of composition of the opus – only the year
	\item Performer: name of the performer
	\item performed: date of performance – only the year
	\item conductedby: name of the person who conducted the performer, this attribute is declared optionnal – thanks to the minOccurs attribute
	\item ComposerNationality: picked from a list of value we enumerate in the scheme
\end{liste}

Here is the resulting XSD scheme:
\begin{Verbatim}[fontsize=\small,formatcom=\color{black!70}]
<?xml version="1.0" encoding="utf-8"?> 
<xs:schema elementFormDefault="qualified"   xmlns:xs="http://www.w3.org/2001/XMLSchema"> 
 <xs:element name="MusicalOpus"> 
   <xs:complexType> 
     <xs:sequence> 
       <xs:element name="Title" type="xs:string" /> 
       <xs:element name="Extent" type="xs:string" /> 
       <xs:element name="Composer" type="xs:string" /> 
       <xs:element name="composed" type="xs:gYear" /> 
       <xs:element name="Performer" type="xs:string" /> 
       <xs:element name="performed" type="xs:gYear"/> 
       <xs:element name="conductedBy" type="xs:string" minOccurs="0"/>  
       <xs:element name="ComposerNationality"> 
         <xs:simpleType> 
           <xs:restriction base="xs:string"> 
             <xs:enumeration value="FR" /> 
             <xs:enumeration value="DE" /> 
             <xs:enumeration value="RU" /> 
             <xs:enumeration value="UK" /> 
             <xs:enumeration value="US" /> 
           </xs:restriction> 
         </xs:simpleType> 
       </xs:element> 
     </xs:sequence> 
   </xs:complexType> 
 </xs:element> 
</xs:schema> 
\end{Verbatim}


And we provide a XML file which states that it conforms to the previous XSD scheme through a xsi:noNamespaceSchemaLocation attribute:
\begin{Verbatim}[fontsize=\small,formatcom=\color{black!70}]
<?xml version="1.0" encoding="utf-8"?> 
<MusicalOpus xmlns:xsi="http://www.w3.org/2001/XMLSchema-instance" 
         xsi:noNamespaceSchemaLocation="MusicalOpus.xsd"> 
  <Title>Festival Overture, The Year 1812</Title> 
  <Extent>14:19</Extent> 
  <Composer>Pyotr Ilyich Tchaikovsky</Composer> 
  <composed>1880</composed> 
  <Performer>Minneapolis Symphony Orchestra</Performer> 
  <performed>1954</performed> 
  <conductedBy>Antal Dorati</conductedBy> 
  <conductedBy>Harold Lawrence</conductedBy> 
  <ComposerNationality>RU</ComposerNationality> 
</MusicalOpus> 
\end{Verbatim}






















\subsection{Ontologie et représentation des connaissances}
\subsubsection*{Resource Description Framework}
\addcontentsline{toc}{subsection}{Resource Description Framework}
The Resource  Description Framework (RDF) is an abstract model which is part of the W3C recommendations for the Semantic Web. 
Let's just bring back to mind how \pc{Tim Berners-Lee} defined it to set RDF back into its context of creation: 

\ciel{
The Semantic Web is not a separate Web but an extension of the current one, in which information is given well-defined meaning, better enabling computers and people to work in cooperation.}

Indeed, RDF aims to describe and link resources – and no more web pages – in a simple, all-purpose and machine-readable way. 
The focus on software agents led to choose a formal semantic and provable inference. 
Thus, such descriptions will foremost benefits to software agents which we'll be able to exploit, process and search into this web of linked data. 

\paragraph{Statements / Triples}
The description consists in making statements that describes or models web resources. The statements are formed as subject-predicate-object sentence called triples that can be represented as a graph – one node/vertex for the subject and the object, and a directed and labeled edge for the predicate. 
\begin{Verbatim}[fontsize=\small,formatcom=\color{black!70}]
Subject		Predicate	    Object	
Tchaikovsky ---- is the Composer of ----> 1812 overture
\end{Verbatim}
Thereby, a set of statements constitute a multigraph, that is a graph in which node/vertex can have multiple ingoing or outgoing edges resulting possibly in loops. 
However, unlike an hypertext the rdf multigraph has labelled edges – also called properties -- connecting a resource with another resource or a literal value.

\paragraph{URI, datatype and literals}
RDF is said to have an URI-based vocabulary, meaning that resources and properties and typed literal are identified by URI reference. 
Indeed, unlike plain literals, typed literals are literals combined with a datatype URI.
Datatypes in RDF are compatible with XML Schema datatypes --which can thus be used as there are-- but any datatype definition conforming to RDF constraints may be used.

Let's rewrite our previous example with Tchaikovsky:
\begin{Verbatim}[fontsize=\small,formatcom=\color{black!70}]
Subject:  http://example.org/Tchaikovsky
Property: http://example.org/Composer
Object:   http://example.org/1812_Overture
\end{Verbatim}

And now an example with a gYear XSD datatype:
\begin{Verbatim}[fontsize=\small,formatcom=\color{black!70}]
Subject:  http://example.org/1812_Overture
Property: http://example.org/composed
Object:   '1880' xsd:gYear
\end{Verbatim}
The following examples make use of the namespace ability to ease the reading. We define here the prefix used for our example and for specific rdf elements:
\begin{liste}
	\item \cd{Example prefix: 'ex:'	Example URI: 'http://example.org'}
	\item \cd{RDF prefix: 'rdf:'	RDF URI:'http://www.w3.org/1999/02/22-rdf-syntax-ns\#'}
\end{liste}
Our previous example could then be written like that:
\cd{ex:1812\_Overture  -- ex:composed -->  '1880' xsd:gYe}


\paragraph{Structured value}
When we want to define a resource composed in fact of several other resources or literals, 
RDF makes us declare an intermediate node – only to conform to the RDF syntax. 
This kind of nodes are called blank nodes because they don't really need to be referenced by an URI. 
As they are only a product of the RDF syntax and don't represent anything in particular they can stay anonymous.

Indeed, when we declare the size of a digital file such as an audio file, we may want to specify the unit. 
Thus considering that the following statement is not sufficient:
\cd{ex:audio\_file\_01  ex:size '33,7'}

So we need to define one blank node and use a particular RDF property called value:
\begin{Verbatim}[fontsize=\small,formatcom=\color{black!70}]
ex:audio_file_01 	-- ex:size -->	_blank_node_01
_blank_node_01	-- rdf:value -->	'33,7'
_blank_node_01	-- ex:unit -->	ex:Mb xsd:decimal
\end{Verbatim}

The value property is not the only one to be used in such context. 
The type property states the nature of a blank node. 
An example will be shown in the next paragraph.

\paragraph{Grouping resources}
% 7.2.3.a  containers
RDF provides two different ways to group things. The first one is called Containers comprises three predefined types which points out some members of the group they define:
\begin{liste}
	\item a bag define a non-ordered group that may include duplicate members. 
	\item a seq define an ordered group that may include duplicate members.
	\item a alt define a group of alternatives resources or literals. 
\end{liste}

Members may be resources or literal, their membership is stated by declaring them as list item (li).
We show here example connected to our information sample. First, a bag of \gui{Performer} having played the \gui{1812 overture}, then a seq describing the content of the audio CD. 
Eventually, an alt group of russian, french and english version of \gui{Tchaikovsky}'s full name.

1) Bag of \gui{Performer}
\begin{Verbatim}[fontsize=\small,formatcom=\color{black!70}]
ex:1812_Overture	-- ex:Performer -->	_blank_node_02
_blank_node_02	-- rdf:type -->		rdf:bag
_blank_node_02	-- rdf:li -->		ex:Minneapolis_S_O
_blank_node_02	-- rdf:li -->		ex:St_Petersburg_Ph
\end{Verbatim}
2) Seq of audio tracks from \gui{Tchaikovsky: 1812 Festival Overture; Capriccio Italien; Beethoven:  Wellington's Victory} CD:
\begin{Verbatim}[fontsize=\small,formatcom=\color{black!70}]
ex:1812_Op49_CD	-- ex:TrackList -->	_blank_node_03
_blank_node_03	-- rdf:type -->		rdf:seq
_blank_node_03	-- rdf:li -->		ex:Op49_audio
_blank_node_03	-- rdf:li -->		ex:Op49_commentary
_blank_node_03	-- rdf:li -->		ex:Capriccio_Italien_audio
_blank_node_03	-- rdf:li -->		ex:Wellington_Op91_audio01
_blank_node_03	-- rdf:li -->		ex:Wellington_Op91_audio02
_blank_node_03	-- rdf:li -->		ex:Op91_commentary
\end{Verbatim}
3) Alt names of \gui{Tchaikovsky}
\begin{Verbatim}[fontsize=\small,formatcom=\color{black!70}]
ex:Tchaikovsky		-- ex:Name -->		_blank_node_04
_blank_node_04	-- rdf:type -->		rdf:alt
_blank_node_04	-- rdf:li -->		'Pyotr Ilyich Tchaikovsky' @en
_blank_node_04	-- rdf:li -->		'Piotr Ilitch Tchaïkovski' @fr
_blank_node_04	-- rdf:li -->		'Пётр Ильич Чайкoвский' @ru
\end{Verbatim}

% 7.2.3.b  collection
Unlike containers, collections can make a closed group definition, that is list all members member of the collection. With a container you can not state that there  is no other member than those you give in your definition. 
Moreover, list have predefined properties to identify the first item, the rest of the list and the end of it – a nil property. 

A rewriting of our tracklist example as a list would be:
\begin{Verbatim}[fontsize=\small,formatcom=\color{black!70}]
ex:1812_Op49_CD	-- ex:TrackList -->	_blank_node_03
_blank_node_03	-- rdf:type -->		rdf:list
_blank_node_03	-- rdf:first -->		ex:Op49_audio
_blank_node_03	-- rdf:rest -->		ex:Op49_commentary
_blank_node_03	-- rdf:rest -->		ex:Capriccio_Italien_audio
_blank_node_03	-- rdf:rest -->		ex:Wellington_Op91_audio01
_blank_node_03	-- rdf:rest -->		ex:Wellington_Op91_audio02
_blank_node_03	-- rdf:rest -->		ex:Op91_commentary
_blank_node_03	-- rdf:rest -->		rdf:nil
\end{Verbatim}

\paragraph{Serialization format}
As an abstract model, RDF statements can be serialized or represented in a variety of form. The most widely known is the \gui{RDF XML} but the W3C also introduced the more readable \gui{Notation} (N3) based on tabular spacing. This last form is closely related to the \gui{Turtle} and \gui{N-Triples} formats.





\subsubsection*{RDF Schema}
\addcontentsline{toc}{subsection}{RDF Schema}
RDFS is the result of 6 years of work from the W3C consortium – from the 1998's first version to the 2004's final recommendation. 
It is formally introduced as a vocabulary description language intended to structure RDF resources. 
Indeed, RDFS presents mechanisms for describing classes of resources, associated properties and also the relationships between properties and other resources. 

These mechanisms are in fact itselves classes and properties that enables us to describe vocabulary or basic ontology. 
Like RDF, RDFS follows a minimalistic approach allowing a relatively basic expressiveness compared to the Ontology Web Language (OWL) – which is built upon it.

\paragraph{RDF(S) Class}
Classes are resources – identified by an URI, described by properties – associated with a set of resources called the class extension. 
Resources in the class extension  are called instances – the rdf:type property may be used to state a resource as an instance of a class. 
Note that a class extension can cointain the class itself as instance – this is why rdfs:Class can be defined as a rdfs:Class in the table below.
SubClasses may be defined by the SubClassOf property. In this case, their extension pertains necessarily to any upper class extension. 
If we state « X » to be the class of all « Opus » and « Y » a sub-class of « X » containing all the « MusicalOpus » ; then every instance of « Y » will be an instance of « X ». 

% \paragraph{Class List}
\begin{table}[ht!]
   \begin{center}
		\begin{tabularx}{400pt}{|l|X|}
		   \hline
		Class name & Comment\\ \hline\hline
		rdfs:Resource & The class resource, everything.\\ \hline
		rdfs:Literal & The class of literal values, e.g. textual strings and integers.\\ \hline
		rdf:XMLLiteral & The class of XML literals values.\\ \hline
		rdfs:Class & The class of classes.\\ \hline
		rdf:Property & The class of RDF properties.\\ \hline
		rdfs:Datatype & The class of RDF datatypes.\\ \hline
		rdf:Statement & The class of RDF statements.\\ \hline
		rdf:Bag & The class of unordered containers.\\ \hline
		rdf:Seq & The class of ordered containers.\\ \hline
		rdf:Alt & The class of containers of alternatives.\\ \hline
		rdfs:Container & The class of RDF containers.\\ \hline
		rdfs:ContainerMembershipProperty & The class of container membership properties, rdf:\_1, rdf:\_2, \dots, all of which are sub-properties of 'member'.\\ \hline
		rdf:List & The class of RDF Lists.\\ \hline
		\end{tabularx}
		\caption{Class list \label{tab:rdfs-classes}}
   \end{center}
\end{table}

\paragraph{RDF(S) Properties}
First of all, let's recall that all properties are instances of the rdf:Property class. A property link a pair of resources, one of them as a subject and the other one as the object. 
RDFS introduces two properties to restrict which can of resources may be linked together. rdfs:domain constraints the subject resource to be an instance of a given class whereas rdfs:range constraints the objet resource likewise. 
SubProperty relationships – stated with rdfs:subPropertyOf – induces that all pairs of resource linked are also linked by the upper property. 
Thus, having a « Contributor » property and a « Performer » sub-property we can state that:
\begin{Verbatim}[fontsize=\small,formatcom=\color{black!70}]
ex:1812\_Overture
ex:Performer
ex:Minneapolis\_SO
\end{Verbatim}
and this would entail:
\begin{Verbatim}[fontsize=\small,formatcom=\color{black!70}]
ex:1812\_Overture
ex:Contributor
ex:Minneapolis\_SO
\end{Verbatim}

% 7.3.2.a  Property List
\begin{table}[ht!]
   \begin{center}
		\begin{tabularx}{450pt}{|l|X|c|c|}
		   \hline
rdf:type & The subject is an instance of a class. & rdfs:Resource & rdfs:Class \\ \hline
rdfs:subClassOf & The subject is a subclass of a class. & rdfs:Class & rdfs:Class \\ \hline
rdfs:subPropertyOf & The subject is a subproperty of a property. & rdf:Property &rdf:Property \\ \hline
rdfs:domain & A domain of the subject property. & rdf:Property & rdfs:Class\\ \hline
rdfs:range & A range of the subject property. & rdf:Property & rdfs:Class\\ \hline
rdfs:label & A human-readable name for the subject. & rdfs:Resource & rdfs:Literal \\ \hline
rdfs:comment & A description of the subject resource. & rdfs:Resource & rdfs:Literal\\ \hline
rdfs:member & A member of the subject resource. & rdfs:Resource & rdfs:Resource\\ \hline
rdf:first & The first item in the subject RDF list. & rdf:List & rdfs:Resource\\ \hline
rdf:rest & The rest of the subject RDF list after the first item. & rdf:List & rdf:List\\ \hline
rdfs:seeAlso & Further information about the subject resource. & rdfs:Resource &rdfs:Resource\\ \hline
rdfs:isDefinedBy & The definition of the subject resource. & rdfs:Resource & rdfs:Resource\\ \hline
rdf:value & Idiomatic property used for structured values (see the RDF Primer for an example of its usage). & rdfs:Resource & rdfs:Resource\\ \hline
rdf:subject & The subject of the subject RDF statement. & rdf:Statement & rdfs:Resource\\ \hline
rdf:predicate & The predicate of the subject RDF statement. & rdf:Statement & rdfs:Resource\\ \hline
rdf:object & The object of the subject RDF statement. & rdf:Statement & rdfs:Resource\\ \hline
		\end{tabularx}
		\caption{Class list \label{tab:rdfs-classes}}
   \end{center}
\end{table}

Eventually, observe these four properties that appear more like annotation than property defining resource: \cd{rdfs:label, rdfs:comment, rdfs:seeAlso, rdfs:isDefinedBy}.





\subsubsection*{Ontology Web Language}
\addcontentsline{toc}{subsection}{Ontology Web Language}
The Ontology Web Language (OWL) is a knowledge representation language intended – as its name states – to build ontology in a web environnement. 
OWL relies on RDF and XML syntax and defines its constructs as extension or subset of RDF/RDFS classes and properties. 
Nevertheless, OWL provides in addition comparaison and cardinality constraints on classes or properties. 
These constructs enables to model domain specific ontology while bringing along generic reasoning tool. 

The W3C created a working group in 2001 and documents became recommendations in 2004.
However, OWL was a revision of earlier work called DAML+OIL initiated conjointly by the « Defense Advanced Research Projects Agency » (DARPA) and the European Union's « Information Society Technologies » (IST) project. 

\paragraph{OWL species}
OWL defines in fact three sub-languages with different level of expressiveness and thus computational efficiency:
\begin{liste}
	\item « OWL Lite » is the simplest, less expressive sub-language. It supports 0..1 cardinality constraints and thus are intended for thesauri and taxonomies migration project.

	\item « OWL DL » gives full expressivity while ensuring computational completeness – all entailments are garanteed to be computed – and decidability – all computations will finish in finite times. Full expressivity means it includes all language constructs but to ensure the rest it needs to restrict their use by some conditions. 

	\item « OWL Full » gives full expressivity without conditions. For instance, a major difference with « OWL DL » is that all resources can be considered as individuals – even Class and Property. There are strong equivalence between « OWL Full » and RDF – RDFS. This comes nevertheless with no computational garantees. 
\end{liste}
% Each level is a sub-level from its predecessor, that is every legal ontology or valid conclusion expressed in « OWL Lite » is a legal ontology or respectively a valid conclusion in « OWL DL » and so on between « OWL DL » and « OWL Full ». 

% 7.4.2  Class
% OWL defines its class like RDFS except that only OWL Full can state that a class is an instance of another class. OWL Lite and DL don't allow a class to be considered at the same time as an individual – thus also as a member of a class extension. 

% 7.4.2.a  Class descriptions
% OWL supports six different kind of class descriptions: 
% 1. a simple class declaration – with a URI reference
% 2. an exhaustive enumeration of the class extension
% 3. definition of a class as a subset of another class depending on property restrictions
% 4. an intersection of several classes descriptions
% 5. an union of several classes descriptions
% 6. a complement of a class description

% 1. Class declaration can be done as follow:
% <owl:Class rdf:ID='Composer' />

% 2. Or if we want to define directly its extension like RDFS Containers, we can write: 
% <owl:Class rdf:ID='Nationality'>
% 	<owl:oneOf rdf:parseType='Collection'>
% 		<owl:Thing rdf:about='ex:French' />
% 		<owl:Thing rdf:about='ex:Russian' />
% 		<owl:Thing rdf:about='ex:English' />
% 	</owl:oneOf>
% </owl:Class>

% 3. We may also use more complex property restrictions, including value or  cardinality restrictions. For this purpose we have a set of properties:

% Value restrictions
% owl:allValuesFrom, owl:someValuesFrom  – with the values being either a class or a datatype but OWL Lite only supports class value.

% owl:hasValue – the value has to be either an individual or a data value. This property is not included in OWL Lite. 

% Cardinality constraints
% owl:maxCardinality, owl:minCardinality, owl:cardinality – the value has to be related to a XML Schema datatype.

% 4,5,6. These constructs are similar to AND, OR, NOT operators acting on classes. Only owl:intersectionOf can be used in some way in OWL Lite, whereas owl:unionOf and owl:complementOf are not included.

% 7.4.2.b  Class axioms
% Classes may be described with the previous properties, class axioms are the three properties that have owl:Class for domain and range: 
% owl:subClassOf enables specialisation to be described. The sub-class' extension set is thus stated as a subset of the class extension set. 

% owl:equivalentClass declares class extension equivalence between two classes descriptions.

% owl:disjointWith establishes that two classes extensions of two classes descriptions have no member in common.

% Note that all owl:Class are sub-classes of the owl:Thing superclass. 

% 7.4.3  Property
% OWL defines four kind of Property that must be mutually disjoint when using OWL DL:

% owl:DatatypeProperty links instance with literal values.
% owl:ObjectProperty define relations between instances.
% owl:AnnotationProperty are intended for human reader. In OWL DL, it is impossible to define restriction or sub-property for Annotation Properties and information will not be taken into account by reasoners.
% owl:OntologyProperty are used for importing ontology and make statements about versionning information. In OWL DL, the same constraints hold as those specified for Annotation Properties. 

% 7.4.3.a  Property declaration
% We take as an example the definition of creation relationships on two different level of specialisation. First, a « createdBy » relation between a « Person » and an « Opus ». Second, a music related « composedBy » relation, involving a « Composer » and a « MusicalOpus ».

% Class declaration
% <owl:Class rdf:ID='Person' />
% <owl:Class rdf:ID='Composer'>
% 	<rdfs:subClassOf rdf:resource='Person' />
% </owl:Class>
% <owl:Class rdf:ID='Opus' />
% <owl:Class rdf:ID='MusicalOpus'>
% 	<rdfs:subClassOf rdf:resource='Opus' />
% </ow:Class>

% Property declaration
% <owl:ObjectProperty rdf:ID='createdBy'>
% 	<rdfs:domain rdf:resource='Opus' />
% 	<rdfs:range rdf:resource='Person' />
% </owl:ObjectProperty>
% <owl:ObjectProperty rdf:ID='composedBy'>
% 	<rdfs:domain rdf:resource='MusicalOpus' />
% 	<rdfs:range rdf:resource='Composer' />
% 	<rdfs:subPropertyOf rdf:resource='createdBy />
% </owl:ObjectProperty>

% Note the use of rdfs:domain, rdfs:range, rdfs:subPropertyOf and rdfs:subClassOf properties within OWL statements.

% Now, if we want to take advantage of Property restrictions to define our « Composer » class, we can state for instance that « Composer » individuals are precisely those « Person » individuals who have « composed » at least one « MusicalOpus ». This lead us to write the following declaration: 

% <owl:Class rdf:ID='Composer'>
% 	<rdfs:subClassOf>
% 		<owl:intersectionOf rdf:parseType='Collection'>
% 			<owl:Class rdf:ID='Person' />
% 			<owl:Restriction>
% 				<owl:onProperty rdf:resource='composedBy' />
% 				<owl:minCardinality rdf:datatype='\&xsd;nonNegativeInteger'>1</owl:minCardinality>
% 			</owl:Restriction>
% 		</owl:intersectionOf >
% 	</rdfs:subClassOf>	
% </owl:Class>

% Note that the intersectionOf property requires a list of Classes declarations which can be given either by owl:Class or owl:Restriction – which is defined as a sub-class of owl:Class. 

% 7.4.3.b  Property characteristics
% Properties can also be defined as:
% transitive: P(x,y) and P(y,z) implies P(x,z). For instance, if x is located in y and y is located in z, then x is located in z. 
% symetric: P(x,y) if and only if P(y,x). If x is next to y, then y must be next to x. 
% functionnal: P(x,y) and P(x,z) implies y = z. The property has only one value for each individual it applies to. 
% inverseOf: P1(x,y) if and only if P2(y,x). If x is the « fatherOf » y, then y is the « childOf » x, thus making « fatherOf » the inverseOf « childOf » property. 
% inverseFunctional: P(y,x) and P(z,x) implies y = z.


% 7.4.4  Individual
% Ontology is not only about Classes and Properties, it enables us to state some facts about individuals. In order to describe them, we state them as Class instance, make use of Property or assert facts about their individuality. 

% Let's use our previous ontology description to decribe « Tchaikovsky » and the « 1812 overture ».

% <Person rdf:ID='Tchaikovsky' />
% <MusicalOpus rdf:ID='1812_Overture'>
% 	<composedBy rd:resource='Tchaikovsky' />
% </MusicalOpus>

% From this description, we can entail that:
% « Tchaikovsky » is not only a « Person », it is also a « Composer » because he composed the « 1812 overture ». As he composed it, we can also say that he has « created » it. 

% As OWL is a web language, it has rejected the « unique name » assumption and hence need to deal with identity uniqueness – and thus URI reference. In order to do so, three constructs have been defined:
% owl:sameAs states that two URI reference refer in fact to the same individual, meaning we can merge their definition and the related conclusions. 
% owl:differentFrom declares that two URI reference refer to different individuals. 
% owl:AllDifferent provides an idiom for stating that a list of individuals are all different. 

% Now, to demonstrate the use of sameAs property and to refer to our information sample, we define several « Tchaikovsky » individuals with different spelling. Then, we declare several MusicalOpus, « Cappricio italien » and « Ромео и Джульетта » which is the russian spelling for « Romeo and Juliet ».

% <Person rdf:ID='Tchaïkovski' xml:lang='fr' />
% <Person rdf:ID='Чайкoвский' xml:lang='ru' />

% <MusicalOpus rdf:ID='Capriccio_italien' xml:lang='fr'>
% 	<composedBy rd:resource='Tchaïkovski' />
% </MusicalOpus>
% <MusicalOpus rdf:ID='Ромео и Джульетта' xml:lang='ru'>
% 	<composedBy rd:resource='Чайкoвский' />
% </MusicalOpus>

% If we change the first statements by:
% <Person rdf:ID='Tchaïkovski' xml:lang='fr'>
% 	<owl:sameAs rdf:resource='Tchaikovsky'/>
% </Person>
% <Person rdf:ID='Чайкoвский' xml:lang='ru'>
% 	<owl:sameAs rdf:resource='Tchaikovsky'/>
% </Person>

% We can entail that there are the same person and that he has composed all three « MusicalOpus » described. 
















\subsection{Thésaurus et vocabulaires structurés}
\subsubsection{Besoins en modélisation}
%%%%%%%%%%%%%%%%%%%%%%%%%%%%%%%%%%%%%%%% à revoir
%La mise en place d'une telle application nécessite de représenter le vocabulaire de la réalisation audiovisuelle dans toutes ses variations possibles et de le documenter suffisamment afin de le rendre compréhensible pour des novices. 
Cet objectif nous amène à considérer la construction d'une ressource termino-ontologique.
L'ontologie permet de représenter les concepts partagés par les professionels de la réalisation audiovisuelle et la terminologie permet de capturer les différentes formes d'expression associées à ces concepts. 

La spécificité de notre problématique est de considérer la collaboration de communautés hétérogènes par leur degré de compréhension des concepts ou leur utilisation de la terminologie. 
Ceci nous amène à envisager la terminologie comme un moyen d'associer à des éléments ontologiques (concept, relation, instances) une chaîne lexicale ou des ressources média.
Chaque chaîne ou ressource s'adresse en particulier à une communauté dont les membres partagent une capacité d'interprétation commune. 
Il n'existe donc plus une terminologie de référence par langue, mais des terminologies pour chaque communauté d'utilisateurs. 
On remarquera que notre acception de la terminologie sert bien à normaliser les pratiques linguistiques entre les membres d'une même organisation. 
En plus de cela, elle permet de fixer la manière de s'adresser à d'autres communautés.

Par ailleurs, les types de réalisations sont divers et nécessitent des concepts spécifiques pour être décrits. Une fiction se structure en séquences et en scènes alors que les documentaires ou magazines d'information se composent de sujets. 
La variabilité des types de contenu à filmer implique donc de pouvoir étendre le fond conceptuel initial pour représenter de nouveaux usages. 
De la même manière, la collaboration avec de nouveaux partenaires nécessite de pouvoir ajouter de nouvelles terminologies au fond conceptuel existant. 
Ontologie et terminologie doivent se gérer de manière indépendante. 
À partir de ces besoins, nous définissons maintenant les exigences en terme de modélisation. 

Nos besoins en modélisation peuvent être exprimés par les assertions suivantes:
\begin{enumerate}
	\item[(A1)] la variabilité des pratiques linguistiques des organisations et des communautés implique d'associer plusieurs termes à un même concept. Il n'y a pas de choix des termes préférés par une communauté mais une \textit{correspondance} entre les termes d'une ou plusieurs communautés, quels que soient la langue et le code d'écriture utilisé.
	
	\item[(A2)] la variabilité de compréhension des communautés implique d'associer des explications (chaîne lexicale) ou des illustrations (ressource média) aux concepts afin d'en enrichir la \textit{documentation}. 
	
	\item[(A3)] la variabilité des cas de collaboration implique de pouvoir étendre la conceptualisation initiale ou la terminologie pour s'adapter à de nouvelles pratiques ou de nouvelles communautés. Cela implique une gestion et une \textit{évolution} indépendante de l'ontologie et de la terminologie. 
\end{enumerate}



\subsubsection*{Simple Knowledge Organization System}
\addcontentsline{toc}{subsection}{Simple Knowledge Organization System}
\e{
SKOS est un langage de représentation de vocabulaires structurés et de thésaurus qui repose sur RDF et OWL. 
Son objectif est de représenter tout type de SOC en vue de le publier sur le web de données,  liées et ouvertes (Linked Open Data). 
En témoigne les travaux et méthodes de conversion proposés par \cite{Summers2008} ou \cite{VanAssem2006} et les applications de gestion de thésaurus développés autour de SKOS, comme celle de \cite{Schandl2010}.
On notera que l'objectif de publication semble pousser vers une représentation minimale mais extensible de SOC déjà construits.
}

% \subsubsection*{SKOS}
\paragraph{Concept et Etiquette}
SKOS centre son modèle sur les concepts (\cd{skos:Concept}) et considère les étiquettes comme des propriétés de ces derniers. 
On distingue trois types d'étiquettes: 
\begin{itemize} 
	\item les étiquettes préférées (\cd{skos:prefLabel}) qui sont uniques par langue et servent de référence.
	\item les étiquettes alternatives (\cd{skos:altLabel}) qui servent de synonymes pour l'étiquette de référence. 
	\item les étiquettes cachées (\cd{skos:hiddenLabel}) qui servent à la récupération d'erreurs de frappes les plus courantes. 
\end{itemize}
Chacune de ces propriétés est formalisée comme une instance de \cd{owl:Anno\-tationProperty}, ce qui permet de l'attacher dans les faits à tout type d'éléments ontologiques, et pas seulement à des concepts. 
Les valeurs lexicales portées par ces étiquettes sont formalisées comme des \cd{rdf:PlainLiteral}, ce qui permet de spécifier la langue et l'alphabet utilisés. 
Par exemple, la chaîne "higashi"@ja-Latn correspond à un mot japonais écrit avec l'alphabet latin.

\paragraph{Documentation}
Différentes notes de documentation existent afin de :
\begin{itemize}
	\item définir un concept (\cd{skos:definition}), expliciter son contexte d'usage (\cd{skos:scopeNote}) ou donner des exemples (\cd{skos:example})
	\item spécifier l'historique de sa signification (\cd{skos:historyNote}), les changements effectués (\cd{skos:changeNote}) ou à faire (\cd{skos:editorialNote})
\end{itemize}
L'ensemble de ces notes est défini comme une spécialisation de \cd{skos:note}, formalisé comme une \cd{owl:annotationProperty}. 
De cette manière, les notes peuvent servir à porter de la documentation écrite (comme les étiquettes), pointer vers des ressources RDF ou des documents identifiés par une URI. 
Cela permet ainsi de prévoir l'extension de ces notes à des besoins plus spécifiques.


\paragraph{Groupes et relations entre Concepts}
Les concepts peuvent être regroupés dans des schémas conceptuels (\cd{skos:\-ConceptScheme} et relation \cd{skos:inScheme}) et structurés par différentes relations:
\begin{itemize}
	\item des relations de structuration hiérarchiques (\cd{skos:broader}, \cd{skos:na\-rrower}) ou associatives (\cd{skos:related})
	\item des relations de correspondances entre concepts de schémas différents, soit une relation d'équivalence exacte (\cd{skos:exactMatch}), une équivalence approximative (\cd{skos:closeMatch}), des relations hiérarchiques (\cd{skos:broadMatch}, \cd{skos:narrowMatch}) ou associative \cd{skos:rela\-ted\-Match}).
\end{itemize}


\paragraph{SKOS-XL}
SKOS-XL est une extension de SKOS développée courant 2008 pour proposer une modélisation alternative au vocabulaire de base et favoriser des extensions plus fines. 
Dans SKOS-XL les termes ne sont plus portés par les concepts mais deviennent des éléments à part entière (\cd{skosxl:Label}). 

Les relations d'attachement entre termes et concepts sont analogues aux attributs de SKOS mais les formalisent comme des instances de \cd{owl:objectProperty}. 
Le domaine de ces relations n'est pas défini ce qui permet de les associer à n'importe quelle ressource RDF, et donc en particulier aux concepts SKOS mais également à des ConceptScheme. 
Si cette dernière possibilité permet de créer des groupes de termes, elle introduit une confusion sur la sémantique des ConceptScheme (groupe de concepts, de termes, de triplets ?). 
Les étiquettes portent une seule chaîne lexicale avec les mêmes possibilités que pour SKOS grâce à l'attribut \cd{skos:literalForm}. 
L'indépendance des étiquettes permet également de spécifier des relations entre eux comme la synonymie, la traduction, etc. \cite{Pastor2009a}. 
Cette possibilité est ouverte par la relation générique \cd{skosxl:labelRelation}. 


Dans SKOS, la gestion de plusieurs jargons métiers dans une même conceptualisation [A1] est rendue difficile par la caractérisation simple des termes par la langue. 
Ainsi, même si on accroche plusieurs étiquettes au même concept, on ne sait pas les sélectionner pour les présenter à l'une ou l'autre communauté. 
Cela implique un dédoublement des concepts et donc des schémas conceptuels nécessaires (un par communauté).%, voir figure \ref{fig:skos}. 
%implique une représentation plus fine par rapport à SKOS ce qui 
Avec le découplage terme-concept de SKOS-XL, on peut gérer terme et concept de manière séparés sans pour autant avoir de primitives spécifiques pour regrouper les termes par jargon ou code d'écriture, voir figure \ref{fig:skosxl}. 
Une solution consisterait à regrouper les termes dans des ConceptScheme (SKOS-XL l'autorise).
Les ConceptScheme serviraient alors à la fois à regrouper les concepts (AV-Scheme) et les termes spécifiques aux organisations (RTBF-Scheme, VRT-Scheme). 
Cependant, si cette modélisation permet de gérer vocabulaire métier et conceptualisation de manière séparés c'est au prix d'un flottement sur la sémantique de ConceptScheme. 
Le support d'un nouveau jargon peut donc se faire sans toucher à la conceptualisation [A3] grâce à la permissivité de l'extension SKOS-XL. 
L'extension ou la mise en correspondance de la conceptualisation est facilitée par les relations sémantiques entre concepts.


\subsubsection*{ISO 25964-1}
\addcontentsline{toc}{subsection}{ISO 25964-1}
Cette norme propose une modélisation terme-concept similaire à SKOS-XL mais se concentre sur la représentation des thésaurus. 
Elle se fonde sur des modèles pré-existants, le méta-modèle \cite{Vandenbussche2009} ainsi que la norme BS 8723. 
L'originalité par rapport à SKOS est de considérer la composition de termes ou de concepts et d'enrichir la description des éléments du modèle par des attributs Dublin Core.

Les termes se distinguent entre termes préférés simples ou composés et termes non préférés simples. 
La caractérisation des termes porte également sur l'appartenance à une langue à laquelle s'ajoutent des attributs de dates, une définition ainsi que des notes d'historique et de révision. 
Des relations sémantiques entre termes sont également considérées en particulier l'équivalence composée, la synonymie, l'abréviation, l'acronyme, etc. 
Le thésaurus est considéré comme l'élément central décrit par l'ensemble des quinze attributs originaux du Dublin Core ainsi que des notes d'historique pour la maintenance. 
Sur les questions de documentation et de groupement de concepts, il existe une similarité importante avec les primitives de SKOS (note, groupe de concepts, etc.).\\

%\textbf{Discussion}\\
Les apports de la norme ISO 25964 par rapport à SKOS concernent davantage les pratiques de création et de maintenance de thésaurus que la gestion des jargons métiers [A1] et l'illustration de concepts par des ressources média [A2]. 
Ainsi, les manques par rapport à nos besoins sont similaires. L'attention portée sur les détails de description de chaque élément du modèle est tout à fait convaincante. 
L'écart avec l'aspect épuré et synthétique de SKOS s'explique certainement par l'écart entre leurs objectifs. 
Alors que SKOS vise la publication de tout type de SOC, l'ISO 25946 se concentre sur la construction et l'évolution des seuls thésaurus. 
La norme ne s'est pas encore attelée aux questions d'interopérabilité et de correspondance avec d'autres vocabulaires. Ce travail est en cours et sera dévoilé dans la seconde partie de la norme (ISO 25946-2). 
De notre point de vue, il manque toujours un moyen de regrouper des termes indépendamment des concepts pour ajouter des jargons à une conceptualisation existante [A3].


\subsection*{Bilan}
L'étude des standards et normes de références précédentes ne semble pas apporter de solution complètement satisfaisante pour l'ensemble de nos besoins. 
En effet, les approches restent centrées sur le concept et sa structuration auquel on intègre (SKOS) ou rattache (SKOS-XL, ISO 25946-1) ensuite les termes. 
Ces derniers sont représentés de manière plus ou moins fine (gestion des compositions dans ISO 25946 absente de SKOS). 
Dans tous les cas, la préférence d'un terme s'établit uniquement sur l'appartenance à une langue et non par rapport à une communauté de jargon. % mais sont caractérisés de manière identique dans les deux modèles (appartenance à une langue). 
De ce fait, ces modèles se limitent à représenter un seul jargon de référence par thésaurus et suppose ainsi l'existence d'une communauté homogène dans sa compréhension et dont on cherche à normaliser l'usage linguistique. %Dans notre cas, la collaboration entre communautés hétérogènes dans leur compréhension des concepts et leur utilisation de la langue exige de pouvoir gérer plusieurs jargons. C'est pourquoi 
Nous proposons dans la suite un modèle Multi-Jargon afin d'associer plusieurs jargons métiers et des explications à une conceptualisation commune. %basé sur des concepts originaux 


\subsection*{Discussion}
\addcontentsline{toc}{subsection}{Discussion}


% % \cleardoublepage

% %%%%%%%%%%%%%%%%%%%%%%%%%%%%%%%%%%%%%%%%%%%%%%%%%%%%%%%%%%%%%%%%%%%%%%%%%%%%%%%%%%%%%%%%%%%%%%%%%%%
\chapter{Modélisations de l'audiovisuel (m,i)}\label{chap:mav}
%
% Il existe de nombreux modèles, schémas et standards pour décrire les divers aspects de la chaîne de production audiovisuelle et des objets qui y sont construits. 
% Parmi ces modèles, certains sont issus d'une refléxion générale sur la description des ressources numériques. 
% L'exemple le plus emblématique étant le schéma de métadonnées de la \pc{Dublin Core Metadata Initiative} (\cite{DCMIUsageBoard2010}) qui doit servir à décrire toutes ressources sur le Web.
% De tels modèles ne suffisent pas à décrire les objets audiovisuels, il ont donc fait l'objet de spécialisation, par exemple \cite{Hunter1999}.
% D'autres adoptent une approche générale qui englobe l'ensemble des contenus multimédia.
% Le \pc{Moving Picture Experts Group} (MPEG) est certainement l'organisation qui a le plus porté cette vision avec leurs standards MPEG-7 et MPEG-21.
% Enfin, il y a des modèles développés spécifiquement par des membres de l'industrie pour répondre aux besoins de l'audiovisuel ou de la télévision.
% Il s'agit par exemple d'organisation comme le \pc{TV Anytime Forum}, la \pc{Society of Motion Picture and Television Engineers} (SMPTE) et l'\pc{Union Européenne de Radio-télévision} (UER ou EBU en anglais)

Les problèmes principaux qui se posent à la production audiovisuelle se situent dans la modélisation des objets qu'elle produit et des connaissances qui y sont associées (\ref{sec:pmetiers}). 
Le besoin d'autonomiser tous les objets de la chaîne audiovisuelle, en vue de les réutiliser dans de nouveaux contextes d'exploitations, exige en effet de réévaluer les modélisations sur ces deux points (\ref{sec:scien}) : 
\begin{liste}
	\item[(A)] \g{la modélisation des objets construits au fil de la chaîne de production audiovisuelle}.
	Il s'agit de modéliser non seulement le produit final et ses composants, mais également tous les produits intermédiaires de la chaîne.
	On entend par là tous les fragments de contenu construits ou transformés au cours de la chaîne de production (les prises de vue du tournage, le montage et la séquence monté, le programme prêt à diffuser, la version pour DVD, le résumé pour le journal télévisé etc.).
	Le fait qu'ils participent ou non à la composition d'un produit final n'implique pas qu'ils ne soient pas exploitable dans d'autres contextes.
	De la même manière, les fragments peuvent être transformés à différents niveaux (technique, esthétique, éditorial etc.) pour les adapter à d'autres usages. 

	De ce fait, la condition pour rendre ces fragments autonomes et réutilisables est de les modéliser en tant qu'éléments documentaires à part entière.	
	Mais identifier des fragments documentaires après coup ne suffit pas. 
	Il faut les modéliser dès que possible, afin de rendre compte de leur statut dans le processus de construction. 
	Ce faisant, il devient alors possible de reprendre ce processus, et de l'adapter aux besoins d'un nouveau contexte d'exploitation. 
	La modélisation de l'objet audiovisuel doit donc se faire sur différents niveaux et de manière progressive, en suivant les opérations meneés au cours de la chaîne de production.\\

	% On s'intéresse donc aux types d'objets audiovisuels modélisés, au niveau d'abstraction et de fragmentation proposé pour rendre compte de la composition de ces objets et de leur construction.\\
	% modèle de composition (\cite{Stockinger2007})

	\item[(B)] \g{la modélisation des connaissances construites sur ces objets}.
:

\end{liste}

Sur ce point, il s'agit de clarifier la nature des connaissances que l'on attache aux objets et leur pertinence vis-à-vis des usages que l'on prend en compte. 
Nous considérons trois types de connaissances à associer aux objets : 
\begin{liste}
	\item les connaissances sur les objets ; leur \e{représentation matérielle} (stockage, encodage, format etc.) ; leur \e{contenu} (ce qui est vu ou entendu par le lecteur).

	\item les connaissances liées à la chaîne de production ; la \e{spécification de la forme et du contenu} (que l'on retrouve dans les documents de pré-production) ; le \e{contexte de production} au sens large, incluant les contributeurs et leurs contributions à la chaîne ; le \e{cadre d'exploitation}  qui détaille l'usage de ces objets (type de distribution, droits et propriété intellectuelle, type de réutilisation et transformations opérées pour la réaliser, etc.).

	\item des connaissances issues de l'analyse du contenu des objets audiovisuels. 
	Par exemple, une analyse rhétorique du contenu permettra de mettre à jour la logique argumentative ou discursive (\cite{Gaillard2008}). 
	Ainsi, une multitude d'analyses peuvent être menées, chacune selon une grille d'analyse du contenu propre. 
	On examinera alors si les modélisations permettent d'ajouter de nouvelles échelles de fragmentation et d'y adjoindre des informations.
\end{liste}


\cite{ThiBui2003} propose 4 types de descriptions d'un contenu :
\begin{liste}
	\item \e{description syntaxique d'ordre sensoriel} : il s'agit de caractériser le signal et la manière dont il peut être perçu.
	Pour les aspects visuels, la couleur, la forme, la luminosité, le mouvement, la position etc.
	Pour les aspects sonores, la tonalité, le rythme etc.
   
	\item \e{description syntaxique d'ordre structurel} : le contenu peut être découpé en éléments de base. 
	Pour un contenu audiovisuel, il peut s'agir d'un découpage temporel (image ou segment temporel), d'un découpage visuel (portion de l'image) ou bien encore d'un mélange des deux.
	Le caractère syntaxique s'oppose au caractère sémantique et indique que le découpage est indépendant de sa signification. 
	Le découpage se fait donc de manière arbitraire. 
	Ilne correspond pas forcément à des objets signifiants pour un humain, comme un plan, une scène ou bien une table que l'on verrait à l'écran.
	Il est toujours difficile de trancher le passage d'un élément syntaxique à un objet signifiant car cela dépend de l'application que l'on considère. 

	\item \e{description sémantique d'ordre structurel} : le contenu est décrit par des objets signifiants. 
	Du point de vue temporel, on distinguera les scènes, les chansons des moments parlés, les refrains des couplets etc.
	Du point de vue spatial, il s'agit d'objets comme une chaise, ou bien de regroupements d'objets.

	\item \e{description sémantique des objets du monde narratif} 
\end{liste}

% Nous présenterons d'abord un scénario de réutilisation pour clarifier les usages que nous visons ainsi que les besoins en modélisation (\ref{sec:cdc-av}).
% Notre état de l'art sera nourri par un examen préalable des définitions de l'objet audiovisuel  (section \ref{sec:dav}).


% On peut tenter de distinguer entre différents approches et objets de modélisation : 
% \begin{liste}
% 	\item \e{les modélisations de l'objet audiovisuel} : celles qui traitent de la composition d'un objet audiovisuel fini, qui détaillent la manière de représenter sa structure interne, ou bien la manière dont on l'a groupé avec d'autres objets.
% 	\item 
% \end{liste}
 

% Certains se concentrent sur la description des caractéristiques du signal, des évènements que montrent le contenu ou encore de la manière dont ces contenus sont produits, échangés, adaptés etc.

% L'objectif de ce chapitre est d'examiner les modèles de l'audiovisuel existants en regard de nos  de réutilisation des objets audiovisuels. 
% modéliser les objets de la chaîne de production et les connaissances associées

% les objets de la chaîne de production
% les connaissances associées à ces objets permettant de les rendre autonome dans leur circulation et leur réutilisation.

% Plusieurs sous-problèmes pour réaliser l'autonomie des objets audiovisuels :  
% nous examinerons la manière dont on peut définir un objet, un document, un contenu audiovisuel. 
% comment modéliser ces objets pour gérer leur circulation 
% comment modéliser ces objets pour faciliter leur réutilisation


% d'un exemple de réutilisation d'objets audiovisuels (\ref{sec:cdc-av}). 
% À 
% , qu'il s'agisse de clarifier les notions d'objet ou de document audiovisuel puis d'investir les problèmes de leur gestion et de leur description. 
% Nous poserons d'abord un exemple de réutilisation d'objet audiovisuels comme élément de base de notre réflexion (\ref{sec:ex-reuse}).

% \e{
% Qu'est-ce qu'un objet audiovisuel et particulier comment peut-on aborder le document audiovisuel ? (\ref{sec:dav})
% Comment les produits de la chaîne audiovisuelle sont gérées par les systèmes informatiques, quelles opérations sont menées sur ces objets ? (\ref{sec:gest})
% Comment décrit-on les objets audiovisuels, comment s'organisent la construction ou la récolte de ces informations dans la chaîne de production ? (\ref{sec:desc})}

\section{Cahier des charges fonctionnel}\label{sec:cdc-av}
% \addcontentsline{toc}{section}{Cahier des charges fonctionnel}



\subsection{Scénario de réutilisation multiple}\label{sec:ex-reuse}
Pour bien saisir la finesse des différentes opérations possibles, nous proposons de prendre un exemple.
Imaginons une chaîne de télévision qui souhaite réaliser la captation d'un évènement culturel (par exemple un opéra, une pièce de théâtre, un concert etc.). 
Les producteurs de la chaîne sont intéressés par trois types de contenus qui seront ensuite exploités de quatre manières différentes (voir Figure \ref{img:intro:reuse}):
\begin{listenum}
	\item[a.] la \e{captation de l'évènement} en tant que tel.
	\item[b.] des \e{entrevues avec l'équipe} (metteur en scène, talents sur scène, programmateur etc.). 
	\item[c.] des \e{commentaires du public} avant ou après l'évènement.\\

	\item une partie de tous les types de contenu sera utilisée pour construire un sujet destiné à un \e{journal télévisé}. 
	\item un montage raccourci de l'évènement et des commentaires du public seront utilisés pour produire une \e{bande-annonce diffusée sur le Web}.
	\item un montage de la captation de l'évènement, des bonus comprenant les entrevues avec l'équipe ainsi que la bande-annonce utilisant les commentaires spectateurs seront intégrés dans le \e{DVD}.	 
	\item tout ou partie du contenu filmé pourra être transmis ou vendu à des \e{organisations tierces}. 
\end{listenum}

\begin{figure}[ht!]
\centering
\includegraphics[width=0.7\textwidth]{images/UC-Tahnhauser-v1fr.png}
\caption{Modèle de la production classique comparé avec une production avec réutilisation}
\label{img:intro:reuse}
\end{figure}

Chaque cas de réutilisation tire sa matière première d'à peu près la même base de contenu filmé, mais en tire partie d'une manière propre à chaque forme d'exploitation visée. 
En effet, chaque audience a ses attentes, de même qu'il existe des contraintes techniques spécifiques pour chaque contexte d'exploitation. 
%En effet, il existe des contraintes techniques et des attentes spécifiques à chaque contexte d'exploitation. 

Ces spécificités exigent des variations dans la qualité de l'encodage, le format d'encapsulation utilisé, le montage réalisé, l'habillage du contenu etc. 
Par exemple, les contraintes de diffusion sur le Web implique d'encoder la vidéo dans un format spécifique et de multiples résolutions, généralement plus petites que pour la diffusion télévisée. 
Ensuite, le montage d'une bande-annonce possède un rythme généralement plus rapide que celui des bonus de DVD. 
Finalement, les cas d'exploitation gérés par la chaîne de télévision posséderont un habillage spécifique (logo de la chaîne, message d'annonces etc.) que ne partageront pas forcément les versions vendu à des organisations tierces. 

L'exemple des commentaires du public -- voir la Figure \ref{img:intro:reuse-process} -- permet de montrer à quels moments des transformations doivent être effectuées afin de produire les différentes formes d'exploitation :
\begin{liste} 
	\item[$\bullet$] On considère que deux commentaires de spectateur ont été tournés. 
	\item[$\bullet$] Un des commentaires est intégré au montage du journal télévisé, alors que les deux sont utilisés pour créer la bande-annonce. La bande-annonce est elle-même intégrée au montage du DVD. 
	\item[$\bullet$] Au moment de la finition, l'encodage de la bande-annonce est adapté à la qualité DVD et Web. De même, le journal télévisé est encodé à la fois pour une diffusion en définition standard (SD) et haute-définition (HD).
\end{liste}


\begin{figure}[ht!]
\centering
\includegraphics[width=0.8\textwidth]{images/EX-Content-Production-v7fr.png}
\caption{Étapes et transformations des contenus pour chaque forme d'exploitation des commentaires des spectateurs}
\label{img:intro:reuse-process}
\end{figure}

% Dans cet exemple, on distingue deux types d'opérations effectuées sur le contenu ; 
% la sélection de séquences au moment du montage qui correspond à une décision éditoriale (quel contenu va-t-on présenter à l'audience ?) ; 
% la tranformation de l'enregistrement du contenu qui correspond à des choix techniques (quelle méthode d'enregistrement va-t-on utiliser ?).
% Afin de préciser la nature de ces opérations, nous présentons différentes approches de la réutilisation des contenus.


\subsection{Besoins en modélisation (n,i,t)}
Le scénario d'usage que nous venons de voir présente un exemple de production incluant directement plusieurs cadres d'exploitation pour des contenus produits en collaboration avec deux organisations professionelles et des amateurs. 
Ce scénario permet d'illustrer les échanges 



The details of this use case specifies more than a simple reuse of material. It specifies the kind of processing that support repurposing and which defines thus the modeling requirements for the audiovisual document:
%Such exploitation cases implies various kinds of operations: 
\begin{itemize}
	\item the \textit{reencoding} of edited material to fit the technical parameters proper to each distribution medium/channel (news report distributed by channel broadcasting and internet).
	
	\item the reuse of shooting materials in two distinct editorial structure (\textit{resequencing} of the opinion shot in the website and news report editing).

	\item the reuse of a part of an editorial structure into another editorial structure (\textit{repurposing} of the public comment editing into the DVD bonus editing). 
\end{itemize}


\section{Qu'est-ce qu'un objet audiovisuel ?}\label{sec:dav}


\subsection{Essence, contenu, asset}
\cite{Cox2006} : Essence + Metadata = Content ; \cite{Austerberry2004} Asset = Content + Rights to use it
Définition de Media Asset etc. de \cite{Furht2008}.

\subsection{Le document audiovisuel}
[Quelles sont les manières de représenter les objets/contenus/documents audiovisuels, quelle sont les différences entre ces notions.
\cite{Morizet-mahoudeaux2005a} ;  Voir thèse Charhad 2005. Voir AAF.]

]

\paragraph{Functional Requirements for Bibliographic Records}
FRBRoo est un modèle conceptuel développé par (\cite{Aalberg2008})

Il vise à faciliter l’échange d’information entre les bibliothèques numériques et les musées. 
Il permet de représenter les personnes participant aux différentes étapes de construction d’un objet culturel, depuis l’idée jusqu’à la réalisation matérielle.
Chaque objet culturel possède trois niveaux de modélisation :
\begin{liste}
	\item le niveau des idées ou des oeuvres (\g{Work}) n’ayant pas pris corps dans une matérialité externe à un sujet (par exemple une mélodie ou une histoire qui nous reste dans la tête). 

	\item le niveau des formes d'expression (\g{Expression}) où l'on distingue parmi toutes les formes possibles pour exprimer une idée (une nouvelle écrite, ses traductions, une adaptation de nouvelle en scénario, une lecture de cette nouvelle etc.).
	On se situe à un niveau intermédiaire qui définit des formes abstraites de  réalisation.
	Il faut préciser qu'on parle de forme abstraite dans le sens où il n'existe pas de réalisation concrète, ce qui n'empêche pas de les définir précisement et donc de distinguer de multiples variantes d'expressions :

	\ciel{
	the form of expression is an inherent characteristic of the expression, any change in form (e.g., from alpha-numeric notation to spoken word, a poem created in capitals and rendered in lower case) is a new expression. Similarly, changes in the intellectual conventions or instruments that are employed to express a work (e.g., translation from one language to another) result in the creation of a new expression.} 
	
	\item le niveau des réalisations concrètes comme les porteurs physique d’information (\g{Information Carrier}) portant les expressions (livre, partition, cd-rom etc.). 
	À ce niveau, il faut également distinguer entre l’original (\g{Manifestation Singleton}) et les copies manufacturées (\g{Item}) issues d’un modèle de publication (\g{Manifestation Product Type}). % à rapprocher de la notion de Media Profile dans MPEG-7
\end{liste}




% \paragraph{Sciences de l'information et de la communication}
% [\cite{Leleu-merviela} : le document comportent à la fois des dimensions sémiotiques (signes et sens), techniques (enregistrements, codages et transmission de signaux) et des dimensions médiatiques (socialisation et diffusion).]

% Avec le numérique : niveau des données (enregistrement en binaire, inaccessible et illisible pour l'humain), niveau du texte (une structure organisée de parties informationnelles), niveau de surface (actualisation effective ou affichage au sens large). 

% À la surface : \e{scénique} (manière de transposer des données en une réalité concrète) et \e{scénation} (manière de restituer temporellement à l'utilisateur les fragments d'un document, \ciel{la structure organisée d’événements et/ou d’états avec lesquels l’utilisateur est effectivement mis en interaction.}).

% \ciel{
% Cependant en numérique, les fragments existaient, au moins potentiellement, dans la mémoire de la machine, ce n’est que leur actualisation sur l’écran et la forme qu’elle prend qui se construit dans l’ici et maintenant de l’interaction. 
% Celle-ci est donc nécessairement volatile. De plus, elle change à chaque fois.
% Ainsi c’est l’affichage, [\dots] qui varie, mais non le document lui-même tel qu’il est mémorisé au niveau des données.}

% \ciel{
% conserver, retrouver l’information n’est pas suffisant. 
% Pour qu’elle puisse être utile, il faut qu’elle puisse être exploitée, c’est-à-dire traitée et rapprochée d’autres de façon à produire de l’information nouvelle. 
% Produire du sens n’est, pour l’essentiel, que rapprocher des informations disparates jamais rassemblées auparavant.} (\cite{Balpe1990})

% % Deux pistes proposées par SLM : 
% Il est alors possible de construire des assemblage cohérent de fragments le temps d'une consultation (d'un affichage) par un utilisateur (documents virtuels personnalisables).
% Plus on a de connaissance sur son activité, ses tâches, ses compétences propres, plus il est alors possible de rendre cette assemblage pertinent. 

% Il est aussi possible de mettre à profit la description des documents pour construire des notions de voisinage indépendamment du profilage des utilisateurs. 
% La proximité entre deux documents pourra s'évaluer d'autant de manière qu'il y a de critères descriptifs.
% Ainsi, des informations auparavant éparpillées dans des documents papier différents pourraient être regroupés. 

\newpage
\section{Circulation et réutilisation des objets audiovisuels}\label{sec:gest}

% [Voir MXF, voir AAF ? \cite{Cox2006}
% [Identifiant : hors du cadre de la thèse, dépendant des choix applicatifs des organisations qui utilisent notre modèle. Plusieurs solutions peuvent être implementés via OWL, les URI pouvant être transformé.]

\e{
Si la promesse du numérique de faciliter la manipulation et la circulation des fichiers semble bien s'être réalisée, il n'est pas si évident de l'articuler avec les besoins de la production audiovisuelle (\ref{sec:besoins}).
Ce que l'on nomme la réutilisation des objets audiovisuels recouvre en réalité diverses pratiques et qui repose plus sur la notion d'objet métier ou d'objet numérique que sur la notion informatique de fichier.
Ainsi, la production souhaite récupérer des contenus existants ou produits par d'autres pour les intégrer dans sa propre chaîne de production, ou bien de réutiliser des contenus dans de nouveaux cadres d'exploitations (variation des modes de consommation, de distribution, de public etc.) quelque soit la manière dont l'informatique représente ces objets.}

\e{
Ces opérations qui semblaient a priori plus simple dans un environnement numérique sont en fait plus compliquées qu'il n'y paraît. 
Le numérique impose le calcul et l'explicitation des informations.
Or toutes les informations construites durant la chaîne de production ne sont pas encore intégrées dans les systèmes informatiques actuels.
Lorsque ces informations s'échangent sur papier, à l'oral, par mail ou dans des fichiers non-structurés, le lien avec les objets audiovisuels est alors bien souvent rompu, ce qui entraîne une limitation des traitements réalisables sur ces objets.}

\e{
Dès lors que l'on s'applique à structurer et associer ces informations aux objets audiovisuels, on ouvre la possibilité de récupérer, manipuler, transformer ces objets de nouvelles manières. 
Ainsi augmentés d'un supplément de contexte, les objets gagnent un supplément de manipulabilité susceptible de satisfaire aux besoins de la production audiovisuelle.
Une des solutions développée et utilisée dans l'industrie de la production audiovisuelle est le format conteneur qui encapsule divers types de données en un seul fichier. 
Ainsi, ces formats permettent d'associer de multiples types de fichiers multimédia avec d'autres types d'informations.}

\e{
Cette section a d'abord un souci de clarification des usages et des solutions adoptées. Nous nous intéresserons d'abord aux pratiques de réutilisations (\ref{sec:reuse}), puis nous expliquerons leurs impacts sur la chaîne de production audiovisuelle (\ref{sec:rechaine}). 
Enfin, nous présenterons des formats conteneurs qui assurent le transport des contenus et des informations associées le long de la chaîne de production (\ref{sec:wrapper}).}





%%%%%%%%%%%%%%%%%%%%%%%%%%%%%%%%%%%%%%%%%%%%%%%
\subsection{Caractériser la réutilisation}\label{sec:reuse}
% \subsubsection{Caractérisations de la réutilisation}\label{sec:caracs-reuse}
Nous avons vu grâce à l'exemple de la section \ref{sec:ex-reuse} à quel moment et dans quel type d'opérations la réutilisation pouvait se concrétiser. 
Nous proposons maintenant d'examiner la manière dont différentes communautés scientifiques  abordent la notion de réutilisation. 
Il s'agit de clarifier les hypothèses et les techniques proposées par chacune de ces communautés, et ainsi identifier les éléments pris en compte dans leur représentation du monde.  % Correction ?

\paragraph{Multimédia et Signal}
Prenons d'abord le cas de la communauté multimédia très orientée analyse et traitement du signal. 
Dans ce cadre, les constats mis en avant sont largement les mêmes que ceux que nous avons présentés précédemment (voir section \ref{sec:motiv}, multiplication et diversification des terminaux de lecture et des réseaux de communication, transformation des usages) :
 
\ciel{ 
Hundreds of device profiles are available for accessing online content and more announced everyday. These devices are connected through a wide variety of networks [\dots] As before, the issue of usage scenarios --activity type, user age and gender, time available, and prior knowledge of the subject matter-- continues to exist.} (\cite{Singh2004}).

Un point diffère cependant, le \gui{problème} de la variabilité des usages est considéré comme de même nature que la variabilité des technologies pour transférer et lire le contenu. 
En effet, l'approche de la réutilisation privilégiée par cette communauté consiste en une transformation automatique du contenu en fonction des paramètres d'un scénario de distribution et de lecture : 

\ciel{
Fundamental to this approach is the need to maintain a single copy of the content in its original form and to repurpose the content to fit the desired scenario in real time and in an automated fashion. [\dots] the next step in the repurposing process is to describe the content so that it can be understood and processed to fit delivery requirements --whether they're technical or usage based.} (\cite{Singh2004}).

L'approche automatique est justifiée par la difficulté à maintenir et gérer différentes versions d'un même contenu, en plus d'être coûteux et chronophage.
Ainsi, la décision humaine est simplement reportée au niveau du paramétrage du système de supervisation des opérations techniques.\\


\paragraph{Ingénierie Documentaire}
Dans la communauté de l'ingénierie documentaire, le principe est de pouvoir modéliser distinctement le message que l'auteur souhaite transmettre et la forme dans laquelle ce message se donne à voir par un lecteur. 
Cette tradition, que l'on pourrait faire remonter à la fin des années 60 avec la création du \e{Generalized Markup Language} (\cite{Goldfarb}) ancêtre des SGML, HTML, XML et consorts, repose sur le balisage d'un contenu source. 
Il s'agit alors d'identifier des fragments de contenu ainsi que leur structuration pour mieux les manipuler, quelque soit les opérations effectuées sur ces fragments (transformation, indexation, réécriture etc. \cite[chap.5.2]{Bachimont2004}). 
Les langages de modélisation documentaires tels que \e{Document Type Definition} ou \e{XML Schema} (\cite{Fallside2004}) permettent de contrôler par une grammaire les systèmes de balises construits en vue de formaliser des usages documentaires. 

Nous noterons le développement récent des \gui{chaînes éditoriales}, ces systèmes qui opérationnalisent l'hypothèse de base de l'ingénierie documentaire reformulée par \cite{Crozat2004} de la sorte : \ciel{tout contenu numérique consiste en une ressource qu’un calcul permet de publier dynamiquement sous différentes formes contextualisées}. 

Ces systèmes se concentrent ainsi sur le maintien d'une ressource de base que l'on peut transformer ensuite de diverses manières, soit par une transformation technique que l'on pourra automatisée, soit par une transformation manuelle réglée sur les usages visés (\cite{Crozat2011}) : 
\begin{liste}
	\item le polymorphisme \ciel{consiste en la possibilité technique de disposer d'une source unique de contenu et de la transformer à volonté selon les supports et mises en formes désirés}. Dans ce cas, on établit une séparation entre le fond (la source documentaire) et les formes de publication qui permet de mettre en place une production multi-support.

	\item la réutilisation \ciel{par référence (sans duplication d'information) consiste en la possibilité technique de désassembler et de ré-assembler des fragments de contenu afin de les partager entre plusieurs documents}. Dans ce cas, l'opération repose sur une modélisation séparée du scénario (la structuration) et le contenu.

	\item la ré-éditorialisation est une \ciel{remise en contexte de fragments issus d'un fonds documentaire, par leur ré-agencement au sein d'un nouveau document, leur augmentation par une création de contenus spécifiques et leur publication sur un nouveau support et/ou pour un nouveau public}.

	% \item[T] l'\e{intégration multimédia} est l'exploitation de la propriété héritée du numérique et du codage binaire de permettre l'inscription sur le même support de formes sémiotiques différentes (texte, image, audio, vidéo, ...), afin de composer des contenus multimédia.
\end{liste}

Notons que les chaînes éditoriales s'orientent vers des pratiques de ré-éditorialisation qui sont réalisées manuellement plutôt que de manière automatique.
Les définitions données du polymorphisme et de la réutilisation sont des définitions d'opérations techniques plutôt que des pratiques en tant que telle. 
Ces opérations sont donc permises et prises en charges par les chaînes éditoriales mais ne constituent par leur horizon d'usage.
 % et paramétrées par des règles définies par un utilisateur.
Il semble donc que l'Ingénierie documentaire traditionnelle et le courant lié aux chaînes éditoriales s'intéressent tous deux à des opérations techniques similaires (le polymorphisme et la réutilisation) mais visent des usages distincts qui ne posent pas les mêmes problèmes :
\begin{liste}
	\item D'un côté, il s'agit d'instrumenter d'automatiser des réécritures, entre objets multimédia mais aussi entre documents structurés en XML, base de données etc. 
	Un exemple classique est la création de compte-rendu (ou reporting) qui s'effectue en extrayant des données de diverses sources puis en les intégrant dans de nouveaux documents.

	\item De l'autre on vise à fournir un nouvel environnement de travail aux métiers de l'édition (auteur, éditeur, graphiste etc.) qui permet de passer d'une production artisanale à une production multi-support, réutilisable, ré-éditorialisable. 
	Dans ce cadre, on s'intéresse plus à la création de documents non automatisable telle que les supports pédagogiques par exemple (\cite{Crozat2007}).\\
\end{liste}


\paragraph{Sémiotique Audiovisuelle}
% transition définition Stockinger
Alors que les approches précédentes se concentrent sur des techniques et des outils particuliers, l'approche sémiotique que nous présentons ici propose un point de vue plus général pour définir les différents types de réutilisations existants. 

La sémiotique s'intéresse aux signes pour étudier les activités humaines associées, qu'il s'agisse des producteurs (et de leur intention de communication), des lecteurs (et de leur interprétation des signes produits) ou des relations entre producteur et lecteur (c'est-à-dire des conventions qu'ils partagent). 
Selon \cite{Peirce1978} le \gui{signe} est composé de trois éléments ; le \e{représentamen}, ce qui représente et qu'on pourrait rapprocher de la notion de signifiant chez \cite{DeSaussure1995} ; l'\e{objet}, ce qui est représenté ; l'\e{interprétant} qui produit la relation entre les deux premiers éléments. 

En sémiotique, le signe fait donc toujours l'objet d'une interprétation de la part d'un lecteur qui mobilise un ensemble de conventions pour tenter d'extraire un sens --qui n'est pas forcément l'intention qu'a voulu exprimé l'auteur.
La transmission d'un contenu ne suffit pas en soi à garantir la réussite de la communcation, celle-ci est toujours suceptible d'échouer (soit par un défaut d'expression, un défaut de convention, un défaut d'interprétation). 

Dans ce cadre théorique, la réutilisation de contenu ne se limite pas à une transformation technique (conversion de formats d'encapsulation, de taille d'image, d'encodage) mais se conçoit comme une \gui{adaptation culturelle} \parencite{Stockinger2007} d'une ressource vis-à-vis d'un contexte qui comprend à la fois un usage et une communauté cible. 
Les contenus n'ont donc pas de valeur en soi, mais une valeur d'usage pour une communauté. 
La réutilisation, l'adaptation culturelle ou encore la republication interviennent alors lorsque les contenus sources ne satisfont pas à leur utilisation ou leur communauté de lecteurs future : 

\ciel{
La \e{republication} (en anglais re-authoring ou re-purposing) recouvre un ensemble d'activités visant à réutiliser un corpus de documents numériques (textuels, audiovisuels, visuels, etc.) pour des usages spécifiques auxquels les documents sources, dans leur forme initiale, ne peuvent que partiellement répondre} \parencite{Stockinger2007b}.


On l'aura compris, ce processus englobe des opérations techniques et éditoriales et place les conventions des communautés dans une position centrale. 
Pour ce qui est de caractériser des communautés d'utilisateurs, \pc{Stockinger} se réfère à des sociologues dont \cite{Bourdieu}, et propose différents critères de regroupement :
% la notion d'habitus développée par
\begin{liste}
	\item le temps ou l'espace occupé
	\item les activités et les objectifs recherchés
	\item les attentes et les intérêts
	\item les compétences linguistiques
	\item et de manière générale les connaissances ou les représentations
\end{liste}

Une fois une communauté cible identifiée, il est alors possible ; (1) de définir le type et la forme de contenu qui est pertinent (utilisable, utile, compréhensible, acceptable etc. par ces utilisateurs) ; (2) les outils nécessaires pour effectuer les opérations propres à adapter le contenu aux besoins de la communauté cible :

\ciel{
La republication est donc un processus, parfois très complexe, d'adaptation d'un document ou d'un corpus de documents sources à des usages spécifiques. Ce processus d'adaptation peut concerner tous les plans constitutifs d'un document (Stockinger, 1981, 1999 et 2003), c'est-à-dire aussi bien le plan du contenu que celui de l'expression. Il s'accomplit à travers un ensemble d'activités intellectuelles et de gestes techniques et en référence à des modèles ou genres de publications qui intègrent les contraintes typiques des contextes et des communautés d'usage auxquels un document ou un corpus de documents republiés est destiné} \parencite{Stockinger2007b}.

% opérations : (traitement linguistique, restructuration, rééditorialisation).
La republication repose donc sur une représentation des communautés d'utilisateurs, de leur capacités d'interprétation ainsi que sur une représentation des contenus dont elles disposent habituellement. 
La republication se définit selon \parencite{Stockinger2007} suivant les critères suivants : 
\begin{liste}
	\item \e{les opérations à effectuer} ; sélection, réorganisation, ajout d'explications, ajout d'éléments complémentaires, traduction, mise en lien avec d'autres ressources, modification de la forme d'expression, création de nouveau contenu etc.
	\item \e{le type} (image, texte, objet audiovisuel etc.) \e{et le genre} (journal télévisé, émission etc.) \e{de ressources à traiter}. 
	\item \e{l'objectif de la réutilisation} ; le contexte d'usage, la communauté cible, le genre de la future publication, le format de distribution etc.
	\item \e{les ressources à disposition pour effectuer la republication} ; les personnes, les outils, le budget, les ressources intellectuelles.
\end{liste}
Cette approche générale de la réutilisation n'est pas purement intellectuelle puisqu'elle se concrétise également dans des développements logiciels. 
En effet, un logiciel nommé \gui{Atelier Sémiotique} se développe dans le cadre de l'\gui{Atelier de Sémiotique Audiovisuelle} et par l'intermédiaire de divers projets (Saphir, Logos) et partenaires (INA, ESCoM, MSH de Paris).

\paragraph{Discussion}
% Notons que cette définition, par rapport à celle des approches précédentes, décrit de manière plus globale ce qu'est la réutilisation en citant de nombreux et nouveaux éléments à prendre en compte. 
% L'analyse proposé par la sémiotique audiovisuelle propose une définition plus générale de ce qu'est la réutilisation. 
L'analyse proposée par \pc{Stockinger} propose une définition générale de la réutilisation qui englobe les pratiques présentées précédemment. 
En effet, l'ingénierie documentaire et la communauté multimédia se concentrent sur la construction d'outils pour automatiser certaines transformations ou réécritures de contenu.
En se concentrant sur un éventail de techniques, ces approches se prêtent plus à certains cas d'usages et visent des objectifs différents. %(comme la ré-éditorialisation pour l'ingénierie documentaire ou bien la construction automatique de compte-rendu pour la communauté multimédia).
% parler de C2M qui vise à une chaîne éditoriale multimédia ?  

\begin{figure}[ht!]
\centering
\includegraphics[width=0.75\textwidth]{images/Reuse-v1.png}
\caption{Les différentes pratiques de réutilisations}
\label{img:intro:reuse}
\end{figure}

Nous proposons donc de définir la terminologie suivante pour distinguer entre trois niveaux successifs de réutilisation, chacun visant à créer un nouveau document mais suivant des opérations différentes (voir Figure \ref{img:intro:reuse}). 
La premier critère distinctif est l'automatisation de la transformation, le second critère repose sur la création originale de contenu plutôt que la réorganisation d'un existant : 
\begin{liste}
	\item le \eg{retraitement} (repurposing) qui se caractérise par une automatisation de la transformation opérée sur le contenu (quelque soit son type), c'est-à-dire que cette transformation est effectué par un logiciel lui-même paramétré par un humain. 
	Ces transformations visent à modifier la forme d'expression du document, extraire des fragments de contenu de différentes sources pour les aggréger dans un nouveau document, ou bien encore réorganiser automatiquement la structure du contenu. 
	Le retraitement dépasse le polymorphisme en ce sens où il est possible de gérer de multiple sources de contenus pour construire dynamiquement un nouveau document. 
	Dans le cas de l'audiovisuel, il s'agit des pratiques de réencodage, de changements de format d'encapsulation etc.
	Pour reprendre l'exemple précédent (\ref{sec:ex-reuse}) d'un contenu TV pour une diffusion Web, ou bien encore la création automatique de résumé de rencontres sportives etc. \\

	\item la \eg{rééditorialisation} (reediting) se caractérise par une transformation (manuelle et automatique) de contenus existants. 
	Le document doit s'adapter à un nouveau contexte de lecture (genre éditoriale, public, forme d'expression etc.).
	La transformation du contenu nécessite une compréhension du nouveau contexte de lecture et consiste en des opérations de réorganisation, de mise en relation avec d'autres contenus, de traduction etc. 
	Ces opérations ne se limitent pas à une transformation de la forme d'expression du document (retraitement) mais ne constituent pas une création originale de contenus (réécriture). 
	Simplement, on réutilise divers contenus existants pour créer un nouveau document.  
	La ré-éditorialisation repose donc sur le polymorphisme et la réutilisation au sens de \cite{Crozat2011}.
	Dans le cas de l'audiovisuel, il s'agit typiquement de pratiques de re-montage et de nouvelles sélections de contenu. 
	Pour reprendre l'exemple précédent, il s'agit de monter différemment une séquence initialement prévue pour un journal TV et qui doit s'insérer dans un DVD etc. \\

	 
	\item la \eg{réécriture} (reauthoring) se caractérise par une transformation de contenus existants accompagnée d'une création de contenu original. 
	L'ajout de contenu sert à satisfaire soit aux attentes spécifiques du nouveau public cible, aux contraintes d'un nouveau genre éditorial (commentaires, explications, exemples etc.) soit à la création d'une version augmentée d'un document existant (pas de changement de public cible, mais de nouvelles attentes).
	Dans le cas de l'audiovisuel, il s'agit par exemple de la construction d'un documentaire à partir de vidéo d'archives, la création originale étant le commentaire proposé.\\	 
\end{liste}

Les pratiques de réutilisations sont donc chevillées aux dimensions techniques, éditoriales et sémiotiques du contenu audiovisuel. 
Leur mise en place pose également des problèmes dans l'organisation de la chaîne de production audiovisuelle et son informatisation
% Il faut donc élargir le champ de la modélisation des contenus à une dimension sémiotique et éditoriale et faire le lien avec le déroulement de la production.








%%%%%%%%%%%%%%%%%%%%%%%%%%%%%%%%%%%%%%%%%%%%%%%
\subsection{Évolutions de la chaîne de production}\label{sec:rechaine}
\e{
Les changements introduits par la réutilisation dans la chaîne de production sont donc plus vastes qu'une simple adaptation technique à de nouveaux modes de distribution. %(canal de diffusion + terminal de lecture). 
Il s'agit également de prendre en compte l'audience visée pour affiner encore plus l'adaptation du contenu à ses futurs consomateurs/lecteurs.
L'objectif est de favoriser le développement de variantes d'un même programme soit par la restructuration du contenu (retraitement ou rééditorialisation) ou par l'ajout de contenus (réécriture). 
L'introduction d'acteurs tiers dans une chaîne de production pour fabriquer ou fournir du contenu ne peut se faire sans une plus grande maîtrise des contenus et une meilleure description de ces derniers dès la pré-production.
En effet, le client qui souhaite déléguer la fabrication de contenu à un fournisseur tiers doit d'abord définir ses attentes. 
À l'inverse, si le fournisseur connaît son contenu le client lui a besoin d'un descriptif pour sélectionner les fragments les plus pertinents.
Ainsi, les chaînes de production des clients et des fournisseurs doivent évoluer pour gérer (fournir/acquérir) non pas juste du contenu, mais des descriptions (adjointes ou pas à du contenu) facilitant le travail de leurs partenaires (fabrication ou réutilisation).
}
% Le producteur-diffuseur rentre alors dans une dynamique d'adaptation de ces contenus.

Comme nous l'avons constaté en \ref{sec:electro}, le développement de l'électronique offre de nouvelles opportunités de production mixte, soit avec des amateurs, soit avec d'autres professionnels. 
Cependant, une organisation souhaitant profiter de ces opportunités devra réussir d'abord à encadrer ses partenaires et clarifier avec eux les termes de leurs accords. 
Ce qui auparavant pouvait se résoudre \e{de visu} ou de manière informelle doit maintenant être explicité afin de clarifier la demande, c'est-à-dire le contenu souhaité. 
Qu'il s'agisse de passer commande, ou bien de rechercher dans des bases existantes, cette étape s'apparente à la définition du besoin, à l'écriture d'un cahier des charges, ou dans les termes propres à la production audiovisuelle, au \e{Scripting} défini en \ref{sec:preprod}. 
Maintenant que la fabrication du contenu est déléguée à des tiers, il reste cependant à récupérer le résultat et à vérifier qu'il satisfait à la demande initiale. 
Cette dernière étape nommée généralement \e{Acquisition} constitue un travail à part entière puisqu'il s'agit de \gui{faire rentrer} le contenu dans les \gui{cases} du système d'information et de gestion des contenus. 
En plus des questions de formats informatiques, s'ajoute souvent le problème de la description des contenus et de leur classification en vue de leur utilisation future.  
L'acquisition dépend grandement des conventions établies avec le fabricant/livreur de contenu et impacte directement sur le temps passé à faire le \e{derushing}. 

Côté client, on transforme d'abord la phase de scripting en l'expression d'une \pc{Commande} ou d'une \pc{Requête} ce qui permet de déléguer la fabrication des contenus à un tiers.
Ensuite, on vérifie par \pc{Acquisition} du contenu que le résultat correspond bien à la demande et on procède aux ajustements nécessaires (si besoin) pour satisfaire aux contraintes de notre système (voir Figure \ref{img:intro:evochain}). 

\begin{figure}[ht!]
\centering
\includegraphics[width=\textwidth]{images/Workflow-Thesis-v6.png}
\caption{Ouverture des chaînes de production du client et du fournisseur}
\label{img:intro:evochain}
\end{figure}

Du côté des fournisseurs de contenu, il existe une distinction entre les chaînes du fabricant et du fournisseur du contenu :  

\begin{liste}
	\item \e{déléguer la fabrication à des contributeurs tiers} : l'utilisation du scripting pour définir la \pc{Commande} de contenu attendu semble une solution satisfaisante, à condition que le vocabulaire utilisé soit normé et rattaché à une conceptualisation de manière à éviter la confusion ou les différences d'interprétations. 
    Lorsqu'il s'agit d'amateurs, la situation se complique car on ne peut pas s'appuyer sur une conceptualisation commune de la production audiovisuelle pour clarifier la commande. 
    De plus, le manque d'expérience et l'ignorance des usages du métier impliquent non seulement de documenter les concepts par des mots et des définitions, mais aussi d'expliquer ce qu'il faut faire durant la phase de \pc{Fabrication}. 
    En d'autres termes, en travaillant avec des amateurs, les professionnels ne se retrouvent non pas à écarter la confusion entre des mots se reférant au même concept, mais à expliquer les opérations auxquelles ces concepts font référence. 
    De même pour l'acquisition, s'il s'agit surtout de se mettre d'accord entre professionnels, travailler avec des amateurs semble plus difficile de prime abord. 
    Les notions de formats d'encodage et d'encapsulation sont souvent confuses ou se mélangent, de même que la description des contenus peut s'avérer compliquée à réaliser sans expérience préalable. 
    Tout du moins, il faut remarquer que la description de la demande initiale sert de description a minima du contenu produit, même si les variations ou les écarts ne sont pas forcément indiqués.
    Le cas échéant, une phase d'\pc{Indexation} peut être nécessaire pour décrire le contenu suivant les exigences du client.\\

	\item \e{rechercher des contenus existants depuis les bases professionnelles} : l'utilisation du vocabulaire de l'écriture audiovisuelle pour définir une \pc{Requête} nécessite une indexation utilisant ce même vocabulaire, ou alors une manière de traduire la requête d'un langage à l'autre (par alignement des vocabulaires par exemple). 
	De même, il faut pouvoir s'accorder sur le niveau de fragmentation recherché (programme complet, séquences, scène, frame etc.), le format du contenu, les descriptions ou les métadonnées à fournir etc.
	Ainsi, la \pc{Livraison} de contenu ne consiste pas en un simple transfert de fichier, mais représente le moment où l'on teste l'interopérabilité entre les systèmes et les formats. 
	Cette étape est d'autant plus cruciale qu'elle se répercute directement sur la phase d'acquisition pour le client. 
	Tout ce qui n'a pas pu être résolu à la livraison (côté fournisseur) devra l'être au moment de l'acquisition dans le système (côté client).\\
	% un vocabulaire de requêtes, s'accorder sur les niveaux de fragmentation, le format de livraison, le contenu de la livraison
\end{liste}



Finalement, il nous faut encore éclairer à quels moments dans la chaîne de production les différentes pratiques de réutilisation sont réalisées (voir définition en \ref{sec:reuse}).
De manière générale, on considère que la réutilisation commence à la phase de pré-production, au moment du \pc{Planning} et du \pc{Scripting} où l'on spécifie les nouvelles formes et formats d'exploitation (voir Figure \ref{img:intro:reutilisation}). 
Mais chaque pratique opére à différents étapes de la chaîne :
\begin{liste}
	\item pour le \e{retraitement}, les variations sur la forme d'expression du document se réalisent en phase de \pc{Finition}. 
	C'est à ce moment que l'original et les variantes sont encodés et encapsulés dans les formats correspondants à leur mode de distribution. 
	Lorsqu'il y a manipulation de la structure des contenus, ces opérations (automatisées) se réalisent à la phase de \pc{Montage}.

	\item pour la \e{rééditorialisation}, le travail commence en phase de \pc{Derushing}, par la sélection des séquences de contenu à ajouter ou à retirer du contenu original. 
	La grande différence avec le retraitement, c'est que cette sélection s'effectue manuellement sur les contenus à disposition. 
	Ensuite, on opére un nouveau \pc{Montage} qui peut également impliquer une \pc{Finition} différente.

	\item pour la \e{réécriture}, la grande différence avec les autres pratiques est que l'on ajoute une nouvelle phase de \pc{Fabrication}. 
	Qu'il s'agisse de création originale ou de récupération de contenu chez un fournisseur tiers, la réécriture consiste à utiliser de nouveaux contenus. 
	Ensuite, on sélectionne en \pc{Derushing} les séquences qui permettront de réaliser un nouveau \pc{Montage}. 
	Finalement, plusieurs \pc{Finition} sont à envisager suivant les cas d'exploitation.
\end{liste}
% conséquences de la réutilisation dans la chaîne : à quel étape ça se joue

\begin{figure}[ht!]
\centering
\includegraphics[width=0.8\textwidth]{images/Workflow-Reuse-v1.png}
\caption{Les différentes formes de réutilisation et leur mise en oeuvre dans la chaîne de production audiovisuelle.}
\label{img:intro:reutilisation}
\end{figure}








\subsection{Formats conteneurs pour l'audiovisuel} \label{sec:wrapper}
% intro
Les formats conteneurs sont des formats de fichiers qui encapsulent des contenus de toutes sortes (audio, vidéo, texte, image etc.). 
Un des exemples le plus connu pour la vidéo est le format AVI de Microsoft, souvent confondu avec un format de compression. 
Ces formats ont la particularité d'associer aux contenus audio-visuels des informations annexes, le plus souvent sous la forme de métadonnées.
L'intérêt de ces formats est de constituer un objet numérique qui structure l'enregistrement du contenu et de ses informations annexes et fournit ainsi une manière unique d'y accéder (\cite{Ferreira2010}).
Sans cela, les informations seraient dispersées et le lien avec le contenu devrait se faire via des références. 
Cette force constitue également un inconvénient lorsqu'il n'est pas possible de faire évoluer le modèle d'information ou bien d'en changer en fonction du type de production ou d'exploitation envisagé.

Dans le cas de la production télévisuelle, deux formats liés et complémentaires ont été progressivement adoptés par l'industrie.  
Il s'agit du \pc{Material eXchange Format} (MXF) et de l'\pc{Advanced Authoring Format} (AAF) que nous présentons par la suite. 
Leur particularité est de pouvoir intégrer des schémas de métadonnées propres aux besoins de l'industrie, mais aussi de pouvoir en utiliser d'autres. 
Comme ces formats servent de référence à l'industrie, il est particulièrement intéressant pour nous de comprendre leur modélisation de l'objet audiovisuel, de voir comment ils gèrent les résultats intermédiaires de la chaîne de production ou quels genres d'informations ils embarquent avec le contenu.


\subsubsection{Material eXchange Format}
% qui / quand
MXF est un format conteneur ouvert développé depuis le milieu des années 90 par des membres de l'industrie et standardisé en 2004 par la \pc{Society of Motion Picture and Television Engineers} (SMPTE).
% objectif
Son objectif est de favoriser les échanges de contenus audio-visuels finis en les associant à d'autres données ou métadonnées (\cite{Devlin2002}).
Ces métadonnées sont structurés par le schéma \pc{Descriptive Metadata Scheme-1} (DMS-1, ) que nous présenterons en détails dans la suite de la section. 

% description
Voici d'abord les principales caractéristiques de MXF en tant que format (\cite{Ferreira2010a}) : 
\begin{liste} 
	\item \e{indépendant d'un système propriétaire}. 
	Le standard se veut avant tout un format qui fonctionne sur tout systèmes. 
	Ainsi, il définit une organisation des données bit par bit qui repose notamment sur le système de \pc{Key-Length-Value} (KLV, clé-longueur-valeur). 
	La clé donne un identificateur de l'élément à suivre, la longueur précise la taille de la valeur à suivre et la valeur contient les données de l'élément.
	Comme dans tout format de fichier, on retrouve la définition d'en-tête et de fin de fichier. 
	L'en-tête contient des métadonnées de description du contenu (\pc{Header Metadata}), de sa structuration (\pc{Partition Metadata}) et une table d'association entre un timecode et une position dans le flux binaire du fichier (\pc{Index Table}).

	\item \e{indépendant des méthodes de compression du contenu utilisées}.
	MXF définit une structure d'encapsulation et d'accès au contenu nommé \pc{Essence Container} (EC) qui permet de transporter du contenu sans le transformer ou bien de faire référence à des fichiers externes.  
	MXF possède des transpositions permettant de synchroniser différents flux de contenus, quelque soit leur encodage. 
	Chaque type de contenu est traité à part, ainsi les EC sont composés de \pc{Content Package}, eux-mêmes décomposé en \pc{Picture Item} (piste vidéo), \pc{SoundItem} (piste audio), \pc{Data Item} (télétexte, sous-titre etc.), \pc{Compound Item} (contenu audiovisuel encodé comme un seul contenu) et \pc{System Item} (autres données comme les timecode etc.).
	La Figure \ref{img:mxf-content} montre deux méthodes d'encapsulation de ces essences.


	\item \e{diffusable en flux continu ou bien par fichier}. 
	Suivant la méthode d'organisation de l'EC décrit ci-dessus, le contenu d'un fichier MXF peut être visionné au cours de son transfert (streaming). 
	Cette caractéristique est particulièrement importante dans le cadre de la diffusion télévisuelle et s'applique à tout les types de contenu d'un MXF (audio-visuel, mais aussi sous-titre ou métadonnées etc.). 
	Naturellement, le fichier MXF peut également être transféré en FTP.

	\item \e{une organisation des contenus indépendante de leur visionnage}. 
	En effet, MXF définit à part l'organisation des contenus encapsulés et la manière de les visionner.
	Le \pc{Header Metadata}	d'un fichier MXF contient une partie \pc{File Pacakage} (FP) qui décrit la manière dont les fichiers de contenus sont encapsulés dans le MXF. 
	Cette description détaille les méthodes d'encodage pour toutes les pistes de contenus (Track) du fichier, de même que les timecode originaux.
	Cependant, si le FP décrit les sources d'un fichier MXF, il existe également un \pc{Material Package} (MP) décrivant la manière dont celles-ci doivent être visionnées.
	Il s'agit là de définir un montage simplifié qui explicite quelle partie et dans quel ordre jouer les contenus sources, à la manière d'une \ciel{Edit Decision List}. 

	\item \e{encapsulation conjointe des contenus et des métadonnées}. 
	Comme nous l'avons vu, un fichier MXF contient un \pc{Header Metadata} transportant des métadonnées propres à l'ensemble du fichier ainsi que des métadonnées propres à chaque paquet de données (Package). 
	Ces éléments sont donc intégrés dans la structure du fichier, au même titre que le contenu.
\end{liste}


\begin{figure}[ht!]
\centering
\includegraphics[width=0.9\textwidth]{images/MXF-ContentPackage.png}
\caption{Deux méthodes d'encapsulation des essences en MXF : image par image (haut, pour le streaming), par séquence vidéo (bas).}
\label{img:mxf-content}
\end{figure}

\paragraph{Adoption et usage}
Du fait de sa large adoption par l'industrie de la télévision, il favorise l'interopérabilité entre les systèmes, souvent propriétaires, des producteurs, diffuseurs, chaînes de télévision etc. (\cite{Ferreira2010}, \cite{Devlin2002}). 
Cependant, MXF n'est pas fait pour gérer les résultats intermédiaires de la chaîne de production. 
Il a été spécifiquement conçu pour favoriser la circulation des programmes finis, indépendamment de la manière dont les contenus sont matériellement enregistrés et structurés.
De ce fait, MXF se positionne comme un format utilisé en fin de chaîne de production, à la diffusion des programmes ou bien dans le cas d'échanges entre professionnels.

\subsubsection{Description Metadata Scheme-1}
Le schéma DMS-1 a été standardisé par le SMPTE en 2004 (\cite{Smpte2004}).
Il propose trois schémas de description (\pc{Framework}), chacun proposant une perspective de description particulière : 
\begin{liste}
	\item le \pc{Production Framework} propose une description du fichier MXF en tant que résultat d'une production. 
	Les informations qu'il regroupe s'appliquent donc au fichier en entier (identification, propriété intellectuelle, droits et contrats, projet, format de publication, format d'image, récompense) mais aussi au contenu de l'objet audiovisuel (évènements relatés, période historique, annotation).

	\item le \pc{Clip Framework} aborde la description du point de vue de la création du matériel audio-visuel, c'est-à-dire des séquences de contenu encapsulées dans le MXF. 
	On retrouve des informations liées à la production (projet, droits et contrat) mais surtout des éléments pour décrire les essences (format de l'image, sous-titre, script utilisé, matériel utilisé et paramètrage, opérations de transformation des essences) et une description par plan.

	\item le \pc{Scene Framework} propose une vision éditoriale du contenu en le découpant en scène et plan. 
	Ces éléments sont ensuite décrits en terme d'évènements, en précisant les participants et les lieux où ils se déroulent etc.
\end{liste}

\paragraph{Framework et ensemble de métadonnées}
Ces \pc{Framework} sont composés de petits ensembles de métadonnées, parfois partagés,  que l'on attachent au \pc{Header Metadata} d'un fichier MXF. 
Par exemple, l'ensemble \pc{Titles} est commun au trois \pc{Framework} et se compose des métadonnées suivantes ; \e{Extended Text Language Code} ; \e{Main Title} ; \e{Secondary Title} ; \e{Working Title} ; \e{Original Title} ; \e{Version Title}. 
De nombreux autres ensembles sont partagés, comme les annotations, la description des lieux, des participants, des organisations etc. 
Ces ensembles prennent alors un sens différent en fonction du \pc{Framework} auquel ils sont associés. 
Par exemple, on distingue les participants à la production, à la création d'une séquence, en tant que présentateur ou acteur (respectivement pour le \pc{Production}, \pc{Clip} et \pc{Scene Framework}). 
De même, pour les lieux il peut s'agir d'un lieu où se trouve l'organisme producteur, du lieu de tournage, du lieu où se déroule l'action qui est différent du lieu de tournage dans le cas d'une fiction (respectivement pour le \pc{Production}, \pc{Clip} et \pc{Scene Framework}). 
Ainsi, la distinction entre éléments réels (issu du \pc{Clip Framework}) ou fictifs (issu du \pc{Scene Framework}) n'est pas clairement spécifié. 
De manière générale, il semble regrettable que les mêmes ensembles de métadonnées soient utilisés pour décrire des objets différents.
Cela propage ainsi une certaine confusion sur le plan sémantique. 

\paragraph{Framework et Package MXF}
Comme pour les ensembles, les \pc{Framework} peuvent s'attacher à un ou plusieurs des \pc{Package} de MXF (\pc{File Package}, \pc{Material Package} etc.). 
Par exemple, le \pc{Clip Framework} appliqué au \pc{Material Package} décrit ce qui est nécessaire au visionnage du contenu (format d'image prévu).
S'il s'appliquait au \pc{File Package}, les informations correspondrait aux informations de création (format d'image original).
Là encore, l'objet de la description change légèrement et si les mêmes éléments de description peuvent être utilisé, il semblerait plus clair de préciser la nature de ces informations. 

Nous remarquerons que DMS-1 utilise la notion de plan de deux manière différentes qui peuvent sembler ambigüe. 
Ainsi, le plan est utilisé à la fois dans le \pc{Clip Framework} et le \pc{Scene Framework}. 
D'après les auteurs, cela correspond à la nature duale des plans, à la fois élément factuel et éditorial. 
Ce choix implique alors que chaque plan peut être décrit à deux endroits à la fois, selon deux perspectives différentes (descriptive de ce qui est perçu, ou bien pour nommer le plan par rapport au script par exemple).

\paragraph{Multilingue et thésaurus}
Concernant la gestion des vocabulaires et des langues, DMS-1 prévoit que certains ensembles de métadonnées soient décrites par un code de langue et puissent faire référence à un élément d'un thésaurus.
Cette perspective est particulièrement intéressante vis-à-vis des besoins que nous avons exprimés dans le chapitre précédent (\ref{sec:bm}).
Cependant, le lien avec un thésaurus externe est limité car il ne s'applique qu'aux ensembles et non à chaque métadonnée de l'ensemble.

\paragraph{Travaux liés}
\cite{Marcos2009} ont construit une ontologie basée sur le schéma DMS-1, MPEG-7 et une ontologie de domaine pour construire un système de Media Asset Management. 
Ce système, développé dans le cadre du projet européen RUSHES, a pour objectif d'aggréger des informations récoltées pendant la production par différentes sources, puis de les associer aux objets audiovisuels et proposer des services de recherche d'information et d'accès au contenu (\cite{Gorka2008}). 
Les objets audiovisuels considérés sont les prises de vue brutes, nommées \ciel{rush} dans le milieu audiovisuel.

L'approche consiste à transformer ou relier des informations de bas-niveau en une indexation sémantique à l'aide des ontologies développées.
Les services sémantiques proposés par le système sont les suivants : 
\begin{liste}
	\item \e{transformation de formats des données échangées pendant la production}. 
	Les annotations recueillies à partir des équipements de tournage et après analyse automatique du contenu sont transformés du format DMS-1 en un autre format utilisé par le système d'indexation.
	\item \e{sémantisation de l'analyse automatique du contenu}. 
	Les auteurs prennent l'exemple d'une reconnaissance des visages qui permet d'identifier le nombre de personnes présents dans une séquence.
	Ces personnes peuvent alors être intégrées à une base de connaissances.
	De même, l'analyse permet de détecter les changements de plans.
	\item \e{recherche, découverte, annotation de séquences}. 
	La recherche peut se faire soit à l'aide de mots-clés, soit à l'aide des concepts de l'ontologie, qui permettent alors d'enrichir les requêtes, de proposer des recommandations etc. 
	De plus, l'ontologie peut être également utilisé pour relier les annotations manuelles avec des concepts ou des éléments de la base de connaissances.
\end{liste}

Ces travaux poussent ainsi l'utilisation de DMS-1 en tant que schéma de description des prises de vue, juste après leur production, et non pas seulement des programmes finis, en fin de chaîne. 
Ce changement de granularité montre l'importance du \pc{Clip Framework} et du \pc{Scene Framework} qui permettent d'attacher la description à ces objets intermédiaires de la production.
Ceci est d'autant plus pertinent que DMS-1 prévoit déjà de décrire, en partie, les participants de la chaîne et leurs contributions.

L'originalité de l'approche se situe également dans la transformation des résultats d'une analyse automatique en objet sémantique. 
Les exemples de l'analyse des visages et de la détection de plan sont éclairants mais les auteurs ne fournissent pas d'indication pour généraliser le procédé à d'autres types d'information.



\subsubsection{Advanced Authoring Format}
% \cite{Gilmer2002} 
% \cite{Austerberry2004}
% qui 
AAF est un format conteneur développé principalement par l'\ciel{Advanced Media Workflow Association} (AMWA) en collaboration avec d'autres organismes tels que le SMPTE et l'EBU.
% objectif, portée et usage 
Ses objectifs sont similaires à celui de MXF, à la différence qu'AAF vise à favoriser les échanges de contenus à l'intérieur de la chaîne de production (\cite{Austerberry2004}).
AAF s'occupe plus particulièrement des informations utilisées au moment de la post-production par les applications de montage : 

\ciel{The traditional workflow – based around tape interchange, isolated non-linear editing and authoring tools, and ad-hoc metadata systems – is being recast as a more integrated networked system with a consistent approach to the format and interchange of essence and metadata.} (\cite{Gilmer2002})

% description
Le modèle de MXF présenté précédemment est en réalité une sous-partie de celui d'AAF. 
On retrouve donc les mêmes principes et fonctions, dont les \pc{Packages} qui portent les descriptions et les \pc{Items} qui encapsulent les contenus. 
Parmi les éléments supplémentaires dans AAF qui le destine particulièrement à une utilisation dans la chaîne de production., nous trouvons :
\begin{liste}
	\item le \pc{Physical Source Package} permet de référencer des contenus enregistré sur d'autres mediums que les disques durs (cassette vidéo, bande 35mm etc.).

	\item le \pc{Composition Package} permet de définir la manière dont le contenu doit être visionné en termes d'ordre (comme MXF) mais aussi en terme d'effets, de transition ou de composition des flux de contenu (ce que l'on appelle des EDL complexes).
	
	\item le \pc{Dictionary} qui permet d'intégrer des définitions des métadonnées autres que celles du dictionnaire du SMPTE dans le AAF.
\end{liste}

De plus, AAF se différencie par l'utilisation de la technologie \e{Structured Storage} de Microsoft pour gérer l'organisation des données (plutôt que la méthode du KLV).
De ce fait, AAF ne permet pas la diffusion en continu (streaming) de ses contenus.
Cependant, les deux formats utilisent le même modèle de structuration, ce qui permet aux applications d'effectuer des transformations de l'un à l'autre aisément, et particulièrement du AAF vers le MXF en suivant le déroulement de la chaîne.
Leurs différences les prédisposent néanmoins à des usages complémentaires. 
AAF se positionne comme un format pour la post-production qui conserve toutes les sources et le master alors que MXF, avec son modèle simplifié et ses capacités de diffusion en continu, est particulièrement intéressant pour les échanges de programmes finis.


\paragraph{SMPTE Metadata Dictionary}
Ce dictionnaire est un gigantesque registre de toutes les métadonnées utilisées par l'industrie télévisuelle. 
Régulièrement mis à jour, la dernière version disponible (\cite{SMPTE2010}) comporte 1476 métadonnées distribuées dans 499 catégories et sous-catégories.
La nature des métadonnées est très diverse, puisqu'on trouve des identificateurs, des informations administratives, interprétative, paramétriques, liées au processus etc.
Le dictionnaire donne une identification unique à chaque métadonnée et donne une définition ainsi que le codage utilisé pour la valeur. 
Malgré sa taille imposante, il est utilisé par les membres de l'industrie et notamment dans DMS-1.



\subsubsection*{Discussion}
\addcontentsline{toc}{subsection}{Discussion}
Les formats conteneurs MXF et AAF reposent sur le schéma de description DMS-1 ainsi que des dictionnaires de métadonnées développés par l'industrie.
L'originalité de ces formats est d'associer directement au matériel audiovisuel plusieurs perspectives de modélisation (\pc{Framework}) [\g{B1 : autonomie}].
L'objet audiovisuel est ainsi modélisé en tant que résultat d'une production qu'il faut valoriser commercialement (\pc{Production}) ; 
décomposé en éléments narratif (plan, scène) faisant partie d'un ensemble documentaire (\pc{Scene}) et dont on décrit le contexte historique (\pc{Production}) et les évènements réels (\pc{Clip}) ou fictifs (\pc{Scene}) qui s'y déroulent ; matériel audiovisuel construit pendant la production dont on décrit les caractéristiques techniques (\pc{Clip}).
Cette pluralité des points de vue pourrait permettre de modéliser les produits intermédiaires de la chaîne tout en leur associant des métadonnées. 
Ce remplissage progressif ne peut intervenir qu'après la production du matériel, et encapsulé dans le format AAF. 
C'est donc les informations de la post-production qui y sont capturées, puis transmises sous une forme simplifiée à un MXF qui symbolise le produit final de la chaîne.
Les produits intermédiaires ne sont donc pas représentés pour eux-mêmes, mais en tant que partie du produit final.
L'approche de modélisation est donc intéressante, mais le couplage matériel et métadonnées empêche de fragmenter la modélisation et de la commencer dès le début de la production.

Parmi les descriptions associées au matériel audiovisuel [\g{B2 : réutilisabilité}], on note un certaine confusion sur le plan sémantique. 
Ainsi, des mêmes ensembles de métadonnées sont utilisées pour représenter des éléments fictifs ou rééls, tandis que d'autres peuvent prendre un sens différents suivant le \pc{Package} ou le \pc{Framework} auxquels ils sont associés.
Ainsi, il ne s'agit pas de critiquer la modélisation qui distingue élément réél et fictif, ou bien encore le plan prévu dans le script et le plan tourné, mais bien la représentation confuse qui en est faite dans DMS-1.

Nous en concluons que les formats conteneurs sont adaptés à la circulation de documents audiovisuels finis mais dont la représentation d'une seule pièce et parfois confuse ne couvre pas nos besoins pour la réutilisation de fragments documentaires.

\input{CONTENU/1-analyse/7-description}

% \cleardoublepage




%%%%%%%%%%%%%%%%%%%%%%%%%%%%%%%%%%%%%%%%%%%%%%%%%%%%%%%%%%%%%%%%%%%%%%%%%%%%%%%%%%%%%%%%%%%%%%%%%%%
%%%%%%%%%%%%%%%%%%%%%%%%%%%%%%%%%%%%%%%%%%%%%%%%%%%%%%%%%%%%%%%%%%%%%%%%%%%%%%%%%%%%%%%%%%%%%%%%%%%
\part*{Contribution}
\addcontentsline{toc}{part}{Contribution}

\chapter{Approche et modélisation}\label{chap:mod}
\section{Principes de l'approche}\label{sec:principes}
\section{Modélisation conceptuelle}\label{sec:concept}

\chapter{Mise en oeuvre}\label{chap:op}
\section{Choix d'un langage}\label{sec:ln}
\section{Opérationnalisation}\label{sec:op}


% \cleardoublepage


%%%%%%%%%%%%%%%%%%%%%%%%%%%%%%%%%%%%%%%%%%%%%%%%%%%%%%%%%%%%%%%%%%%%%%%%%%%%%%%%%%%%%%%%%%%%%%%%%%%
%%%%%%%%%%%%%%%%%%%%%%%%%%%%%%%%%%%%%%%%%%%%%%%%%%%%%%%%%%%%%%%%%%%%%%%%%%%%%%%%%%%%%%%%%%%%%%%%%%%
\part*{Discussion}
\addcontentsline{toc}{part}{Discussion}
\chapter{Applications et validation}\label{chap:app}
% Le modèle évolue en fonction des besoins des membres du projet MediaMap. Les développeurs des applications sont susceptible d"étendre ou d'affiner le modèle suivant les besoins qu'ils rencontrent. Il ne s'agit pas encore d'un standard figé mais bien d'un effort en cours et en collaboration avec les utilisateurs. Nous nous efforçons donc d'adopter une modélisation et un vocabulaire proche de leur contexte de travail en vue de privilégier l'adoption du modèle, son utilisation et sa réappropriation. Ensuite, il s'agit de formaliser la modélisation afin de pouvoir l'opérationnaliser et développer des applications qui produisent des descriptions suivant ce modèle ou se nourrissent d'elles pour restituer une perspective métier à ces utilisateurs. 
\section{Applications du projet MediaMap}\label{sec:app}
\section{Expérimentation du projet MediaMap}\label{sec:xp}
\section{Validation}\label{sec:val}

\chapter*{Conclusion}\label{chap:cc}
\addcontentsline{toc}{chapter}{Conclusion}

% \chapter{}\label{c:}
% \section{}\label{s:}

% \cleardoublepage

%%%%%%%%%%%%%%%%%%%%%%%%%%%%%%%%%%%%%%%%%%%%%%%%%%%%%%%%%%%%%%%%%%%%%%%%%%%%%%%%%%%%%%%%%%%%%%%%%%%
%%%%%%%%%%%%%%%%%%%%%%%%%%%%%%%%%%%%%%%%%%%%%%%%%%%%%%%%%%%%%%%%%%%%%%%%%%%%%%%%%%%%%%%%%%%%%%%%%%%
%%%%%%%%%%%%%%%%%%%%%%%%%%%%%%%%%%%%%%%%%%%%%%%%%%%%%%%%%%%%%%%%%%%%%%%%%%%%%%%%%%%%%%%%%%%%%%%%%%%
%%%%%%%%%%%%%%%%%%%%%%%%%%%%%%%%%%%%%%%%%%%%%%%%%%%%%%%%%%%%%%%%%%%%%%%%%%%%%%%%%%%%%%%%%%%%%%%%%%%
%%%%%%%%%%%%%%%%%%%%%%%%%%%%%%%%%%%%%%%%%%%%%%%%%%%%%%%%%%%%%%%%%%%%%%%%%%%%%%%%%%%%%%%%%%%%%%%%%%%
%%%%%%%%%%%%%%%%%%%%%%%%%%%%%%%%%%%%%%%%%%%%%%%%%%%%%%%%%%%%%%%%%%%%%%%%%%%%%%%%%%%%%%%%%%%%%%%%%%%
%%%%%%%%%%%%%%%%%%%%%%%%%%%%%%%%%%%%%%%%%%%%%%%%%%%%%%%%%%%%%%%%%%%%%%%%%%%%%%%%%%%%%%%%%%%%%%%%%%%
%%%%%%%%%%%%%%%%%%%%%%%%%%%%%%%%%%%%%%%%%%%%%%%%%%%%%%%%%%%%%%%%%%%%%%%%%%%%%%%%%%%%%%%%%%%%%%%%%%%
%%%%%%%%%%%%%%%%%%%%%%%%%%%%%%%%%%%%%%%%%%%%%%%%%%%%%%%%%%%%%%%%%%%%%%%%%%%%%%%%%%%%%%%%%%%%%%%%%%%
%%%%%%%%%%%%%%%%%%%%%%%%%%%%%%%%%%%%%%%%%%%%%%%%%%%%%%%%%%%%%%%%%%%%%%%%%%%%%%%%%%%%%%%%%%%%%%%%%%%
%%%%%%%%%%%%%%%%%%%%%%%%%%%%%%%%%%%%%%%%%%%%%%%%%%%%%%%%%%%%%%%%%%%%%%%%%%%%%%%%%%%%%%%%%%%%%%%%%%%
% \appendix
% \part*{Annexes}
% \rehead{\headingsf\scshape Annexe~\thechapter}
% \noappendicestocpagenum
% \addappheadtotoc

% \cleardoublepage

%%%%%%%%%%%%%%%%%%%%%%%%%%%%%%%%%%%%%%%%%%%%%%%%%%%%%%%%%%%%%%%%%%%%%%%%%%%%%%%%%%%%%%%%%%%%%%%%%%%
%%%%%%%%%%%%%%%%%%%%%%%%%%%%%%%%%%%%%%%%%%%%%%%%%%%%%%%%%%%%%%%%%%%%%%%%%%%%%%%%%%%%%%%%%%%%%%%%%%%
%%%%%%%%%%%%%%%%%%%%%%%%%%%%%%%%%%%%%%%%%%%%%%%%%%%%%%%%%%%%%%%%%%%%%%%%%%%%%%%%%%%%%%%%%%%%%%%%%%%
% \chapter{D'un Web à l'autre : les paradigmes de la lecture informatique}\label{a:webs}
% \KOMAoptions{twoside=no}

% \pagestyle{empty}

% \input{PERRON/GARDE}

% \newpage

% ~

% \cleardoublepage

% \pagestyle{empty}

% \input{PERRON/REMERCIEMENTS}

% \cleardoublepage

% \KOMAoptions{twoside=yes}

\frontmatter

\pagestyle{scrheadings}

\shorttableofcontents{Sommaire}{2}

\cleardoublepage


%%%%%%%%%%%%%%%%%%%%%%%%%%%%%%%%%%%%%%%%%%%%%%%%%%%%%%%%%%%%%%%%%%%%%%%%%%%%%%%%%%%%%%%%%%%%%%%%%%%
%%%%%%%%%%%%%%%%%%%%%%%%%%%%%%%%%%%%%%%%%%%%%%%%%%%%%%%%%%%%%%%%%%%%%%%%%%%%%%%%%%%%%%%%%%%%%%%%%%%
\mainmatter

% \pagestyle{empty}

% ~

% \bigskip

% \vspace{11em}

% \bigskip

% \epigraphii{La totalité est la non vérité.}{Adorno, \ita{Minima Moralia}}

% \cleardoublepage
\addcontentsline{toc}{part}{État de l'Art}
\pagestyle{empty}

~
\bigskip

\vspace{11em}

\bigskip

\epigraphii{Se demander si un ordinateur peut penser ... est aussi intéressant que de se demander si un sous-marin peut nager.}{ Edsger Wybe Dijkstra, \ita{The threats to computing science}}

\pagestyle{scrheadings}

%%%%%%%%%%%%%%%%%%%%%%%%%%%%%%%%%%%%%%%%%%%%%%%%%%%%%%%%%%%%%%%%%%%%%%%%%%%%%%%%%%%%%%%%%%%%%%%%%%%
%%%%%%%%%%%%%%%%%%%%%%%%%%%%%%%%%%%%%%%%%%%%%%%%%%%%%%%%%%%%%%%%%%%%%%%%%%%%%%%%%%%%%%%%%%%%%%%%%%%
\part*{Exposition}
\addcontentsline{toc}{part}{Exposition}

%%%%%%%%%%%%%%%%%%%%%%%%%%%%%%%%%%%%%%%%%%%%%%%%%%%%%%%%%%%%%%%%%%%%%%%%%%%%%%%%%%%%%%%%%%%%%%%%%%%
%%%%%%%%%%%%%%%%%%%%%%%%%%%%%%%%%%%%%%%%%%%%%%%%%%%%%%%%%%%%%%%%%%%%%%%%%%%%%%%%%%%%%%%%%%%%%%%%%%%
% \section*{Préambule (n)}
% \addcontentsline{toc}{section}{Préambule}



%%%%%%%%%%%%%%%%%%%%%%%%%%%%%%%%%%%%%%%%%%%%%%%%%%%%%%%%%%%%%%%%%%%%%%%%%%%%%%%%%%%%%%%%%%%%%%%%%%%
%%%%%%%%%%%%%%%%%%%%%%%%%%%%%%%%%%%%%%%%%%%%%%%%%%%%%%%%%%%%%%%%%%%%%%%%%%%%%%%%%%%%%%%%%%%%%%%%%%%
\chapter{Introduction}\label{chap:intro}
% \epigraphii{On peut s'étonner que les actes spontanés par lesquels l'homme a mis en forme sa vie, se sédimentent au dehors et y mènent l'existence anonyme des choses. La civilisation à laquelle je participe existe pour moi avec évidence dans les ustensiles qu'elle se donne.}{Merleau-Ponty\\\hfill\ita{Phénoménologie de la perception}}

%%%%%%%%%%%%%%%%%%%%%%%%%%%%%%%%%%%%%%%%%%%%%%%%%%%%%%%%%%%%%%%%%%%%%%%%%%%%%%%%%%%%%%%%%%%%%%%%%%%
% \section{Contexte}\label{chap:contexte}
Le travail de thèse dont nous rendons compte dans ce mémoire s'est déroulé dans le cadre du projet MediaMap\footnote{Voir \url{http://www.mediamapproject.org/}} soutenu par le cluster européen Eureka Celtic et la Direction Générale de la Compétitivité, de l'Industrie et
des Services (DGCIS) du ministère de l'Economie, des finances et de l'industrie.
La participation à ce projet de recherche et développement nous a imprégnés de connaissances sur la production audiovisuelle.
%, les besoins qui émergent des tendances actuelles et les problèmes rencontrés pour les combler. 



%Nous nous intéresserons ensuite (\g{Chapitre \ref{chap:problo}}) aux 


%%%%%%%%%%%%%%%%%%%%%%%%%%%%%%%%%%%%%%%%%%%%%%%%%%%%%%%%%%%%%%%%%%%%%%%%%%%%%%%%%%%%%%%%%%%%%%%%%%%
\section{L'impact du numérique sur l'audiovisuel}\label{sec:motiv}
\e{
La révolution numérique initiée depuis une trentaine d'années en conjonction avec la révolution électronique et informatique, s'est progressivement imposée à tous types d'information et de contenus. 
%jusqu'à devenir prépondérante et indispensable dans un monde informatisé.
Ces mouvements d'informatisation des pratiques et de numérisation de l'information impactent les organisations et les métiers en redistribuant les tâches entre humains et machines. 
Dans le cadre de cette thèse, nous nous pencherons sur le cas de l'audiovisuel et de la numérisation de ce contenus et des informations associées.}

%L'audiovisuel représente à plus d'un titre un objet singulier.
%l'exploitation des contenus numériques est en passe de s'étendre à toutes les étapes du cycle de vie d'un objet audiovisuel. % ? pourquoi exploitation ?
Après une informatisation des étapes de postproduction (logiciels de montage, effet spéciaux etc.) puis des équipements de captation (caméra, micro etc.) et de la distribution du côté des diffuseurs, nous avons connu une véritable explosion d'appareils destinés au grand public (lecteur multimédia portable, appareil photo, téléphone portable, dictaphone etc.).


Plusieurs grands chantiers s'ouvrent désormais dans ce mouvement d'informatisation des étapes de la chaîne de production :
\begin{liste}
	\item L'archivage des contenus de manière à les faire rentrer dans l'histoire malgré la dégradation inexorable des supports. 
	Il ne s'agit pas simplement de leur permettre de survivre jusqu'à la prochaine génération d'appareils électroniques, mais également de garantir sa \ciel{lisibilité technique et culturelle}, (\cite[p.~12]{Bachimont2000}).

	\item La préproduction des contenus qui est presque inexistante et qui permettrait de récolter des informations sur les contenus avant même leur fabrication. 
	L'enjeu plus général est d'initier l'indexation des contenus au moment du \e{Scripting} et de la continuer tout au long de la chaîne, chaque étape pouvant rajouter des informations supplémentaires ou bien réévaluer les anciennes.
\end{liste}

À côté de la numérisation et de l'informatisation, il ne faudrait pas oublier le développement considérable des réseaux de télécommunications qui a grandement favorisé les échanges de contenus de manière illégale ou légale, de pairs à pairs (\e{Peer to Peer}), entre diffuseur et spectateurs (\e{Business to Consumers}) ou entre acteurs professionnels de la chaîne (\e{Business to Business}). 
Ainsi, numérisation et informatisation sont dorénavant implictement associées aux facilités de transfert de ces réseaux. 

Sans tenter de tirer toutes les conséquences de ces révolutions technologiques (électronique et informatisation, numérisation, mise en réseaux) sur les usages des spectateurs et des acteurs de la chaîne de production audiovisuelle, on retiendra trois constats important qui charpentent notre réflexion : 
\begin{liste}
	\item \eg{il existe un nombre croissant de contenus audiovisuels en circulation.}
	L'offre et la demande augmentent de même que les capacités de transfert des réseaux et la généralisation des appareils électroniques de captation et de visionnage.

	\item \eg{les pratiques en lien avec les contenus audiovisuels se développent et s'individualisent.}
	La consommation de ces contenus se joue de plus en plus à un niveau individuel depuis l'apparition d'appareils personnels de communication et de visionnage. 
	De même, la production de contenus est facilité par l'informatisation et les progrès des appareils électroniques de captation. 
	La création de contenus n'est donc plus l'apanage de grandes équipes spécialisées et lourdement équipées.
	
	\item \eg{la numérisation complique le maintien de l'unicité des contenus audiovisuels}. 
	%Le principe même du numérique (\ciel{ça a été manipulé} [Bachimont2010]) repose sur la représentation de l'information et le calcul. 
	L'environnement numérique, par rapport à l'analogique, est plus propice à la copie, le transfert, la fragmentation et la manipulation des contenus ramenés invariablement à une donnée muette quelle que soit sa nature (texte, son, image animée ou non etc.) ou sa signification.
	Ainsi coupé des sens et du sens, les contenus semblent perdre leur identité dans le \gui{monomédia} numérique (\cite[p.~13]{Bachimont2000}) dans lequel on a peine à les gérer pour ce qu'ils représentent. 	

	%plus manipulables au sens où chaque visionnage construit une forme perceptive du contenu 

	%est une reconstruction de leur forme perceptible par des appareils de . Il y a donc par définition une certaine instabilité 
	%mobile, versatile, altérable, mouvant, variable, instable
\end{liste} 

%Le développement de l'électronique a ainsi favorisé une  le nombre de contenus audiovisuels, à favoriser leur circulation
%On constate ainsi une augmentation des sources de contenus ainsi qu'une intensification de leur circulation qui ont des conséquences sur les usages des spectateurs et les acteurs de la chaîne de production.
Nous allons maintenant préciser cet argumentaire et présenter les enjeux posés par la numérisation, l'informatisation et le développement de l'électronique.




%%%%%%%%%%%%%%%%%%%%%%%%%%%%%%%%%%%%%%%%%%%%%%%
\subsection{La numérisation des contenus audiovisuels}\label{sec:num}
Le contenu audiovisuel est avant tout un objet temporel, c'est-à-dire un flux d'images et de sons qui s'écoule en un temps donné. 
Un des enjeux liés à l'audiovisuel constitue alors de trouver une technologie d'enregistrement et de manipulation de ces flux.

\paragraph{L'enregistrement analogique du flux}
Les techniques d'enregistrement analogique reposent sur la conversion continue du flux original en un signal analogue dont les variations s'effectuent sur une échelle physique différente, par exemple une fréquence sonore transformée en une tension électrique. 
L'enregistrement s'effectue sur différents types de supports, cassettes magnétiques, films argentiques etc. 
La caractéristique commune de ces supports est que l'accès au contenu enregistré ne peut se faire que de manière linéaire ou séquentielle, c'est-à-dire qu'il faut avancer ou rembobiner la bande jusqu'au point de départ avant de pouvoir commencer la lecture. 
Ainsi, chaque opération de manipulation de ces contenus compte toujours un temps pas forcément négligeable consacré à l'alignement avec le point de départ désiré. 
Pour bien s'en rendre compte, il suffit de se souvenir du temps qu'il fallait pour rembobiner la cassette de votre film préféré, puis du temps passé en avance rapide pour passer les inévitables publicités précédant le film.

\paragraph{Numérisation et délinéarisation de l'accés}
Avec le numérique, cette expérience autrefois familière a complètement disparu et se voit remplacée par un accès immédiat à n'importe quel moment du contenu audio-visuel. 
La numérisation des contenus consiste à discrétiser le flux en un ensemble de valeurs que l'on convertit ensuite en flux binaire. Ce flux est ensuite enregistré sur des supports de mémoire magnétiques (disquettes 3'1/2, disques durs etc.) optiques (CD, DVD etc.) ou électronique (mémoire vive dite RAM, mémoire flash etc.). 
Seules ces dernières garantissent un accès arbitraire ou délinéarisé aux données stockées en mémoire (n'importe quelle donnée, à n'importe quel moment). 
Les supports magnétiques et optiques proposent un accés séquentiel comme l'analogique, mais plus rapide et surtout que l'on peut coupler avec les mémoires électroniques afin de se rapprocher de leurs performances.

\paragraph{Fragmentation}
Par ailleurs, au-delà des avantages liées aux temps d'accès au contenu, le numérique facilite également sa fragmentation et sa manipulation.
Contrairement à l'analogique, le numérique permet de représenter de manière arbitraire tout type d'information puis d'effectuer des calculs sur ces représentations. 
Ainsi, on peut associer au contenu audiovisuel, d'autres contenus de différentes natures pour les enrichir et faciliter son exploitation ultérieure.

\paragraph{Discrétisation}
La discrétisation du flux audiovisuel quant-à-elle remet en cause la temporalité du contenu et permet de ce fait une fragmentation plus aisée et plus fine. 
Lorsque l'analogique effectue une transformation continue et analogue, le numérique définit une fréquence d'échantillonnage et quantifie les valeurs de la source sur une échelle arbitraire finie. 
C'est donc véritablement la fréquence d'échantillonnage qui constitue la première unité de réprésentation de l'information au-dessus du bit.
C'est donc en décidant du nombre de pixels pour représenter une image, ou aux nombres de valeurs par secondes prises pour représenter un son que l'on décide d'une première échelle de fragmentation. 
% Premier niveau de manip technique, ensuite il y a des USI
Toute autre fragmentation à l'échelle supérieure est potentiellement (re)constructible par calcul, pour autant qu'on possède une méthode opérationnalisable. 
%De même que chaque élément de cette fragmentation, chaque unité, devient adressable et donc manipulable par calcul. 
%\cite{Bachimont2000} parle ainsi d'utm, usi ...
% COUPURE DU SENS ?

% \g{== Révision à faire, introduire les UTM et USI ==}
\paragraph{Numériser, c'est informatiser le métier}
Parmi toutes les fragmentations possibles, il convient alors de déterminer leur pertinence par rapport aux types de calculs que l'on souhaite réaliser à chaque étape de la chaîne de production. 
Il s'agit ici de contrôler les calculs à effectuer en suivant les règles du métier qui en disposera.
Chaque étape ayant ses objectifs propres, les calculs effectués varient et s'opèrent à différents niveaux de fragmentation. 
Or la numérisation des contenus et de l'information s'effectue toujours en vue de leur manipulation par des programmes informatiques.
Ainsi, la numérisation entraîne toujours une informatisation qui impacte le métier dans son organisation et ses pratiques parce qu'il transforme les possibilités techniques qui le concerne.
% L'enjeu est donc double, d'une part identifier les niveaux de fragmentation pertinents pour chaque étape de la chaîne de production, et d'autre part de se donner les moyens de reconstruire la cohérence . 

% numériser => (a) fragmenter + (b) mettre en réseau => (a) besoin de conserver la cohérence de l'ensemble + (b) besoin d'autonomiser pour une future situation d'usage





%%%%%%%%%%%%%%%%%%%%%%%%%%%%%%%%%%%%%%%%%%%%%%%
\subsection{Le développement de l'électronique et des réseaux de télécommunications}\label{sec:electro}
L’explosion des appareils multimédia et des possibilités de transférer des contenus par les réseaux de télécommunication a promu de nouvelles pratiques de consommation et d'échanges des contenus tant chez les professionnels de l'audiovisuel que dans le grand public.
% 
Du côté des professionnels, on voit ainsi l’émergence de systèmes de production qui utilisent le réseau pour faire transiter les contenus entre les systèmes d'information de leurs différents départements. 
Ces systèmes reprennent les principes d'architecture multi-tiers utilisés sur le Web avec les contenus représentés par des fichiers ou des flux binaires, d'où leur appelation de \e{file-based production system}.
% 
Ce genre de système favorise également l’échange de fichiers entre organisations, puisque l’architecture permet d'exposer les données stockées de la même manière sur un réseau interne (intranet) ou externe (extranet, Internet).
% 
Du côté du grand public, les appareils portables acquièrent de plus en plus de connectivité avec leur environnement et les réseaux. 
De simples lecteurs à brancher en USB, les appareils sont passés au stade communicant avec la 3G, le Wifi ou le Bluetooth. 
La fonction d'échange se banalise et inversement, les appareils communicant comme les téléphones deviennent eux-même des stations multimédia à part entière, musique, photo, vidéo, courriel, sms etc.\\


Un autre facteur à noter, est que ces appareils portables sont personnels, c’est-à-dire qu’ils sont majoritairement utilisés par un seul individu contrairement à l’usage du téléviseur qui était et reste encore largement un objet collectif. 
Une autre distinction fondamentale se situe dans le fait que ce sont des appareils informatiques qui fonctionnent entièrement dans le numérique. 
Les possibilités d’interaction en font non plus de simples terminaux de lecture, mais potentiellement de véritables instruments de création et de communication. En effet, l'insertion de données et la capture de contenus (photos, sons, textos etc.) est prévu et le couplage avec des plates-formes de publication (réseaux sociaux, dépôts de contenus, CMS etc.) est de mieux en mieux réalisé.

De ce fait, en plus des capacités toujours plus importante de consommation de contenus, ces appareils favorisent eux aussi leur circulation ou la circulation d’information annexes. 
Le partage d’opinions s’est considérablement développé avec la vague d'applications Web dites sociales qui permettaient aux utilisateurs de créer du contenu sans avoir à maîtriser les arcanes techniques du Web. 
Cet ajout d’opinion, même s’il est parfois réduit au minimum à une marque d'appréciation, implique ainsi l’utilisateur dans le processus de diffusion du contenu en le portant à l'attention d'autres utilisateurs (ses contacts dans le cas des réseaux sociaux, ses lecteurs dans le cas d'un weblog ou autres CMS, un ensemble d'utilisateurs anonymes dans le cas de services sociaux tels que Delicious\footnote{Delicious : \url{http://delicious.com/} est une plate-forme de sauvegarde, d'indexation par mot-clé et de partage de marques-pages. Pour l'utilisateur il s'agit soit de sauvegarder et d'indexer ses marques-pages, soit de découvrir les marques-pages correspondant à tel ou tel mot(s)-clé(s) déjà sauvegardées par la communauté. Les grandes tendances d'indexation sont ainsi accessibles à tous, tout en permettant à chacun de développer son système d'indexation personnelle -- adaptation française du mot anglais \e{folksonomy}. Il est également possible de partager directement ses trouvailles avec d'autres utilisateurs par un système d'abonnement et de notification.} ou Digg \footnote{Digg : \url{http://digg.com/} est un site de marque-page social qui fonctione sur le principe du vote. Un utilisateur peut proposer une page qui est alors soumise aux votes des autres utilisateurs. Suivant le succès de la page, celle-ci sera mise en avant sur la page principale de Digg, ou bien mise de côté avec le reste des pages moins populaires, et finira par être supprimée.}).\\


Les appareils numériques multimédia possèdent également des capteurs de plus en plus performant et de moins en moins coûteux qui permettent au grand public de découvrir de nouvelles activités de création (photographie et retouche d’image, tournage et montage vidéo, prise de son et mixage audio etc.).
Cet abaissement du coût d’entrée dans la production a favorisé l'émergence d’une production amateur hétéroclite qui va du passant prenant une photo d’un évènement se déroulant devant ses yeux jusqu’à l’amateur qui pratique par amour mais avec l'exigence d’un professionnel. 
Cette production amateur rentre alors en concurrence avec la production professionnelle, voire la remplace dans certains cas (lorsqu’un passant est le seul témoin d’un évènement inattendu par exemple). 
La concurrence est d'autant plus forte depuis l’apparition de plates-formes de partages qui facilitent la distribution des contenus. 
De manière générale, l’opposition classique entre producteurs et consommateurs se brouille et les professionnels cherchent de plus en plus à mettre à contribution les amateurs dans leurs processus. 
On se dirige ainsi vers un modèle où professionnels et amateurs contribuent à divers degrés et divers moments au cycle de vie des contenus.


Avec l'informatisation des appareils multimédia s’est introduit la possibilité de personnaliser la communication avec l'utilisateur, d'y ajouter de l'interactivité et de connecter les utilisateurs entre eux. 
Ces nouvelles possibilités transforment les attendus et les pratiques du grand public. 
De ce fait, cela impacte les rapports avec les professionnels qui tendent à vouloir intégrer les contributions externes à leurs propres productions. 
Ainsi, il ne s’agit plus simplement de produire des contenus qui s'adressent indistinctement aux masses, mais de trouver des moyens de personnaliser son offre, de recommander des contenus, de faciliter la récupération de contenus, de diversifier les occasions de consommer ou de contribuer.








%%%%%%%%%%%%%%%%%%%%%%%%%%%%%%%%%%%%%%%%%%%%%%%
\section{Le projet MediaMap}\label{sec:mm}
% %%%%%%%%%%%%%%%%%%%%%%%%%
\paragraph{Objectifs}
Le projet MediaMap vise à développer des modèles et des applications pour promouvoir la production audiovisuelle collaborative articulant contenus professionnels et amateurs. 
En particulier, l'ambition est d'intégrer les amateurs et leurs contenus dans la chaîne de production professionnelle en améliorant d'une part la qualité technique et éditoriale des contenus fabriqués, et d'autre part en facilitant la collaboration et l'intercompréhension entre les différents acteurs de la chaîne.

La piste de travail retenue a été de construire des ontologies capables de représenter et décrire les contenus au fur et à mesure de leur processus de production. 
Ces informations serviraient de base de connaissances pour de nouvelles applications s'intégrant dès le début à la chaîne de production audiovisuelle, c'est-à-dire dès la conception du contenu. 
Les partenaires du projet ont ainsi développé des applications d'organisation du processus de production, de description du contenu, d'assistance au tournage ainsi qu'un moteur de recherche utilisant ces ontologies comme modèle d'information de référence.

%%%%%%%%%%%%%%%%%%%%%%%%%
\paragraph{Composition}
Le projet MediaMap a rassemblé une dizaine d'entreprises ainsi que deux équipes de recherche de l'Université de Technologie de Compiègne :
\begin{liste}
	\item l'équipe de recherche \pc{Information Connaissance Interaction} (ICI) chargée de la partie modélisation qui a abouti à la construction d'ontologies.

	\item l'équipe de recherche \pc{Automatique, Systèmes Embarqués, Robotique} (ASER) chargée de la conception d 'algorithmes d'analyse d'images et de vidéos.
\end{liste}


Parmi les entreprises du consortium, on compte les deux grandes chaînes de télévision publiques belges ainsi que de nombreuses PME belge ou française qui apportent leurs expertises dans différents domaines :
\begin{liste}
	\item \pc{BelgaVox} qui gère un des plus grands stocks d'archives audiovisuelles belges et produit des documentaires.

	\item \pc{Exalead} qui est un éditeur de solution de recherche pour les entreprises.

	\item \pc{Kane Consulting} qui propose des analyses du marché et des usages aux acteurs de la production audiovisuelle.

	\item \pc{Memnon} qui est spécialisé dans la numérisation, la documentation et l'archivage de contenus audio et vidéo.

	\item \pc{Perfect Memory} qui s'est créé pendant le projet afin d'accompagner les solutions du projet sur le marché grand public et prospecte également le marché professionnel.

	\item \pc{Skema} qui développe des applications de production de contenu audiovisuel amateur pour mobiles et caméras.

	\item \pc{Solution 2.0} qui est une agence de conception et de réalisation de plate-forme Web.

	\item la \pc{Radio-Télévision Belge de la communauté Française} (RTBF).

	\item \pc{Vitec Multimédia} qui développe et manufacture du matériel vidéo numérique.

	\item la \pc{Vlaamse Radio- en Televisieomroep} (VRT).
\end{liste}









%%%%%%%%%%%%%%%%%%%%%%%%%%%%%%%%%%%%%%%%%%%%%%%
\section*{Organisation du mémoire}\label{sec:plan}
\addcontentsline{toc}{section}{Organisation du mémoire}

\paragraph{Exposition}
\e{
La première partie de ce mémoire a pour objectif de présenter les problèmes qui se posent à la production audiovisuelle depuis son avancée vers la numérisation. 
Elle nous permet également de préciser la manière dont nous posons le problème, à la fois en terme métiers et avec nos lunettes de scientifiques.}

Le chapitre \g{\ref{chap:intro}. Introduction} nous sert à rappeler le contexte technologique général qui s'impose au monde de l'audiovisuel. 
En effet, la production audiovisuelle se dirige progressivement vers une numérisation et une mise en réseau de ses produits ainsi qu'une informatisation de ses pratiques qui n'est pas sans conséquences. 

Le chapitre \g{\ref{chap:problo}. Problématisation} nous permet de préciser comment ces tendances impactent le monde de l'audiovisuel.
Nous nous appuyerons sur l'étude du fonctionnement de la chaîne de production audiovisuelle classique (\ref{sec:prod}) pour dresser un bilan des attentes des professionnels vis-à-vis du numérique (\ref{sec:besoins}).
% En particulier, 
Nous précisons alors la manière dont nous posons le problème sur le plan métier, ce qui nous amènera à définir le problème scientifique. 
Sur le plan métier, le défi posé par le numérique consiste à passer d'une vision à l'échelle du document à une vision à l'échelle du fragment. 
En effet, le numérique favorise la fragmentation et la circulation des contenus qu'il s'agit alors de rendre autonome pour en permettre l'exploitation (\ref{sec:pmetiers}). 
Sur le plan scientifique, nous posons le problème en terme de modélisation des objets audiovisuels et des connaissances associées afin de construire une compréhension commune et dynamique pour tous les acteurs impliqués dans la production audiovisuelle (\ref{sec:scien}).
% Nous concluons en expliquant comment nous mobilisons diverses disciplines scientifiques pour construire une réponse aux problèmes posés (\ref{sec:posd}).
% proposent d'une part des outils et des méthodes de modélisation, et d'autre part proposent des modélisations de l'audiovisuel (\ref{sec:posd}).


\paragraph{État de l'art}
\e{
La deuxième partie de ce mémoire vise à étudier des outils, méthodes et langages de modélisation. Elle permet également d'étudier les modélisations existantes des objets audiovisuels, mais égalements des connaissances métiers qui y sont associées afin de faciliter leur exploitation.}

Le chapitre \g{\ref{chap:omod}. Outils de modélisation} commence par clarifier les besoins de modélisations à partir d'un scénario de production collaborative impliquant des acteurs professionnels et amateurs (\ref{sec:cdcf}).
Ces besoins ne se situent pas seulement au niveau de la modélisation conceptuelle, mais également sur le plan des jargons utilisées pour présenter ces concepts à des contributeurs de la production.
Ainsi, nous examinons les définitions des concepts de \gui{systèmes d'organisation de connaissances} (SOC) et en particulier les relations qu'entretiennent \gui{terminologie} et d'\gui{ontologie} (\ref{sec:defs}).
Nous étudions ensuite les langages de structuration et de représentation des connaissances qui permettent de modéliser ces deux types de SOC (\ref{sec:mods}).

Le chapitre \g{\ref{chap:mav}. Modélisations de l'audiovisuel} a pour objectif de mettre en rapport les représentations des professionnels de la production avec diverses communautés scientifiques, en vue de clarifier la définition d'un objet audiovisuel (\ref{sec:dav}) et de sa réutilisation (\ref{sec:gest}).
Cette étude s'appuie sur la poursuite du scénario d'usage du chapitre précédent, et met en exergue la nécessité de fragmenter la modélisation des objets audiovisuels pour favoriser leur réutilisation (\ref{sec:cdc-av}).
Nous étudions ensuite les solutions utilisées dans l'industrie pour gérer la circulation des programmes (\ref{sec:wrapper}) et les décrire (\ref{sec:desc}).
Cette dernière partie analyse les méthodes de fabrication de ces description ainsi que leur nature, puis les modélisations développées.
Une attention particulière est portée à de MPEG-7 (\ref{sec:mpeg7}), qui tient lieu de référence à de nombreux travaux de formalisation sous la forme d'ontologie (\ref{sec:mpeg7etc}).
Nous présentons également des approches de description plus proche de la perspective de la production audiovisuelle (\ref{sec:insitu}).
% à revoir


\paragraph{Contribution}
\e{La troisième partie de ce mémoire présente notre contribution conceptuelle et informatique aux problèmes de modélisations que nous avons soulevés.
Nous détaillons nos choix de représentation pour opérationnaliser notre contribution en une ontologie informatique.}

Le chapitre \g{\ref{chap:mod}. Approche et modélisation} revient sur les langages et les modélisations étudiés dans le chapitre précédent et introduit les principes de notre approche (\ref{sec:principes}).
Notre positionnement au sein de la chaîne de production audiovisuelle, nous permet de modéliser le déroulement de la chaîne, la définition d'une structure documentaire première, puis les fragments audiovisuels construits ainsi que les connaissances qui s'y rapportent.
Nous détaillons la modélisation conceptuelle en parties, chacune correspondant à un besoin fonctionnel, en expliquant leur mise en relation par des exemples (\ref{sec:concept}).

Le chapitre \g{\ref{chap:op}. Mise en oeuvre} présente la représentation informatique de notre conceptualisation.
Nous argumentons d'abord nos choix de langage et montrons comment nous les utilisons (\ref{sec:ln}). 
Nous détaillons ensuite la structuration de notre ontologie et son articulation avec des thésaurus et des bases de faits (\ref{sec:op}).

\paragraph{Discussion}
\e{La dernière partie de ce mémoire montre comment notre contribution est utilisé dans le cadre du projet MediaMap et ouvre la discussion sur ce travail de thèse.}

Le chapitre \g{\ref{chap:app}. Applications et expérimentations} introduit les diverses applications qui ont été développé (\ref{sec:app}) par nos partenaires et les expérimentations qu'elles ont permis de mener (\ref{sec:xp}).
Il s'agit d'éclairer l'appropriation de notre travail dans le cadre de scénarios de production audiovisuelle collaborative.
En particulier, nous expliquons quelle partie de l'ontologie est mobilisée par les applications pour construire ou bien intégrer des connaissances sur la production, ses contributeurs et ses produits.

La \g{Conclusion} remet en perspective notre contribution et les applications développées par rapport aux problèmes métiers et scientifiques posés.
Nous ouvrons également la discussion sur la poursuite de nos recherches et l'avenir des applications du projet MediaMap.

\chapter{Problématisation}\label{chap:problo}
\minitoc
\section{La production audiovisuelle}\label{sec:metier}
% faire le point sur ce que le numérique pourrait apporté à la production audiovisuelle
\e{
L'objectif de cette première section est de donner des éléments de compréhension du métier de la production audiovisuelle.
Dans un premier temps, nous rappelons comment s'organise classiquement la fabrication des objets audiovisuels (\ref{sec:prod}).
Ensuite, nous détaillons les notions et des mots utilisées par les professionnels pour parler de l'objet audiovisuel à construire (\ref{sec:docvoc}).
Enfin, à partir de ces éléments nous précisons les besoins que rencontrent ces professionnels avec l'émergence du numérique et de la mise en réseau (\ref{sec:besoins}). 
}


%%%%%%%%%%%%%%%%%%%%%%%%%%%%%%%%%%%%%%%%%%%%%%%
\subsection{Déroulement de la chaîne de production}\label{sec:prod}
% \addcontentsline{toc}{subsection}{La production audiovisuelle}
La création de documents audiovisuels est une entreprise collective qui suit généralement ce qu'on appelle la chaîne de production audiovisuelle. 
Organisée de manière linéaire, cette chaîne peut se décomposer en 4 grandes étapes -- voir Figure \ref{img:intro:chaine}.

\begin{liste} 
	\item \g{Préproduction} : Cette première étape consiste à construire une ébauche du futur document audiovisuel de manière à prévoir les moyens à engager pour le réaliser.

	\item \g{Production} : Cette étape vise à tourner plusieurs prises pour chaque partie du document et à commencer à faire le tri entre elles.

	\item \g{Postproduction} : L'objectif est d'assembler les prises et de les retoucher de manière à former un document cohérent et adapté à une audience et un mode de distribution.

	\item \g{Exploitation} : Une fois le document achevé, on valorise sa construction par une distribution auprès d'une audience ainsi qu'un archivage qui permettra de le réutiliser ultérieurement.
\end{liste}

\begin{figure}[ht!]
\centering
\includegraphics[width=\textwidth]{images/Workflow-Thesis-v0.png}
\caption{La chaîne de production audiovisuelle classique}
\label{img:intro:chaine}
\end{figure}


%%%%%%%%%%%%%%%%%%%%%%%%%
\subsubsection*{Préproduction}\label{sec:preprod}
L'objectif de cette étape est de construire une ébauche du futur document audiovisuel, de manière à prévoir les moyens à engager pour le réaliser. 
On distingue deux phases de préparation, l'écriture ou \e{Scripting} et le \e{Planning} ou planification.

À partir d'une idée, l'écriture du document se déroule en plusieurs étapes où l'on fixe progressivement le message à faire passer ainsi que sa forme audiovisuelle. 
Une fiction par exemple s'écrit à partir d'un résumé de l'histoire, puis on développe les scènes, les personnages, les lieux, les dialogues, etc. jusqu'à arriver à la manière dont cette histoire sera racontée à l'écran. 
On fixe ainsi le type de plan à filmer, les mouvements de caméra, le type de lumière, etc. 
Dans certaines grosses productions, on aura même recours au dessin pour aider à la représentation visuelle.


Toutes ces informations sur le contenu et la forme du futur document audiovisuel servent à estimer le temps nécessaire et les moyens à engager. 
L'estimation du coût d'une production est un élément essentiel pour la production audiovisuelle. 
En effet, dès le départ le producteur doit estimer la rentabilité du futur document afin de voir quels moyens il peut engager. 
Il s'agit bel et bien d'investissement, parfois très lourd, surtout en comparaison avec d'autres industries culturelles – musique, littérature, radio, presse – à l'exception récente du jeu vidéo. 
Contrainte budgétaire et forme esthétique sont donc en négociation dans cette première étape.


%%%%%%%%%%%%%%%%%%%%%%%%%
\subsubsection*{Production}
Une fois l'ébauche et les moyens déterminés, la production vise à réaliser chaque partie du document une à une puis à les assembler en un montage cohérent.

La phase de \e{Fabrication} consiste à capter du réel que l'on a mis en scène. La captation d'un évènement est réalisée grâce à des appareils d'enregistrement (caméra, microphones, etc.). 
La mise en scène du réel se construit à partir d'un ensemble de techniques, d'équipements et d'accessoires (lumière, costume, décors, maquillage, etc.) qui permettent d'obtenir l'image et le son souhaités. 
La technique est ainsi mobilisée dans un objectif esthétique. 
Dans le cas d'une fiction, la fabrication du contenu se fait en fonction du planning de mobilisation des personnes et des équipements plutôt que suivant l'ordre chronologique de l'histoire. 
On regroupe ainsi le tournage des scènes dans tel lieu ou avec tel équipement afin de réduire les coûts. Au final, la fabrication
produit des séquences vidéo et audio qu'il s'agit ensuite de redécouper pour mieux les assembler.

La phase de \e{Derushing} consiste à examiner les séquences réalisées pendant la fabrication et à les trier en vue de faciliter le montage. 
Par exemple, le tournage d'une fiction produit des séquences vidéo comprenant plusieurs prises d'une même scène et souvent des séquences captées par des caméras ayant des angles de prise de vue différents. 
Il faut donc redécouper les séquences en prises puis regrouper les prises d'une même scène. 
Le montage se fera d'autant plus facilement qu'on saura également identifier la qualité et les avantages de chaque prise d'une même scène.


%%%%%%%%%%%%%%%%%%%%%%%%%
\subsubsection*{Postproduction}
Lorsque le contenu audiovisuel est fabriqué et trié, il reste à le structurer en un document et à le conditionner en fonction de sa future exploitation.

La phase de \e{Montage} consiste à agencer des petites séquences de vidéo et d'audio pour construire la structure du document audiovisuel.
L'agencement des plans, leur durée et la transition entre ces plans constituent les ressorts esthétiques propres à l'audiovisuel. 
Ils participent à la transmission du message en ceci qu'ils servent de raccord entre les plans, comme le souligne le réalisateur \pc{Sergueï Mikhaïlovitch Eisenstein}\footnote{L'origine de cette fameuse citation est assez obscur, on la retrouve dans de nombreux documents, dont cet article --\cite{Montage et Réalisme}-- datant des années 60 et extrait de la revue québecoise \gui{Séquence : La revue du cinéma}.} :

\begin{cico}
Le montage est l 'art d'exprimer ou de signifier par le rapport de deux plans juxtaposés de telle sorte que cette juxtaposition fasse naître l'idée ou exprime quelque chose qui n'est contenu dans aucun des deux plans pris séparément. L'ensemble est supérieur à la somme des parties.
\end{cico}

Après le montage, une phase de \e{Finition} est nécessaire pour intégrer d'autres ressources au document en fonction de sa distribution. 
Pour une diffusion antenne d'un reportage, on ajoute le logo de la chaîne de télévision, un jingle, le nom des intervenants ou des titres, etc. 
Une diffusion sur DVD nécessitera l'ajout des conditions
légales d'usages, l'intégration d'un menu de navigation, etc. 
La distribution détermine également un format d'encapsulation (.avi, .mkv, .ogg etc.) et un encodage du contenu audiovisuel (MPEG-2, H.264, MPEG-4, theora vorbis etc.). 
Le résultat de cette phase de post-production est de fournir des documents prêts à l'usage et dans certains cas plusieurs variantes pour chacun des modes de distribution envisagés.


%%%%%%%%%%%%%%%%%%%%%%%%%
\subsubsection*{Exploitation}
Une fois le document achevé, on valorise sa construction par une distribution auprès d'une audience ainsi qu'un archivage qui permettra de le réutiliser ultérieurement.

La phase de \e{Distribution} consiste à rendre le document matériellement accessible à une audience. 
Il s'agit d'un transfert qui peut faire l'objet d'une transaction commerciale ou s'appuyer sur d'autres types de modèles économiques (publicité entre autres). 
La nature du transfert varie et porte à la fois sur les modalités d'accès au contenu et les droits d'usages.

La phase d'\e{Archivage} consiste le plus souvent à stocker le contenu diffusé afin de pouvoir le réutiliser tel quel plus tard, soit en le rediffusant, soit en le vendant à un autre diffuseur. 
C'est aussi généralement la phase où l'on construit, récupère et attache des descriptions du contenu au document audiovisuel. 
En effet, l'archivage n'a de sens que s'il permet de retrouver, voire redécouvrir, les documents archivés.
\ciel{Au plan économique, un film est un bien informationnel, d'expérience, caractérisé par une très forte densité d'informations. Son exploitation s'organise autour de versions différentes, distribuées sur des marchés distincts par des acteurs spécialisés.} (\cite{Blanc2006}).%Gilles Le Blanc - Innovations numériques, distribution et différenciation  : le cas de la projection numérique dans le cinéma.




%<TODO
%TODO>
\subsection{Documents et vocabulaire de la production (n)}\label{sec:docvoc}
\e{
Dans cette section, nous présentons des définitions utilisées dans le milieu professionnel et tirées du \gui{Dictionnaire technique du Cinéma (\cite{Pinel2008})} afin de présenter les principaux documents et le vocabulaire utilisé pendant la phase de préproduction. 
Il s'agit ainsi de mettre en exergue la manière dont se construit une description textuelle de l'objet audiovisuel en devenir ainsi que le vocabulaire utilisé pour faire cette description.}
% Des éléments qui nous serviront à mieux cerner les problèmes qui se posent à la production audiovisuelle.

\paragraph{La notion de plan}
La notion de plan peut se présenter de diverses manières suivant le point de vue adopté. 
Du point de vue technique, il s'agit d'une série d'images (ou photogrammes) qui sont enregistré par un appareil de capatation (une caméra) au cours d'une même prise de vue : \ciel{série de photogrammes enregistrés au cours d'une même prise} (\cite{Pinel2008}).
Il s'agit donc de l'enregistrement qui est effectué entre le moment où l'on presse sur le bouton pour lancer et celui où l'on arrête l'enregistrement.
Cette définition, certes robuste, ne permet pas pour autant de caractériser les plans ou de les comparer entre eux. 
Ainsi, on s'intéresse au point de vue de l'écriture filmique et de la réalisation qui considère non seulement l'action d'enregistrement, mais également la manière dont il est effectué : 

\ciel{
Fragment de temps et d'espace enregistré d'un seul tenant, selon un point de vue déterminé, et donnant à la projection le sentiment de la continuité d'une même \e{image en mouvement}.} (\cite{Pinel2008})

Cette définition complète la précédente en considérant le rapport au sujet (le point de vue) et la temporalité du plan et son rapport à un ensemble d'autres plans (la continuité). 
Elle permet d'envisager les plans comme des éléments de base que l'on assemblera ensuite pour construire un objet audiovisuel : 

\ciel{
La notion de plan est apparue [\dots] lorsqu'on a abandonné le point de vue unique du tableau pour envisager le sujet sous différents angles et à différentes distances et lorsque, par la grâce du montage, on a mis en relation ces plans entre eux.
[\dots]
Si le photogramme représente l'unité technique de la prise de vues, la scène et la séquence les unités narratives de l'oeuvre cinématographique, le plan est la cellule fondamentale de l'écriture du film, de sa préparation jusqu'à la copie standard.} (\cite{Pinel2008})

Dans cette citation, on voit aussi émerger l'idée qu'il existe différents niveaux d'analyse dans l'objet audiovisuel : 
\begin{liste}
	\item un \g{niveau technique} avec le photogramme mais également le pixel dans le numérique.
	\item un \g{niveau narratif} avec la scène (unité de temps et de lieu dans l'histoire) et la séquence (suite de scènes constituant une action dramatique autonome ou distincte).
	\item \g{un niveau dont le plan est l'unité de base qui sert tout au long de la chaîne de production}. 
	On parle d'unité, car il s'agit du résultat de base d'une prise de vue, c'est-à-dire de la fabrication (production) de l'objet audiovisuel. 
	La pré-production, étape de préparation du tournage, utilise donc naturellement cette unité. 
	De même, le montage consiste à organiser ces plans pour former un ensemble cohérent, quitte à les ajuster (raccourcir, allonger, modification du cadrage etc.). 
	Ainsi, il semble que cette unité servent non seulement d'unité de travail de référence pour la chaîne de production\footnote{La production sonore d'un objet audiovisuel ne s'organise pas forcément de la même manière que celle de la production de l'image. Néanmoins, par la force des choses, la construction de l'image prime bien souvent sur celle du son et son unité de base sert donc de référence même pour la production sonore.}, mais également du premier niveau signifiant propre à l'audiovisuel (l'image seul pouvant être rattaché à la photographie).
\end{liste}

Ces distinctions nous permettent de définir le plan selon les caractéristiques suivantes : 
\begin{liste}
	\item \g{l'échelle relative du cadre par rapport au(x) sujet(s)} (personnages, objets etc.).
	C'est ce qui permet de définir un ensemble de \gui{valeurs de plan}, gros plan, plan américain etc. que nous définirons par la suite.
	
	\item \g{l'angle de la prise de vue} (plongée, contre-plongée, cadre incliné etc.)

	\item \g{le mouvement de la caméra} et d'autres paramètres de son objectif (panoramique, travelling, rotation, zoom, focale, focus etc.)

	\item \g{l'articulation des plans entre eux}, d'une part en terme de durée, mais aussi en terme de transition et d'impression de continuité entre les plans. 
	Par exemple, il existe des règles de cadrage et de montage pour aider les spectateurs à situer les personnes sur un plateau\footnote{La règle des 180° oblige ainsi à maintenir les mêmes relations est-à-gauche/droite-de entre les personnes, de manière à ce que le spectateur puisse se souvenir des positions des interlocuteurs sur un plateau. En inversant ces relations topographiques, on donne l'impression au spectateurs que les personnes ont échangé leurs places alors que c'est juste la caméra qui a changé de point de vue. Il s'agit donc d'une règle très importante pour assurer la continuité et la compréhension du spectateur.}.
\end{liste}


\paragraph{Valeurs de plan utilisées dans un script}
La valeur de plan est un des éléments le plus utilisé pour distinguer les plans entre eux, notamment au moment de l'écriture. 
Plutôt que de préciser les paramètres optiques de la caméra (considérés comme des détails très difficile à préciser à l'avance), le réalisateur préfère parler d'un type de plan pour donner une idée générale de l'image à obtenir. 
Le Tableau \ref{tab:vplans} présente les principales valeurs de plans utilisées par les professionnels, avec leur abréviations et leur(s) dénomination(s) anglaise(s) tandis que la Figure \ref{img:intro:plans} en propose une illustration.

\begin{table}[ht!]
   \begin{center}
		\begin{tabularx}{\textwidth}{|p{100pt}|X|p{100pt}|}
		   \hline
	\pc{Dénomination française} & \pc{Défintion} & \pc{Dénomination anglaise} \\ \hline\hline
 	\g{très gros plan (t.g.p.)} & plan cadrant une partie du visage, un détail du corps (un oeil, une bouche, un doigt etc.) ou le détail d'un objet. & extreme close-up, e.c.u. ; big close-up, b.c.u.\\ \hline

 	\g{gros plan (g.p.)} & plan isolant un visage, généralement cadré à la hauteur du noeud de cravate, ou un autre détail du corps (plan de détail ; insert), voire \e{tout ou partie d'un petit objet}. & close-up, c.u.\\ \hline

	\g{plan rapproché} & plan cadrant le(s) personnage(s) au niveau de la taille (plan rapproché taille, p.r.t.) ou de la poitrine (plan rapproché poitrine, p.r.p.). & medium close-up, m.c.u.\\ \hline
	
	\g{plan ceinture} & plan coupant les personnages au niveau de la ceinture & belt shot\\ \hline 

	\g{plan américain} (p.a.) & plan coupant les personnages à mi-cuisse & american shot ; medium close-shot, m.c.s.\\ \hline
	
	\g{plan moyen (p.m.) ou plan en pied} & plan présentant le(s) personnage(s) en pied. Il existe également des variations de ce plan qui sont nommées \e{serré} (aussi nommé plan américain large) ou \e{large} et qui font varier légèrement le cadrage. & medium shot, middle-shot, mid-shot, m.s. ; full shot, f.s.\\ \hline
	

	\g{plan de demi-ensemble (p.d.e., 1/2e.)} & plan mettant en place les personnages dans leur milieu en cadrant une bonne partie du décor & medium-long shot, m.l.s.\\ \hline

	\g{plan d'ensemble (p.e.)} & plan cadrant l'ensemble du décor construit & long shot, l.s.\\ \hline
	
	\g{plan de grand ensemble (p.g.e.)} & plan cadrant l'ensemble du décor construit de grande envergure. & very long shot, v.l.s.\\ \hline

	\g{plan général (p.g.)} & plan couvrant un vaste ensemble qui situe le décor construit dans son cadre : le décor dans le décor. & master shot ; extreme long shot, e.l.s\\ \hline 
		\end{tabularx}
		\caption{Valeurs de plans : du plus précis au plus général \label{tab:vplans}}
   \end{center}
\end{table}

% \begin{liste}
% 	\item \g{très gros plan} (t.g.p.) : \ciel{plan cadrant une partie du visage, un détail du corps (un oeil, une bouche, un doigt etc.) ou le détail d'un objet}. [extreme close-up, e.c.u. ; big close-up, b.c.u.]

% 	\item \g{gros plan} (g.p.) : \ciel{plan isolant un visage, généralement cadré à la hauteur du noeud de cravate, ou un autre détail du corps} (plan de détail ; insert), voire \ciel{tout ou partie d'un petit objet}. [close-up, c.u.]

% 	\item \g{plan rapproché} : \ciel{plan cadrant le(s) personnage(s) au niveau de la taille (plan rapproché taille, p.r.t.) ou de la poitrine (plan rapproché poitrine, p.r.p.).} [medium close-up, m.c.u.]
	
% 	\item \g{plan ceinture} : plan coupant les personnages au niveau de la ceinture [belt shot] 

% 	\item \g{plan américain} (p.a.) : \ciel{plan coupant les personnages à mi-cuisse} [american shot ; medium close-shot, m.c.s.]
	
% 	\item \g{plan moyen} (p.m.) ou \g{plan en pied} : \ciel{plan présentant le(s) personnage(s) en pied.} [medium shot, middle-shot, mid-shot, m.s. ; full shot, f.s.]
% 	Il existe également des variations de ce plan qui sont nommées \ciel{serré} (aussi nommé plan américain large) ou \ciel{large} et qui font varier légèrement le cadrage.

% 	\item \g{plan de demi-ensemble} (p.d.e., 1/2e.): \ciel{plan mettant en place les personnages dans leur milieu en cadrant une bonne partie du décor}. [medium-long shot, m.l.s.]

% 	\item \g{plan d'ensemble} (p.e.) : \ciel{plan cadrant l'ensemble du décor construit}. [long shot, l.s.]
	
% 	\item \g{plan de grand ensemble} (p.g.e.) : \ciel{plan cadrant l'ensemble du décor construit de grande envergure}. [very long shot, v.l.s.]

% 	\item \g{plan général} (p.g.) : \ciel{plan couvrant un vaste ensemble qui situe le décor construit dans son cadre : le décor dans le décor}. [master shot ; extreme long shot, e.l.s] 
% \end{liste}
%TODO:source

\begin{figure}[ht!]
\centering
\includegraphics[width=0.4\textwidth]{./images/ValeurPlan-v1.png}
\caption{Différentes valeurs de plan pour le cadrage d'un personnage à l'écran}
\label{img:intro:plans}
\end{figure}


\paragraph{Quelques documents de (pré)production}
%TODO:description + source
La pré-production se repose sur différents types de documents qui permettent de faire émerger progressivement la structure narrative ou documentaire, le découpage en plans et tous les détails de réalisation nécessaire à une bonne préparation du tournage. 
On notera que chacun de ces documents constitue un jalon dans la préparation du projet et que le script, résultat final de cette écriture, constitue une sorte de cahier des charges de l'objet audiovisuel à fabriquer.
\begin{liste}
	\item \g{sujet} : \ciel{matière première du film enrichie et développée lors de la préparation puis mise en forme au cours de la réalisation et du montage.} 
	
	\item \g{synopsis} : \ciel{exposé sommaire en quelques lignes, voire en quelques pages, du contenu dramatique ou documentaire d'un film}. 
	À noter que ce document est également utilisé plus tard dans la chaîne de production, notamment pour être transmis aux journalistes ou aux archivistes.
	De plus, il constitue la première mise en forme narrative du contenu du film, à la différence du sujet qui ne se constitue que de quelques idées directrices. 

	\item en cas d'adaptation d'une oeuvre littéraire en un objet audiovisuel, on développe un \g{traitement} : 
	\ciel{Travail littéraire préparatoire effectué à partir d'une oeuvre pré-existante ou d'une oeuvre originale pour assurer sa transmutation en termes cinématographiques.}

	\item lorsqu'on développe un objet audiovisuel original, à défaut de traitement on peut parler de \g{scénario} :
	\ciel{description de l'action d'un film épousant la forme \e{littéraire} du récit, rendant compte des articulations narratives et comportant une ébauche des dialogues, quelquefois la description plus précise de certaines scènes-clefs.}

	\item lorsque le besoin de préciser encore la construction de la narration, les auteurs peuvent construire une \g{continuité (dialoguée)} : \ciel{étape de la préparation écrite du film qui permet d'enrichir le traitement en développant chronologiquement les fragments d'action, en mettant au point le détail de chaque scène et en précisant le dialogue.}

	\item \g{plan de tournage} : \ciel{ultime travail de préparation effectué par le réalisateur avant le tournage. Il consiste à fragmenter la continuité en unités cinématographiques de temps et d'espace : les plans.}

	\item \g{script} : \ciel{dernière mouture du scénario, guide complet du tournage}.La Figure \ref{img:intro:script} présente un exemple de script extrait d'un film récent écrit et réalisé par Quentin Tarantino.
\end{liste}

\begin{figure}[ht!]
\centering
\includegraphics[width=0.7\textwidth]{images/ScriptExample-v1.png}
\caption{Extrait du script de Kill Bill écrit et réalisé par Quentin Tarantino}
\label{img:intro:script} 
\end{figure}

% des réécritures de documents et des mises à jours qui pourraient être réalisées par des machines, des informations qui pourraient être transmises automatiquement à travers un réseau numérique d'information


% un vocabulaire bien défini qui fait l'objet de nombreux dictionnaires, donc prêt à être formalisé


% des équipes qui sont réduites lors de tournage en extérieur




%%%%%%%%%%%%%%%%%%%%%%%%%%%%%%%%%%%%%%%%%%%%%%%%%%%%%%%%%%%%%%%%%%%%%%%%%%%%%%%%%%%%%%%%%%%%%%%%%%%
\subsection{Besoins Métiers (m)}\label{sec:besoins}
\e{
Face à une numérisation qui fragmente les contenus, une mise en réseau qui facilite la fabrication amateur, intensifie la circulation de ces fragments (\ref{sec:motiv}), la production audiovisuelle rencontre de nouveaux défis qui remettent en jeu son organisation et sa manière de se représenter le monde. 
Ainsi d'une part la fragmentation ne doit pas mettre en péril la cohérence de l'ensemble et d'autre part, la circulation des contenus ne doit pas compromettre l'exploitation future du tout ou des parties. 
L'ouverture d'une chaîne de production à des acteurs tiers implique de clarifier les attendus de chacun. 
Lorsqu'il s'agit de faire fabriquer ou de récupérer du contenu, il devient nécessaire pour le client de décrire la commande de contenu au fournisseur, de même que le fournisseur doit décrire à son client le contenu livré pour faciliter son exploitation. 
Ainsi, on souhaite adjoindre aux contenus des descriptions qui permettent de faciliter leur recherche et leur manipulation.
}


Pour les professionnels de la production audiovisuelle, le défi porte à la fois sur l'organisation de leur chaîne de production et sur la gestion de leurs produits :
\begin{liste}
	\item[(1)] \g{comment transformer la chaîne de production afin de l'ajuster à la diversification des formes de fabrication et de distribution des contenus, mais aussi aux changements dans les pratiques de consommation des audiences ?}

	\item[(2)] \g{comment passer d'une gestion de fichiers à une gestion de contenus audiovisuels considérés comme des objets numériques fragmentés dont il s'agit de garantir l'autonomie dès leur conception et jusque dans leurs différents cadres d'exploitation ?}
\end{liste}

% Problèmes métiers ? Tant que ça ne devient pas une solution, mais des tendances à prendre en compte 

On peut ensuite détailler ces défis en objectifs plus précis :
\begin{liste}
	\item[(1a)] \e{accorder contribution amateur et production professionnelle pour fabriquer ou valoriser du contenu.}

	L'amélioration croissante des capteurs des appareils multimédia ajoutée aux capacités de communication offrent au grand public de plus en plus de manières de participer aux processus de fabrication ou de diffusion des contenus.
	Les possibilités accrues de participation au processus médiatique (participation à l'émission, envoi de contenu, propagation via ses contacts, commentaires etc.) valorisent le spectateur, le contenu et la plate-forme de diffusion (comme d'une certaine manière peut le faire le bouche à oreille).

	De manière générale, l'intégration de contenus externes dans une chaîne de production professionnelle ne s'envisage  qu'à partir d'un certain niveau de qualité du contenu livré.  
	Paradoxalement, dans certains cas les signes d'une production amateur (tremblements, caméra à l'épaule etc.) peuvent être revendiquées comme des marque de style qui suggère une collaboration avec le public ou une proximité avec une réalité éloignée des images diffusées par les médias.
	Ainsi, les professionnels souhaitent encadrer plus ou moins fortement la production amateur par des indications, recommendations, obligations.\\


	\item[(1b)] \e{créer de nouvelles étapes dans la chaîne visant à réutiliser les contenus existants et les adpater à de nouveaux modes de consommation.}

	L'augmentation de l'offre de contenus accessibles aux spectateurs (chaînes, enregistrements, balladodiffusion, vidéo à la demande etc.) se traduit par une mise en concurrence accrue des contenus diffusés par les professionnels.
	Le contrôle de l'offre n'étant plus atteignable, il faut adopter de nouvelles stratégies de valorisation des contenus produits ou diffusés pour maintenir leur visibilité et leur rentabilité. 
	Une autre approche consiste à fournir un service de recommandation aux spectateurs et ainsi rentrer dans une démarche de fidélisation. 

	Par ailleurs, l'augmentation des terminaux de lecture multimédia et leur portabilité offrent de plus en plus d'occasions aux spectateurs de consommer des contenus. 
	Par exemple, les situations de mobilités peuvent impliquer des capacités de transfert diminués, un écran plus petit, des temps de disponibilités plus courts etc.
	Il semble alors que la production doivent évoluer pour fournir de nouveaux formats ou des formes retravaillées de contenus existants.

	Dans tous les cas, cela implique de se consacrer à des tâches d'éditorialisation des contenus pour répondre aux exigences et aux attentes de ces nouveaux modes de consommation.\\
% \end{liste}


% \begin{liste}
	\item[(2a)] \e{gérer l'intégration de contenu externes, les variations d'un même contenu pour les rattacher à un même objet numérique.} 
	% représentation

	L'utilisation de contenus provenant de sources externes de même que la production de multiples variations d'un même contenu augmente le nombre de ressources à gérer. 
	De plus, les relations entre ces différentes ressources nécessitent d'être clarifiées et explicitées dans le système de gestion. 
	% chaîne éditoriale ? 
	
	Il ne s'agit plus simplement de gérer des fichiers mais un ensemble de fichiers et de données qui constituent un ensemble cohérent et fragmenté que l'on nomme un objet numérique. 
	Cet objet doit intégrer à la fois les diverses sources qui le composent mais aussi des variations correspondants aux exploitations visées, des descriptions et tout ce qui permet de garantir son autonomie. 
	Il doit également s'agir d'un objet \e{métier} car son statut, son organisation, sa sémantique correspondent à la vision d'un métier, à la manière dont il pense le monde. 

	% Ainsi, on ne souhaite plus gérer des fichiers mais des objets numériques qui doivent acquérir un statut, une sémantique correspondant à la manière dont les métiers de la production les considèrent.\\
	% qui possède une valeur et une sémantique propre à un contexte d'usage. 	


	\item[(2b)] \e{associer des descriptions aux contenus pour faciliter leur exploitation dans un environnement numérique.}
	% description

	L'augmentation des contenus audiovisuels en circulation, la diversification de leurs modes d'exploitation compliquent la gestion des contenus.  
	Afin de favoriser la réutilisation de ces contenus, il faut pouvoir leur attacher des informations pertinentes pour les professionnels qui les manipulent. 
	La description du contenu peut varier suivant les besoins de chaque métier impliqué dans la chaîne de production. 
	Les opérations n'étant pas les mêmes, les descriptions de ces opérations varient donc également et sont nécessaires pour faciliter la réutilisation du contenu. 
	%de manière à faciliter modalités d'exploitation envisagées  

	Lorsque la réutilisation et production s'entremêlent, il est également nécessaire de construire les descriptions en même temps que le contenu. 
	De cette manière on récupère ou on réévalue l'information à mesure de l'avancée dans la chaîne. 

	Cela nécessite d'informatiser l'étape de pré-production de la chaîne et de modéliser les informations utilisées par les professionnels. 
	%commencer plus tôt, avoir plusieurs niveaux/types de description, raccrocher les bons éléments à la représentation de l'objet numérique
	
	%et embarquent des descriptions explicitant leurs modalités d'exploitation.

\end{liste}

\e{
En guise de synthèse, nous pouvons dire qu'il s'agit de constituer des objets audiovisuels autonomes dans les chaînes de production audiovisuelle. 
Nous précisons ce caractère autonome, car ces objets seront porteurs de leur propre description et associés à des connaissances sur l'organisation de la chaîne de production dans lesquels ils évoluent.
Ainsi, ces objets audiovisuels pourront être (ré)introduits à n'importe quelle étape d'une chaîne de production et fourniront aux contributeurs concernés des informations propres à faciliter leur (ré)utilisation.
}
	% \item \eg{valoriser et éditorialiser les contenus existants pour les rendres plus visibles, plus attrayants auprès des audiences ciblées.}
	% L'augmentation de l'offre de contenus accessibles aux spectateurs (chaînes, enregistrements, balladodiffusion, vidéo à la demande etc.) se traduit par une mise en concurrence accrue des contenus diffusés par les professionnels.

	% Le contrôle de l'offre n'étant plus atteignable, il faut adopter de nouvelles stratégies de valorisation des contenus produits ou diffusés pour maintenir leur visibilité et leur rentabilité. 
	% Une autre approche consiste à fournir un service de recommandation aux spectateurs et ainsi rentrer dans une démarche de fidélisation. 
	% Cela implique de se consacrer à des tâches d'éditorialisation des contenus pour des audiences plus ciblées.
	
	% \item \eg{produire de nouvelles formes de contenus ou adapter les rmes existantes pour satisfaires aux nouveaux modes de distribution/consommation.}
	% L'augmentation des terminaux de lecture multimédia et leur portabilité offrent de plus en plus d'occasions aux spectateurs de consommer des contenus. Par exemple, les situations de mobilités peuvent impliquer des capacités de transfert diminués, un écran plus petit, des temps de disponibilités plus courts etc.
	% Il semble alors s'ouvrir une place pour de nouveaux formats ou des formes retravaillés de contenus existants. 
	% Il s'agit de faire de la production multi-support et d'adapter les contenus en fonction des conditions de distribution et de l'audience visé (réutilisation).
	
	% \item \eg{articuler la contribution amateur avec la chaîne de production professionnelle.}
	% L'amélioration croissante des capteurs des appareils multimédia ajoutée aux capacités de communication offrent au grand public de plus en plus de manières de participer aux processus de fabrication ou de diffusion des contenus.
	% Les possibilités accrues de participation au processus médiatique (participation à l'émission, envoi de contenu, propagation via ses contacts, commentaires etc.) valorisent le spectateur, le contenu et la plate-forme de diffusion.

	% Cependant, il faut être capable d'intégrer ces contributions externes au sein de la production professionnelle en les encadrant plus ou moins fortement, par des indications, recommendations ou des contraintes.

	% \item \eg{gérer les objets audiovisuels dès le début et tout au long de leur cycle de vie.}
	% %
	
	% Une circulation plus importante des contenus implique de trouver un moyen de gérer non plus des fichiers mais des objets numériques qui unifient plusieurs variations d'un même contenu et embarquent des descriptions explicitant leurs modalités d'exploitation.



% numériser => (a) fragmenter + (b) mettre en réseau => (a) besoin de conserver la cohérence de l'ensemble + (b) besoin d'autonomiser pour une future situation d'usage








%%%%%%%%%%%%%%%%%%%%%%%%%%%%%%%%%%%%%%%%%%%%%%%%%%%%%%%%%%%%%%%%%%%%%%%%%%%%%%%%%%%%%%%%%%%%%%%%%%%
\newpage
\section{Problèmes}\label{sec:prob}



%%%%%%%%%%%%%%%%%%%%%%%%%%%%%%%%%%%%%%%%%%%%%%%%%%%%%%%%%%%%%%%%%%%%%%%%%%%%%%%%%%%%%%%%%%%%%%%%%%%
\subsection{Problèmes métiers (m)}\label{sec:pmetiers}
% Nos champs d'applications sont : 
% Prenant acte des besoins de la production audiovisuelle nous distinguons deux problèmes métiers (1) identifier le(s) niveau(x) de fragmentation et (2) le(s) type(s) de description susceptibles de favoriser la fabrication mixte, la circulation et la réutilisation des contenus.
% Ces problèmes remettent en cause à la fois la représentation classique des contenus et leur description. 
% de manière à faciliter (1) la production mixte amateur-professionnel et (2) leur réutilisation dans de nouveaux contextes d'exploitation.
% l'informatisation du début de la chaîne de production, préproduction 
% l'inscription formelle de l'écriture audiovisuelle ?

\e{
L'objectif central qui se pose à la production audiovisuelle est de constituer des objets audiovisuels autonomes et donc (ré)utilisable à n'importe quel étape de la chaîne. 
Or, dans la chaîne de production classique les programmes n'émergent qu'à la fin de la chaîne et sont gérés d'une pièce. 
Il n'y a pas forcément de place pour les éléments de contenu intermédiaires, et ce sont justement à la modélisation de ces fragments que l'on s'attaque. 
Ces fragments doivent devenir des éléments documentaires qui possèdent leur unité propre de même que les objets finis que sont les programmes. 
Une des difficultés réside dans cette articulation entre des fragments et le tout ou l'ensemble que constitue les programmes. 
Par ailleurs, il existe des problèmes sous-jacents à cette fragmentation documentaire : 
}
\begin{liste}
	\item \e{identifier quels niveaux de fragments peuvent prétendre à ce genre de transformation}.
	Si le numérique permet de fragmenter à l'envie, il faut cependant prendre en compte les pratiques du métier pour identifier les niveaux de fragmentation pertinents ou déjà utilisé dans le métier mais non modélisé.
	Par exemple, la prise de vue est le résultat d'une activité de tournage, pour autant elle ne constitue pas un objet éditorial comme peut l'être une interview.
	De plus, il s'agit également de déterminer comment manipuler ces fragments comme des objets à part entière sans compromettre l'articulation de l'ensemble. 
	Par exemple, une interview peut s'intégrer dans un journal télévisé ou bien un reportage dans des versions plus ou moins courtes. 
	Pour autant, il s'agit du fruit d'une même activité, simplement le montage, et donc le résultat, est différent suivant le programme dans lequel l'interview s'insère.\\
	
	\item \e{identifier quelles informations et connaissances doivent être rattacher à ces fragments pour les rendre autonome}.
	D'une manière similaire à la démarche pour les objets entiers, certaines connaissances, notamment relatives au contexte de production, doivent être attachées au fragment pour garantir sa réutilisation et sa cohérence. 
	En reprenant l'exemple de la prise de vue, on peut lui associer le bout de script qui a prescrit ce qu'elle devait montrer. 
	Si l'on pousse encore cette logique, il faut également incorporer le document qui définit le programme pour lequel on a tourné cette prise de vue, la personne qui l'a effectué, les équipements utilisés etc. 
	Ainsi, on étend la modélisation de l'objet audiovisuel à son contexte de production et tout ce qui renforce les possibilités de recherche, de manipulation, de gestion et de transformation de ces objets. 
	De plus, s'ajoute à cela la question de la collecte de ces informations. 
	En effet, la saisie de ces informations au sein d'un système d'information et leur utilisation par les acteurs de la chaîne n'est pas une simple formalité.
	Ce problème pousse également dans le sens d'une modélisation plus contextuelle, de manière à proposer un environnement de travail adapté et utilisable aux contributeurs de la chaîne.\\
	
	% \item \e{}.
\end{liste}


% Notre problème se situe dans le croisement de la représentation des contenus et la représentation des activités humaines qui construisent, manipulent, éditent, transforment, publient et documentent ces contenus. 
% Cet angle de recherche nous amène donc à considérer non pas le contenu audiovisuel dans son ensemble, une fois terminé et validé, mais la construction de tout ces fragments qui le composent. 
% Il nous faut aussi considérer, non pas seulement le contenu audiovisuel, mais aussi d'autres informations, d'autres documents qui constituent son contexte de production, dans un sens très général. 
% Chaque prise de vue constitue donc un objet à représenter en tant que tel, tout autant que le bout de script qui a prescrit ce que cette prise de vue devait montrer. 
% Si l'on pousse encore cette logique, on peut alors représenter de même le document qui définit le programme pour lequel on a tourné cette prise de vue, la personne qui l'a effectué, les équipements utilisés etc. 
% Ainsi, on étend la représentation du contenu à son contexte de production entendu comme toutes les informations qui renforceront les possibilités de recherche, de manipulation, de gestion et de transformation de ces contenus. 

%%%%%%%%%%%%%%%%%%%%%%%%%%%%%%%%%%%%%%%%%%%%%%%
\subsection{Problèmes scientifiques (m)}\label{sec:scien}
Au fur et à mesure que la circulation des contenus s'intensifie, il y a un besoin grandissant de faciliter l'échange d'information tant à la fois sur le plan informatique, que sur le plan humain. 
De plus, l'ouverture de la chaîne de production à de nouveaux contributeurs (amateurs et professionnels) ne fait qu'accentuer la disparité des connaissances et des systèmes utilisés. 
Afin de construire une compréhension commune à tous les contributeurs au cycle de vie, on fabrique un modèle conceptuel capable d'intégrer et de mettre en relation leurs connaissances. 
Il s'agit là d'un apport par rapport à la situation existante où le consensus n'existait pas, ou alors de manière éphémère, locale au sein d'une équipe.
L'objectif est de fluidifier les échanges d'information et de contenus en formalisant les connaissances utilisées pour :
\begin{liste}
% modéliser tous les objets de la chaîne de production audiovisuelle
	\item[(A)] \g{modéliser les objets construits au fil de la chaîne de production audiovisuelle}.
	\item[(B)] \g{modéliser les connaissances sur ces objets} (descriptions, contexte de production, contribution au cycle de vie).
	% \item[(B)] décrire les objets audiovisuels
	% \item[(C)] représenter la contribution de chacun des acteurs au cycle de vie des objets audiovisuels
\end{liste}

\e{
Ainsi, notre problème de recherche général s'articule autour de la modélisation des connaissances et des informations que les contributeurs construisent, utilisent, échangent au cours du cycle de vie des objets audiovisuels. 
Cette modélisation constitue une première étape dans la mise en place d'un système d'information servant à mieux gérer les objets audiovisuels et médier la communication entre systèmes informatiques tout autant qu'entre contributeurs humains.
Après avoir numérisé les contenus audiovisuels, on souhaite transformer chaque élément les composant en objet documentaire et documenter leur cycle de vie.\\}



\g{(A)} Le problème est d'organiser la gestion des objets audiovisuels en proposant une modélisation capable de faciliter leur identification, leur manipulation et leur réutilisation tout au long de leur cycle de vie. 
En particulier,	l'objet audiovisuel professionnel est produit de manière collective, chaque contributeur apportant un élément à l'ensemble. 
Ces contributions doivent donc pouvoir être identifiées comme appartenant à un ensemble, de même que chaque élément doit pouvoir être considéré pour soi afin d'être intégré dans un autre ensemble (réutilisation).

Pour cela on adopte une représentation des différents niveaux d'abstraction des objets audiovisuels numériques de façon à rétablir les liens entre les différentes versions ou copies d'un même contenu, quelque soit la nature des variations entre elles (encodage, format d'encapsulation, montage, finition, langue etc.).
La distinction entre différents niveaux de modélisation (technique, esthétique, éditorial etc.) doit permettre de construire une représentation dynamique de l'objet audiovisuel qui suit l'avancement du processus de production.\\
% La distinction entre différents niveaux de modélisation (technique, esthétique, éditorial etc.) doit permettre de construire une représentation de l'objet audiovisuel au fur et à mesure de l'avancement du processus de production.\\


\g{(B)} Le problème est d'attacher plusieurs types de connaissances aux objets audiovisuels de manière à les rendre autonomes dans leur circulation et leur réutilisation. 
Afin de faciliter l'échange d'information et la réutilisation des objets audiovisuels entre différents contextes, il faut modéliser des connaissances sur ces objets qui sont parfois déjà existantes mais non formalisées, ou bien qu'il faut rendre compréhensibles.
En effet, l'échange d'information dans la production audiovisuelle est primordial et s'effectue entre métiers et organisations différents, voire avec des amateurs. 
En particulier, on souhaite s'appuyer sur le vocabulaire de l'écriture filmique utilisé dans des documents de préproduction pour spécifier les résultats attendus de la production.
L'information contenue dans ces documents est importante mais repose sur des conventions plus ou moins tacites qu'il faut expliciter pour les professionnels, expliquer pour les amateurs.
La formalisation des ces éléments devra donc pouvoir être lu et modifié tout au long du cycle de vie des objets audiovisuels par tout types de contributeurs.

% échange d'info, adaptation par l'explicitation du vocabulaire et la contribution au cycle de vie
La formalisation de l'écriture filmique permettra d'adapter la présentation de l'information en fonction des connaissances, de l'implication du contributeur dans la chaîne (rôle, tâche, niveau de compétences etc.), de son référentiel professionnel ou linguistique. 
Il s'agit alors d'établir des correspondances entre les connaissances connues par le lecteur d'une information et celles utilisées par la personne qui l'a exprimé.
Ainsi, d'une part on explicite l'expression de l'information, ce qui permet l'adpatation, et facilite son interprétation ultérieure.
Le processus prend tout son sens lorsqu'il s'agit de traduire un concept de la réalisation audiovisuelle pour guider un amateur dans son tournage.
% Par exemple, l'action écrite par l'auteur et transformé en scène par le réalisateur, doit ensuite être tourné un caméraman, des acteurs etc. 


% réutilisation, attachement des connaissances pertinentes pour manipuler ou exploiter chaque fragment ou l'objet en entier
De plus, les descriptions utilisées, en plus de permettre de spécifier le résultat attendu de la production, doivent permettre de faciliter la recherche, la manipulation et la réutilisation des objets audiovisuels.
Pour cela, il faut articuler ces connaissances aux objets et aux fragments qui les composent. 
Le vocabulaire de l'écriture filmique sera également précieux, puisqu'il nous donne une unité de base, le plan, ainsi que ces caractéristiques qui permettent de le distinguer des autres. 
La recherche dans des dépôts de contenus se ferra ainsi de manière similaire à la commande de contenu à d'autres contributeurs, qu'ils soient professionnels ou amateurs.
./



% Il s'agit donc de définir un modèle de description susceptible d'être utilisé par les contributeurs professionnels ou amateurs, à toutes les étapes de la chaîne de production. 


% Afin de décrire les contenus, on souhaite s'appuyer sur le vocabulaire de l'écriture audiovisuelle utilisé dans les différentes étapes de la chaîne et notamment dès la préproduction. 
% Cette écriture repose sur un vocabulaire des techniques de réalisation audiovisuelle (prise de vue, transition, composition de l'image etc.) renvoyant à des effets largement connus dans le milieu de l'audiovisuel et chez les cinéphiles. 
% L'écriture est utilisée avant la fabrication du contenu pour la spécifier, puis pendant la fabrication pour enregistrer les différences. 
% Une formalisation de ce vocabulaire permettrait de construire une description textuelle d'un contenu à partir d'une description objective de la réalisation (réglages des appareils, position des acteurs etc.).\\



% \g{(C)} Le problème est de représenter et faciliter l'échange d'information entre des contributeurs hétérogènes dans leurs connaissances et leur implication dans la chaîne de production.
% Une première diffculté réside dans l'articulation entre les connaissances des contributeurs qui expriment l'information et ceux qui l'interprèteront.
% La seconde diffculté consiste dans l'articulation des représentations du cycle de vie, de l'objet audiovisuel et des descriptions qui leurs sont associées. 
% En effet, chaque contributeur peut participer à la constitution d'informations associées au contenu en cours de sa production.
% Ces informations varient en fonction de l'implication du contributeur dans la chaîne (rôle, tâche, niveau de compétences etc.). 
% De plus, les informations construites à un moment sont susceptibles d'être utilisées plus tard dans la chaîne, par un contributeur ne partageant pas forcément les mêmes connaissances ou le même référentiel professionnel ou linguistique.

% On cherche alors à réaliser une adaptation de la forme d'expression de ces informations afin de faciliter le déroulement du processus de production. 
% Dans un premier temps, on explicite les connaissances utilisées par un premier utilisateur pour exprimer une information. 
% Ensuite, on établit une correspondance avec les connaissances connues d'un autre utilisateur et on adapte au besoin la forme de d'expression de cette information pour faciliter son interprétation. 
% L'adpatation qui en résulte prend tout son sens lorsqu'il s'agit de traduire un concept de la réalisation audiovisuelle pour guider un amateur dans son tournage.


%%%%%%%%%%%%%%%%%%%%%%%%%%%%%%%%%%%%%%%%%%%%%%%
\section{Positionnement Disciplinaire (n,i)}\label{sec:posd}
[Ingénierie des connaissances ; Media Asset Management ; Gestion électronique de Documents ; Ingénierie documentaire]
% ingénierie des connaissances (représentation des connaissances) ingénierie des inscriptions numériques de connaissances, dont les documents
% ingénierie documentaire (modélisation des documents propres à la production audiovisuelle)
% indexation et gestion des connaissances (description des contenus audiovisuels)
% la ged s'occupe de la gestion de documents, nous proposons de gérer des fragments de documents, de gérer leur construction en plusieurs étapes, par plusieurs acteurs et dans le cadre de différentes missions.

% Pour définir cet ensemble d'informations qui forment le contexte de production, nous nous sommes appuyés sur les partenaires du projet MediaMap. 



% des réécritures de documents et des mises à jours qui pourraient être réalisées par des machines, des informations qui pourraient être transmises automatiquement à travers un réseau numérique d'information
% un vocabulaire bien défini qui fait l'objet de nombreux dictionnaires, donc prêt à être formalisé
% des équipes qui sont réduites lors de tournage en extérieur

% à mettre dans le pos. disciplinaire, comment on aborde les problèmes posées
% [Nous avons ainsi dégagé plusieurs perspectives métiers qui nous ont servi de guide pour identifier les échanges d'informations les plus importants ainsi que le vocabulaire utilisé pour les exprimer. 
% Chacune de ces perspective possède un objectif propre et des spécificités, cependant il apparaît qu'un langage commun est utilisé par tous les acteurs de la production. 
% En se concentrant sur la description d'un contenu existant ou à venir, ce langage permet à ces acteurs de communiquer entre eux. Le réalisateur qui spécifie un attendu dans son script, les caméraman qui réalisent le cadrage, les opérateurs lumières etc. 
% Tous utilisent ce langage pour imaginer le résultat à produire et en déduire les gestes à opérer. 
% Les usages n'étant jamais complètement figé, chaque organisation développe ses propres idiomatismes de langage. 
% Dans ce cas, la collaboration entre organisations impliquent de pouvoir réaliser des ajustements dans l'expression de la description du contenu. 
% De même, la collaboration avec des contributeurs amateurs soulève un problème de compréhension de ce langage (et donc de l'attendu) mais aussi de connaissances des gestes à opérer (pour produire le résultat attendu).
% Ainsi, à mesure que la circulation des contenus s'intensifie, que les besoins de collaboration augmentent, naît un besoin grandissant d'explicitation des échanges d'information afin de dégager une vue d'ensemble de la chaîne de production, de ses acteurs, de leurs interactions, de leurs produits. 

% Notre proposition consiste à modéliser ces éléments et à en informatiser l'accès de manière à fluidifier les échanges de contenus et faciliter la compréhension des informations afférentes.] 


% \cleardoublepage





%%%%%%%%%%%%%%%%%%%%%%%%%%%%%%%%%%%%%%%%%%%%%%%%%%%%%%%%%%%%%%%%%%%%%%%%%%%%%%%%%%%%%%%%%%%%%%%%%%%
%%%%%%%%%%%%%%%%%%%%%%%%%%%%%%%%%%%%%%%%%%%%%%%%%%%%%%%%%%%%%%%%%%%%%%%%%%%%%%%%%%%%%%%%%%%%%%%%%%%
\part*{État de l'Art}
%%%%%%%%%%%%%%%%%%%%%%%%%%%%%%%%%%%%%%%%%%%%%%%%%%%%%%%%%%%%%%%%%%%%%%%%%%%%%%%%%%%%%%%%%%%%%%%%%%%
%%%%%%%%%%%%%%%%%%%%%%%%%%%%%%%%%%%%%%%%%%%%%%%%%%%%%%%%%%%%%%%%%%%%
%%%%%%%%%%%%%%%%%%%%%%%%%%%%%%%%%%%%%%%%%%%%%%%%%%%%%%%%%%%%%%%%%%%%
%%%%%%%%%%%%%%%%%%%%%%%%%%%%%%%%%%%%%%%%%%%%%%%%%%%%%%%%%%%%%%%%%%%%
%%%%%%%%%%%%%%%%%%%%%%%%%%%%%%%%%%%%%%%%%%%%%%%%%%%%%%%%%%%%%%%%%%%%
\chapter{Outils de modélisation}\label{chap:omod}
\minitoc

% pourquoi on parle de SOC ? Parce qu'on souhaite représenter des objets audiovisuels, leur production, leur description et que ceci ne peut se faire sans une représentation du vocabulaire utilisé dans la production audiovisuelle. 

Dans le cadre de la production audiovisuelle collaborative, qui implique à la fois des amateurs et des professionnels, un des enjeux que nous avons noté (\ref{sec:scien}) est de rendre plus compréhensible l'échange d'informations entre contributeurs. 
D'une part, nous avons des practiciens professionnels utilisant un ou plusieurs vocabulaires métiers suffisamment définis pour que l'on puisse en faire des dictionnaires (\cite{Journot2008}; \cite{Pinel2008}). 
Ces personnes font usage de la langue d'une manière précise, régie par des conventions et portée par une conception de la production audiovisuelle et de ses objets, que l'on suppose stabilisée, au moins localement (au sein d'un même pays, d'une même école de pensée, organisation, équipe etc.).
Toutefois, on s'attend à des variations dans les usages de la langue, au même titre que l'on suppose qu'il existe des variations entre pratiques des gens du métiers.
Cependant, il semble qu'il s'agit seulement de variations et qu'il soit possible alors d'identifier les éléments communs et distincts.

D'autre part, nous avons les amateurs, qui n'ont pas le support de ces conventions travaillées au quotidien.
Ils sont, au mieux, des intermittents éclairés qui ont saisi le sens d'éléments de langage propres aux métiers (par des lectures, des rencontres, des formations etc.). 
Il faut donc supposer qu'il y a tout à expliquer à ces amateurs, plutôt que de parier sur leur compréhension innée des métiers de la production audiovisuelle.
En particulier s'il s'agit de demander du contenu à des amateurs sous la forme d'un script de tournage, il faudra alors trouver un moyen d'expliciter cette commande et d'expliquer ou d'assister sa réalisation. 

Ainsi, l'écart entre collaborateurs (qu'ils soient tous professionnels, ou un mélange d'amateurs et de professionnels) peut se situer sur différents niveaux : 
\begin{liste} 
	\item au niveau du vocabulaire métier, comme les mots utilisés dans le script pour désigner tel ou tel type de plan etc. 
	Dans le cas d'amateurs, il faut supposer que ces mots sont inconnus ou méconnus ; dans le cas de professionnels, on préfèrera expliciter le vocabulaire afin d'éviter toute confusion. 
	
	\item au niveau des connaissances et de la manière de conceptualiser le métier, le cycle de production et ses objets. 
	Là encore, un amateur ne connaît pas ou très peu les détails classiques des méthodes de production professionnelle. 
	De plus, la production audiovisuelle étant organisée en projets distincts, les méthodes peuvent fortement varier entre la production d'un documentaire et d'une émission de variétés. 
	Chaque genre, chaque équipe aura donc ses propres objets, ses propres méthodes qu'il faut alors expliciter aux autres professionnels pour s'assurer de leur collaboration. 

	\item sur le plan pratique, il faut également remarquer que les compétences peuvent également varier fortement entre métiers, suivant les genres de production audiovisuelle. 
	Ainsi, en plus d'expliciter les échanges d'informations, il serait également souhaitable de proposer une assistance aux collaborateurs amateurs ou professionnels pour s'assurer que le résultat produit correspond bien à l'attente initiale. 	
\end{liste}


Nous présentons dans une première section un exemple de commande de tournage qui illustre ces différents écarts en impliquant des communautés d'amateurs et de professionnels (\ref{sec:cdcf}). 
Ce scénario d'usage nous permet de préciser les besoins en modélisation exprimés précédemment (\ref{sec:prob}).
Notamment, il apparaît nécessaire de représenter à la fois la ou les conceptualisation(s), le(s) vocabulaire(s) utilisé(s) ainsi que les résultats attendus par les acteurs de la chaîne de production audiovisuelle. 
Ainsi, nous nous intéresserons à l'utilisation de divers Systèmes d'Organisations de Connaissances (SOC, \cite{Zacklad2010}) pour mettre en place un partage d'information normalisée. 
Après un retour sur les définitions principales que nous utiliserons, nous examinerons les langages, modèles et normes existants qui permettent de représenter des SOC (\ref{sec:defs}).
La distinction entre terminologie et ontologie nous permet de détailler le fonctionnement d'une méthode de construction d'ontologie différentielle (\ref{sec:construction}).
Enfin, nous présenterons les langages permettant de représenter ces SOCs (\ref{sec:mods}). 

%Dans une seconde partie, nous examinerons les modèles de l'audiovisuel existants.




%%%%%%%%%%%%%%%%%%%%%%%%%%%%%%%%%%%%%%%%%%%%%%%%%%%%%%%%%%%%%%%%%%%%
%%%%%%%%%%%%%%%%%%%%%%%%%%%%%%%%%%%%%%%%%%%%%%%%%%%%%%%%%%%%%%%%%%%%
\section{Cahier des charges fonctionnel (n)}\label{sec:cdcf}


%%%%%%%%%%%%%%%%%%%%%%%%%%%%%%%%%%%%%%%%%%%%%%%%%%%%%%%%%%%%%%%%%%%%
\subsection{Scénario de commande de tournage}\label{sec:scenar}
Considérons comme cas d'étude une commande de tournage en vue de réaliser des reportages sur des évènements culturels de type concert ou opéra. 
Il met en jeu trois communautés en collaboration :

\begin{itemize}
	\item la RTBF (Radio Télévision Belge Francophone) établit des commandes de contenu dans un jargon métier propre. Son objectif est d'externaliser dès que possible la réalisation de la commande. Cela implique une compréhension commune sur le contenu à réaliser qui passe par un accord sur la manière de décrire la commande. 
	
	\item le contenu commandé est tourné soit par la VRT (Radio-Télévision Flamande) qui utilise un jargon différent de la RTBF, soit des amateurs qui ne connaissent pas les concepts de la réalisation audiovisuelle. Dans le premier cas, la conceptualisation est commune, seuls les termes changent. Dans le second cas, il s'agit d'expliquer et d'illustrer les concepts utilisés.\\
\end{itemize}

Le développement d'une application d'assistant de tournage pour guider les amateurs paraît souhaitable pour amener le contenu filmé à un niveau de qualité exploitable. 
Il faut cependant faire la distinction entre les propositions de dépôt spontané de contenu (comme le pratique une chaîne d'information telle que BFM\footnote{La rubrique témoins BFM permet à un utilisateur de déposer des photos ou vidéos sur le site. 
Après modération, le contenu est diffusé et peut même faire l'objet d'une vente. Voir http$:$//temoins.bfmtv.com/}) et les appels à contribution où le professionnel passe commande auprès d'amateurs en détaillant ses exigences. 

Dans ce dernier cadre, on souhaite fournir un plan de tournage au caméraman afin de guider sa prise de vue. 
Le plan de tournage est construit à partir de recommandations rédigées par un réalisateur (position, cadrage, lumière, etc.) utilisant un vocabulaire métier. 
L'originalité de l'application est d'adapter l'information présentée au caméraman suivant ses capacités (amateur, professionnel) ou son employeur si c'est un professionnel travaillant dans une tierce organisation. 
On suppose ainsi que malgré quelques variations dans le vocabulaire utilisé, les professionnels de l'audiovisuel utilisent les mêmes concepts pour décrire le contenu. 
Par exemple, la notion de cadrage fait appel à des concepts de valeurs de plan indiquant la portion visible d'un personnage à l'écran (voir Figure \ref{img:intro:script}, page \pageref{img:intro:script}).
Un \textit{plan américain} indique ainsi que le personnage principal est cadré de la tête jusqu'au dessus des genoux. 
Le terme est utilisé en Europe en rappel à son emploi caractéristique dans les films américains des années 1910-1940, notamment dans les westerns où il permettait de montrer l'ensemble du pistolet à la ceinture des personnages\footnote{Roger Boussinot, l'Encyclopédie du Cinéma, Bordas.}. 
Ce cadrage est aussi appelé \textit{plan 3/4} et en anglais 3/4 shot, medium-long shot ou american shot pour traduire l'expression popularisée en Europe. 
Si le terme utilisé varie suivant le lieu et la littérature de référence, la définition de ce type de cadrage est sans équivoque. 

%=============
% \begin{figure}[htb]
% \centering
% \includegraphics[width=0.5\textwidth]{./images/ValeurPlan-v1.png}
% \caption{Différentes valeurs de plan pour le cadrage d'un personnage à l'écran}
% \label{fig:cadrage}
% \end{figure}
%=============

Les amateurs quand à eux ignorent ces concepts et n'ont pas été initiés à ces pratiques. Ils ont donc besoin d'explications et d'illustrations pour comprendre les recommandations du réalisateur. 
Dans le cas du cadrage, une illustration graphique est d'autant plus pertinente. 
L'enjeu se situe donc dans la collaboration entre un prescripteur et un opérateur qui doivent s'accorder sur le contenu à produire malgré la différence de vocabulaire. 
% Un exemple des différences de présentation entre amateur et professionnel est illustré figure \ref{fig:prescription}.


% %%=============
% \begin{figure}[htb]
% \centering
% \includegraphics[width=0.3\textwidth]{./images/ShootingRecommandation-v1.png}
% \caption{Exemple de prescription de tournage à destination de professionnels (en haut) ou d'amateurs (en bas)}
% \label{fig:eda:prescription}
% \end{figure}
% %%=============


%%%%%%%%%%%%%%%%%%%%%%%%%%%%%%%%%%%%%%%%%%%%%%%%%%%%%%%%%%%%%%%%%%%%
\subsection{Besoins en modélisation}\label{sec:bm}
La mise en place d'une telle application nécessite de représenter le vocabulaire de la réalisation audiovisuelle dans toutes ses variations possibles et de le documenter suffisamment afin de le rendre compréhensible pour des novices. 
Cet objectif nous amène à considérer la construction d'une ressource termino-ontologique. L'ontologie permet de représenter les concepts partagés par les professionels de la réalisation audiovisuelle et la terminologie permet de capturer les différentes formes d'expression associées à ces concepts. 

La spécificité de notre problématique est de considérer la collaboration de communautés hétérogènes par leur degré de compréhension des concepts ou leur utilisation de la terminologie. 
Ceci nous amène à envisager la terminologie comme un moyen d'associer à des éléments ontologiques (concept, relation, instances) une chaîne lexicale ou des ressources média. 
Chaque chaîne ou ressource s'adresse en particulier à une communauté dont les membres partagent une capacité d'interprétation commune. 
Il n'existe donc plus une terminologie de référence par langue, mais des terminologies pour chaque communauté d'utilisateurs. 
On remarquera que notre acception de la terminologie sert bien à normaliser les pratiques linguistiques entre les membres d'une même organisation. 
En plus de cela, elle permet de fixer la manière de s'adresser à d'autres communautés.

Par ailleurs, les types de réalisations sont divers et nécessitent des concepts spécifiques pour être décrits. 
Une fiction se structure en séquences et en scènes alors que les documentaires ou magazines d'information se composent de sujets. 
La variabilité des types de contenu à filmer implique donc de pouvoir étendre le fond conceptuel initial pour représenter de nouveaux usages. 
De la même manière, la collaboration avec de nouveaux partenaires nécessite de pouvoir ajouter de nouvelles terminologies au fond conceptuel existant. 
Ontologie et terminologie doivent se gérer de manière indépendante. A partir de ces besoins, nous définissons maintenant les exigences en terme de modélisation. 

Nos besoins en modélisation peuvent être exprimés par les assertions suivantes:
\begin{enumerate}
	\item[(\e{A1}]\e{: multi-jargon}) la variabilité des pratiques linguistiques des organisations et des communautés implique d'associer plusieurs termes à un même concept. 
	Il n'y a pas de choix des termes préférés par une communauté mais une \e{correspondance} entre les termes d'une ou plusieurs communautés, quels que soient la langue, le jargon et le code d'écriture utilisé.
	
	\item[(\e{A2}]\e{: documentation}) la variabilité de compréhension des communautés implique d'associer des explications (chaîne lexicale) ou des illustrations (ressource média) aux concepts afin d'en enrichir la documentation. 
	
	\item[(\e{A3}]\e{: gestion, évolution}) la variabilité des cas de collaboration implique de pouvoir étendre la conceptualisation initiale ou la terminologie pour s'adapter à de nouvelles pratiques ou de nouvelles communautés. 
	Cela implique une gestion et une évolution indépendante de l'ontologie et de la terminologie. 
\end{enumerate}


Dans le cas d'une demande de cadrage en plan américain, la demande est d'abord exprimée dans le jargon de la RTBF puis traduite dans le jargon de la VRT (plan américain pour la RTBF, plan 3/4 pour la VRT) [\g{A1 : multi-jargon}]. 
Ensuite, pour les amateurs, la terminologie est enrichie par des illustrations [\g{A2 : documentation}]. Enfin, un nouveau concept de cadrage est ajouté (plan américain large ou plan moyen serré) [\g{A3 : gestion, évolution}] en vue d'une nouvelle coopération avec la VRT. En plus de cela, le problème de la langue (français et flamand) s'ajoute à la question des jargons métiers. 



\section{Les Systèmes d'Organisation de Connaissances (m)}\label{chap:defs}
%Distinguer entre terme et concept ; dictionnaire, thésaurus, ontologies etc.

\e{
L'objectif de cette section est de clarifier ce qui appartient au domaine de la  linguistique et ce qui relève du domaine conceptuel, en vue d'identifier les notions qui nous serviront à spécifier une solution au cahier des charges dressés dans la section précédente. 
La confusion qui nous intéresse concerne principalement la définition des ontologies par rapport à d'autres SOC tels ques les thésaurus, notamment du fait qu'on utilise parfois les mêmes langages pour les représenter.}

La définition des SOC proposée par \cite{Zacklad2010}, étend celle de \cite{Hodge2000} à \ciel{l'ensemble des formes d'écritures codifiées participant à la description documentaire primaire ou secondaire d'une situation}. 
L'ensemble défini par \citeauthor{Hodge2000} comprend ainsi tout type de :
\begin{liste}
	\item \e{liste de termes} (fichiers d'autorités, glossaires, dictionnaires, répertoires géographiques)
	\item \e{schème de classification/catégorisation} (vedettes-matières, taxonomie)
	\item \e{schème qui se structure par le types de relations qui unit ses membres} (thésaurus, réseaux sémantiques, ontologies). 
\end{liste}

À cela \citeauthor{Zacklad2010} souhaite ajouter des modes de description du contenu émergents ou plus faiblement codifiés comme par exemple les folksonomies. 
Les SOC qui nous intéressent en particulier sont les schèmes structurés par types de relations. 
% pourquoi ? 


\subsection{Thésaurus, terminologie, ontologie}\label{sec:tto}
Dans cette section, nous nous reposerons majoritairement sur les définitions de \cite{bachimont:icc}. Concernant le thésaurus, l'auteur écrit :

\g{Thésaurus :} 
\ciel{
Une organisation de libellés linguistiques selon des relations d'hyperonymie et d'hyponimie. 
Les libellés sont également reliés par des relations dites d'association, qui sont de nature quelconque. 
Même si en pratique les libellés d'un thésaurus correspondent souvent à des termes du domaine, ce n'est pas nécessairement systématique.}

Cette définition situe clairement les thésaurus comme faisant partie du cadre de la linguistique. 
Il s'agit d'un ensemble de mots structurés et reliés suivant leur \e{signification}, c'est à dire leur sens normé ou commun à plusieurs contextes d'usage particuliers (à l'inverse du sens, qui lui varie suivant les usages, \cite{Roche2005}). 
\citeauthor{bachimont:icc} finit sa définition en comparant les mots issus d'un thésaurus aux termes. La distinction se joue à deux niveaux, la stabilité d'écriture du terme (niveau linguistique) et le fait qu'un terme renvoit à un concept (niveau conceptuel) : 

\g{Terme :} 
\ciel{
Une unité linguistique dont le signifié est un concept, c'est-à-dire un signifié normé. 
Le terme se manifeste linguistiquement par une stabilité et régularité de sa forme signifiante.
En particulier, un terme possède des contextes d'occurrence réguliers, obéissant à des canevas morpho-syntaxiques typiques. 
La détection de ces canevas est à la base des outils de détection des termes en corpus. 
Un terme peut posséder des variantes terminologiques.
Dans une optique normative, on détermine une forme préférée.}

Ainsi, plus qu'un repérage des mots (signifiant) utilisés dans un domaine donné, la terminologie s'attache à identifier les signifiés correspondants. 
Au-delà des débats sur les méthodes utilisées pour constituer les couples signifiant-signifié\footnote{L'approche \e{sémasiologique} (initiée par \cite{Bourigault1994}) s'appuie sur l'analyse linguistique d'un corpus de textes pour repérer les couplages signifiant-signifié ainsi que l'organisation conceptuelle sous-jacente. L'identification de ces \e{désignations} est ensuite validée par des experts du domaine. 
Dans une optique différente, l'approche \e{onomasiologique} prend comme appui la modélisation conceptuelle pour nommer ensuite les concepts. On parle alors de \e{dénominations} dont l'objectif est de refléter sans ambiguïté la structure conceptuelle dont elles sont issues.
Une critique faite à la première approche par \cite{Roche2006} est que les relations identifiées entre désignations sont purement linguistiques (hyper/hyponymie, méronymie etc.) et ne se rattachent pas à une structure conceptuelle. Ainsi, l'analyse de texte n'est qu'une première étape dans la constitution d'une terminologie, elle permet d'identifier les usages des mots, mais pas de les raccorder à des concepts. De même, la constitution du corpus va également grandement influer sur les résultats de l'analyse et pose alors un problème de réutilisabilité.}, on cherche à repérer la modélisation conceptuelle sous-jacente d'un domaine, de manière à pouvoir adosser chaque terme à un concept.
À noter que l'inverse n'est pas forcément valable, car il existe des concepts qui n'ont pas d'appelation usuelle et que l'on doit alors désigner par une phrase. 
Une autre conséquence de cette définition est que plusieurs mots peuvent être adossés au même concept.
Il devient alors important de pouvoir expliciter cette équivalence et éventuellement de spécifier un signifié préféré pour le terme.

Cette définition du terme préfigure ainsi la relation qu'entretient la terminologie avec l'ontologie pour \cite{bachimont:icc} : 

\g{Terminologie :} 
\ciel{un recensement et une organisation d'unités linguistiques à l'usage stabilisé et attesté, dont le signifié correspond à un concept du domaine.
La terminologie est l'organisation des termes du domaine.
La terminologie est la face linguistique de l'ontologie, qui en est le côté conceptuel. 
Il n'y a pas une stricte correspondance cependant entre ontologie et terminologie : si tout terme doit correspondre à un concept de l'ontologie, tout concept n'a pas forcément d'usage linguistique régulier attesté.}

De son côté, \cite[\S 2.4]{Roche2005} parle de manière similaire \ciel{[d']un système de termes reflétant une modélisation conceptuelle, [...] plus généralement dénommé \e{système notionnel} [qui] trouve sa raison d'être dans la façon dont nous appréhendons les objets du monde.}
\citeauthor{Roche2005} précise que si les systèmes notionnels ne relèvent pas de la linguistique, ils ne dépassent pas forcément le cadre d'une langue, sauf \ciel{pour des communautés de pratique dont les langues d'usage partagent la même conceptualisation du monde.}.
La distinction est ainsi faite entre les mots d'usage (qui peuvent être polysémiques) et les termes dont on spécifié une forme préférée (signifiant) et qu'on adosse à une signification (signifié normé).  

Concernant les particularismes qui peuvent exister dans chaque communauté, \citeauthor{Roche2005} propose de s'éloigner d'une vision purement normalisatrice. 
Ainsi, sur le plan linguistique, il est possible de rattacher les différents mots d'usages et de préciser leur contextes d'utilisation.
Sur le plan conceptuel, l'auteur propose de constituer des \gui{terminologies régionales} que l'on cherchera ensuite à mettre en correspondance. 




\paragraph{Ontologie}
Concernant les ontologies, nous nous limiterons aux définitions proposés dans le cadre de l'ingénierie des connaissances (IC). Les travaux de \cite{Charlet2002} nous rappelle qu'il existe de multiples définitions : 

Pour \cite{Gruber1993} : \ciel{Une ontologie est une spécification explicite d'une conceptualisation.}

 La définition proposée par \cite{Uschold1996} nous permet de préciser de quoi se compose une conceptualisation et en quoi une ontologie la spécifie : 

\ciel{Une ontologie implique ou comprend une certaine vue du monde par rapport à un domaine donné. Cette vue est souvent conçue comme
un ensemble de concepts -- e.g. entités, attributs, processus --, leurs définitions
et leurs interrelations. On appelle cela une conceptualisation. [...]
Une ontologie peut prendre différentes formes mais elle inclura nécessairement
un vocabulaire de termes et une spécification de leur signification. [...]
Une ontologie est une spécification rendant partiellement compte d’une conceptualisation.}

\citeauthor{Charlet2002} en conclut qu'une ontologie est une conceptualisation, c'est-à-dire un ensemble de concepts et de relations dont on cherche à normer la signification. 
Pour faire de la conceptualisation un objet informatique, il faut spécifier une théorie logique dotée d'un vocabulaire (les concepts et les relations), à la manière des travaux de \cite{Guarino1995}.

% Roche puis Babache
Pour \citeauthor{Roche2005}, une ontologie est équivalente au système notionnel des terminologies, d'où la relation forte établie par les chercheurs en IC : 

\ciel{définie pour un objectif donné et un domaine particulier, une ontologie est pour l'ingénierie des connaissances une représentation d'une modélisation d'un domaine partagée par une communauté d'acteurs. Objet informatique défini à l'aide d'un formalisme de représentation, elle se compose principalement d'un ensemble de concepts définis en compréhension, de relations et de propriétés logiques.} (\cite{Roche2005})

\citeauthor{Bachimont2000a} insiste sur le fait qu'on utilise une sémantique donnée (différentielle, référentielle, psychologique, distributionnelle, conceptuelle etc., \cite{bachimont:hdr}) pour établir la signification des concepts de l'ontologie. Chaque sémantique propose un point de vue particulier qui permet de faire correspondre une signification propre à chaque unité d'expression : 

\ciel{une ontologie est la signature fonctionnelle et relationnelle, munie de sa sémantique, d'un langage formel de représentation et manipulation des connaissances.} (\cite{Bachimont2000a})

Les ontologies se construisent ainsi en s'adossant à des théories, et ce sont ces théories qui fixent des principes pour déterminer la signification des unités linguistiques qu'elles emploient et chargent d'un sens bien précis (sémantique). 

\paragraph{Sémantiques et ontologies}
Pour bien cerner les conséquences de cette définition, voici quelques sémantiques décrites par \cite{bachimont:hdr} qui se distinguent dans leur manière d'expliciter la signification d'une unité d'expression : 
\begin{liste}
	\item \e{sémantique différentielle} : la signification d'une unité consiste en l'identité et la différence par rapport aux autres unités linguistiques de la langue. On reste donc dans le cadre de la linguistique. La différenciation des  unités peut se faire par différentes méthodes ; par observation empirique d'un corpus de texte (\e{sémantique distributionnelle}) ; ou bien suivant une modélisation de la signification des concepts du domaine établie par des experts par exemple (\e{sémantique conceptuelle}). 

	\item \e{sémantique référentielle} : la signification d'une unité est l'objet auquel elle fait référence, dans un univers extralinguistique. Ici, on s'attache à une théorie propre à cet univers et qui explicite la définition des objets.
	
	\item \e{sémantique psychologique} : la signification d'une unité est la représentation mentale que l'on s'en fait. Là encore, il s'agit de suivre une théorie, mais dans le champ de la psychologie. 
\end{liste}
Cette liste de sémantiques permet de comprendre la grande variabilité des ontologies qu'il est possible de construire. 


\paragraph{Classification d'ontologie}
Il peut également être utile de définir des propriétés pour distinguer des "genres" d'ontologies, non pas en fonction de la sémantique utilisé, mais en fonction de l'usage que l'on souhaite en faire. 
\cite{Oberle2006} propose une classification qui repose sur trois propriétés : 
\begin{liste}
	\item l'\g{objectif} de l'ontologie (purpose) où l'on distingue entre deux objectifs, servir de référence ou bien être utilisé dans un cas d'application :
	\begin{liste}
		\item l'\e{ontologie de référence} vise à établir un consensus entre des agents (humains, machines) d'une même communauté, ou bien à servir d'explication et de langages communs avec des agents de communautés différentes. 
		\item l'\e{ontologie d'application} qui se limite à un cas d'application et suit des contraintes et simplifications propres.
	\end{liste}
	La différence réside dans l'arbitrage entre l'expressivité de la représentation et sa décidabilité (\cite{Borgo2002}). 
	Typiquement, une référence n'est consulté qu'occasionnellement et se doit d'être la plus exhaustive possible alors qu'une ontologie d'application doit servir à faire des raisonnements à l'exécution. 

	\item l'\g{expressivité} de l'ontologie (expressiveness) où l'on considère un engagement plus ou moins fort sur le formalisme de représentation :
	\begin{liste}
		\item l'\e{ontologie légère} (lightweight) qui peut se limiter à une hiérarchie de concepts bien connus dans une communauté avec quelques relations. 
		L'apport se situe alors dans la structuration des connaissances qui clarifie leur signification plus qu'il ne les établit.
		\item l'\e{ontologie lourde} (heavyweight) qui vise à exclure toute ambiguïté terminologique et conceptuelle.
		Pour cela, la formalisation se veut beaucoup plus contrainte et détaillée afin de forcer une interprétation. 
	\end{liste}

	\item la \g{spécificité} de l'ontologie qui peut se limiter à une domaine ou s'étendre à un ensemble de domaines voire plusieurs champs disciplinaires : 
	\begin{liste}
		\item l'\e{ontologie générique} (generic, upper/top level) contient des concepts utilisés dans de nombreux champs disciplinaires (évènements, processus etc.)
		\item l'\e{ontologie noyau} (core) comporte des concepts qui se situent à la croisée de plusieurs domaines. 
		La distinction avec une ontologie générique se fait car elle comporte des concepts utilisable quelque soit le domaine.
		De même, on distingue les ontologies noyaux des ontologies de domaine car les premières comportent des éléments réutilisables dans des plusieurs domaines proches. 
		\item l'\e{ontologie de domaine} contient des concepts propre à un domaine, et bien souvent des éléments plus génériques extraits de domaine différents.
		Cependant, l'ontologie de domaine présente généralement des éléments plus spécifiques, propres au domaine concerné. 
	\end{liste}
\end{liste}
\subsection{Méthodes de construction d'ontologie}\label{sec:construction}
Nous avons défini dans la section précédente (\ref{sec:tto}) ce qu'est une ontologie, présenté des exemples de sémantiques et montré comment il était possible de classifier ces ontologies suivant leur usage. 
Nous abordons maintenant les méthodes de construction d'ontologies.


\paragraph{Méthode d'\cite{Uschold1996}}
\citeauthor{Uschold1996} ont défini une méthode de construction à partir de leur expérience de développement d'ontologies en entreprises. 
Elle se décompose en quatre étapes :
\begin{listenum}
	\item Une phase de \g{conception} qui vise à identifier le domaine concerné, le but et la portée de l'ontologie.
	\item Une phase de \g{construction} qui se décompose en trois étapes ; définir les concepts clés et les relations entre ces concepts ; expliciter la représentation de la conceptualisation dans un langage formel ; intégrer des connaissances d'autres ontologies.
	\item Une phase d'\g{évaluation} de l'ontologie construite.
	\item Une phase de \g{documentation} qui doit expliciter les décisions effectuées aux étapes précédentes afin de faciliter la réutilisation de l'ontologie.
\end{listenum}


\paragraph{Methontology}
La méthode proposée par le Laboratoire d'Intelligence Artificielle (LAI) de l'université Polytechnique de Madrid a pour particularité d'intégrer le développement de l'ontologie à une méthodologie de gestion de projet \parcite{Fernandez1997, Blazquez1998}. La méthodologie distingue trois types d'activités se déroulant en parallèle, et dont les deux premières servent à soutenir la construction de l'ontologie :
\begin{liste}
	\item Les activités de \g{gestion de projet}, notamment la planification avant-projet puis le \e{contrôle de la qualité} des résultats produits. 
	
	\item Les activités de \g{support} qui concernent l'\e{acquisition} des connaissances du domaine ; l'\e{intégration} de connaissances d'autres ontologies ; la \e{documentation} de l'ontologie et de sa production ; la \e{gestion de version} des résultats produits ; l'\e{évaluation} technique de la construction de l'ontologie ainsi que de sa documentation. 

	\item Les activités de \g{développement technique} qui permettent de construire l'ontologie par étape. 
	Dans un premier temps, la \e{spécification} définit l'objectif de l'ontologie, les applications et les utilisateurs concernés ; puis la \e{conceptualisation} structure les connaissances du domaine, qui sont ensuite formalisées (étape de \e{formalisation}) et enfin représentées dans un langage informatique (étape d'\e{implémentation}). La séquence se poursuit par une étape de \e{maintenance}.
\end{liste}


% Ontospec (Kassel)
% Guarino et Welty
% voir livre BB pour la justification du manque de sémantique

\paragraph{Archonte}
La méthode \pc{Archonte} (\pc{arch}itecture for \pc{ont}ological \pc{e}laborating), proposée par \cite{Bachimont2000a}, met en avant l'importance cruciale donnée au choix d'une sémantique dans la construction d'une ontologie.
La méthode repose sur trois étapes successives qui aboutissent à une ontologie computationnelle, exprimée dans un langage opérationnel de représentation des connaissances.
% , et à partir duquel on peut effectuer des inférences. 
% Précisons maintenant les étapes de cette méthode ainsi que les résultats obtenus :
% \begin{liste}
Le point de départ de la méthodologie est constitué d'expressions linguistiques (signifiant) issues du domaine considéré.
L'intérêt de disposer d'un tel ensemble de traces linguistiques est qu'elles servent à exprimer des concepts (signifiés) ou des connaissances sur le monde. 
Ainsi, on se retrouve avec un corpus de candidats-termes dont la signification peut être source d'ambiguïtés et dont on cherche à clarifier l'interprétation.\\

\g{[1.]} La première étape de cette méthode (aussi appelée \e{normalisation sémantique}) consiste à établir un \g{engagement sémantique}  qui précise la manière de mener l'interprétation des candidats-termes et de construire une première structure de connaissances. 
Pour cela, on fixe d'abord un contexte de référence, la tâche ou le problème qui a poussé à l'élaboration de l'ontologie, qui permet de cadrer l'interprétation des candidats-termes.

Ensuite, pour préciser l'interprétation on s'appuie sur la sémantique différentielle afin d'expliciter les différences et les similarités entre une notion et son voisinage direct (notion parente, notions soeurs) : 

	\ciel{
	La méthodologie que nous proposons ici repose sur l'organisation générale des unités en un réseau d'identités et de différences.
	Ce sont les propriétés structurelles de ce réseau qui permettent de contraindre l'interprétation des unités définies dans le réseau : la position d'une unité dans le réseau prescrit comment la comprendre et lui prescrit une signification qui pourra dès lors lui être associée, quel que soit le contexte où elle se rencontre.} (\cite[p.139]{bachimont:icc})

Cette caractérisation des notions par leur voisinage repose sur quatre relations à expliciter : 
\begin{liste}
	\item la \gui{communauté avec le parent} (\ciel{similarity with parent}) : pourquoi la notion hérite des proprités de son parent.
	\item la \gui{différence avec le parent} (\ciel{difference with parent}) : en quoi la notion est différente de son parent.
	\item la \gui{différence avec les soeurs} (\ciel{difference with siblings}) : en quoi un notion est différente de ses notions soeurs.
	\item la \gui{communauté avec les soeurs} (\ciel{similarity with siblings}) : quelle est la propriété que partagent les notion soeurs -- dont on distingue plusieurs valeurs exclusives, une par soeur.\\ 		
\end{liste}
% \end{liste}

Cette première étape aboutit à la construction d'un \g{arbre ontologique différentiel} qui structure un ensemble de notions de manière hiérarchique et non ambiguië par rapport à un contexte de référence.
Les candidats-termes sont structurés par des prescriptions interprétatives et deviennent ainsi des primitives de modélisation.\\

% \begin{liste}
\g{[2.]} L'\g{engagement ontologique} consiste à munir l'ontologie différentielle d'une sémantique formelle extensionnelle.
Rappelons que cette sémantique définit les concepts par leur extension, c'est-à-dire tous les individus qu'ils désignent parmi un ensemble de référence. 
Il s'agit donc de relier des primitives dotées d'une signification linguistique normalisée à des concepts désignant un ensemble de référents (ou individus).
Pour cela, il faut adjoindre à l'ontologie différentielle un modèle référentiel : 

	\ciel{
	l'ontologie référentielle obéit aux contraintes sémantiques de l'ontologie différentielle : [s]a structure arborescente se retrouve dans l'ontologie référentielle et lui donne son squelette.
	Chaque relation de spécialisation sémantique au niveau différentiel se traduit par une spécialisation d'extension au niveau référentiel.} (\cite[p.148]{bachimont:icc})

Ce changement de sémantique permet d'enrichir l'ontologie de nouveaux concepts et d'en modifier la structuration. 
En effet, on peut désormais avoir recours à des opérations ensemblistes (réunion, intersection, complémentaire) qui composent le sens des concepts et permettent ainsi de définir de nouveaux concepts.
L'ajout de ces \gui{concepts définis} modifie également la structure de l'ontologie, qui passe d'une arborescence à une structure en treillis, c'est-à-dire admettant l'héritage multiple. 
Par exemple, une primitive différentielle de \cd{mandat politique} spécialisée en concepts de \cd{député} et de \cd{maire} ne permet pas de représenter de double mandat.
Par contre, une définition extensionnelle permet de définir le concept de \cd{député-maire} simplement par l'intersection des extensions de ces concepts\footnote{Pour plus de détails sur cet exemple, se reporter à l'exemple donné par \cite[p.149]{bachimont:icc}}.
% \end{liste}

À l'issue de cette étape on obtient donc une \g{ontologie référentielle}, c'est-à-dire un treillis de concepts définis par une sémantique référentielle.\\

% \begin{liste}
\g{[3.]} L'\g{engagement computationnel} vise à doter les concepts de l'ontologie référentielle d'une signification en termes d'opérations informatiques.
Pour cela, il faut d'abord choisir un langage opérationnel de représentation des connaissances qui détermine l'expressivité et les opérations de calculs à disposition pour élaborer une version informatique de l'ontologie.
Nous présentons quelques uns de ces langages dans la section \ref{sec:onto-mc}.
La transposition dans un langage a des conséquences au niveau de l'expressivité et de la décidabilité du modèle.
% \end{liste}

Nous obtenons une \g{ontologie computationelle} qui est une version de l'ontologie référentielle exploitable informatiquement.





\subsection{La validation en Ingénierie des Connaissances}\label{sec:valid-ic}
En suivant l'analyse proposée par \cite{Bachimont2004}, il en découle que  l'Ingénierie des Connaissances (IC) \ciel{exprime les connaissances d'un domaine dans un langage de modélisation et l'opérationnalise en un système}. 
En d'autres termes, la modélisation de l'IC porte sur les concepts utilisés par les membres de ce domaine pour penser et établir des connaissances sur le monde, mais pas directement sur le monde.
Ainsi, les modèles de l'IC n'ont pas pour vocation à \ciel{prédire quoi que ce soit sur le monde ni sur la connaissance}, mais plutôt d'\ciel{instrumenter le travail intellectuel, l'exercice de la pensée, le travail de la connaissance}. 
Dans cette perspective, \citeauthor{Bachimont2004} nous propose de théoriser l'IC comme \ciel{une ingénierie des inscriptions numériques des connaissances qui vise à instrumenter le travail cognitif associé à ces inscriptions}. 
        
Les inscriptions possèdent une double dimension ; \e{matérielle} (et donc manipulable par des techniques de calcul logique) ; \e{sémiotique} (et donc interprétable selon des conventions propres à une situation d'usage).  
En d'autres termes, le système d'IC permet d'agir de manière prédictible sur les inscriptions de connaissances, ces actions produisant de nouvelles inscriptions qui donnent matière à penser à l'utilisateur. 
        
Il y a donc plusieurs éléments à valider en IC, les calculs qui seront faits sur les inscriptions (on teste le comportement du système informatique) puis l'interprétation de ces inscriptions (on évalue le gain apporté par le système et les inscriptions qu'il fournit à l'utilisateur selon une situation d'usage). 
La modélisation prise en charge par l'IC ne porte donc ni sur le monde, ni sur l'activité cognitive et ne peut être validée uniquement par le formalisme de ses inscriptions. 
        
Les inscriptions de connaissances doivent être considérées sous deux angles : d'un point de vue \e{nomographique} (on formalise la manipulation symbolique des inscriptions pour prévoir/définir le comportement du système) et \e{idiographique} (on décrit le sens des manipulations symboliques et des inscriptions produites par rapport aux normes, conventions, concepts du domaine).



% Au final, on construit un objet informatique en suivant une méthode de construction qui nous guide pour spécifier le comportement de l'ontologie. 

% RTO





\subsection*{Discussion sur la méthodologie suivie}
\addcontentsline{toc}{subsection}{Discussion}
% pourquoi on choisit celle de BB, et pourquoi on ne l'utilise pas globalement, mais seulement localement sur des patrons d'utilisation précis.



Notre étude des méthodes de construction d'ontologie ne vise pas à être exhaustive, pour cela nous renvoyons à \cite{Gomez-Perez2004}.
Il faut cependant remarquer que nous avons écarté les approches d'acquisition des connaissances à partir d'un corpus de texte, comme la méthode \g{Terminae} développée par \cite{Aussenac-Gilles2003}.
Dans cette approche, l'analyse linguistique d'un corpus permet de repérer des candidats-termes (\pc{Syntex}), d'effectuer des regroupements de contexte (\pc{Upery}) et d'identifier des relations (\pc{Yakwa}) afin d'accompagner la modélisation conceptuelle.
Or, dans notre contexte de travail, les documents professionnels qui décrivent en détails la production audiovisuelle sont rares (peu d'organisation ont les ressources de les produire) et constituent des ressources de valeur auquelles il est difficile d'avoir accès. 
De plus, la notion de document audiovisuel est largement absente des ouvrages généralistes, ce qui fonde précisément l'intérêt de notre travail de recherche.
\citeauthor{Uschold1996} proposent un point de vue global sur le processus de construction d'ontologie, qui reste un point de référence dans le domaine.
L'approche de \pc{Methontology} clarifie cependant le déroulement de la construction et s'efforce de l'intégrer dans une vision de gestion de projet assez exhaustive. 
Si cette vision est intéressante, il n'est pas forcément possible de l'appliquer telle quelle en pratique, du fait de contraintes spécifiques d'un projet.
Par ailleurs, \pc{Archonte} détaille une méthode de formalisation logique qui permet de positionner la conceptualisation sur le plan sémantique. 
De ce fait, elle précise la manière de formaliser une conceptualisation et peut s'intégrer de manière complémentaire aux autres méthodes exposées.

Dans notre cas, nous avons suivi une méthodologie \e{ad-hoc}, nécessaire pour suivre les contraintes d'un projet de recherche et développement impliquant de multiples partenaires.
Les aspects de gestion de projet (tels que décrits dans \pc{Methontology}) étaient donc largement assujetis à l'avancement du projet MediaMap.
De même, la spécification et l'acquisition des connaissances et la conceptualisation reposent principalement sur un dialogue avec des experts du domaine (dans le cas du projet MediaMap ce fût principalement les membres de la RTBF et de la VRT) dans le cadre de réunions générales et de séminaires plus focalisés.
Les autres aspects de la construction de l'ontologie ont été réalisés par les membres de l'équipe de recherche ICI, en s'inspirant des séquences de développement technique proposées par \pc{Methontology}.

Sur le plan sémantique, l'étendue de notre modélisation nous pousse à introduire des concepts de haut-niveau afin d'articuler diverses connaissances se rapportant à la production audiovisuelle (l'organisation du processus, ses contributeurs, ses résultats, et leur description par un vocabulaire professionnel et compréhensible par des amateurs).
Ces ontologies génériques de haut-niveau proposent des fondements théoriques important et des patrons de conception détaillés (au sens de \cite{Isaac2005}) pour modéliser certaines situations. 
On pense par exemple, au patron \ciel{Description \& Situations} de l'ontologie DOLCE \parcite{Gangemi2005}. 
Pour autant, la modélisation ne doit pas perdre de vue notre perspective applicative, nécessaire à l'adoption et la compréhension du modèle par des utilisateurs du domaine.
Ainsi, \citeauthor{Isaac2005} proposent d'adapter la structure de ces patrons aux besoins descriptifs de l'application.
Cela consiste à simplifier la structure des patrons en créant des \e{raccourcis relationnels}, quitte à faire disparaître les subtilités de représentation.
L'enjeu d'une telle approche est de faire co-exister la forme simplifiée (adaptée aux usages du domaine et aux besoins expressifs de l'application) et la forme de haut-niveau qui la rend réutilisable dans d'autres domaines.
De notre point de vue, notre modélisation se construit à partir de tels patrons de conception, qu'il s'agira ensuite d'articuler par des relations. 
Nous avons appliqué la méthodologie \pc{Archonte} pour formaliser la sémantique de ces patrons, qui sont ensuite regroupés pour former une seule ontologie.
En effet, nous considérons que le point important est de justifier la modélisation de ces parties et de leur mise en relation, plutôt que de la structure de l'ensemble.






% est-ce qu'on a adapté des patrons de conception générique aux notions du domaine, pour simplifier la modélisation (modélisation réduite) ? VOIR \cite{Isaac2005}









\section{Langages de représentations}\label{s:mods}

% \subsection{Langages de balisages }
\subsection{eXtended Markup Language (?)}
The eXtended Markup Language (XML) aims to give a hierarchical structure to  text in a machine-, yet human-readable way. It is widely used to store or exchange information as it also supports Unicode.
XML is formally defined as a Standard Generalized Markup Language's subset (SGML) designed to improve parser efficiency. Work on XML began in 1996 and it became a W3C Recommendation in early 1998.

\paragraph{Mark-up}
This is achieved by adding mark-up elements that are easily noticed as they begin with '<' and end with a '>'. Mark-up elements are used to enclose unicode text, and give thus a mean to identify them and possibly to process it. As its name indicates, it is said extensible because we can define our own mark-up elements and writes a line like that:

\begin{Verbatim}[fontsize=\small,formatcom=\color{black!70}]
opening mark-up element 			enclosing mark-up element
<structural_element>Some unicode text inside</structural_element>
\end{Verbatim}

Attributes can be defined for each mark-up element. 
For instance, the xml:lang attributes indicates the natural language used to write the enclosed text. The « 1812 Overture » full title can be written like that:
\begin{Verbatim}[fontsize=\small,formatcom=\color{black!70}]
<title xml:lang='ru'>Торжественная увертюра 1812 года, Toržestvennaja uvertjura 1812 goda</title>
<title xml:lang='fr'>Ouverture Solennelle, L'Année 1812, Op. 49</title>
\end{Verbatim}

\paragraph{Syntax}
XML does not only enclose text with mark-up elements. It also enables to imbricate mark-up elements in such a way that the elements conforms to a tree structure. 

\begin{Verbatim}[fontsize=\small,formatcom=\color{black!70}]
<element>
	<sub-element>Example of text</sub-element>
</element>
\end{Verbatim}

Other syntaxic rules have been defined to enables conforming parser to process XML file. Any file conspuing to these rules is said to be well-formed.

\paragraph{Schema}
Furthermore, if XML provides us with a syntax we also have the ability to makes purpose-specific XML-based mark-up languages – i.e. define constraints on structure, mark-up elements or even datatyping definition. Indeed, several schema languages exists and are used to encode documents or serialize text data according to a particular schema. XML files complying with a schema – i.e. conforming to the constraints defined in the schema – are said to be valid.

\paragraph{Namespace}
When creating schemas, ambiguity problems usually arise and namespace declaration can take care of that. Indeed, it provides an abstract container for XML elements and attributes and gives to their name a scope. As each namespace is identified by an URI, the ambiguity between identically named elements or attributes from differents namespace can be resolved. 

Therefore, I can declare my own « title » element and simultaniously use the « DC Terms » property title. We show here a complete example with xml heading:
\begin{Verbatim}[fontsize=\small,formatcom=\color{black!70}]
<?xml version="1.0" encoding="UTF-8"?>
<ex:musical_opus xmlns:dc="http://purl.org/dc/elements/1.1/"
    			 xmlns:ex="http://example.org">
	<dc:title>1812 Overture</dc:title>
    <ex:title>Festival Overture, The Year 1812</ex:title>
</ex:musical_opus>
\end{Verbatim}

In this example, dc:title and ex:title denotes two different elements for which we can give informally different meaning – ex:title indicating the complete name, dc:title a short version. 

\subsubsection*{XML Schema}
This W3C recommendation was published in 2001 and is one of several xml schema languages\footnote{We can cite, the old and very simple \gui{Document Type Defintion}(DTD) as well as the major rival of XML Schema, namely \gui{Relax NG}.}. It is often called XSD in reference to its files suffix – '.xsd'.

XSD can define imbrication, quantification and naming rules for xml elements and attributes – in order to enable vocabulary and content model validation. XSD also supports namespace so parts of other schemas can be included or imported. 

\paragraph{Datatypes}
But one of the main and most criticized characteristic of XSD is DataType validation. It can be applied to elements or attributes to constraint their content.  DataType definition must use XSD primitive or derived datatypes – see the scheme for a detailled hierarchy. This dependence upon specific datatypes is the source of many criticism. 

Derived datatypes can be built by restriction – of the permitted values set –, list – declaration of values –, or union – between several types. 
As an complete example, we define a XSD schema describing « MusicalOpus » as a list of XML elements named:
\begin{liste}
	\item Title: title of the musical opus
	\item Extent: length or duration – as a string
	\item Composer: name of the composer
	\item composed: date of composition of the opus – only the year
	\item Performer: name of the performer
	\item performed: date of performance – only the year
	\item conductedby: name of the person who conducted the performer, this attribute is declared optionnal – thanks to the minOccurs attribute
	\item ComposerNationality: picked from a list of value we enumerate in the scheme
\end{liste}

Here is the resulting XSD scheme:
\begin{Verbatim}[fontsize=\small,formatcom=\color{black!70}]
<?xml version="1.0" encoding="utf-8"?> 
<xs:schema elementFormDefault="qualified"   xmlns:xs="http://www.w3.org/2001/XMLSchema"> 
 <xs:element name="MusicalOpus"> 
   <xs:complexType> 
     <xs:sequence> 
       <xs:element name="Title" type="xs:string" /> 
       <xs:element name="Extent" type="xs:string" /> 
       <xs:element name="Composer" type="xs:string" /> 
       <xs:element name="composed" type="xs:gYear" /> 
       <xs:element name="Performer" type="xs:string" /> 
       <xs:element name="performed" type="xs:gYear"/> 
       <xs:element name="conductedBy" type="xs:string" minOccurs="0"/>  
       <xs:element name="ComposerNationality"> 
         <xs:simpleType> 
           <xs:restriction base="xs:string"> 
             <xs:enumeration value="FR" /> 
             <xs:enumeration value="DE" /> 
             <xs:enumeration value="RU" /> 
             <xs:enumeration value="UK" /> 
             <xs:enumeration value="US" /> 
           </xs:restriction> 
         </xs:simpleType> 
       </xs:element> 
     </xs:sequence> 
   </xs:complexType> 
 </xs:element> 
</xs:schema> 
\end{Verbatim}


And we provide a XML file which states that it conforms to the previous XSD scheme through a xsi:noNamespaceSchemaLocation attribute:
\begin{Verbatim}[fontsize=\small,formatcom=\color{black!70}]
<?xml version="1.0" encoding="utf-8"?> 
<MusicalOpus xmlns:xsi="http://www.w3.org/2001/XMLSchema-instance" 
         xsi:noNamespaceSchemaLocation="MusicalOpus.xsd"> 
  <Title>Festival Overture, The Year 1812</Title> 
  <Extent>14:19</Extent> 
  <Composer>Pyotr Ilyich Tchaikovsky</Composer> 
  <composed>1880</composed> 
  <Performer>Minneapolis Symphony Orchestra</Performer> 
  <performed>1954</performed> 
  <conductedBy>Antal Dorati</conductedBy> 
  <conductedBy>Harold Lawrence</conductedBy> 
  <ComposerNationality>RU</ComposerNationality> 
</MusicalOpus> 
\end{Verbatim}






















\subsection{Ontologie et représentation des connaissances}
\subsubsection*{Resource Description Framework}
\addcontentsline{toc}{subsection}{Resource Description Framework}
The Resource  Description Framework (RDF) is an abstract model which is part of the W3C recommendations for the Semantic Web. 
Let's just bring back to mind how \pc{Tim Berners-Lee} defined it to set RDF back into its context of creation: 

\ciel{
The Semantic Web is not a separate Web but an extension of the current one, in which information is given well-defined meaning, better enabling computers and people to work in cooperation.}

Indeed, RDF aims to describe and link resources – and no more web pages – in a simple, all-purpose and machine-readable way. 
The focus on software agents led to choose a formal semantic and provable inference. 
Thus, such descriptions will foremost benefits to software agents which we'll be able to exploit, process and search into this web of linked data. 

\paragraph{Statements / Triples}
The description consists in making statements that describes or models web resources. The statements are formed as subject-predicate-object sentence called triples that can be represented as a graph – one node/vertex for the subject and the object, and a directed and labeled edge for the predicate. 
\begin{Verbatim}[fontsize=\small,formatcom=\color{black!70}]
Subject		Predicate	    Object	
Tchaikovsky ---- is the Composer of ----> 1812 overture
\end{Verbatim}
Thereby, a set of statements constitute a multigraph, that is a graph in which node/vertex can have multiple ingoing or outgoing edges resulting possibly in loops. 
However, unlike an hypertext the rdf multigraph has labelled edges – also called properties -- connecting a resource with another resource or a literal value.

\paragraph{URI, datatype and literals}
RDF is said to have an URI-based vocabulary, meaning that resources and properties and typed literal are identified by URI reference. 
Indeed, unlike plain literals, typed literals are literals combined with a datatype URI.
Datatypes in RDF are compatible with XML Schema datatypes --which can thus be used as there are-- but any datatype definition conforming to RDF constraints may be used.

Let's rewrite our previous example with Tchaikovsky:
\begin{Verbatim}[fontsize=\small,formatcom=\color{black!70}]
Subject:  http://example.org/Tchaikovsky
Property: http://example.org/Composer
Object:   http://example.org/1812_Overture
\end{Verbatim}

And now an example with a gYear XSD datatype:
\begin{Verbatim}[fontsize=\small,formatcom=\color{black!70}]
Subject:  http://example.org/1812_Overture
Property: http://example.org/composed
Object:   '1880' xsd:gYear
\end{Verbatim}
The following examples make use of the namespace ability to ease the reading. We define here the prefix used for our example and for specific rdf elements:
\begin{liste}
	\item \cd{Example prefix: 'ex:'	Example URI: 'http://example.org'}
	\item \cd{RDF prefix: 'rdf:'	RDF URI:'http://www.w3.org/1999/02/22-rdf-syntax-ns\#'}
\end{liste}
Our previous example could then be written like that:
\cd{ex:1812\_Overture  -- ex:composed -->  '1880' xsd:gYe}


\paragraph{Structured value}
When we want to define a resource composed in fact of several other resources or literals, 
RDF makes us declare an intermediate node – only to conform to the RDF syntax. 
This kind of nodes are called blank nodes because they don't really need to be referenced by an URI. 
As they are only a product of the RDF syntax and don't represent anything in particular they can stay anonymous.

Indeed, when we declare the size of a digital file such as an audio file, we may want to specify the unit. 
Thus considering that the following statement is not sufficient:
\cd{ex:audio\_file\_01  ex:size '33,7'}

So we need to define one blank node and use a particular RDF property called value:
\begin{Verbatim}[fontsize=\small,formatcom=\color{black!70}]
ex:audio_file_01 	-- ex:size -->	_blank_node_01
_blank_node_01	-- rdf:value -->	'33,7'
_blank_node_01	-- ex:unit -->	ex:Mb xsd:decimal
\end{Verbatim}

The value property is not the only one to be used in such context. 
The type property states the nature of a blank node. 
An example will be shown in the next paragraph.

\paragraph{Grouping resources}
% 7.2.3.a  containers
RDF provides two different ways to group things. The first one is called Containers comprises three predefined types which points out some members of the group they define:
\begin{liste}
	\item a bag define a non-ordered group that may include duplicate members. 
	\item a seq define an ordered group that may include duplicate members.
	\item a alt define a group of alternatives resources or literals. 
\end{liste}

Members may be resources or literal, their membership is stated by declaring them as list item (li).
We show here example connected to our information sample. First, a bag of \gui{Performer} having played the \gui{1812 overture}, then a seq describing the content of the audio CD. 
Eventually, an alt group of russian, french and english version of \gui{Tchaikovsky}'s full name.

1) Bag of \gui{Performer}
\begin{Verbatim}[fontsize=\small,formatcom=\color{black!70}]
ex:1812_Overture	-- ex:Performer -->	_blank_node_02
_blank_node_02	-- rdf:type -->		rdf:bag
_blank_node_02	-- rdf:li -->		ex:Minneapolis_S_O
_blank_node_02	-- rdf:li -->		ex:St_Petersburg_Ph
\end{Verbatim}
2) Seq of audio tracks from \gui{Tchaikovsky: 1812 Festival Overture; Capriccio Italien; Beethoven:  Wellington's Victory} CD:
\begin{Verbatim}[fontsize=\small,formatcom=\color{black!70}]
ex:1812_Op49_CD	-- ex:TrackList -->	_blank_node_03
_blank_node_03	-- rdf:type -->		rdf:seq
_blank_node_03	-- rdf:li -->		ex:Op49_audio
_blank_node_03	-- rdf:li -->		ex:Op49_commentary
_blank_node_03	-- rdf:li -->		ex:Capriccio_Italien_audio
_blank_node_03	-- rdf:li -->		ex:Wellington_Op91_audio01
_blank_node_03	-- rdf:li -->		ex:Wellington_Op91_audio02
_blank_node_03	-- rdf:li -->		ex:Op91_commentary
\end{Verbatim}
3) Alt names of \gui{Tchaikovsky}
\begin{Verbatim}[fontsize=\small,formatcom=\color{black!70}]
ex:Tchaikovsky		-- ex:Name -->		_blank_node_04
_blank_node_04	-- rdf:type -->		rdf:alt
_blank_node_04	-- rdf:li -->		'Pyotr Ilyich Tchaikovsky' @en
_blank_node_04	-- rdf:li -->		'Piotr Ilitch Tchaïkovski' @fr
_blank_node_04	-- rdf:li -->		'Пётр Ильич Чайкoвский' @ru
\end{Verbatim}

% 7.2.3.b  collection
Unlike containers, collections can make a closed group definition, that is list all members member of the collection. With a container you can not state that there  is no other member than those you give in your definition. 
Moreover, list have predefined properties to identify the first item, the rest of the list and the end of it – a nil property. 

A rewriting of our tracklist example as a list would be:
\begin{Verbatim}[fontsize=\small,formatcom=\color{black!70}]
ex:1812_Op49_CD	-- ex:TrackList -->	_blank_node_03
_blank_node_03	-- rdf:type -->		rdf:list
_blank_node_03	-- rdf:first -->		ex:Op49_audio
_blank_node_03	-- rdf:rest -->		ex:Op49_commentary
_blank_node_03	-- rdf:rest -->		ex:Capriccio_Italien_audio
_blank_node_03	-- rdf:rest -->		ex:Wellington_Op91_audio01
_blank_node_03	-- rdf:rest -->		ex:Wellington_Op91_audio02
_blank_node_03	-- rdf:rest -->		ex:Op91_commentary
_blank_node_03	-- rdf:rest -->		rdf:nil
\end{Verbatim}

\paragraph{Serialization format}
As an abstract model, RDF statements can be serialized or represented in a variety of form. The most widely known is the \gui{RDF XML} but the W3C also introduced the more readable \gui{Notation} (N3) based on tabular spacing. This last form is closely related to the \gui{Turtle} and \gui{N-Triples} formats.





\subsubsection*{RDF Schema}
\addcontentsline{toc}{subsection}{RDF Schema}
RDFS is the result of 6 years of work from the W3C consortium – from the 1998's first version to the 2004's final recommendation. 
It is formally introduced as a vocabulary description language intended to structure RDF resources. 
Indeed, RDFS presents mechanisms for describing classes of resources, associated properties and also the relationships between properties and other resources. 

These mechanisms are in fact itselves classes and properties that enables us to describe vocabulary or basic ontology. 
Like RDF, RDFS follows a minimalistic approach allowing a relatively basic expressiveness compared to the Ontology Web Language (OWL) – which is built upon it.

\paragraph{RDF(S) Class}
Classes are resources – identified by an URI, described by properties – associated with a set of resources called the class extension. 
Resources in the class extension  are called instances – the rdf:type property may be used to state a resource as an instance of a class. 
Note that a class extension can cointain the class itself as instance – this is why rdfs:Class can be defined as a rdfs:Class in the table below.
SubClasses may be defined by the SubClassOf property. In this case, their extension pertains necessarily to any upper class extension. 
If we state « X » to be the class of all « Opus » and « Y » a sub-class of « X » containing all the « MusicalOpus » ; then every instance of « Y » will be an instance of « X ». 

% \paragraph{Class List}
\begin{table}[ht!]
   \begin{center}
		\begin{tabularx}{400pt}{|l|X|}
		   \hline
		Class name & Comment\\ \hline\hline
		rdfs:Resource & The class resource, everything.\\ \hline
		rdfs:Literal & The class of literal values, e.g. textual strings and integers.\\ \hline
		rdf:XMLLiteral & The class of XML literals values.\\ \hline
		rdfs:Class & The class of classes.\\ \hline
		rdf:Property & The class of RDF properties.\\ \hline
		rdfs:Datatype & The class of RDF datatypes.\\ \hline
		rdf:Statement & The class of RDF statements.\\ \hline
		rdf:Bag & The class of unordered containers.\\ \hline
		rdf:Seq & The class of ordered containers.\\ \hline
		rdf:Alt & The class of containers of alternatives.\\ \hline
		rdfs:Container & The class of RDF containers.\\ \hline
		rdfs:ContainerMembershipProperty & The class of container membership properties, rdf:\_1, rdf:\_2, \dots, all of which are sub-properties of 'member'.\\ \hline
		rdf:List & The class of RDF Lists.\\ \hline
		\end{tabularx}
		\caption{Class list \label{tab:rdfs-classes}}
   \end{center}
\end{table}

\paragraph{RDF(S) Properties}
First of all, let's recall that all properties are instances of the rdf:Property class. A property link a pair of resources, one of them as a subject and the other one as the object. 
RDFS introduces two properties to restrict which can of resources may be linked together. rdfs:domain constraints the subject resource to be an instance of a given class whereas rdfs:range constraints the objet resource likewise. 
SubProperty relationships – stated with rdfs:subPropertyOf – induces that all pairs of resource linked are also linked by the upper property. 
Thus, having a « Contributor » property and a « Performer » sub-property we can state that:
\begin{Verbatim}[fontsize=\small,formatcom=\color{black!70}]
ex:1812\_Overture
ex:Performer
ex:Minneapolis\_SO
\end{Verbatim}
and this would entail:
\begin{Verbatim}[fontsize=\small,formatcom=\color{black!70}]
ex:1812\_Overture
ex:Contributor
ex:Minneapolis\_SO
\end{Verbatim}

% 7.3.2.a  Property List
\begin{table}[ht!]
   \begin{center}
		\begin{tabularx}{450pt}{|l|X|c|c|}
		   \hline
rdf:type & The subject is an instance of a class. & rdfs:Resource & rdfs:Class \\ \hline
rdfs:subClassOf & The subject is a subclass of a class. & rdfs:Class & rdfs:Class \\ \hline
rdfs:subPropertyOf & The subject is a subproperty of a property. & rdf:Property &rdf:Property \\ \hline
rdfs:domain & A domain of the subject property. & rdf:Property & rdfs:Class\\ \hline
rdfs:range & A range of the subject property. & rdf:Property & rdfs:Class\\ \hline
rdfs:label & A human-readable name for the subject. & rdfs:Resource & rdfs:Literal \\ \hline
rdfs:comment & A description of the subject resource. & rdfs:Resource & rdfs:Literal\\ \hline
rdfs:member & A member of the subject resource. & rdfs:Resource & rdfs:Resource\\ \hline
rdf:first & The first item in the subject RDF list. & rdf:List & rdfs:Resource\\ \hline
rdf:rest & The rest of the subject RDF list after the first item. & rdf:List & rdf:List\\ \hline
rdfs:seeAlso & Further information about the subject resource. & rdfs:Resource &rdfs:Resource\\ \hline
rdfs:isDefinedBy & The definition of the subject resource. & rdfs:Resource & rdfs:Resource\\ \hline
rdf:value & Idiomatic property used for structured values (see the RDF Primer for an example of its usage). & rdfs:Resource & rdfs:Resource\\ \hline
rdf:subject & The subject of the subject RDF statement. & rdf:Statement & rdfs:Resource\\ \hline
rdf:predicate & The predicate of the subject RDF statement. & rdf:Statement & rdfs:Resource\\ \hline
rdf:object & The object of the subject RDF statement. & rdf:Statement & rdfs:Resource\\ \hline
		\end{tabularx}
		\caption{Class list \label{tab:rdfs-classes}}
   \end{center}
\end{table}

Eventually, observe these four properties that appear more like annotation than property defining resource: \cd{rdfs:label, rdfs:comment, rdfs:seeAlso, rdfs:isDefinedBy}.





\subsubsection*{Ontology Web Language}
\addcontentsline{toc}{subsection}{Ontology Web Language}
The Ontology Web Language (OWL) is a knowledge representation language intended – as its name states – to build ontology in a web environnement. 
OWL relies on RDF and XML syntax and defines its constructs as extension or subset of RDF/RDFS classes and properties. 
Nevertheless, OWL provides in addition comparaison and cardinality constraints on classes or properties. 
These constructs enables to model domain specific ontology while bringing along generic reasoning tool. 

The W3C created a working group in 2001 and documents became recommendations in 2004.
However, OWL was a revision of earlier work called DAML+OIL initiated conjointly by the « Defense Advanced Research Projects Agency » (DARPA) and the European Union's « Information Society Technologies » (IST) project. 

\paragraph{OWL species}
OWL defines in fact three sub-languages with different level of expressiveness and thus computational efficiency:
\begin{liste}
	\item « OWL Lite » is the simplest, less expressive sub-language. It supports 0..1 cardinality constraints and thus are intended for thesauri and taxonomies migration project.

	\item « OWL DL » gives full expressivity while ensuring computational completeness – all entailments are garanteed to be computed – and decidability – all computations will finish in finite times. Full expressivity means it includes all language constructs but to ensure the rest it needs to restrict their use by some conditions. 

	\item « OWL Full » gives full expressivity without conditions. For instance, a major difference with « OWL DL » is that all resources can be considered as individuals – even Class and Property. There are strong equivalence between « OWL Full » and RDF – RDFS. This comes nevertheless with no computational garantees. 
\end{liste}
% Each level is a sub-level from its predecessor, that is every legal ontology or valid conclusion expressed in « OWL Lite » is a legal ontology or respectively a valid conclusion in « OWL DL » and so on between « OWL DL » and « OWL Full ». 

% 7.4.2  Class
% OWL defines its class like RDFS except that only OWL Full can state that a class is an instance of another class. OWL Lite and DL don't allow a class to be considered at the same time as an individual – thus also as a member of a class extension. 

% 7.4.2.a  Class descriptions
% OWL supports six different kind of class descriptions: 
% 1. a simple class declaration – with a URI reference
% 2. an exhaustive enumeration of the class extension
% 3. definition of a class as a subset of another class depending on property restrictions
% 4. an intersection of several classes descriptions
% 5. an union of several classes descriptions
% 6. a complement of a class description

% 1. Class declaration can be done as follow:
% <owl:Class rdf:ID='Composer' />

% 2. Or if we want to define directly its extension like RDFS Containers, we can write: 
% <owl:Class rdf:ID='Nationality'>
% 	<owl:oneOf rdf:parseType='Collection'>
% 		<owl:Thing rdf:about='ex:French' />
% 		<owl:Thing rdf:about='ex:Russian' />
% 		<owl:Thing rdf:about='ex:English' />
% 	</owl:oneOf>
% </owl:Class>

% 3. We may also use more complex property restrictions, including value or  cardinality restrictions. For this purpose we have a set of properties:

% Value restrictions
% owl:allValuesFrom, owl:someValuesFrom  – with the values being either a class or a datatype but OWL Lite only supports class value.

% owl:hasValue – the value has to be either an individual or a data value. This property is not included in OWL Lite. 

% Cardinality constraints
% owl:maxCardinality, owl:minCardinality, owl:cardinality – the value has to be related to a XML Schema datatype.

% 4,5,6. These constructs are similar to AND, OR, NOT operators acting on classes. Only owl:intersectionOf can be used in some way in OWL Lite, whereas owl:unionOf and owl:complementOf are not included.

% 7.4.2.b  Class axioms
% Classes may be described with the previous properties, class axioms are the three properties that have owl:Class for domain and range: 
% owl:subClassOf enables specialisation to be described. The sub-class' extension set is thus stated as a subset of the class extension set. 

% owl:equivalentClass declares class extension equivalence between two classes descriptions.

% owl:disjointWith establishes that two classes extensions of two classes descriptions have no member in common.

% Note that all owl:Class are sub-classes of the owl:Thing superclass. 

% 7.4.3  Property
% OWL defines four kind of Property that must be mutually disjoint when using OWL DL:

% owl:DatatypeProperty links instance with literal values.
% owl:ObjectProperty define relations between instances.
% owl:AnnotationProperty are intended for human reader. In OWL DL, it is impossible to define restriction or sub-property for Annotation Properties and information will not be taken into account by reasoners.
% owl:OntologyProperty are used for importing ontology and make statements about versionning information. In OWL DL, the same constraints hold as those specified for Annotation Properties. 

% 7.4.3.a  Property declaration
% We take as an example the definition of creation relationships on two different level of specialisation. First, a « createdBy » relation between a « Person » and an « Opus ». Second, a music related « composedBy » relation, involving a « Composer » and a « MusicalOpus ».

% Class declaration
% <owl:Class rdf:ID='Person' />
% <owl:Class rdf:ID='Composer'>
% 	<rdfs:subClassOf rdf:resource='Person' />
% </owl:Class>
% <owl:Class rdf:ID='Opus' />
% <owl:Class rdf:ID='MusicalOpus'>
% 	<rdfs:subClassOf rdf:resource='Opus' />
% </ow:Class>

% Property declaration
% <owl:ObjectProperty rdf:ID='createdBy'>
% 	<rdfs:domain rdf:resource='Opus' />
% 	<rdfs:range rdf:resource='Person' />
% </owl:ObjectProperty>
% <owl:ObjectProperty rdf:ID='composedBy'>
% 	<rdfs:domain rdf:resource='MusicalOpus' />
% 	<rdfs:range rdf:resource='Composer' />
% 	<rdfs:subPropertyOf rdf:resource='createdBy />
% </owl:ObjectProperty>

% Note the use of rdfs:domain, rdfs:range, rdfs:subPropertyOf and rdfs:subClassOf properties within OWL statements.

% Now, if we want to take advantage of Property restrictions to define our « Composer » class, we can state for instance that « Composer » individuals are precisely those « Person » individuals who have « composed » at least one « MusicalOpus ». This lead us to write the following declaration: 

% <owl:Class rdf:ID='Composer'>
% 	<rdfs:subClassOf>
% 		<owl:intersectionOf rdf:parseType='Collection'>
% 			<owl:Class rdf:ID='Person' />
% 			<owl:Restriction>
% 				<owl:onProperty rdf:resource='composedBy' />
% 				<owl:minCardinality rdf:datatype='\&xsd;nonNegativeInteger'>1</owl:minCardinality>
% 			</owl:Restriction>
% 		</owl:intersectionOf >
% 	</rdfs:subClassOf>	
% </owl:Class>

% Note that the intersectionOf property requires a list of Classes declarations which can be given either by owl:Class or owl:Restriction – which is defined as a sub-class of owl:Class. 

% 7.4.3.b  Property characteristics
% Properties can also be defined as:
% transitive: P(x,y) and P(y,z) implies P(x,z). For instance, if x is located in y and y is located in z, then x is located in z. 
% symetric: P(x,y) if and only if P(y,x). If x is next to y, then y must be next to x. 
% functionnal: P(x,y) and P(x,z) implies y = z. The property has only one value for each individual it applies to. 
% inverseOf: P1(x,y) if and only if P2(y,x). If x is the « fatherOf » y, then y is the « childOf » x, thus making « fatherOf » the inverseOf « childOf » property. 
% inverseFunctional: P(y,x) and P(z,x) implies y = z.


% 7.4.4  Individual
% Ontology is not only about Classes and Properties, it enables us to state some facts about individuals. In order to describe them, we state them as Class instance, make use of Property or assert facts about their individuality. 

% Let's use our previous ontology description to decribe « Tchaikovsky » and the « 1812 overture ».

% <Person rdf:ID='Tchaikovsky' />
% <MusicalOpus rdf:ID='1812_Overture'>
% 	<composedBy rd:resource='Tchaikovsky' />
% </MusicalOpus>

% From this description, we can entail that:
% « Tchaikovsky » is not only a « Person », it is also a « Composer » because he composed the « 1812 overture ». As he composed it, we can also say that he has « created » it. 

% As OWL is a web language, it has rejected the « unique name » assumption and hence need to deal with identity uniqueness – and thus URI reference. In order to do so, three constructs have been defined:
% owl:sameAs states that two URI reference refer in fact to the same individual, meaning we can merge their definition and the related conclusions. 
% owl:differentFrom declares that two URI reference refer to different individuals. 
% owl:AllDifferent provides an idiom for stating that a list of individuals are all different. 

% Now, to demonstrate the use of sameAs property and to refer to our information sample, we define several « Tchaikovsky » individuals with different spelling. Then, we declare several MusicalOpus, « Cappricio italien » and « Ромео и Джульетта » which is the russian spelling for « Romeo and Juliet ».

% <Person rdf:ID='Tchaïkovski' xml:lang='fr' />
% <Person rdf:ID='Чайкoвский' xml:lang='ru' />

% <MusicalOpus rdf:ID='Capriccio_italien' xml:lang='fr'>
% 	<composedBy rd:resource='Tchaïkovski' />
% </MusicalOpus>
% <MusicalOpus rdf:ID='Ромео и Джульетта' xml:lang='ru'>
% 	<composedBy rd:resource='Чайкoвский' />
% </MusicalOpus>

% If we change the first statements by:
% <Person rdf:ID='Tchaïkovski' xml:lang='fr'>
% 	<owl:sameAs rdf:resource='Tchaikovsky'/>
% </Person>
% <Person rdf:ID='Чайкoвский' xml:lang='ru'>
% 	<owl:sameAs rdf:resource='Tchaikovsky'/>
% </Person>

% We can entail that there are the same person and that he has composed all three « MusicalOpus » described. 
















\subsection{Thésaurus et vocabulaires structurés}
\subsubsection{Besoins en modélisation}
%%%%%%%%%%%%%%%%%%%%%%%%%%%%%%%%%%%%%%%% à revoir
%La mise en place d'une telle application nécessite de représenter le vocabulaire de la réalisation audiovisuelle dans toutes ses variations possibles et de le documenter suffisamment afin de le rendre compréhensible pour des novices. 
Cet objectif nous amène à considérer la construction d'une ressource termino-ontologique.
L'ontologie permet de représenter les concepts partagés par les professionels de la réalisation audiovisuelle et la terminologie permet de capturer les différentes formes d'expression associées à ces concepts. 

La spécificité de notre problématique est de considérer la collaboration de communautés hétérogènes par leur degré de compréhension des concepts ou leur utilisation de la terminologie. 
Ceci nous amène à envisager la terminologie comme un moyen d'associer à des éléments ontologiques (concept, relation, instances) une chaîne lexicale ou des ressources média.
Chaque chaîne ou ressource s'adresse en particulier à une communauté dont les membres partagent une capacité d'interprétation commune. 
Il n'existe donc plus une terminologie de référence par langue, mais des terminologies pour chaque communauté d'utilisateurs. 
On remarquera que notre acception de la terminologie sert bien à normaliser les pratiques linguistiques entre les membres d'une même organisation. 
En plus de cela, elle permet de fixer la manière de s'adresser à d'autres communautés.

Par ailleurs, les types de réalisations sont divers et nécessitent des concepts spécifiques pour être décrits. Une fiction se structure en séquences et en scènes alors que les documentaires ou magazines d'information se composent de sujets. 
La variabilité des types de contenu à filmer implique donc de pouvoir étendre le fond conceptuel initial pour représenter de nouveaux usages. 
De la même manière, la collaboration avec de nouveaux partenaires nécessite de pouvoir ajouter de nouvelles terminologies au fond conceptuel existant. 
Ontologie et terminologie doivent se gérer de manière indépendante. 
À partir de ces besoins, nous définissons maintenant les exigences en terme de modélisation. 

Nos besoins en modélisation peuvent être exprimés par les assertions suivantes:
\begin{enumerate}
	\item[(A1)] la variabilité des pratiques linguistiques des organisations et des communautés implique d'associer plusieurs termes à un même concept. Il n'y a pas de choix des termes préférés par une communauté mais une \textit{correspondance} entre les termes d'une ou plusieurs communautés, quels que soient la langue et le code d'écriture utilisé.
	
	\item[(A2)] la variabilité de compréhension des communautés implique d'associer des explications (chaîne lexicale) ou des illustrations (ressource média) aux concepts afin d'en enrichir la \textit{documentation}. 
	
	\item[(A3)] la variabilité des cas de collaboration implique de pouvoir étendre la conceptualisation initiale ou la terminologie pour s'adapter à de nouvelles pratiques ou de nouvelles communautés. Cela implique une gestion et une \textit{évolution} indépendante de l'ontologie et de la terminologie. 
\end{enumerate}



\subsubsection*{Simple Knowledge Organization System}
\addcontentsline{toc}{subsection}{Simple Knowledge Organization System}
\e{
SKOS est un langage de représentation de vocabulaires structurés et de thésaurus qui repose sur RDF et OWL. 
Son objectif est de représenter tout type de SOC en vue de le publier sur le web de données,  liées et ouvertes (Linked Open Data). 
En témoigne les travaux et méthodes de conversion proposés par \cite{Summers2008} ou \cite{VanAssem2006} et les applications de gestion de thésaurus développés autour de SKOS, comme celle de \cite{Schandl2010}.
On notera que l'objectif de publication semble pousser vers une représentation minimale mais extensible de SOC déjà construits.
}

% \subsubsection*{SKOS}
\paragraph{Concept et Etiquette}
SKOS centre son modèle sur les concepts (\cd{skos:Concept}) et considère les étiquettes comme des propriétés de ces derniers. 
On distingue trois types d'étiquettes: 
\begin{itemize} 
	\item les étiquettes préférées (\cd{skos:prefLabel}) qui sont uniques par langue et servent de référence.
	\item les étiquettes alternatives (\cd{skos:altLabel}) qui servent de synonymes pour l'étiquette de référence. 
	\item les étiquettes cachées (\cd{skos:hiddenLabel}) qui servent à la récupération d'erreurs de frappes les plus courantes. 
\end{itemize}
Chacune de ces propriétés est formalisée comme une instance de \cd{owl:Anno\-tationProperty}, ce qui permet de l'attacher dans les faits à tout type d'éléments ontologiques, et pas seulement à des concepts. 
Les valeurs lexicales portées par ces étiquettes sont formalisées comme des \cd{rdf:PlainLiteral}, ce qui permet de spécifier la langue et l'alphabet utilisés. 
Par exemple, la chaîne "higashi"@ja-Latn correspond à un mot japonais écrit avec l'alphabet latin.

\paragraph{Documentation}
Différentes notes de documentation existent afin de :
\begin{itemize}
	\item définir un concept (\cd{skos:definition}), expliciter son contexte d'usage (\cd{skos:scopeNote}) ou donner des exemples (\cd{skos:example})
	\item spécifier l'historique de sa signification (\cd{skos:historyNote}), les changements effectués (\cd{skos:changeNote}) ou à faire (\cd{skos:editorialNote})
\end{itemize}
L'ensemble de ces notes est défini comme une spécialisation de \cd{skos:note}, formalisé comme une \cd{owl:annotationProperty}. 
De cette manière, les notes peuvent servir à porter de la documentation écrite (comme les étiquettes), pointer vers des ressources RDF ou des documents identifiés par une URI. 
Cela permet ainsi de prévoir l'extension de ces notes à des besoins plus spécifiques.


\paragraph{Groupes et relations entre Concepts}
Les concepts peuvent être regroupés dans des schémas conceptuels (\cd{skos:\-ConceptScheme} et relation \cd{skos:inScheme}) et structurés par différentes relations:
\begin{itemize}
	\item des relations de structuration hiérarchiques (\cd{skos:broader}, \cd{skos:na\-rrower}) ou associatives (\cd{skos:related})
	\item des relations de correspondances entre concepts de schémas différents, soit une relation d'équivalence exacte (\cd{skos:exactMatch}), une équivalence approximative (\cd{skos:closeMatch}), des relations hiérarchiques (\cd{skos:broadMatch}, \cd{skos:narrowMatch}) ou associative \cd{skos:rela\-ted\-Match}).
\end{itemize}


\paragraph{SKOS-XL}
SKOS-XL est une extension de SKOS développée courant 2008 pour proposer une modélisation alternative au vocabulaire de base et favoriser des extensions plus fines. 
Dans SKOS-XL les termes ne sont plus portés par les concepts mais deviennent des éléments à part entière (\cd{skosxl:Label}). 

Les relations d'attachement entre termes et concepts sont analogues aux attributs de SKOS mais les formalisent comme des instances de \cd{owl:objectProperty}. 
Le domaine de ces relations n'est pas défini ce qui permet de les associer à n'importe quelle ressource RDF, et donc en particulier aux concepts SKOS mais également à des ConceptScheme. 
Si cette dernière possibilité permet de créer des groupes de termes, elle introduit une confusion sur la sémantique des ConceptScheme (groupe de concepts, de termes, de triplets ?). 
Les étiquettes portent une seule chaîne lexicale avec les mêmes possibilités que pour SKOS grâce à l'attribut \cd{skos:literalForm}. 
L'indépendance des étiquettes permet également de spécifier des relations entre eux comme la synonymie, la traduction, etc. \cite{Pastor2009a}. 
Cette possibilité est ouverte par la relation générique \cd{skosxl:labelRelation}. 


Dans SKOS, la gestion de plusieurs jargons métiers dans une même conceptualisation [A1] est rendue difficile par la caractérisation simple des termes par la langue. 
Ainsi, même si on accroche plusieurs étiquettes au même concept, on ne sait pas les sélectionner pour les présenter à l'une ou l'autre communauté. 
Cela implique un dédoublement des concepts et donc des schémas conceptuels nécessaires (un par communauté).%, voir figure \ref{fig:skos}. 
%implique une représentation plus fine par rapport à SKOS ce qui 
Avec le découplage terme-concept de SKOS-XL, on peut gérer terme et concept de manière séparés sans pour autant avoir de primitives spécifiques pour regrouper les termes par jargon ou code d'écriture, voir figure \ref{fig:skosxl}. 
Une solution consisterait à regrouper les termes dans des ConceptScheme (SKOS-XL l'autorise).
Les ConceptScheme serviraient alors à la fois à regrouper les concepts (AV-Scheme) et les termes spécifiques aux organisations (RTBF-Scheme, VRT-Scheme). 
Cependant, si cette modélisation permet de gérer vocabulaire métier et conceptualisation de manière séparés c'est au prix d'un flottement sur la sémantique de ConceptScheme. 
Le support d'un nouveau jargon peut donc se faire sans toucher à la conceptualisation [A3] grâce à la permissivité de l'extension SKOS-XL. 
L'extension ou la mise en correspondance de la conceptualisation est facilitée par les relations sémantiques entre concepts.


\subsubsection*{ISO 25964-1}
\addcontentsline{toc}{subsection}{ISO 25964-1}
Cette norme propose une modélisation terme-concept similaire à SKOS-XL mais se concentre sur la représentation des thésaurus. 
Elle se fonde sur des modèles pré-existants, le méta-modèle \cite{Vandenbussche2009} ainsi que la norme BS 8723. 
L'originalité par rapport à SKOS est de considérer la composition de termes ou de concepts et d'enrichir la description des éléments du modèle par des attributs Dublin Core.

Les termes se distinguent entre termes préférés simples ou composés et termes non préférés simples. 
La caractérisation des termes porte également sur l'appartenance à une langue à laquelle s'ajoutent des attributs de dates, une définition ainsi que des notes d'historique et de révision. 
Des relations sémantiques entre termes sont également considérées en particulier l'équivalence composée, la synonymie, l'abréviation, l'acronyme, etc. 
Le thésaurus est considéré comme l'élément central décrit par l'ensemble des quinze attributs originaux du Dublin Core ainsi que des notes d'historique pour la maintenance. 
Sur les questions de documentation et de groupement de concepts, il existe une similarité importante avec les primitives de SKOS (note, groupe de concepts, etc.).\\

%\textbf{Discussion}\\
Les apports de la norme ISO 25964 par rapport à SKOS concernent davantage les pratiques de création et de maintenance de thésaurus que la gestion des jargons métiers [A1] et l'illustration de concepts par des ressources média [A2]. 
Ainsi, les manques par rapport à nos besoins sont similaires. L'attention portée sur les détails de description de chaque élément du modèle est tout à fait convaincante. 
L'écart avec l'aspect épuré et synthétique de SKOS s'explique certainement par l'écart entre leurs objectifs. 
Alors que SKOS vise la publication de tout type de SOC, l'ISO 25946 se concentre sur la construction et l'évolution des seuls thésaurus. 
La norme ne s'est pas encore attelée aux questions d'interopérabilité et de correspondance avec d'autres vocabulaires. Ce travail est en cours et sera dévoilé dans la seconde partie de la norme (ISO 25946-2). 
De notre point de vue, il manque toujours un moyen de regrouper des termes indépendamment des concepts pour ajouter des jargons à une conceptualisation existante [A3].


\subsection*{Bilan}
L'étude des standards et normes de références précédentes ne semble pas apporter de solution complètement satisfaisante pour l'ensemble de nos besoins. 
En effet, les approches restent centrées sur le concept et sa structuration auquel on intègre (SKOS) ou rattache (SKOS-XL, ISO 25946-1) ensuite les termes. 
Ces derniers sont représentés de manière plus ou moins fine (gestion des compositions dans ISO 25946 absente de SKOS). 
Dans tous les cas, la préférence d'un terme s'établit uniquement sur l'appartenance à une langue et non par rapport à une communauté de jargon. % mais sont caractérisés de manière identique dans les deux modèles (appartenance à une langue). 
De ce fait, ces modèles se limitent à représenter un seul jargon de référence par thésaurus et suppose ainsi l'existence d'une communauté homogène dans sa compréhension et dont on cherche à normaliser l'usage linguistique. %Dans notre cas, la collaboration entre communautés hétérogènes dans leur compréhension des concepts et leur utilisation de la langue exige de pouvoir gérer plusieurs jargons. C'est pourquoi 
Nous proposons dans la suite un modèle Multi-Jargon afin d'associer plusieurs jargons métiers et des explications à une conceptualisation commune. %basé sur des concepts originaux 


\subsection*{Discussion}
\addcontentsline{toc}{subsection}{Discussion}


% % \cleardoublepage

% %%%%%%%%%%%%%%%%%%%%%%%%%%%%%%%%%%%%%%%%%%%%%%%%%%%%%%%%%%%%%%%%%%%%%%%%%%%%%%%%%%%%%%%%%%%%%%%%%%%
\chapter{Modélisations de l'audiovisuel (m,i)}\label{chap:mav}
%
% Il existe de nombreux modèles, schémas et standards pour décrire les divers aspects de la chaîne de production audiovisuelle et des objets qui y sont construits. 
% Parmi ces modèles, certains sont issus d'une refléxion générale sur la description des ressources numériques. 
% L'exemple le plus emblématique étant le schéma de métadonnées de la \pc{Dublin Core Metadata Initiative} (\cite{DCMIUsageBoard2010}) qui doit servir à décrire toutes ressources sur le Web.
% De tels modèles ne suffisent pas à décrire les objets audiovisuels, il ont donc fait l'objet de spécialisation, par exemple \cite{Hunter1999}.
% D'autres adoptent une approche générale qui englobe l'ensemble des contenus multimédia.
% Le \pc{Moving Picture Experts Group} (MPEG) est certainement l'organisation qui a le plus porté cette vision avec leurs standards MPEG-7 et MPEG-21.
% Enfin, il y a des modèles développés spécifiquement par des membres de l'industrie pour répondre aux besoins de l'audiovisuel ou de la télévision.
% Il s'agit par exemple d'organisation comme le \pc{TV Anytime Forum}, la \pc{Society of Motion Picture and Television Engineers} (SMPTE) et l'\pc{Union Européenne de Radio-télévision} (UER ou EBU en anglais)

Les problèmes principaux qui se posent à la production audiovisuelle se situent dans la modélisation des objets qu'elle produit et des connaissances qui y sont associées (\ref{sec:pmetiers}). 
Le besoin d'autonomiser tous les objets de la chaîne audiovisuelle, en vue de les réutiliser dans de nouveaux contextes d'exploitations, exige en effet de réévaluer les modélisations sur ces deux points (\ref{sec:scien}) : 
\begin{liste}
	\item[(A)] \g{la modélisation des objets construits au fil de la chaîne de production audiovisuelle}.
	Il s'agit de modéliser non seulement le produit final et ses composants, mais également tous les produits intermédiaires de la chaîne.
	On entend par là tous les fragments de contenu construits ou transformés au cours de la chaîne de production (les prises de vue du tournage, le montage et la séquence monté, le programme prêt à diffuser, la version pour DVD, le résumé pour le journal télévisé etc.).
	Le fait qu'ils participent ou non à la composition d'un produit final n'implique pas qu'ils ne soient pas exploitable dans d'autres contextes.
	De la même manière, les fragments peuvent être transformés à différents niveaux (technique, esthétique, éditorial etc.) pour les adapter à d'autres usages. 

	De ce fait, la condition pour rendre ces fragments autonomes et réutilisables est de les modéliser en tant qu'éléments documentaires à part entière.	
	Mais identifier des fragments documentaires après coup ne suffit pas. 
	Il faut les modéliser dès que possible, afin de rendre compte de leur statut dans le processus de construction. 
	Ce faisant, il devient alors possible de reprendre ce processus, et de l'adapter aux besoins d'un nouveau contexte d'exploitation. 
	La modélisation de l'objet audiovisuel doit donc se faire sur différents niveaux et de manière progressive, en suivant les opérations meneés au cours de la chaîne de production.\\

	% On s'intéresse donc aux types d'objets audiovisuels modélisés, au niveau d'abstraction et de fragmentation proposé pour rendre compte de la composition de ces objets et de leur construction.\\
	% modèle de composition (\cite{Stockinger2007})

	\item[(B)] \g{la modélisation des connaissances construites sur ces objets}.
:

\end{liste}

Sur ce point, il s'agit de clarifier la nature des connaissances que l'on attache aux objets et leur pertinence vis-à-vis des usages que l'on prend en compte. 
Nous considérons trois types de connaissances à associer aux objets : 
\begin{liste}
	\item les connaissances sur les objets ; leur \e{représentation matérielle} (stockage, encodage, format etc.) ; leur \e{contenu} (ce qui est vu ou entendu par le lecteur).

	\item les connaissances liées à la chaîne de production ; la \e{spécification de la forme et du contenu} (que l'on retrouve dans les documents de pré-production) ; le \e{contexte de production} au sens large, incluant les contributeurs et leurs contributions à la chaîne ; le \e{cadre d'exploitation}  qui détaille l'usage de ces objets (type de distribution, droits et propriété intellectuelle, type de réutilisation et transformations opérées pour la réaliser, etc.).

	\item des connaissances issues de l'analyse du contenu des objets audiovisuels. 
	Par exemple, une analyse rhétorique du contenu permettra de mettre à jour la logique argumentative ou discursive (\cite{Gaillard2008}). 
	Ainsi, une multitude d'analyses peuvent être menées, chacune selon une grille d'analyse du contenu propre. 
	On examinera alors si les modélisations permettent d'ajouter de nouvelles échelles de fragmentation et d'y adjoindre des informations.
\end{liste}


\cite{ThiBui2003} propose 4 types de descriptions d'un contenu :
\begin{liste}
	\item \e{description syntaxique d'ordre sensoriel} : il s'agit de caractériser le signal et la manière dont il peut être perçu.
	Pour les aspects visuels, la couleur, la forme, la luminosité, le mouvement, la position etc.
	Pour les aspects sonores, la tonalité, le rythme etc.
   
	\item \e{description syntaxique d'ordre structurel} : le contenu peut être découpé en éléments de base. 
	Pour un contenu audiovisuel, il peut s'agir d'un découpage temporel (image ou segment temporel), d'un découpage visuel (portion de l'image) ou bien encore d'un mélange des deux.
	Le caractère syntaxique s'oppose au caractère sémantique et indique que le découpage est indépendant de sa signification. 
	Le découpage se fait donc de manière arbitraire. 
	Ilne correspond pas forcément à des objets signifiants pour un humain, comme un plan, une scène ou bien une table que l'on verrait à l'écran.
	Il est toujours difficile de trancher le passage d'un élément syntaxique à un objet signifiant car cela dépend de l'application que l'on considère. 

	\item \e{description sémantique d'ordre structurel} : le contenu est décrit par des objets signifiants. 
	Du point de vue temporel, on distinguera les scènes, les chansons des moments parlés, les refrains des couplets etc.
	Du point de vue spatial, il s'agit d'objets comme une chaise, ou bien de regroupements d'objets.

	\item \e{description sémantique des objets du monde narratif} 
\end{liste}

% Nous présenterons d'abord un scénario de réutilisation pour clarifier les usages que nous visons ainsi que les besoins en modélisation (\ref{sec:cdc-av}).
% Notre état de l'art sera nourri par un examen préalable des définitions de l'objet audiovisuel  (section \ref{sec:dav}).


% On peut tenter de distinguer entre différents approches et objets de modélisation : 
% \begin{liste}
% 	\item \e{les modélisations de l'objet audiovisuel} : celles qui traitent de la composition d'un objet audiovisuel fini, qui détaillent la manière de représenter sa structure interne, ou bien la manière dont on l'a groupé avec d'autres objets.
% 	\item 
% \end{liste}
 

% Certains se concentrent sur la description des caractéristiques du signal, des évènements que montrent le contenu ou encore de la manière dont ces contenus sont produits, échangés, adaptés etc.

% L'objectif de ce chapitre est d'examiner les modèles de l'audiovisuel existants en regard de nos  de réutilisation des objets audiovisuels. 
% modéliser les objets de la chaîne de production et les connaissances associées

% les objets de la chaîne de production
% les connaissances associées à ces objets permettant de les rendre autonome dans leur circulation et leur réutilisation.

% Plusieurs sous-problèmes pour réaliser l'autonomie des objets audiovisuels :  
% nous examinerons la manière dont on peut définir un objet, un document, un contenu audiovisuel. 
% comment modéliser ces objets pour gérer leur circulation 
% comment modéliser ces objets pour faciliter leur réutilisation


% d'un exemple de réutilisation d'objets audiovisuels (\ref{sec:cdc-av}). 
% À 
% , qu'il s'agisse de clarifier les notions d'objet ou de document audiovisuel puis d'investir les problèmes de leur gestion et de leur description. 
% Nous poserons d'abord un exemple de réutilisation d'objet audiovisuels comme élément de base de notre réflexion (\ref{sec:ex-reuse}).

% \e{
% Qu'est-ce qu'un objet audiovisuel et particulier comment peut-on aborder le document audiovisuel ? (\ref{sec:dav})
% Comment les produits de la chaîne audiovisuelle sont gérées par les systèmes informatiques, quelles opérations sont menées sur ces objets ? (\ref{sec:gest})
% Comment décrit-on les objets audiovisuels, comment s'organisent la construction ou la récolte de ces informations dans la chaîne de production ? (\ref{sec:desc})}

\section{Cahier des charges fonctionnel}\label{sec:cdc-av}
% \addcontentsline{toc}{section}{Cahier des charges fonctionnel}



\subsection{Scénario de réutilisation multiple}\label{sec:ex-reuse}
Pour bien saisir la finesse des différentes opérations possibles, nous proposons de prendre un exemple.
Imaginons une chaîne de télévision qui souhaite réaliser la captation d'un évènement culturel (par exemple un opéra, une pièce de théâtre, un concert etc.). 
Les producteurs de la chaîne sont intéressés par trois types de contenus qui seront ensuite exploités de quatre manières différentes (voir Figure \ref{img:intro:reuse}):
\begin{listenum}
	\item[a.] la \e{captation de l'évènement} en tant que tel.
	\item[b.] des \e{entrevues avec l'équipe} (metteur en scène, talents sur scène, programmateur etc.). 
	\item[c.] des \e{commentaires du public} avant ou après l'évènement.\\

	\item une partie de tous les types de contenu sera utilisée pour construire un sujet destiné à un \e{journal télévisé}. 
	\item un montage raccourci de l'évènement et des commentaires du public seront utilisés pour produire une \e{bande-annonce diffusée sur le Web}.
	\item un montage de la captation de l'évènement, des bonus comprenant les entrevues avec l'équipe ainsi que la bande-annonce utilisant les commentaires spectateurs seront intégrés dans le \e{DVD}.	 
	\item tout ou partie du contenu filmé pourra être transmis ou vendu à des \e{organisations tierces}. 
\end{listenum}

\begin{figure}[ht!]
\centering
\includegraphics[width=0.7\textwidth]{images/UC-Tahnhauser-v1fr.png}
\caption{Modèle de la production classique comparé avec une production avec réutilisation}
\label{img:intro:reuse}
\end{figure}

Chaque cas de réutilisation tire sa matière première d'à peu près la même base de contenu filmé, mais en tire partie d'une manière propre à chaque forme d'exploitation visée. 
En effet, chaque audience a ses attentes, de même qu'il existe des contraintes techniques spécifiques pour chaque contexte d'exploitation. 
%En effet, il existe des contraintes techniques et des attentes spécifiques à chaque contexte d'exploitation. 

Ces spécificités exigent des variations dans la qualité de l'encodage, le format d'encapsulation utilisé, le montage réalisé, l'habillage du contenu etc. 
Par exemple, les contraintes de diffusion sur le Web implique d'encoder la vidéo dans un format spécifique et de multiples résolutions, généralement plus petites que pour la diffusion télévisée. 
Ensuite, le montage d'une bande-annonce possède un rythme généralement plus rapide que celui des bonus de DVD. 
Finalement, les cas d'exploitation gérés par la chaîne de télévision posséderont un habillage spécifique (logo de la chaîne, message d'annonces etc.) que ne partageront pas forcément les versions vendu à des organisations tierces. 

L'exemple des commentaires du public -- voir la Figure \ref{img:intro:reuse-process} -- permet de montrer à quels moments des transformations doivent être effectuées afin de produire les différentes formes d'exploitation :
\begin{liste} 
	\item[$\bullet$] On considère que deux commentaires de spectateur ont été tournés. 
	\item[$\bullet$] Un des commentaires est intégré au montage du journal télévisé, alors que les deux sont utilisés pour créer la bande-annonce. La bande-annonce est elle-même intégrée au montage du DVD. 
	\item[$\bullet$] Au moment de la finition, l'encodage de la bande-annonce est adapté à la qualité DVD et Web. De même, le journal télévisé est encodé à la fois pour une diffusion en définition standard (SD) et haute-définition (HD).
\end{liste}


\begin{figure}[ht!]
\centering
\includegraphics[width=0.8\textwidth]{images/EX-Content-Production-v7fr.png}
\caption{Étapes et transformations des contenus pour chaque forme d'exploitation des commentaires des spectateurs}
\label{img:intro:reuse-process}
\end{figure}

% Dans cet exemple, on distingue deux types d'opérations effectuées sur le contenu ; 
% la sélection de séquences au moment du montage qui correspond à une décision éditoriale (quel contenu va-t-on présenter à l'audience ?) ; 
% la tranformation de l'enregistrement du contenu qui correspond à des choix techniques (quelle méthode d'enregistrement va-t-on utiliser ?).
% Afin de préciser la nature de ces opérations, nous présentons différentes approches de la réutilisation des contenus.


\subsection{Besoins en modélisation (n,i,t)}
Le scénario d'usage que nous venons de voir présente un exemple de production incluant directement plusieurs cadres d'exploitation pour des contenus produits en collaboration avec deux organisations professionelles et des amateurs. 
Ce scénario permet d'illustrer les échanges 



The details of this use case specifies more than a simple reuse of material. It specifies the kind of processing that support repurposing and which defines thus the modeling requirements for the audiovisual document:
%Such exploitation cases implies various kinds of operations: 
\begin{itemize}
	\item the \textit{reencoding} of edited material to fit the technical parameters proper to each distribution medium/channel (news report distributed by channel broadcasting and internet).
	
	\item the reuse of shooting materials in two distinct editorial structure (\textit{resequencing} of the opinion shot in the website and news report editing).

	\item the reuse of a part of an editorial structure into another editorial structure (\textit{repurposing} of the public comment editing into the DVD bonus editing). 
\end{itemize}


\section{Qu'est-ce qu'un objet audiovisuel ?}\label{sec:dav}


\subsection{Essence, contenu, asset}
\cite{Cox2006} : Essence + Metadata = Content ; \cite{Austerberry2004} Asset = Content + Rights to use it
Définition de Media Asset etc. de \cite{Furht2008}.

\subsection{Le document audiovisuel}
[Quelles sont les manières de représenter les objets/contenus/documents audiovisuels, quelle sont les différences entre ces notions.
\cite{Morizet-mahoudeaux2005a} ;  Voir thèse Charhad 2005. Voir AAF.]

]

\paragraph{Functional Requirements for Bibliographic Records}
FRBRoo est un modèle conceptuel développé par (\cite{Aalberg2008})

Il vise à faciliter l’échange d’information entre les bibliothèques numériques et les musées. 
Il permet de représenter les personnes participant aux différentes étapes de construction d’un objet culturel, depuis l’idée jusqu’à la réalisation matérielle.
Chaque objet culturel possède trois niveaux de modélisation :
\begin{liste}
	\item le niveau des idées ou des oeuvres (\g{Work}) n’ayant pas pris corps dans une matérialité externe à un sujet (par exemple une mélodie ou une histoire qui nous reste dans la tête). 

	\item le niveau des formes d'expression (\g{Expression}) où l'on distingue parmi toutes les formes possibles pour exprimer une idée (une nouvelle écrite, ses traductions, une adaptation de nouvelle en scénario, une lecture de cette nouvelle etc.).
	On se situe à un niveau intermédiaire qui définit des formes abstraites de  réalisation.
	Il faut préciser qu'on parle de forme abstraite dans le sens où il n'existe pas de réalisation concrète, ce qui n'empêche pas de les définir précisement et donc de distinguer de multiples variantes d'expressions :

	\ciel{
	the form of expression is an inherent characteristic of the expression, any change in form (e.g., from alpha-numeric notation to spoken word, a poem created in capitals and rendered in lower case) is a new expression. Similarly, changes in the intellectual conventions or instruments that are employed to express a work (e.g., translation from one language to another) result in the creation of a new expression.} 
	
	\item le niveau des réalisations concrètes comme les porteurs physique d’information (\g{Information Carrier}) portant les expressions (livre, partition, cd-rom etc.). 
	À ce niveau, il faut également distinguer entre l’original (\g{Manifestation Singleton}) et les copies manufacturées (\g{Item}) issues d’un modèle de publication (\g{Manifestation Product Type}). % à rapprocher de la notion de Media Profile dans MPEG-7
\end{liste}




% \paragraph{Sciences de l'information et de la communication}
% [\cite{Leleu-merviela} : le document comportent à la fois des dimensions sémiotiques (signes et sens), techniques (enregistrements, codages et transmission de signaux) et des dimensions médiatiques (socialisation et diffusion).]

% Avec le numérique : niveau des données (enregistrement en binaire, inaccessible et illisible pour l'humain), niveau du texte (une structure organisée de parties informationnelles), niveau de surface (actualisation effective ou affichage au sens large). 

% À la surface : \e{scénique} (manière de transposer des données en une réalité concrète) et \e{scénation} (manière de restituer temporellement à l'utilisateur les fragments d'un document, \ciel{la structure organisée d’événements et/ou d’états avec lesquels l’utilisateur est effectivement mis en interaction.}).

% \ciel{
% Cependant en numérique, les fragments existaient, au moins potentiellement, dans la mémoire de la machine, ce n’est que leur actualisation sur l’écran et la forme qu’elle prend qui se construit dans l’ici et maintenant de l’interaction. 
% Celle-ci est donc nécessairement volatile. De plus, elle change à chaque fois.
% Ainsi c’est l’affichage, [\dots] qui varie, mais non le document lui-même tel qu’il est mémorisé au niveau des données.}

% \ciel{
% conserver, retrouver l’information n’est pas suffisant. 
% Pour qu’elle puisse être utile, il faut qu’elle puisse être exploitée, c’est-à-dire traitée et rapprochée d’autres de façon à produire de l’information nouvelle. 
% Produire du sens n’est, pour l’essentiel, que rapprocher des informations disparates jamais rassemblées auparavant.} (\cite{Balpe1990})

% % Deux pistes proposées par SLM : 
% Il est alors possible de construire des assemblage cohérent de fragments le temps d'une consultation (d'un affichage) par un utilisateur (documents virtuels personnalisables).
% Plus on a de connaissance sur son activité, ses tâches, ses compétences propres, plus il est alors possible de rendre cette assemblage pertinent. 

% Il est aussi possible de mettre à profit la description des documents pour construire des notions de voisinage indépendamment du profilage des utilisateurs. 
% La proximité entre deux documents pourra s'évaluer d'autant de manière qu'il y a de critères descriptifs.
% Ainsi, des informations auparavant éparpillées dans des documents papier différents pourraient être regroupés. 

\newpage
\section{Circulation et réutilisation des objets audiovisuels}\label{sec:gest}

% [Voir MXF, voir AAF ? \cite{Cox2006}
% [Identifiant : hors du cadre de la thèse, dépendant des choix applicatifs des organisations qui utilisent notre modèle. Plusieurs solutions peuvent être implementés via OWL, les URI pouvant être transformé.]

\e{
Si la promesse du numérique de faciliter la manipulation et la circulation des fichiers semble bien s'être réalisée, il n'est pas si évident de l'articuler avec les besoins de la production audiovisuelle (\ref{sec:besoins}).
Ce que l'on nomme la réutilisation des objets audiovisuels recouvre en réalité diverses pratiques et qui repose plus sur la notion d'objet métier ou d'objet numérique que sur la notion informatique de fichier.
Ainsi, la production souhaite récupérer des contenus existants ou produits par d'autres pour les intégrer dans sa propre chaîne de production, ou bien de réutiliser des contenus dans de nouveaux cadres d'exploitations (variation des modes de consommation, de distribution, de public etc.) quelque soit la manière dont l'informatique représente ces objets.}

\e{
Ces opérations qui semblaient a priori plus simple dans un environnement numérique sont en fait plus compliquées qu'il n'y paraît. 
Le numérique impose le calcul et l'explicitation des informations.
Or toutes les informations construites durant la chaîne de production ne sont pas encore intégrées dans les systèmes informatiques actuels.
Lorsque ces informations s'échangent sur papier, à l'oral, par mail ou dans des fichiers non-structurés, le lien avec les objets audiovisuels est alors bien souvent rompu, ce qui entraîne une limitation des traitements réalisables sur ces objets.}

\e{
Dès lors que l'on s'applique à structurer et associer ces informations aux objets audiovisuels, on ouvre la possibilité de récupérer, manipuler, transformer ces objets de nouvelles manières. 
Ainsi augmentés d'un supplément de contexte, les objets gagnent un supplément de manipulabilité susceptible de satisfaire aux besoins de la production audiovisuelle.
Une des solutions développée et utilisée dans l'industrie de la production audiovisuelle est le format conteneur qui encapsule divers types de données en un seul fichier. 
Ainsi, ces formats permettent d'associer de multiples types de fichiers multimédia avec d'autres types d'informations.}

\e{
Cette section a d'abord un souci de clarification des usages et des solutions adoptées. Nous nous intéresserons d'abord aux pratiques de réutilisations (\ref{sec:reuse}), puis nous expliquerons leurs impacts sur la chaîne de production audiovisuelle (\ref{sec:rechaine}). 
Enfin, nous présenterons des formats conteneurs qui assurent le transport des contenus et des informations associées le long de la chaîne de production (\ref{sec:wrapper}).}





%%%%%%%%%%%%%%%%%%%%%%%%%%%%%%%%%%%%%%%%%%%%%%%
\subsection{Caractériser la réutilisation}\label{sec:reuse}
% \subsubsection{Caractérisations de la réutilisation}\label{sec:caracs-reuse}
Nous avons vu grâce à l'exemple de la section \ref{sec:ex-reuse} à quel moment et dans quel type d'opérations la réutilisation pouvait se concrétiser. 
Nous proposons maintenant d'examiner la manière dont différentes communautés scientifiques  abordent la notion de réutilisation. 
Il s'agit de clarifier les hypothèses et les techniques proposées par chacune de ces communautés, et ainsi identifier les éléments pris en compte dans leur représentation du monde.  % Correction ?

\paragraph{Multimédia et Signal}
Prenons d'abord le cas de la communauté multimédia très orientée analyse et traitement du signal. 
Dans ce cadre, les constats mis en avant sont largement les mêmes que ceux que nous avons présentés précédemment (voir section \ref{sec:motiv}, multiplication et diversification des terminaux de lecture et des réseaux de communication, transformation des usages) :
 
\ciel{ 
Hundreds of device profiles are available for accessing online content and more announced everyday. These devices are connected through a wide variety of networks [\dots] As before, the issue of usage scenarios --activity type, user age and gender, time available, and prior knowledge of the subject matter-- continues to exist.} (\cite{Singh2004}).

Un point diffère cependant, le \gui{problème} de la variabilité des usages est considéré comme de même nature que la variabilité des technologies pour transférer et lire le contenu. 
En effet, l'approche de la réutilisation privilégiée par cette communauté consiste en une transformation automatique du contenu en fonction des paramètres d'un scénario de distribution et de lecture : 

\ciel{
Fundamental to this approach is the need to maintain a single copy of the content in its original form and to repurpose the content to fit the desired scenario in real time and in an automated fashion. [\dots] the next step in the repurposing process is to describe the content so that it can be understood and processed to fit delivery requirements --whether they're technical or usage based.} (\cite{Singh2004}).

L'approche automatique est justifiée par la difficulté à maintenir et gérer différentes versions d'un même contenu, en plus d'être coûteux et chronophage.
Ainsi, la décision humaine est simplement reportée au niveau du paramétrage du système de supervisation des opérations techniques.\\


\paragraph{Ingénierie Documentaire}
Dans la communauté de l'ingénierie documentaire, le principe est de pouvoir modéliser distinctement le message que l'auteur souhaite transmettre et la forme dans laquelle ce message se donne à voir par un lecteur. 
Cette tradition, que l'on pourrait faire remonter à la fin des années 60 avec la création du \e{Generalized Markup Language} (\cite{Goldfarb}) ancêtre des SGML, HTML, XML et consorts, repose sur le balisage d'un contenu source. 
Il s'agit alors d'identifier des fragments de contenu ainsi que leur structuration pour mieux les manipuler, quelque soit les opérations effectuées sur ces fragments (transformation, indexation, réécriture etc. \cite[chap.5.2]{Bachimont2004}). 
Les langages de modélisation documentaires tels que \e{Document Type Definition} ou \e{XML Schema} (\cite{Fallside2004}) permettent de contrôler par une grammaire les systèmes de balises construits en vue de formaliser des usages documentaires. 

Nous noterons le développement récent des \gui{chaînes éditoriales}, ces systèmes qui opérationnalisent l'hypothèse de base de l'ingénierie documentaire reformulée par \cite{Crozat2004} de la sorte : \ciel{tout contenu numérique consiste en une ressource qu’un calcul permet de publier dynamiquement sous différentes formes contextualisées}. 

Ces systèmes se concentrent ainsi sur le maintien d'une ressource de base que l'on peut transformer ensuite de diverses manières, soit par une transformation technique que l'on pourra automatisée, soit par une transformation manuelle réglée sur les usages visés (\cite{Crozat2011}) : 
\begin{liste}
	\item le polymorphisme \ciel{consiste en la possibilité technique de disposer d'une source unique de contenu et de la transformer à volonté selon les supports et mises en formes désirés}. Dans ce cas, on établit une séparation entre le fond (la source documentaire) et les formes de publication qui permet de mettre en place une production multi-support.

	\item la réutilisation \ciel{par référence (sans duplication d'information) consiste en la possibilité technique de désassembler et de ré-assembler des fragments de contenu afin de les partager entre plusieurs documents}. Dans ce cas, l'opération repose sur une modélisation séparée du scénario (la structuration) et le contenu.

	\item la ré-éditorialisation est une \ciel{remise en contexte de fragments issus d'un fonds documentaire, par leur ré-agencement au sein d'un nouveau document, leur augmentation par une création de contenus spécifiques et leur publication sur un nouveau support et/ou pour un nouveau public}.

	% \item[T] l'\e{intégration multimédia} est l'exploitation de la propriété héritée du numérique et du codage binaire de permettre l'inscription sur le même support de formes sémiotiques différentes (texte, image, audio, vidéo, ...), afin de composer des contenus multimédia.
\end{liste}

Notons que les chaînes éditoriales s'orientent vers des pratiques de ré-éditorialisation qui sont réalisées manuellement plutôt que de manière automatique.
Les définitions données du polymorphisme et de la réutilisation sont des définitions d'opérations techniques plutôt que des pratiques en tant que telle. 
Ces opérations sont donc permises et prises en charges par les chaînes éditoriales mais ne constituent par leur horizon d'usage.
 % et paramétrées par des règles définies par un utilisateur.
Il semble donc que l'Ingénierie documentaire traditionnelle et le courant lié aux chaînes éditoriales s'intéressent tous deux à des opérations techniques similaires (le polymorphisme et la réutilisation) mais visent des usages distincts qui ne posent pas les mêmes problèmes :
\begin{liste}
	\item D'un côté, il s'agit d'instrumenter d'automatiser des réécritures, entre objets multimédia mais aussi entre documents structurés en XML, base de données etc. 
	Un exemple classique est la création de compte-rendu (ou reporting) qui s'effectue en extrayant des données de diverses sources puis en les intégrant dans de nouveaux documents.

	\item De l'autre on vise à fournir un nouvel environnement de travail aux métiers de l'édition (auteur, éditeur, graphiste etc.) qui permet de passer d'une production artisanale à une production multi-support, réutilisable, ré-éditorialisable. 
	Dans ce cadre, on s'intéresse plus à la création de documents non automatisable telle que les supports pédagogiques par exemple (\cite{Crozat2007}).\\
\end{liste}


\paragraph{Sémiotique Audiovisuelle}
% transition définition Stockinger
Alors que les approches précédentes se concentrent sur des techniques et des outils particuliers, l'approche sémiotique que nous présentons ici propose un point de vue plus général pour définir les différents types de réutilisations existants. 

La sémiotique s'intéresse aux signes pour étudier les activités humaines associées, qu'il s'agisse des producteurs (et de leur intention de communication), des lecteurs (et de leur interprétation des signes produits) ou des relations entre producteur et lecteur (c'est-à-dire des conventions qu'ils partagent). 
Selon \cite{Peirce1978} le \gui{signe} est composé de trois éléments ; le \e{représentamen}, ce qui représente et qu'on pourrait rapprocher de la notion de signifiant chez \cite{DeSaussure1995} ; l'\e{objet}, ce qui est représenté ; l'\e{interprétant} qui produit la relation entre les deux premiers éléments. 

En sémiotique, le signe fait donc toujours l'objet d'une interprétation de la part d'un lecteur qui mobilise un ensemble de conventions pour tenter d'extraire un sens --qui n'est pas forcément l'intention qu'a voulu exprimé l'auteur.
La transmission d'un contenu ne suffit pas en soi à garantir la réussite de la communcation, celle-ci est toujours suceptible d'échouer (soit par un défaut d'expression, un défaut de convention, un défaut d'interprétation). 

Dans ce cadre théorique, la réutilisation de contenu ne se limite pas à une transformation technique (conversion de formats d'encapsulation, de taille d'image, d'encodage) mais se conçoit comme une \gui{adaptation culturelle} \parencite{Stockinger2007} d'une ressource vis-à-vis d'un contexte qui comprend à la fois un usage et une communauté cible. 
Les contenus n'ont donc pas de valeur en soi, mais une valeur d'usage pour une communauté. 
La réutilisation, l'adaptation culturelle ou encore la republication interviennent alors lorsque les contenus sources ne satisfont pas à leur utilisation ou leur communauté de lecteurs future : 

\ciel{
La \e{republication} (en anglais re-authoring ou re-purposing) recouvre un ensemble d'activités visant à réutiliser un corpus de documents numériques (textuels, audiovisuels, visuels, etc.) pour des usages spécifiques auxquels les documents sources, dans leur forme initiale, ne peuvent que partiellement répondre} \parencite{Stockinger2007b}.


On l'aura compris, ce processus englobe des opérations techniques et éditoriales et place les conventions des communautés dans une position centrale. 
Pour ce qui est de caractériser des communautés d'utilisateurs, \pc{Stockinger} se réfère à des sociologues dont \cite{Bourdieu}, et propose différents critères de regroupement :
% la notion d'habitus développée par
\begin{liste}
	\item le temps ou l'espace occupé
	\item les activités et les objectifs recherchés
	\item les attentes et les intérêts
	\item les compétences linguistiques
	\item et de manière générale les connaissances ou les représentations
\end{liste}

Une fois une communauté cible identifiée, il est alors possible ; (1) de définir le type et la forme de contenu qui est pertinent (utilisable, utile, compréhensible, acceptable etc. par ces utilisateurs) ; (2) les outils nécessaires pour effectuer les opérations propres à adapter le contenu aux besoins de la communauté cible :

\ciel{
La republication est donc un processus, parfois très complexe, d'adaptation d'un document ou d'un corpus de documents sources à des usages spécifiques. Ce processus d'adaptation peut concerner tous les plans constitutifs d'un document (Stockinger, 1981, 1999 et 2003), c'est-à-dire aussi bien le plan du contenu que celui de l'expression. Il s'accomplit à travers un ensemble d'activités intellectuelles et de gestes techniques et en référence à des modèles ou genres de publications qui intègrent les contraintes typiques des contextes et des communautés d'usage auxquels un document ou un corpus de documents republiés est destiné} \parencite{Stockinger2007b}.

% opérations : (traitement linguistique, restructuration, rééditorialisation).
La republication repose donc sur une représentation des communautés d'utilisateurs, de leur capacités d'interprétation ainsi que sur une représentation des contenus dont elles disposent habituellement. 
La republication se définit selon \parencite{Stockinger2007} suivant les critères suivants : 
\begin{liste}
	\item \e{les opérations à effectuer} ; sélection, réorganisation, ajout d'explications, ajout d'éléments complémentaires, traduction, mise en lien avec d'autres ressources, modification de la forme d'expression, création de nouveau contenu etc.
	\item \e{le type} (image, texte, objet audiovisuel etc.) \e{et le genre} (journal télévisé, émission etc.) \e{de ressources à traiter}. 
	\item \e{l'objectif de la réutilisation} ; le contexte d'usage, la communauté cible, le genre de la future publication, le format de distribution etc.
	\item \e{les ressources à disposition pour effectuer la republication} ; les personnes, les outils, le budget, les ressources intellectuelles.
\end{liste}
Cette approche générale de la réutilisation n'est pas purement intellectuelle puisqu'elle se concrétise également dans des développements logiciels. 
En effet, un logiciel nommé \gui{Atelier Sémiotique} se développe dans le cadre de l'\gui{Atelier de Sémiotique Audiovisuelle} et par l'intermédiaire de divers projets (Saphir, Logos) et partenaires (INA, ESCoM, MSH de Paris).

\paragraph{Discussion}
% Notons que cette définition, par rapport à celle des approches précédentes, décrit de manière plus globale ce qu'est la réutilisation en citant de nombreux et nouveaux éléments à prendre en compte. 
% L'analyse proposé par la sémiotique audiovisuelle propose une définition plus générale de ce qu'est la réutilisation. 
L'analyse proposée par \pc{Stockinger} propose une définition générale de la réutilisation qui englobe les pratiques présentées précédemment. 
En effet, l'ingénierie documentaire et la communauté multimédia se concentrent sur la construction d'outils pour automatiser certaines transformations ou réécritures de contenu.
En se concentrant sur un éventail de techniques, ces approches se prêtent plus à certains cas d'usages et visent des objectifs différents. %(comme la ré-éditorialisation pour l'ingénierie documentaire ou bien la construction automatique de compte-rendu pour la communauté multimédia).
% parler de C2M qui vise à une chaîne éditoriale multimédia ?  

\begin{figure}[ht!]
\centering
\includegraphics[width=0.75\textwidth]{images/Reuse-v1.png}
\caption{Les différentes pratiques de réutilisations}
\label{img:intro:reuse}
\end{figure}

Nous proposons donc de définir la terminologie suivante pour distinguer entre trois niveaux successifs de réutilisation, chacun visant à créer un nouveau document mais suivant des opérations différentes (voir Figure \ref{img:intro:reuse}). 
La premier critère distinctif est l'automatisation de la transformation, le second critère repose sur la création originale de contenu plutôt que la réorganisation d'un existant : 
\begin{liste}
	\item le \eg{retraitement} (repurposing) qui se caractérise par une automatisation de la transformation opérée sur le contenu (quelque soit son type), c'est-à-dire que cette transformation est effectué par un logiciel lui-même paramétré par un humain. 
	Ces transformations visent à modifier la forme d'expression du document, extraire des fragments de contenu de différentes sources pour les aggréger dans un nouveau document, ou bien encore réorganiser automatiquement la structure du contenu. 
	Le retraitement dépasse le polymorphisme en ce sens où il est possible de gérer de multiple sources de contenus pour construire dynamiquement un nouveau document. 
	Dans le cas de l'audiovisuel, il s'agit des pratiques de réencodage, de changements de format d'encapsulation etc.
	Pour reprendre l'exemple précédent (\ref{sec:ex-reuse}) d'un contenu TV pour une diffusion Web, ou bien encore la création automatique de résumé de rencontres sportives etc. \\

	\item la \eg{rééditorialisation} (reediting) se caractérise par une transformation (manuelle et automatique) de contenus existants. 
	Le document doit s'adapter à un nouveau contexte de lecture (genre éditoriale, public, forme d'expression etc.).
	La transformation du contenu nécessite une compréhension du nouveau contexte de lecture et consiste en des opérations de réorganisation, de mise en relation avec d'autres contenus, de traduction etc. 
	Ces opérations ne se limitent pas à une transformation de la forme d'expression du document (retraitement) mais ne constituent pas une création originale de contenus (réécriture). 
	Simplement, on réutilise divers contenus existants pour créer un nouveau document.  
	La ré-éditorialisation repose donc sur le polymorphisme et la réutilisation au sens de \cite{Crozat2011}.
	Dans le cas de l'audiovisuel, il s'agit typiquement de pratiques de re-montage et de nouvelles sélections de contenu. 
	Pour reprendre l'exemple précédent, il s'agit de monter différemment une séquence initialement prévue pour un journal TV et qui doit s'insérer dans un DVD etc. \\

	 
	\item la \eg{réécriture} (reauthoring) se caractérise par une transformation de contenus existants accompagnée d'une création de contenu original. 
	L'ajout de contenu sert à satisfaire soit aux attentes spécifiques du nouveau public cible, aux contraintes d'un nouveau genre éditorial (commentaires, explications, exemples etc.) soit à la création d'une version augmentée d'un document existant (pas de changement de public cible, mais de nouvelles attentes).
	Dans le cas de l'audiovisuel, il s'agit par exemple de la construction d'un documentaire à partir de vidéo d'archives, la création originale étant le commentaire proposé.\\	 
\end{liste}

Les pratiques de réutilisations sont donc chevillées aux dimensions techniques, éditoriales et sémiotiques du contenu audiovisuel. 
Leur mise en place pose également des problèmes dans l'organisation de la chaîne de production audiovisuelle et son informatisation
% Il faut donc élargir le champ de la modélisation des contenus à une dimension sémiotique et éditoriale et faire le lien avec le déroulement de la production.








%%%%%%%%%%%%%%%%%%%%%%%%%%%%%%%%%%%%%%%%%%%%%%%
\subsection{Évolutions de la chaîne de production}\label{sec:rechaine}
\e{
Les changements introduits par la réutilisation dans la chaîne de production sont donc plus vastes qu'une simple adaptation technique à de nouveaux modes de distribution. %(canal de diffusion + terminal de lecture). 
Il s'agit également de prendre en compte l'audience visée pour affiner encore plus l'adaptation du contenu à ses futurs consomateurs/lecteurs.
L'objectif est de favoriser le développement de variantes d'un même programme soit par la restructuration du contenu (retraitement ou rééditorialisation) ou par l'ajout de contenus (réécriture). 
L'introduction d'acteurs tiers dans une chaîne de production pour fabriquer ou fournir du contenu ne peut se faire sans une plus grande maîtrise des contenus et une meilleure description de ces derniers dès la pré-production.
En effet, le client qui souhaite déléguer la fabrication de contenu à un fournisseur tiers doit d'abord définir ses attentes. 
À l'inverse, si le fournisseur connaît son contenu le client lui a besoin d'un descriptif pour sélectionner les fragments les plus pertinents.
Ainsi, les chaînes de production des clients et des fournisseurs doivent évoluer pour gérer (fournir/acquérir) non pas juste du contenu, mais des descriptions (adjointes ou pas à du contenu) facilitant le travail de leurs partenaires (fabrication ou réutilisation).
}
% Le producteur-diffuseur rentre alors dans une dynamique d'adaptation de ces contenus.

Comme nous l'avons constaté en \ref{sec:electro}, le développement de l'électronique offre de nouvelles opportunités de production mixte, soit avec des amateurs, soit avec d'autres professionnels. 
Cependant, une organisation souhaitant profiter de ces opportunités devra réussir d'abord à encadrer ses partenaires et clarifier avec eux les termes de leurs accords. 
Ce qui auparavant pouvait se résoudre \e{de visu} ou de manière informelle doit maintenant être explicité afin de clarifier la demande, c'est-à-dire le contenu souhaité. 
Qu'il s'agisse de passer commande, ou bien de rechercher dans des bases existantes, cette étape s'apparente à la définition du besoin, à l'écriture d'un cahier des charges, ou dans les termes propres à la production audiovisuelle, au \e{Scripting} défini en \ref{sec:preprod}. 
Maintenant que la fabrication du contenu est déléguée à des tiers, il reste cependant à récupérer le résultat et à vérifier qu'il satisfait à la demande initiale. 
Cette dernière étape nommée généralement \e{Acquisition} constitue un travail à part entière puisqu'il s'agit de \gui{faire rentrer} le contenu dans les \gui{cases} du système d'information et de gestion des contenus. 
En plus des questions de formats informatiques, s'ajoute souvent le problème de la description des contenus et de leur classification en vue de leur utilisation future.  
L'acquisition dépend grandement des conventions établies avec le fabricant/livreur de contenu et impacte directement sur le temps passé à faire le \e{derushing}. 

Côté client, on transforme d'abord la phase de scripting en l'expression d'une \pc{Commande} ou d'une \pc{Requête} ce qui permet de déléguer la fabrication des contenus à un tiers.
Ensuite, on vérifie par \pc{Acquisition} du contenu que le résultat correspond bien à la demande et on procède aux ajustements nécessaires (si besoin) pour satisfaire aux contraintes de notre système (voir Figure \ref{img:intro:evochain}). 

\begin{figure}[ht!]
\centering
\includegraphics[width=\textwidth]{images/Workflow-Thesis-v6.png}
\caption{Ouverture des chaînes de production du client et du fournisseur}
\label{img:intro:evochain}
\end{figure}

Du côté des fournisseurs de contenu, il existe une distinction entre les chaînes du fabricant et du fournisseur du contenu :  

\begin{liste}
	\item \e{déléguer la fabrication à des contributeurs tiers} : l'utilisation du scripting pour définir la \pc{Commande} de contenu attendu semble une solution satisfaisante, à condition que le vocabulaire utilisé soit normé et rattaché à une conceptualisation de manière à éviter la confusion ou les différences d'interprétations. 
    Lorsqu'il s'agit d'amateurs, la situation se complique car on ne peut pas s'appuyer sur une conceptualisation commune de la production audiovisuelle pour clarifier la commande. 
    De plus, le manque d'expérience et l'ignorance des usages du métier impliquent non seulement de documenter les concepts par des mots et des définitions, mais aussi d'expliquer ce qu'il faut faire durant la phase de \pc{Fabrication}. 
    En d'autres termes, en travaillant avec des amateurs, les professionnels ne se retrouvent non pas à écarter la confusion entre des mots se reférant au même concept, mais à expliquer les opérations auxquelles ces concepts font référence. 
    De même pour l'acquisition, s'il s'agit surtout de se mettre d'accord entre professionnels, travailler avec des amateurs semble plus difficile de prime abord. 
    Les notions de formats d'encodage et d'encapsulation sont souvent confuses ou se mélangent, de même que la description des contenus peut s'avérer compliquée à réaliser sans expérience préalable. 
    Tout du moins, il faut remarquer que la description de la demande initiale sert de description a minima du contenu produit, même si les variations ou les écarts ne sont pas forcément indiqués.
    Le cas échéant, une phase d'\pc{Indexation} peut être nécessaire pour décrire le contenu suivant les exigences du client.\\

	\item \e{rechercher des contenus existants depuis les bases professionnelles} : l'utilisation du vocabulaire de l'écriture audiovisuelle pour définir une \pc{Requête} nécessite une indexation utilisant ce même vocabulaire, ou alors une manière de traduire la requête d'un langage à l'autre (par alignement des vocabulaires par exemple). 
	De même, il faut pouvoir s'accorder sur le niveau de fragmentation recherché (programme complet, séquences, scène, frame etc.), le format du contenu, les descriptions ou les métadonnées à fournir etc.
	Ainsi, la \pc{Livraison} de contenu ne consiste pas en un simple transfert de fichier, mais représente le moment où l'on teste l'interopérabilité entre les systèmes et les formats. 
	Cette étape est d'autant plus cruciale qu'elle se répercute directement sur la phase d'acquisition pour le client. 
	Tout ce qui n'a pas pu être résolu à la livraison (côté fournisseur) devra l'être au moment de l'acquisition dans le système (côté client).\\
	% un vocabulaire de requêtes, s'accorder sur les niveaux de fragmentation, le format de livraison, le contenu de la livraison
\end{liste}



Finalement, il nous faut encore éclairer à quels moments dans la chaîne de production les différentes pratiques de réutilisation sont réalisées (voir définition en \ref{sec:reuse}).
De manière générale, on considère que la réutilisation commence à la phase de pré-production, au moment du \pc{Planning} et du \pc{Scripting} où l'on spécifie les nouvelles formes et formats d'exploitation (voir Figure \ref{img:intro:reutilisation}). 
Mais chaque pratique opére à différents étapes de la chaîne :
\begin{liste}
	\item pour le \e{retraitement}, les variations sur la forme d'expression du document se réalisent en phase de \pc{Finition}. 
	C'est à ce moment que l'original et les variantes sont encodés et encapsulés dans les formats correspondants à leur mode de distribution. 
	Lorsqu'il y a manipulation de la structure des contenus, ces opérations (automatisées) se réalisent à la phase de \pc{Montage}.

	\item pour la \e{rééditorialisation}, le travail commence en phase de \pc{Derushing}, par la sélection des séquences de contenu à ajouter ou à retirer du contenu original. 
	La grande différence avec le retraitement, c'est que cette sélection s'effectue manuellement sur les contenus à disposition. 
	Ensuite, on opére un nouveau \pc{Montage} qui peut également impliquer une \pc{Finition} différente.

	\item pour la \e{réécriture}, la grande différence avec les autres pratiques est que l'on ajoute une nouvelle phase de \pc{Fabrication}. 
	Qu'il s'agisse de création originale ou de récupération de contenu chez un fournisseur tiers, la réécriture consiste à utiliser de nouveaux contenus. 
	Ensuite, on sélectionne en \pc{Derushing} les séquences qui permettront de réaliser un nouveau \pc{Montage}. 
	Finalement, plusieurs \pc{Finition} sont à envisager suivant les cas d'exploitation.
\end{liste}
% conséquences de la réutilisation dans la chaîne : à quel étape ça se joue

\begin{figure}[ht!]
\centering
\includegraphics[width=0.8\textwidth]{images/Workflow-Reuse-v1.png}
\caption{Les différentes formes de réutilisation et leur mise en oeuvre dans la chaîne de production audiovisuelle.}
\label{img:intro:reutilisation}
\end{figure}








\subsection{Formats conteneurs pour l'audiovisuel} \label{sec:wrapper}
% intro
Les formats conteneurs sont des formats de fichiers qui encapsulent des contenus de toutes sortes (audio, vidéo, texte, image etc.). 
Un des exemples le plus connu pour la vidéo est le format AVI de Microsoft, souvent confondu avec un format de compression. 
Ces formats ont la particularité d'associer aux contenus audio-visuels des informations annexes, le plus souvent sous la forme de métadonnées.
L'intérêt de ces formats est de constituer un objet numérique qui structure l'enregistrement du contenu et de ses informations annexes et fournit ainsi une manière unique d'y accéder (\cite{Ferreira2010}).
Sans cela, les informations seraient dispersées et le lien avec le contenu devrait se faire via des références. 
Cette force constitue également un inconvénient lorsqu'il n'est pas possible de faire évoluer le modèle d'information ou bien d'en changer en fonction du type de production ou d'exploitation envisagé.

Dans le cas de la production télévisuelle, deux formats liés et complémentaires ont été progressivement adoptés par l'industrie.  
Il s'agit du \pc{Material eXchange Format} (MXF) et de l'\pc{Advanced Authoring Format} (AAF) que nous présentons par la suite. 
Leur particularité est de pouvoir intégrer des schémas de métadonnées propres aux besoins de l'industrie, mais aussi de pouvoir en utiliser d'autres. 
Comme ces formats servent de référence à l'industrie, il est particulièrement intéressant pour nous de comprendre leur modélisation de l'objet audiovisuel, de voir comment ils gèrent les résultats intermédiaires de la chaîne de production ou quels genres d'informations ils embarquent avec le contenu.


\subsubsection{Material eXchange Format}
% qui / quand
MXF est un format conteneur ouvert développé depuis le milieu des années 90 par des membres de l'industrie et standardisé en 2004 par la \pc{Society of Motion Picture and Television Engineers} (SMPTE).
% objectif
Son objectif est de favoriser les échanges de contenus audio-visuels finis en les associant à d'autres données ou métadonnées (\cite{Devlin2002}).
Ces métadonnées sont structurés par le schéma \pc{Descriptive Metadata Scheme-1} (DMS-1, ) que nous présenterons en détails dans la suite de la section. 

% description
Voici d'abord les principales caractéristiques de MXF en tant que format (\cite{Ferreira2010a}) : 
\begin{liste} 
	\item \e{indépendant d'un système propriétaire}. 
	Le standard se veut avant tout un format qui fonctionne sur tout systèmes. 
	Ainsi, il définit une organisation des données bit par bit qui repose notamment sur le système de \pc{Key-Length-Value} (KLV, clé-longueur-valeur). 
	La clé donne un identificateur de l'élément à suivre, la longueur précise la taille de la valeur à suivre et la valeur contient les données de l'élément.
	Comme dans tout format de fichier, on retrouve la définition d'en-tête et de fin de fichier. 
	L'en-tête contient des métadonnées de description du contenu (\pc{Header Metadata}), de sa structuration (\pc{Partition Metadata}) et une table d'association entre un timecode et une position dans le flux binaire du fichier (\pc{Index Table}).

	\item \e{indépendant des méthodes de compression du contenu utilisées}.
	MXF définit une structure d'encapsulation et d'accès au contenu nommé \pc{Essence Container} (EC) qui permet de transporter du contenu sans le transformer ou bien de faire référence à des fichiers externes.  
	MXF possède des transpositions permettant de synchroniser différents flux de contenus, quelque soit leur encodage. 
	Chaque type de contenu est traité à part, ainsi les EC sont composés de \pc{Content Package}, eux-mêmes décomposé en \pc{Picture Item} (piste vidéo), \pc{SoundItem} (piste audio), \pc{Data Item} (télétexte, sous-titre etc.), \pc{Compound Item} (contenu audiovisuel encodé comme un seul contenu) et \pc{System Item} (autres données comme les timecode etc.).
	La Figure \ref{img:mxf-content} montre deux méthodes d'encapsulation de ces essences.


	\item \e{diffusable en flux continu ou bien par fichier}. 
	Suivant la méthode d'organisation de l'EC décrit ci-dessus, le contenu d'un fichier MXF peut être visionné au cours de son transfert (streaming). 
	Cette caractéristique est particulièrement importante dans le cadre de la diffusion télévisuelle et s'applique à tout les types de contenu d'un MXF (audio-visuel, mais aussi sous-titre ou métadonnées etc.). 
	Naturellement, le fichier MXF peut également être transféré en FTP.

	\item \e{une organisation des contenus indépendante de leur visionnage}. 
	En effet, MXF définit à part l'organisation des contenus encapsulés et la manière de les visionner.
	Le \pc{Header Metadata}	d'un fichier MXF contient une partie \pc{File Pacakage} (FP) qui décrit la manière dont les fichiers de contenus sont encapsulés dans le MXF. 
	Cette description détaille les méthodes d'encodage pour toutes les pistes de contenus (Track) du fichier, de même que les timecode originaux.
	Cependant, si le FP décrit les sources d'un fichier MXF, il existe également un \pc{Material Package} (MP) décrivant la manière dont celles-ci doivent être visionnées.
	Il s'agit là de définir un montage simplifié qui explicite quelle partie et dans quel ordre jouer les contenus sources, à la manière d'une \ciel{Edit Decision List}. 

	\item \e{encapsulation conjointe des contenus et des métadonnées}. 
	Comme nous l'avons vu, un fichier MXF contient un \pc{Header Metadata} transportant des métadonnées propres à l'ensemble du fichier ainsi que des métadonnées propres à chaque paquet de données (Package). 
	Ces éléments sont donc intégrés dans la structure du fichier, au même titre que le contenu.
\end{liste}


\begin{figure}[ht!]
\centering
\includegraphics[width=0.9\textwidth]{images/MXF-ContentPackage.png}
\caption{Deux méthodes d'encapsulation des essences en MXF : image par image (haut, pour le streaming), par séquence vidéo (bas).}
\label{img:mxf-content}
\end{figure}

\paragraph{Adoption et usage}
Du fait de sa large adoption par l'industrie de la télévision, il favorise l'interopérabilité entre les systèmes, souvent propriétaires, des producteurs, diffuseurs, chaînes de télévision etc. (\cite{Ferreira2010}, \cite{Devlin2002}). 
Cependant, MXF n'est pas fait pour gérer les résultats intermédiaires de la chaîne de production. 
Il a été spécifiquement conçu pour favoriser la circulation des programmes finis, indépendamment de la manière dont les contenus sont matériellement enregistrés et structurés.
De ce fait, MXF se positionne comme un format utilisé en fin de chaîne de production, à la diffusion des programmes ou bien dans le cas d'échanges entre professionnels.

\subsubsection{Description Metadata Scheme-1}
Le schéma DMS-1 a été standardisé par le SMPTE en 2004 (\cite{Smpte2004}).
Il propose trois schémas de description (\pc{Framework}), chacun proposant une perspective de description particulière : 
\begin{liste}
	\item le \pc{Production Framework} propose une description du fichier MXF en tant que résultat d'une production. 
	Les informations qu'il regroupe s'appliquent donc au fichier en entier (identification, propriété intellectuelle, droits et contrats, projet, format de publication, format d'image, récompense) mais aussi au contenu de l'objet audiovisuel (évènements relatés, période historique, annotation).

	\item le \pc{Clip Framework} aborde la description du point de vue de la création du matériel audio-visuel, c'est-à-dire des séquences de contenu encapsulées dans le MXF. 
	On retrouve des informations liées à la production (projet, droits et contrat) mais surtout des éléments pour décrire les essences (format de l'image, sous-titre, script utilisé, matériel utilisé et paramètrage, opérations de transformation des essences) et une description par plan.

	\item le \pc{Scene Framework} propose une vision éditoriale du contenu en le découpant en scène et plan. 
	Ces éléments sont ensuite décrits en terme d'évènements, en précisant les participants et les lieux où ils se déroulent etc.
\end{liste}

\paragraph{Framework et ensemble de métadonnées}
Ces \pc{Framework} sont composés de petits ensembles de métadonnées, parfois partagés,  que l'on attachent au \pc{Header Metadata} d'un fichier MXF. 
Par exemple, l'ensemble \pc{Titles} est commun au trois \pc{Framework} et se compose des métadonnées suivantes ; \e{Extended Text Language Code} ; \e{Main Title} ; \e{Secondary Title} ; \e{Working Title} ; \e{Original Title} ; \e{Version Title}. 
De nombreux autres ensembles sont partagés, comme les annotations, la description des lieux, des participants, des organisations etc. 
Ces ensembles prennent alors un sens différent en fonction du \pc{Framework} auquel ils sont associés. 
Par exemple, on distingue les participants à la production, à la création d'une séquence, en tant que présentateur ou acteur (respectivement pour le \pc{Production}, \pc{Clip} et \pc{Scene Framework}). 
De même, pour les lieux il peut s'agir d'un lieu où se trouve l'organisme producteur, du lieu de tournage, du lieu où se déroule l'action qui est différent du lieu de tournage dans le cas d'une fiction (respectivement pour le \pc{Production}, \pc{Clip} et \pc{Scene Framework}). 
Ainsi, la distinction entre éléments réels (issu du \pc{Clip Framework}) ou fictifs (issu du \pc{Scene Framework}) n'est pas clairement spécifié. 
De manière générale, il semble regrettable que les mêmes ensembles de métadonnées soient utilisés pour décrire des objets différents.
Cela propage ainsi une certaine confusion sur le plan sémantique. 

\paragraph{Framework et Package MXF}
Comme pour les ensembles, les \pc{Framework} peuvent s'attacher à un ou plusieurs des \pc{Package} de MXF (\pc{File Package}, \pc{Material Package} etc.). 
Par exemple, le \pc{Clip Framework} appliqué au \pc{Material Package} décrit ce qui est nécessaire au visionnage du contenu (format d'image prévu).
S'il s'appliquait au \pc{File Package}, les informations correspondrait aux informations de création (format d'image original).
Là encore, l'objet de la description change légèrement et si les mêmes éléments de description peuvent être utilisé, il semblerait plus clair de préciser la nature de ces informations. 

Nous remarquerons que DMS-1 utilise la notion de plan de deux manière différentes qui peuvent sembler ambigüe. 
Ainsi, le plan est utilisé à la fois dans le \pc{Clip Framework} et le \pc{Scene Framework}. 
D'après les auteurs, cela correspond à la nature duale des plans, à la fois élément factuel et éditorial. 
Ce choix implique alors que chaque plan peut être décrit à deux endroits à la fois, selon deux perspectives différentes (descriptive de ce qui est perçu, ou bien pour nommer le plan par rapport au script par exemple).

\paragraph{Multilingue et thésaurus}
Concernant la gestion des vocabulaires et des langues, DMS-1 prévoit que certains ensembles de métadonnées soient décrites par un code de langue et puissent faire référence à un élément d'un thésaurus.
Cette perspective est particulièrement intéressante vis-à-vis des besoins que nous avons exprimés dans le chapitre précédent (\ref{sec:bm}).
Cependant, le lien avec un thésaurus externe est limité car il ne s'applique qu'aux ensembles et non à chaque métadonnée de l'ensemble.

\paragraph{Travaux liés}
\cite{Marcos2009} ont construit une ontologie basée sur le schéma DMS-1, MPEG-7 et une ontologie de domaine pour construire un système de Media Asset Management. 
Ce système, développé dans le cadre du projet européen RUSHES, a pour objectif d'aggréger des informations récoltées pendant la production par différentes sources, puis de les associer aux objets audiovisuels et proposer des services de recherche d'information et d'accès au contenu (\cite{Gorka2008}). 
Les objets audiovisuels considérés sont les prises de vue brutes, nommées \ciel{rush} dans le milieu audiovisuel.

L'approche consiste à transformer ou relier des informations de bas-niveau en une indexation sémantique à l'aide des ontologies développées.
Les services sémantiques proposés par le système sont les suivants : 
\begin{liste}
	\item \e{transformation de formats des données échangées pendant la production}. 
	Les annotations recueillies à partir des équipements de tournage et après analyse automatique du contenu sont transformés du format DMS-1 en un autre format utilisé par le système d'indexation.
	\item \e{sémantisation de l'analyse automatique du contenu}. 
	Les auteurs prennent l'exemple d'une reconnaissance des visages qui permet d'identifier le nombre de personnes présents dans une séquence.
	Ces personnes peuvent alors être intégrées à une base de connaissances.
	De même, l'analyse permet de détecter les changements de plans.
	\item \e{recherche, découverte, annotation de séquences}. 
	La recherche peut se faire soit à l'aide de mots-clés, soit à l'aide des concepts de l'ontologie, qui permettent alors d'enrichir les requêtes, de proposer des recommandations etc. 
	De plus, l'ontologie peut être également utilisé pour relier les annotations manuelles avec des concepts ou des éléments de la base de connaissances.
\end{liste}

Ces travaux poussent ainsi l'utilisation de DMS-1 en tant que schéma de description des prises de vue, juste après leur production, et non pas seulement des programmes finis, en fin de chaîne. 
Ce changement de granularité montre l'importance du \pc{Clip Framework} et du \pc{Scene Framework} qui permettent d'attacher la description à ces objets intermédiaires de la production.
Ceci est d'autant plus pertinent que DMS-1 prévoit déjà de décrire, en partie, les participants de la chaîne et leurs contributions.

L'originalité de l'approche se situe également dans la transformation des résultats d'une analyse automatique en objet sémantique. 
Les exemples de l'analyse des visages et de la détection de plan sont éclairants mais les auteurs ne fournissent pas d'indication pour généraliser le procédé à d'autres types d'information.



\subsubsection{Advanced Authoring Format}
% \cite{Gilmer2002} 
% \cite{Austerberry2004}
% qui 
AAF est un format conteneur développé principalement par l'\ciel{Advanced Media Workflow Association} (AMWA) en collaboration avec d'autres organismes tels que le SMPTE et l'EBU.
% objectif, portée et usage 
Ses objectifs sont similaires à celui de MXF, à la différence qu'AAF vise à favoriser les échanges de contenus à l'intérieur de la chaîne de production (\cite{Austerberry2004}).
AAF s'occupe plus particulièrement des informations utilisées au moment de la post-production par les applications de montage : 

\ciel{The traditional workflow – based around tape interchange, isolated non-linear editing and authoring tools, and ad-hoc metadata systems – is being recast as a more integrated networked system with a consistent approach to the format and interchange of essence and metadata.} (\cite{Gilmer2002})

% description
Le modèle de MXF présenté précédemment est en réalité une sous-partie de celui d'AAF. 
On retrouve donc les mêmes principes et fonctions, dont les \pc{Packages} qui portent les descriptions et les \pc{Items} qui encapsulent les contenus. 
Parmi les éléments supplémentaires dans AAF qui le destine particulièrement à une utilisation dans la chaîne de production., nous trouvons :
\begin{liste}
	\item le \pc{Physical Source Package} permet de référencer des contenus enregistré sur d'autres mediums que les disques durs (cassette vidéo, bande 35mm etc.).

	\item le \pc{Composition Package} permet de définir la manière dont le contenu doit être visionné en termes d'ordre (comme MXF) mais aussi en terme d'effets, de transition ou de composition des flux de contenu (ce que l'on appelle des EDL complexes).
	
	\item le \pc{Dictionary} qui permet d'intégrer des définitions des métadonnées autres que celles du dictionnaire du SMPTE dans le AAF.
\end{liste}

De plus, AAF se différencie par l'utilisation de la technologie \e{Structured Storage} de Microsoft pour gérer l'organisation des données (plutôt que la méthode du KLV).
De ce fait, AAF ne permet pas la diffusion en continu (streaming) de ses contenus.
Cependant, les deux formats utilisent le même modèle de structuration, ce qui permet aux applications d'effectuer des transformations de l'un à l'autre aisément, et particulièrement du AAF vers le MXF en suivant le déroulement de la chaîne.
Leurs différences les prédisposent néanmoins à des usages complémentaires. 
AAF se positionne comme un format pour la post-production qui conserve toutes les sources et le master alors que MXF, avec son modèle simplifié et ses capacités de diffusion en continu, est particulièrement intéressant pour les échanges de programmes finis.


\paragraph{SMPTE Metadata Dictionary}
Ce dictionnaire est un gigantesque registre de toutes les métadonnées utilisées par l'industrie télévisuelle. 
Régulièrement mis à jour, la dernière version disponible (\cite{SMPTE2010}) comporte 1476 métadonnées distribuées dans 499 catégories et sous-catégories.
La nature des métadonnées est très diverse, puisqu'on trouve des identificateurs, des informations administratives, interprétative, paramétriques, liées au processus etc.
Le dictionnaire donne une identification unique à chaque métadonnée et donne une définition ainsi que le codage utilisé pour la valeur. 
Malgré sa taille imposante, il est utilisé par les membres de l'industrie et notamment dans DMS-1.



\subsubsection*{Discussion}
\addcontentsline{toc}{subsection}{Discussion}
Les formats conteneurs MXF et AAF reposent sur le schéma de description DMS-1 ainsi que des dictionnaires de métadonnées développés par l'industrie.
L'originalité de ces formats est d'associer directement au matériel audiovisuel plusieurs perspectives de modélisation (\pc{Framework}) [\g{B1 : autonomie}].
L'objet audiovisuel est ainsi modélisé en tant que résultat d'une production qu'il faut valoriser commercialement (\pc{Production}) ; 
décomposé en éléments narratif (plan, scène) faisant partie d'un ensemble documentaire (\pc{Scene}) et dont on décrit le contexte historique (\pc{Production}) et les évènements réels (\pc{Clip}) ou fictifs (\pc{Scene}) qui s'y déroulent ; matériel audiovisuel construit pendant la production dont on décrit les caractéristiques techniques (\pc{Clip}).
Cette pluralité des points de vue pourrait permettre de modéliser les produits intermédiaires de la chaîne tout en leur associant des métadonnées. 
Ce remplissage progressif ne peut intervenir qu'après la production du matériel, et encapsulé dans le format AAF. 
C'est donc les informations de la post-production qui y sont capturées, puis transmises sous une forme simplifiée à un MXF qui symbolise le produit final de la chaîne.
Les produits intermédiaires ne sont donc pas représentés pour eux-mêmes, mais en tant que partie du produit final.
L'approche de modélisation est donc intéressante, mais le couplage matériel et métadonnées empêche de fragmenter la modélisation et de la commencer dès le début de la production.

Parmi les descriptions associées au matériel audiovisuel [\g{B2 : réutilisabilité}], on note un certaine confusion sur le plan sémantique. 
Ainsi, des mêmes ensembles de métadonnées sont utilisées pour représenter des éléments fictifs ou rééls, tandis que d'autres peuvent prendre un sens différents suivant le \pc{Package} ou le \pc{Framework} auxquels ils sont associés.
Ainsi, il ne s'agit pas de critiquer la modélisation qui distingue élément réél et fictif, ou bien encore le plan prévu dans le script et le plan tourné, mais bien la représentation confuse qui en est faite dans DMS-1.

Nous en concluons que les formats conteneurs sont adaptés à la circulation de documents audiovisuels finis mais dont la représentation d'une seule pièce et parfois confuse ne couvre pas nos besoins pour la réutilisation de fragments documentaires.

\input{CONTENU/1-analyse/7-description}

% \cleardoublepage




%%%%%%%%%%%%%%%%%%%%%%%%%%%%%%%%%%%%%%%%%%%%%%%%%%%%%%%%%%%%%%%%%%%%%%%%%%%%%%%%%%%%%%%%%%%%%%%%%%%
%%%%%%%%%%%%%%%%%%%%%%%%%%%%%%%%%%%%%%%%%%%%%%%%%%%%%%%%%%%%%%%%%%%%%%%%%%%%%%%%%%%%%%%%%%%%%%%%%%%
\part*{Contribution}
\addcontentsline{toc}{part}{Contribution}

\chapter{Approche et modélisation}\label{chap:mod}
\section{Principes de l'approche}\label{sec:principes}
\section{Modélisation conceptuelle}\label{sec:concept}

\chapter{Mise en oeuvre}\label{chap:op}
\section{Choix d'un langage}\label{sec:ln}
\section{Opérationnalisation}\label{sec:op}


% \cleardoublepage


%%%%%%%%%%%%%%%%%%%%%%%%%%%%%%%%%%%%%%%%%%%%%%%%%%%%%%%%%%%%%%%%%%%%%%%%%%%%%%%%%%%%%%%%%%%%%%%%%%%
%%%%%%%%%%%%%%%%%%%%%%%%%%%%%%%%%%%%%%%%%%%%%%%%%%%%%%%%%%%%%%%%%%%%%%%%%%%%%%%%%%%%%%%%%%%%%%%%%%%
\part*{Discussion}
\addcontentsline{toc}{part}{Discussion}
\chapter{Applications et validation}\label{chap:app}
% Le modèle évolue en fonction des besoins des membres du projet MediaMap. Les développeurs des applications sont susceptible d"étendre ou d'affiner le modèle suivant les besoins qu'ils rencontrent. Il ne s'agit pas encore d'un standard figé mais bien d'un effort en cours et en collaboration avec les utilisateurs. Nous nous efforçons donc d'adopter une modélisation et un vocabulaire proche de leur contexte de travail en vue de privilégier l'adoption du modèle, son utilisation et sa réappropriation. Ensuite, il s'agit de formaliser la modélisation afin de pouvoir l'opérationnaliser et développer des applications qui produisent des descriptions suivant ce modèle ou se nourrissent d'elles pour restituer une perspective métier à ces utilisateurs. 
\section{Applications du projet MediaMap}\label{sec:app}
\section{Expérimentation du projet MediaMap}\label{sec:xp}
\section{Validation}\label{sec:val}

\chapter*{Conclusion}\label{chap:cc}
\addcontentsline{toc}{chapter}{Conclusion}

% \chapter{}\label{c:}
% \section{}\label{s:}

% \cleardoublepage

%%%%%%%%%%%%%%%%%%%%%%%%%%%%%%%%%%%%%%%%%%%%%%%%%%%%%%%%%%%%%%%%%%%%%%%%%%%%%%%%%%%%%%%%%%%%%%%%%%%
%%%%%%%%%%%%%%%%%%%%%%%%%%%%%%%%%%%%%%%%%%%%%%%%%%%%%%%%%%%%%%%%%%%%%%%%%%%%%%%%%%%%%%%%%%%%%%%%%%%
%%%%%%%%%%%%%%%%%%%%%%%%%%%%%%%%%%%%%%%%%%%%%%%%%%%%%%%%%%%%%%%%%%%%%%%%%%%%%%%%%%%%%%%%%%%%%%%%%%%
%%%%%%%%%%%%%%%%%%%%%%%%%%%%%%%%%%%%%%%%%%%%%%%%%%%%%%%%%%%%%%%%%%%%%%%%%%%%%%%%%%%%%%%%%%%%%%%%%%%
%%%%%%%%%%%%%%%%%%%%%%%%%%%%%%%%%%%%%%%%%%%%%%%%%%%%%%%%%%%%%%%%%%%%%%%%%%%%%%%%%%%%%%%%%%%%%%%%%%%
%%%%%%%%%%%%%%%%%%%%%%%%%%%%%%%%%%%%%%%%%%%%%%%%%%%%%%%%%%%%%%%%%%%%%%%%%%%%%%%%%%%%%%%%%%%%%%%%%%%
%%%%%%%%%%%%%%%%%%%%%%%%%%%%%%%%%%%%%%%%%%%%%%%%%%%%%%%%%%%%%%%%%%%%%%%%%%%%%%%%%%%%%%%%%%%%%%%%%%%
%%%%%%%%%%%%%%%%%%%%%%%%%%%%%%%%%%%%%%%%%%%%%%%%%%%%%%%%%%%%%%%%%%%%%%%%%%%%%%%%%%%%%%%%%%%%%%%%%%%
%%%%%%%%%%%%%%%%%%%%%%%%%%%%%%%%%%%%%%%%%%%%%%%%%%%%%%%%%%%%%%%%%%%%%%%%%%%%%%%%%%%%%%%%%%%%%%%%%%%
%%%%%%%%%%%%%%%%%%%%%%%%%%%%%%%%%%%%%%%%%%%%%%%%%%%%%%%%%%%%%%%%%%%%%%%%%%%%%%%%%%%%%%%%%%%%%%%%%%%
%%%%%%%%%%%%%%%%%%%%%%%%%%%%%%%%%%%%%%%%%%%%%%%%%%%%%%%%%%%%%%%%%%%%%%%%%%%%%%%%%%%%%%%%%%%%%%%%%%%
% \appendix
% \part*{Annexes}
% \rehead{\headingsf\scshape Annexe~\thechapter}
% \noappendicestocpagenum
% \addappheadtotoc

% \cleardoublepage

%%%%%%%%%%%%%%%%%%%%%%%%%%%%%%%%%%%%%%%%%%%%%%%%%%%%%%%%%%%%%%%%%%%%%%%%%%%%%%%%%%%%%%%%%%%%%%%%%%%
%%%%%%%%%%%%%%%%%%%%%%%%%%%%%%%%%%%%%%%%%%%%%%%%%%%%%%%%%%%%%%%%%%%%%%%%%%%%%%%%%%%%%%%%%%%%%%%%%%%
%%%%%%%%%%%%%%%%%%%%%%%%%%%%%%%%%%%%%%%%%%%%%%%%%%%%%%%%%%%%%%%%%%%%%%%%%%%%%%%%%%%%%%%%%%%%%%%%%%%
% \chapter{D'un Web à l'autre : les paradigmes de la lecture informatique}\label{a:webs}
% \KOMAoptions{twoside=no}

% \pagestyle{empty}

% \input{PERRON/GARDE}

% \newpage

% ~

% \cleardoublepage

% \pagestyle{empty}

% \input{PERRON/REMERCIEMENTS}

% \cleardoublepage

% \KOMAoptions{twoside=yes}

\frontmatter

\pagestyle{scrheadings}

\shorttableofcontents{Sommaire}{2}

\cleardoublepage


%%%%%%%%%%%%%%%%%%%%%%%%%%%%%%%%%%%%%%%%%%%%%%%%%%%%%%%%%%%%%%%%%%%%%%%%%%%%%%%%%%%%%%%%%%%%%%%%%%%
%%%%%%%%%%%%%%%%%%%%%%%%%%%%%%%%%%%%%%%%%%%%%%%%%%%%%%%%%%%%%%%%%%%%%%%%%%%%%%%%%%%%%%%%%%%%%%%%%%%
\mainmatter

% \pagestyle{empty}

% ~

% \bigskip

% \vspace{11em}

% \bigskip

% \epigraphii{La totalité est la non vérité.}{Adorno, \ita{Minima Moralia}}

% \cleardoublepage
\addcontentsline{toc}{part}{État de l'Art}
\pagestyle{empty}

~
\bigskip

\vspace{11em}

\bigskip

\epigraphii{Se demander si un ordinateur peut penser ... est aussi intéressant que de se demander si un sous-marin peut nager.}{ Edsger Wybe Dijkstra, \ita{The threats to computing science}}

\pagestyle{scrheadings}

%%%%%%%%%%%%%%%%%%%%%%%%%%%%%%%%%%%%%%%%%%%%%%%%%%%%%%%%%%%%%%%%%%%%%%%%%%%%%%%%%%%%%%%%%%%%%%%%%%%
%%%%%%%%%%%%%%%%%%%%%%%%%%%%%%%%%%%%%%%%%%%%%%%%%%%%%%%%%%%%%%%%%%%%%%%%%%%%%%%%%%%%%%%%%%%%%%%%%%%
\part*{Exposition}
\addcontentsline{toc}{part}{Exposition}

\input{CONTENU/0-expo/1-intro}
\input{CONTENU/0-expo/2-motiv}
\input{CONTENU/0-expo/3-prodav}
\input{CONTENU/0-expo/4-besoins}
\input{CONTENU/0-expo/5-problo}

% \cleardoublepage





%%%%%%%%%%%%%%%%%%%%%%%%%%%%%%%%%%%%%%%%%%%%%%%%%%%%%%%%%%%%%%%%%%%%%%%%%%%%%%%%%%%%%%%%%%%%%%%%%%%
%%%%%%%%%%%%%%%%%%%%%%%%%%%%%%%%%%%%%%%%%%%%%%%%%%%%%%%%%%%%%%%%%%%%%%%%%%%%%%%%%%%%%%%%%%%%%%%%%%%
\part*{État de l'Art}
%%%%%%%%%%%%%%%%%%%%%%%%%%%%%%%%%%%%%%%%%%%%%%%%%%%%%%%%%%%%%%%%%%%%%%%%%%%%%%%%%%%%%%%%%%%%%%%%%%%
\input{CONTENU/1-analyse/0-cdc}
\input{CONTENU/1-analyse/1-ontologie}
\input{CONTENU/1-analyse/2-methode}
\input{CONTENU/1-analyse/3-modele}

% % \cleardoublepage

% %%%%%%%%%%%%%%%%%%%%%%%%%%%%%%%%%%%%%%%%%%%%%%%%%%%%%%%%%%%%%%%%%%%%%%%%%%%%%%%%%%%%%%%%%%%%%%%%%%%
\input{CONTENU/1-analyse/4-cdc-av}
\input{CONTENU/1-analyse/5-dav}
\input{CONTENU/1-analyse/6-gestion}
\input{CONTENU/1-analyse/7-description}

% \cleardoublepage




%%%%%%%%%%%%%%%%%%%%%%%%%%%%%%%%%%%%%%%%%%%%%%%%%%%%%%%%%%%%%%%%%%%%%%%%%%%%%%%%%%%%%%%%%%%%%%%%%%%
%%%%%%%%%%%%%%%%%%%%%%%%%%%%%%%%%%%%%%%%%%%%%%%%%%%%%%%%%%%%%%%%%%%%%%%%%%%%%%%%%%%%%%%%%%%%%%%%%%%
\part*{Contribution}
\addcontentsline{toc}{part}{Contribution}

\chapter{Approche et modélisation}\label{chap:mod}
\section{Principes de l'approche}\label{sec:principes}
\section{Modélisation conceptuelle}\label{sec:concept}

\chapter{Mise en oeuvre}\label{chap:op}
\section{Choix d'un langage}\label{sec:ln}
\section{Opérationnalisation}\label{sec:op}


% \cleardoublepage


%%%%%%%%%%%%%%%%%%%%%%%%%%%%%%%%%%%%%%%%%%%%%%%%%%%%%%%%%%%%%%%%%%%%%%%%%%%%%%%%%%%%%%%%%%%%%%%%%%%
%%%%%%%%%%%%%%%%%%%%%%%%%%%%%%%%%%%%%%%%%%%%%%%%%%%%%%%%%%%%%%%%%%%%%%%%%%%%%%%%%%%%%%%%%%%%%%%%%%%
\part*{Discussion}
\addcontentsline{toc}{part}{Discussion}
\chapter{Applications et validation}\label{chap:app}
% Le modèle évolue en fonction des besoins des membres du projet MediaMap. Les développeurs des applications sont susceptible d"étendre ou d'affiner le modèle suivant les besoins qu'ils rencontrent. Il ne s'agit pas encore d'un standard figé mais bien d'un effort en cours et en collaboration avec les utilisateurs. Nous nous efforçons donc d'adopter une modélisation et un vocabulaire proche de leur contexte de travail en vue de privilégier l'adoption du modèle, son utilisation et sa réappropriation. Ensuite, il s'agit de formaliser la modélisation afin de pouvoir l'opérationnaliser et développer des applications qui produisent des descriptions suivant ce modèle ou se nourrissent d'elles pour restituer une perspective métier à ces utilisateurs. 
\section{Applications du projet MediaMap}\label{sec:app}
\section{Expérimentation du projet MediaMap}\label{sec:xp}
\section{Validation}\label{sec:val}

\chapter*{Conclusion}\label{chap:cc}
\addcontentsline{toc}{chapter}{Conclusion}

% \chapter{}\label{c:}
% \section{}\label{s:}

% \cleardoublepage

%%%%%%%%%%%%%%%%%%%%%%%%%%%%%%%%%%%%%%%%%%%%%%%%%%%%%%%%%%%%%%%%%%%%%%%%%%%%%%%%%%%%%%%%%%%%%%%%%%%
%%%%%%%%%%%%%%%%%%%%%%%%%%%%%%%%%%%%%%%%%%%%%%%%%%%%%%%%%%%%%%%%%%%%%%%%%%%%%%%%%%%%%%%%%%%%%%%%%%%
%%%%%%%%%%%%%%%%%%%%%%%%%%%%%%%%%%%%%%%%%%%%%%%%%%%%%%%%%%%%%%%%%%%%%%%%%%%%%%%%%%%%%%%%%%%%%%%%%%%
%%%%%%%%%%%%%%%%%%%%%%%%%%%%%%%%%%%%%%%%%%%%%%%%%%%%%%%%%%%%%%%%%%%%%%%%%%%%%%%%%%%%%%%%%%%%%%%%%%%
%%%%%%%%%%%%%%%%%%%%%%%%%%%%%%%%%%%%%%%%%%%%%%%%%%%%%%%%%%%%%%%%%%%%%%%%%%%%%%%%%%%%%%%%%%%%%%%%%%%
%%%%%%%%%%%%%%%%%%%%%%%%%%%%%%%%%%%%%%%%%%%%%%%%%%%%%%%%%%%%%%%%%%%%%%%%%%%%%%%%%%%%%%%%%%%%%%%%%%%
%%%%%%%%%%%%%%%%%%%%%%%%%%%%%%%%%%%%%%%%%%%%%%%%%%%%%%%%%%%%%%%%%%%%%%%%%%%%%%%%%%%%%%%%%%%%%%%%%%%
%%%%%%%%%%%%%%%%%%%%%%%%%%%%%%%%%%%%%%%%%%%%%%%%%%%%%%%%%%%%%%%%%%%%%%%%%%%%%%%%%%%%%%%%%%%%%%%%%%%
%%%%%%%%%%%%%%%%%%%%%%%%%%%%%%%%%%%%%%%%%%%%%%%%%%%%%%%%%%%%%%%%%%%%%%%%%%%%%%%%%%%%%%%%%%%%%%%%%%%
%%%%%%%%%%%%%%%%%%%%%%%%%%%%%%%%%%%%%%%%%%%%%%%%%%%%%%%%%%%%%%%%%%%%%%%%%%%%%%%%%%%%%%%%%%%%%%%%%%%
%%%%%%%%%%%%%%%%%%%%%%%%%%%%%%%%%%%%%%%%%%%%%%%%%%%%%%%%%%%%%%%%%%%%%%%%%%%%%%%%%%%%%%%%%%%%%%%%%%%
% \appendix
% \part*{Annexes}
% \rehead{\headingsf\scshape Annexe~\thechapter}
% \noappendicestocpagenum
% \addappheadtotoc

% \cleardoublepage

%%%%%%%%%%%%%%%%%%%%%%%%%%%%%%%%%%%%%%%%%%%%%%%%%%%%%%%%%%%%%%%%%%%%%%%%%%%%%%%%%%%%%%%%%%%%%%%%%%%
%%%%%%%%%%%%%%%%%%%%%%%%%%%%%%%%%%%%%%%%%%%%%%%%%%%%%%%%%%%%%%%%%%%%%%%%%%%%%%%%%%%%%%%%%%%%%%%%%%%
%%%%%%%%%%%%%%%%%%%%%%%%%%%%%%%%%%%%%%%%%%%%%%%%%%%%%%%%%%%%%%%%%%%%%%%%%%%%%%%%%%%%%%%%%%%%%%%%%%%
% \chapter{D'un Web à l'autre : les paradigmes de la lecture informatique}\label{a:webs}
% \input{CONTENU/A1.WEBS/0}

% \cleardoublepage

%%%%%%%%%%%%%%%%%%%%%%%%%%%%%%%%%%%%%%%%%%%%%%%%%%%%%%%%%%%%%%%%%%%%%%%%%%%%%%%%%%%%%%%%%%%%%%%%%%%
%%%%%%%%%%%%%%%%%%%%%%%%%%%%%%%%%%%%%%%%%%%%%%%%%%%%%%%%%%%%%%%%%%%%%%%%%%%%%%%%%%%%%%%%%%%%%%%%%%%
%%%%%%%%%%%%%%%%%%%%%%%%%%%%%%%%%%%%%%%%%%%%%%%%%%%%%%%%%%%%%%%%%%%%%%%%%%%%%%%%%%%%%%%%%%%%%%%%%%%
% \chapter[Supports, outils et espaces --- Les mutations des opérations lectoriales]{Supports, outils et espaces\\\scriptsize{Les mutations des opérations lectoriales}}\label{a:hist}
% \input{CONTENU/A2.HIST/H0}
% \input{CONTENU/A2.HIST/H1}
% \input{CONTENU/A2.HIST/H2}
% \input{CONTENU/A2.HIST/H3}
% \input{CONTENU/A2.HIST/H4}
% \input{CONTENU/A2.HIST/H5}

% \cleardoublepage

%%%%%%%%%%%%%%%%%%%%%%%%%%%%%%%%%%%%%%%%%%%%%%%%%%%%%%%%%%%%%%%%%%%%%%%%%%%%%%%%%%%%%%%%%%%%%%%%%%%
%%%%%%%%%%%%%%%%%%%%%%%%%%%%%%%%%%%%%%%%%%%%%%%%%%%%%%%%%%%%%%%%%%%%%%%%%%%%%%%%%%%%%%%%%%%%%%%%%%%
%%%%%%%%%%%%%%%%%%%%%%%%%%%%%%%%%%%%%%%%%%%%%%%%%%%%%%%%%%%%%%%%%%%%%%%%%%%%%%%%%%%%%%%%%%%%%%%%%%%
% \chapter{Fragments pertinents issus des entretiens avec des lecteurs savants}\label{a:preint}
% \input{CONTENU/A3.PREINT.LEFTOVERS/0}

% \cleardoublepage

%%%%%%%%%%%%%%%%%%%%%%%%%%%%%%%%%%%%%%%%%%%%%%%%%%%%%%%%%%%%%%%%%%%%%%%%%%%%%%%%%%%%%%%%%%%%%%%%%%%
%%%%%%%%%%%%%%%%%%%%%%%%%%%%%%%%%%%%%%%%%%%%%%%%%%%%%%%%%%%%%%%%%%%%%%%%%%%%%%%%%%%%%%%%%%%%%%%%%%%
%%%%%%%%%%%%%%%%%%%%%%%%%%%%%%%%%%%%%%%%%%%%%%%%%%%%%%%%%%%%%%%%%%%%%%%%%%%%%%%%%%%%%%%%%%%%%%%%%%%
% \input{CONTENU/A4.SC01/QuestionnaireSC01} % → Les questions

% % → Quelques macros pour la structuration des questions
% \newcommand{\questionnaireSCUN}[1]{\section{#1}}
% \newcommand{\sectionquestionnaire}[1]{\subsection*{#1}}
% \newcommand{\question}[1]{\par\bigskip\textbf{#1}\par}
% \newcommand{\sousquestion}[1]{\par\medskip\textbf{\textcolor[rgb]{0.0,0.0,0.0}{--- \textbf{\textit{#1}}}}\par} % parcol = 0.73
% \newcommand{\reponse}[1]{\ignorespaces#1}
% \newcommand{\poimp}[1]{\bigskip\textsc{\textbf{#1}}\par}

% \chapter[Questionnaires de retour d'utilisation]{Questionnaires de retour d'utilisation\\{\scriptsize Expérience : «commentaire composé multimédia»}}\label{a:sc01}

% \ita{Note 1 : les réponses des étudiants sont rapportées sans intervention de notre part sur l'orthographe ou la syntaxe.}

% \ita{Note 2 : le prototype a été présenté aux étudiants sous le nom de «Verena».}

% \questionnaireSCUN{Étudiant \as}\input{CONTENU/A4.SC01/as}			% Alain SAAS
% \questionnaireSCUN{Étudiant \cl}\input{CONTENU/A4.SC01/cl}			% Cecile LABORDE
% \questionnaireSCUN{Étudiant \dao}\input{CONTENU/A4.SC01/dao}		% Djamila AIT OUADDA
% \questionnaireSCUN{Étudiant \ds}\input{CONTENU/A4.SC01/ds}			% Delphine SZYMCZAK
% \questionnaireSCUN{Étudiant \dd}\input{CONTENU/A4.SC01/dd}			% Dorine DUFOUR
% \questionnaireSCUN{Étudiant \glb}\input{CONTENU/A4.SC01/glb}		% Gabrielle LE BIHAN
% \questionnaireSCUN{Étudiant \rb}\input{CONTENU/A4.SC01/rb}			% Romain BODINIER

% \cleardoublepage

%%%%%%%%%%%%%%%%%%%%%%%%%%%%%%%%%%%%%%%%%%%%%%%%%%%%%%%%%%%%%%%%%%%%%%%%%%%%%%%%%%%%%%%%%%%%%%%%%%%
%%%%%%%%%%%%%%%%%%%%%%%%%%%%%%%%%%%%%%%%%%%%%%%%%%%%%%%%%%%%%%%%%%%%%%%%%%%%%%%%%%%%%%%%%%%%%%%%%%%
%%%%%%%%%%%%%%%%%%%%%%%%%%%%%%%%%%%%%%%%%%%%%%%%%%%%%%%%%%%%%%%%%%%%%%%%%%%%%%%%%%%%%%%%%%%%%%%%%%%
%%%%%%%%%%%%%%%%%%%%%%%%%%%%%%%%%%%%%%%%%%%%%%%%%%%%%%%%%%%%%%%%%%%%%%%%%%%%%%%%%%%%%%%%%%%%%%%%%%%
%%%%%%%%%%%%%%%%%%%%%%%%%%%%%%%%%%%%%%%%%%%%%%%%%%%%%%%%%%%%%%%%%%%%%%%%%%%%%%%%%%%%%%%%%%%%%%%%%%%
%%%%%%%%%%%%%%%%%%%%%%%%%%%%%%%%%%%%%%%%%%%%%%%%%%%%%%%%%%%%%%%%%%%%%%%%%%%%%%%%%%%%%%%%%%%%%%%%%%%
%%%%%%%%%%%%%%%%%%%%%%%%%%%%%%%%%%%%%%%%%%%%%%%%%%%%%%%%%%%%%%%%%%%%%%%%%%%%%%%%%%%%%%%%%%%%%%%%%%%
%%%%%%%%%%%%%%%%%%%%%%%%%%%%%%%%%%%%%%%%%%%%%%%%%%%%%%%%%%%%%%%%%%%%%%%%%%%%%%%%%%%%%%%%%%%%%%%%%%%
%%%%%%%%%%%%%%%%%%%%%%%%%%%%%%%%%%%%%%%%%%%%%%%%%%%%%%%%%%%%%%%%%%%%%%%%%%%%%%%%%%%%%%%%%%%%%%%%%%%
%%%%%%%%%%%%%%%%%%%%%%%%%%%%%%%%%%%%%%%%%%%%%%%%%%%%%%%%%%%%%%%%%%%%%%%%%%%%%%%%%%%%%%%%%%%%%%%%%%%
%%%%%%%%%%%%%%%%%%%%%%%%%%%%%%%%%%%%%%%%%%%%%%%%%%%%%%%%%%%%%%%%%%%%%%%%%%%%%%%%%%%%%%%%%%%%%%%%%%%

\backmatter

\rehead{\headingsf\scshape\headmark}
\lohead{\headingsf\scshape\headmark}

% \addcontentsline{toc}{chapter}{Bibliographie}
% \nocite{*}
\printbibliography[maxnames=11]

\cleardoublepage
\setcounter{tocdepth}{3}
\tableofcontents


% \cleardoublepage

%%%%%%%%%%%%%%%%%%%%%%%%%%%%%%%%%%%%%%%%%%%%%%%%%%%%%%%%%%%%%%%%%%%%%%%%%%%%%%%%%%%%%%%%%%%%%%%%%%%
%%%%%%%%%%%%%%%%%%%%%%%%%%%%%%%%%%%%%%%%%%%%%%%%%%%%%%%%%%%%%%%%%%%%%%%%%%%%%%%%%%%%%%%%%%%%%%%%%%%
%%%%%%%%%%%%%%%%%%%%%%%%%%%%%%%%%%%%%%%%%%%%%%%%%%%%%%%%%%%%%%%%%%%%%%%%%%%%%%%%%%%%%%%%%%%%%%%%%%%
% \chapter[Supports, outils et espaces --- Les mutations des opérations lectoriales]{Supports, outils et espaces\\\scriptsize{Les mutations des opérations lectoriales}}\label{a:hist}
% \input{CONTENU/A2.HIST/H0}
% \input{CONTENU/A2.HIST/H1}
% \input{CONTENU/A2.HIST/H2}
% \input{CONTENU/A2.HIST/H3}
% \input{CONTENU/A2.HIST/H4}
% \input{CONTENU/A2.HIST/H5}

% \cleardoublepage

%%%%%%%%%%%%%%%%%%%%%%%%%%%%%%%%%%%%%%%%%%%%%%%%%%%%%%%%%%%%%%%%%%%%%%%%%%%%%%%%%%%%%%%%%%%%%%%%%%%
%%%%%%%%%%%%%%%%%%%%%%%%%%%%%%%%%%%%%%%%%%%%%%%%%%%%%%%%%%%%%%%%%%%%%%%%%%%%%%%%%%%%%%%%%%%%%%%%%%%
%%%%%%%%%%%%%%%%%%%%%%%%%%%%%%%%%%%%%%%%%%%%%%%%%%%%%%%%%%%%%%%%%%%%%%%%%%%%%%%%%%%%%%%%%%%%%%%%%%%
% \chapter{Fragments pertinents issus des entretiens avec des lecteurs savants}\label{a:preint}
% \KOMAoptions{twoside=no}

% \pagestyle{empty}

% \input{PERRON/GARDE}

% \newpage

% ~

% \cleardoublepage

% \pagestyle{empty}

% \input{PERRON/REMERCIEMENTS}

% \cleardoublepage

% \KOMAoptions{twoside=yes}

\frontmatter

\pagestyle{scrheadings}

\shorttableofcontents{Sommaire}{2}

\cleardoublepage


%%%%%%%%%%%%%%%%%%%%%%%%%%%%%%%%%%%%%%%%%%%%%%%%%%%%%%%%%%%%%%%%%%%%%%%%%%%%%%%%%%%%%%%%%%%%%%%%%%%
%%%%%%%%%%%%%%%%%%%%%%%%%%%%%%%%%%%%%%%%%%%%%%%%%%%%%%%%%%%%%%%%%%%%%%%%%%%%%%%%%%%%%%%%%%%%%%%%%%%
\mainmatter

% \pagestyle{empty}

% ~

% \bigskip

% \vspace{11em}

% \bigskip

% \epigraphii{La totalité est la non vérité.}{Adorno, \ita{Minima Moralia}}

% \cleardoublepage
\addcontentsline{toc}{part}{État de l'Art}
\pagestyle{empty}

~
\bigskip

\vspace{11em}

\bigskip

\epigraphii{Se demander si un ordinateur peut penser ... est aussi intéressant que de se demander si un sous-marin peut nager.}{ Edsger Wybe Dijkstra, \ita{The threats to computing science}}

\pagestyle{scrheadings}

%%%%%%%%%%%%%%%%%%%%%%%%%%%%%%%%%%%%%%%%%%%%%%%%%%%%%%%%%%%%%%%%%%%%%%%%%%%%%%%%%%%%%%%%%%%%%%%%%%%
%%%%%%%%%%%%%%%%%%%%%%%%%%%%%%%%%%%%%%%%%%%%%%%%%%%%%%%%%%%%%%%%%%%%%%%%%%%%%%%%%%%%%%%%%%%%%%%%%%%
\part*{Exposition}
\addcontentsline{toc}{part}{Exposition}

\input{CONTENU/0-expo/1-intro}
\input{CONTENU/0-expo/2-motiv}
\input{CONTENU/0-expo/3-prodav}
\input{CONTENU/0-expo/4-besoins}
\input{CONTENU/0-expo/5-problo}

% \cleardoublepage





%%%%%%%%%%%%%%%%%%%%%%%%%%%%%%%%%%%%%%%%%%%%%%%%%%%%%%%%%%%%%%%%%%%%%%%%%%%%%%%%%%%%%%%%%%%%%%%%%%%
%%%%%%%%%%%%%%%%%%%%%%%%%%%%%%%%%%%%%%%%%%%%%%%%%%%%%%%%%%%%%%%%%%%%%%%%%%%%%%%%%%%%%%%%%%%%%%%%%%%
\part*{État de l'Art}
%%%%%%%%%%%%%%%%%%%%%%%%%%%%%%%%%%%%%%%%%%%%%%%%%%%%%%%%%%%%%%%%%%%%%%%%%%%%%%%%%%%%%%%%%%%%%%%%%%%
\input{CONTENU/1-analyse/0-cdc}
\input{CONTENU/1-analyse/1-ontologie}
\input{CONTENU/1-analyse/2-methode}
\input{CONTENU/1-analyse/3-modele}

% % \cleardoublepage

% %%%%%%%%%%%%%%%%%%%%%%%%%%%%%%%%%%%%%%%%%%%%%%%%%%%%%%%%%%%%%%%%%%%%%%%%%%%%%%%%%%%%%%%%%%%%%%%%%%%
\input{CONTENU/1-analyse/4-cdc-av}
\input{CONTENU/1-analyse/5-dav}
\input{CONTENU/1-analyse/6-gestion}
\input{CONTENU/1-analyse/7-description}

% \cleardoublepage




%%%%%%%%%%%%%%%%%%%%%%%%%%%%%%%%%%%%%%%%%%%%%%%%%%%%%%%%%%%%%%%%%%%%%%%%%%%%%%%%%%%%%%%%%%%%%%%%%%%
%%%%%%%%%%%%%%%%%%%%%%%%%%%%%%%%%%%%%%%%%%%%%%%%%%%%%%%%%%%%%%%%%%%%%%%%%%%%%%%%%%%%%%%%%%%%%%%%%%%
\part*{Contribution}
\addcontentsline{toc}{part}{Contribution}

\chapter{Approche et modélisation}\label{chap:mod}
\section{Principes de l'approche}\label{sec:principes}
\section{Modélisation conceptuelle}\label{sec:concept}

\chapter{Mise en oeuvre}\label{chap:op}
\section{Choix d'un langage}\label{sec:ln}
\section{Opérationnalisation}\label{sec:op}


% \cleardoublepage


%%%%%%%%%%%%%%%%%%%%%%%%%%%%%%%%%%%%%%%%%%%%%%%%%%%%%%%%%%%%%%%%%%%%%%%%%%%%%%%%%%%%%%%%%%%%%%%%%%%
%%%%%%%%%%%%%%%%%%%%%%%%%%%%%%%%%%%%%%%%%%%%%%%%%%%%%%%%%%%%%%%%%%%%%%%%%%%%%%%%%%%%%%%%%%%%%%%%%%%
\part*{Discussion}
\addcontentsline{toc}{part}{Discussion}
\chapter{Applications et validation}\label{chap:app}
% Le modèle évolue en fonction des besoins des membres du projet MediaMap. Les développeurs des applications sont susceptible d"étendre ou d'affiner le modèle suivant les besoins qu'ils rencontrent. Il ne s'agit pas encore d'un standard figé mais bien d'un effort en cours et en collaboration avec les utilisateurs. Nous nous efforçons donc d'adopter une modélisation et un vocabulaire proche de leur contexte de travail en vue de privilégier l'adoption du modèle, son utilisation et sa réappropriation. Ensuite, il s'agit de formaliser la modélisation afin de pouvoir l'opérationnaliser et développer des applications qui produisent des descriptions suivant ce modèle ou se nourrissent d'elles pour restituer une perspective métier à ces utilisateurs. 
\section{Applications du projet MediaMap}\label{sec:app}
\section{Expérimentation du projet MediaMap}\label{sec:xp}
\section{Validation}\label{sec:val}

\chapter*{Conclusion}\label{chap:cc}
\addcontentsline{toc}{chapter}{Conclusion}

% \chapter{}\label{c:}
% \section{}\label{s:}

% \cleardoublepage

%%%%%%%%%%%%%%%%%%%%%%%%%%%%%%%%%%%%%%%%%%%%%%%%%%%%%%%%%%%%%%%%%%%%%%%%%%%%%%%%%%%%%%%%%%%%%%%%%%%
%%%%%%%%%%%%%%%%%%%%%%%%%%%%%%%%%%%%%%%%%%%%%%%%%%%%%%%%%%%%%%%%%%%%%%%%%%%%%%%%%%%%%%%%%%%%%%%%%%%
%%%%%%%%%%%%%%%%%%%%%%%%%%%%%%%%%%%%%%%%%%%%%%%%%%%%%%%%%%%%%%%%%%%%%%%%%%%%%%%%%%%%%%%%%%%%%%%%%%%
%%%%%%%%%%%%%%%%%%%%%%%%%%%%%%%%%%%%%%%%%%%%%%%%%%%%%%%%%%%%%%%%%%%%%%%%%%%%%%%%%%%%%%%%%%%%%%%%%%%
%%%%%%%%%%%%%%%%%%%%%%%%%%%%%%%%%%%%%%%%%%%%%%%%%%%%%%%%%%%%%%%%%%%%%%%%%%%%%%%%%%%%%%%%%%%%%%%%%%%
%%%%%%%%%%%%%%%%%%%%%%%%%%%%%%%%%%%%%%%%%%%%%%%%%%%%%%%%%%%%%%%%%%%%%%%%%%%%%%%%%%%%%%%%%%%%%%%%%%%
%%%%%%%%%%%%%%%%%%%%%%%%%%%%%%%%%%%%%%%%%%%%%%%%%%%%%%%%%%%%%%%%%%%%%%%%%%%%%%%%%%%%%%%%%%%%%%%%%%%
%%%%%%%%%%%%%%%%%%%%%%%%%%%%%%%%%%%%%%%%%%%%%%%%%%%%%%%%%%%%%%%%%%%%%%%%%%%%%%%%%%%%%%%%%%%%%%%%%%%
%%%%%%%%%%%%%%%%%%%%%%%%%%%%%%%%%%%%%%%%%%%%%%%%%%%%%%%%%%%%%%%%%%%%%%%%%%%%%%%%%%%%%%%%%%%%%%%%%%%
%%%%%%%%%%%%%%%%%%%%%%%%%%%%%%%%%%%%%%%%%%%%%%%%%%%%%%%%%%%%%%%%%%%%%%%%%%%%%%%%%%%%%%%%%%%%%%%%%%%
%%%%%%%%%%%%%%%%%%%%%%%%%%%%%%%%%%%%%%%%%%%%%%%%%%%%%%%%%%%%%%%%%%%%%%%%%%%%%%%%%%%%%%%%%%%%%%%%%%%
% \appendix
% \part*{Annexes}
% \rehead{\headingsf\scshape Annexe~\thechapter}
% \noappendicestocpagenum
% \addappheadtotoc

% \cleardoublepage

%%%%%%%%%%%%%%%%%%%%%%%%%%%%%%%%%%%%%%%%%%%%%%%%%%%%%%%%%%%%%%%%%%%%%%%%%%%%%%%%%%%%%%%%%%%%%%%%%%%
%%%%%%%%%%%%%%%%%%%%%%%%%%%%%%%%%%%%%%%%%%%%%%%%%%%%%%%%%%%%%%%%%%%%%%%%%%%%%%%%%%%%%%%%%%%%%%%%%%%
%%%%%%%%%%%%%%%%%%%%%%%%%%%%%%%%%%%%%%%%%%%%%%%%%%%%%%%%%%%%%%%%%%%%%%%%%%%%%%%%%%%%%%%%%%%%%%%%%%%
% \chapter{D'un Web à l'autre : les paradigmes de la lecture informatique}\label{a:webs}
% \input{CONTENU/A1.WEBS/0}

% \cleardoublepage

%%%%%%%%%%%%%%%%%%%%%%%%%%%%%%%%%%%%%%%%%%%%%%%%%%%%%%%%%%%%%%%%%%%%%%%%%%%%%%%%%%%%%%%%%%%%%%%%%%%
%%%%%%%%%%%%%%%%%%%%%%%%%%%%%%%%%%%%%%%%%%%%%%%%%%%%%%%%%%%%%%%%%%%%%%%%%%%%%%%%%%%%%%%%%%%%%%%%%%%
%%%%%%%%%%%%%%%%%%%%%%%%%%%%%%%%%%%%%%%%%%%%%%%%%%%%%%%%%%%%%%%%%%%%%%%%%%%%%%%%%%%%%%%%%%%%%%%%%%%
% \chapter[Supports, outils et espaces --- Les mutations des opérations lectoriales]{Supports, outils et espaces\\\scriptsize{Les mutations des opérations lectoriales}}\label{a:hist}
% \input{CONTENU/A2.HIST/H0}
% \input{CONTENU/A2.HIST/H1}
% \input{CONTENU/A2.HIST/H2}
% \input{CONTENU/A2.HIST/H3}
% \input{CONTENU/A2.HIST/H4}
% \input{CONTENU/A2.HIST/H5}

% \cleardoublepage

%%%%%%%%%%%%%%%%%%%%%%%%%%%%%%%%%%%%%%%%%%%%%%%%%%%%%%%%%%%%%%%%%%%%%%%%%%%%%%%%%%%%%%%%%%%%%%%%%%%
%%%%%%%%%%%%%%%%%%%%%%%%%%%%%%%%%%%%%%%%%%%%%%%%%%%%%%%%%%%%%%%%%%%%%%%%%%%%%%%%%%%%%%%%%%%%%%%%%%%
%%%%%%%%%%%%%%%%%%%%%%%%%%%%%%%%%%%%%%%%%%%%%%%%%%%%%%%%%%%%%%%%%%%%%%%%%%%%%%%%%%%%%%%%%%%%%%%%%%%
% \chapter{Fragments pertinents issus des entretiens avec des lecteurs savants}\label{a:preint}
% \input{CONTENU/A3.PREINT.LEFTOVERS/0}

% \cleardoublepage

%%%%%%%%%%%%%%%%%%%%%%%%%%%%%%%%%%%%%%%%%%%%%%%%%%%%%%%%%%%%%%%%%%%%%%%%%%%%%%%%%%%%%%%%%%%%%%%%%%%
%%%%%%%%%%%%%%%%%%%%%%%%%%%%%%%%%%%%%%%%%%%%%%%%%%%%%%%%%%%%%%%%%%%%%%%%%%%%%%%%%%%%%%%%%%%%%%%%%%%
%%%%%%%%%%%%%%%%%%%%%%%%%%%%%%%%%%%%%%%%%%%%%%%%%%%%%%%%%%%%%%%%%%%%%%%%%%%%%%%%%%%%%%%%%%%%%%%%%%%
% \input{CONTENU/A4.SC01/QuestionnaireSC01} % → Les questions

% % → Quelques macros pour la structuration des questions
% \newcommand{\questionnaireSCUN}[1]{\section{#1}}
% \newcommand{\sectionquestionnaire}[1]{\subsection*{#1}}
% \newcommand{\question}[1]{\par\bigskip\textbf{#1}\par}
% \newcommand{\sousquestion}[1]{\par\medskip\textbf{\textcolor[rgb]{0.0,0.0,0.0}{--- \textbf{\textit{#1}}}}\par} % parcol = 0.73
% \newcommand{\reponse}[1]{\ignorespaces#1}
% \newcommand{\poimp}[1]{\bigskip\textsc{\textbf{#1}}\par}

% \chapter[Questionnaires de retour d'utilisation]{Questionnaires de retour d'utilisation\\{\scriptsize Expérience : «commentaire composé multimédia»}}\label{a:sc01}

% \ita{Note 1 : les réponses des étudiants sont rapportées sans intervention de notre part sur l'orthographe ou la syntaxe.}

% \ita{Note 2 : le prototype a été présenté aux étudiants sous le nom de «Verena».}

% \questionnaireSCUN{Étudiant \as}\input{CONTENU/A4.SC01/as}			% Alain SAAS
% \questionnaireSCUN{Étudiant \cl}\input{CONTENU/A4.SC01/cl}			% Cecile LABORDE
% \questionnaireSCUN{Étudiant \dao}\input{CONTENU/A4.SC01/dao}		% Djamila AIT OUADDA
% \questionnaireSCUN{Étudiant \ds}\input{CONTENU/A4.SC01/ds}			% Delphine SZYMCZAK
% \questionnaireSCUN{Étudiant \dd}\input{CONTENU/A4.SC01/dd}			% Dorine DUFOUR
% \questionnaireSCUN{Étudiant \glb}\input{CONTENU/A4.SC01/glb}		% Gabrielle LE BIHAN
% \questionnaireSCUN{Étudiant \rb}\input{CONTENU/A4.SC01/rb}			% Romain BODINIER

% \cleardoublepage

%%%%%%%%%%%%%%%%%%%%%%%%%%%%%%%%%%%%%%%%%%%%%%%%%%%%%%%%%%%%%%%%%%%%%%%%%%%%%%%%%%%%%%%%%%%%%%%%%%%
%%%%%%%%%%%%%%%%%%%%%%%%%%%%%%%%%%%%%%%%%%%%%%%%%%%%%%%%%%%%%%%%%%%%%%%%%%%%%%%%%%%%%%%%%%%%%%%%%%%
%%%%%%%%%%%%%%%%%%%%%%%%%%%%%%%%%%%%%%%%%%%%%%%%%%%%%%%%%%%%%%%%%%%%%%%%%%%%%%%%%%%%%%%%%%%%%%%%%%%
%%%%%%%%%%%%%%%%%%%%%%%%%%%%%%%%%%%%%%%%%%%%%%%%%%%%%%%%%%%%%%%%%%%%%%%%%%%%%%%%%%%%%%%%%%%%%%%%%%%
%%%%%%%%%%%%%%%%%%%%%%%%%%%%%%%%%%%%%%%%%%%%%%%%%%%%%%%%%%%%%%%%%%%%%%%%%%%%%%%%%%%%%%%%%%%%%%%%%%%
%%%%%%%%%%%%%%%%%%%%%%%%%%%%%%%%%%%%%%%%%%%%%%%%%%%%%%%%%%%%%%%%%%%%%%%%%%%%%%%%%%%%%%%%%%%%%%%%%%%
%%%%%%%%%%%%%%%%%%%%%%%%%%%%%%%%%%%%%%%%%%%%%%%%%%%%%%%%%%%%%%%%%%%%%%%%%%%%%%%%%%%%%%%%%%%%%%%%%%%
%%%%%%%%%%%%%%%%%%%%%%%%%%%%%%%%%%%%%%%%%%%%%%%%%%%%%%%%%%%%%%%%%%%%%%%%%%%%%%%%%%%%%%%%%%%%%%%%%%%
%%%%%%%%%%%%%%%%%%%%%%%%%%%%%%%%%%%%%%%%%%%%%%%%%%%%%%%%%%%%%%%%%%%%%%%%%%%%%%%%%%%%%%%%%%%%%%%%%%%
%%%%%%%%%%%%%%%%%%%%%%%%%%%%%%%%%%%%%%%%%%%%%%%%%%%%%%%%%%%%%%%%%%%%%%%%%%%%%%%%%%%%%%%%%%%%%%%%%%%
%%%%%%%%%%%%%%%%%%%%%%%%%%%%%%%%%%%%%%%%%%%%%%%%%%%%%%%%%%%%%%%%%%%%%%%%%%%%%%%%%%%%%%%%%%%%%%%%%%%

\backmatter

\rehead{\headingsf\scshape\headmark}
\lohead{\headingsf\scshape\headmark}

% \addcontentsline{toc}{chapter}{Bibliographie}
% \nocite{*}
\printbibliography[maxnames=11]

\cleardoublepage
\setcounter{tocdepth}{3}
\tableofcontents


% \cleardoublepage

%%%%%%%%%%%%%%%%%%%%%%%%%%%%%%%%%%%%%%%%%%%%%%%%%%%%%%%%%%%%%%%%%%%%%%%%%%%%%%%%%%%%%%%%%%%%%%%%%%%
%%%%%%%%%%%%%%%%%%%%%%%%%%%%%%%%%%%%%%%%%%%%%%%%%%%%%%%%%%%%%%%%%%%%%%%%%%%%%%%%%%%%%%%%%%%%%%%%%%%
%%%%%%%%%%%%%%%%%%%%%%%%%%%%%%%%%%%%%%%%%%%%%%%%%%%%%%%%%%%%%%%%%%%%%%%%%%%%%%%%%%%%%%%%%%%%%%%%%%%
% \input{CONTENU/A4.SC01/QuestionnaireSC01} % → Les questions

% % → Quelques macros pour la structuration des questions
% \newcommand{\questionnaireSCUN}[1]{\section{#1}}
% \newcommand{\sectionquestionnaire}[1]{\subsection*{#1}}
% \newcommand{\question}[1]{\par\bigskip\textbf{#1}\par}
% \newcommand{\sousquestion}[1]{\par\medskip\textbf{\textcolor[rgb]{0.0,0.0,0.0}{--- \textbf{\textit{#1}}}}\par} % parcol = 0.73
% \newcommand{\reponse}[1]{\ignorespaces#1}
% \newcommand{\poimp}[1]{\bigskip\textsc{\textbf{#1}}\par}

% \chapter[Questionnaires de retour d'utilisation]{Questionnaires de retour d'utilisation\\{\scriptsize Expérience : «commentaire composé multimédia»}}\label{a:sc01}

% \ita{Note 1 : les réponses des étudiants sont rapportées sans intervention de notre part sur l'orthographe ou la syntaxe.}

% \ita{Note 2 : le prototype a été présenté aux étudiants sous le nom de «Verena».}

% \questionnaireSCUN{Étudiant \as}\input{CONTENU/A4.SC01/as}			% Alain SAAS
% \questionnaireSCUN{Étudiant \cl}\input{CONTENU/A4.SC01/cl}			% Cecile LABORDE
% \questionnaireSCUN{Étudiant \dao}\input{CONTENU/A4.SC01/dao}		% Djamila AIT OUADDA
% \questionnaireSCUN{Étudiant \ds}\input{CONTENU/A4.SC01/ds}			% Delphine SZYMCZAK
% \questionnaireSCUN{Étudiant \dd}\input{CONTENU/A4.SC01/dd}			% Dorine DUFOUR
% \questionnaireSCUN{Étudiant \glb}\input{CONTENU/A4.SC01/glb}		% Gabrielle LE BIHAN
% \questionnaireSCUN{Étudiant \rb}\input{CONTENU/A4.SC01/rb}			% Romain BODINIER

% \cleardoublepage

%%%%%%%%%%%%%%%%%%%%%%%%%%%%%%%%%%%%%%%%%%%%%%%%%%%%%%%%%%%%%%%%%%%%%%%%%%%%%%%%%%%%%%%%%%%%%%%%%%%
%%%%%%%%%%%%%%%%%%%%%%%%%%%%%%%%%%%%%%%%%%%%%%%%%%%%%%%%%%%%%%%%%%%%%%%%%%%%%%%%%%%%%%%%%%%%%%%%%%%
%%%%%%%%%%%%%%%%%%%%%%%%%%%%%%%%%%%%%%%%%%%%%%%%%%%%%%%%%%%%%%%%%%%%%%%%%%%%%%%%%%%%%%%%%%%%%%%%%%%
%%%%%%%%%%%%%%%%%%%%%%%%%%%%%%%%%%%%%%%%%%%%%%%%%%%%%%%%%%%%%%%%%%%%%%%%%%%%%%%%%%%%%%%%%%%%%%%%%%%
%%%%%%%%%%%%%%%%%%%%%%%%%%%%%%%%%%%%%%%%%%%%%%%%%%%%%%%%%%%%%%%%%%%%%%%%%%%%%%%%%%%%%%%%%%%%%%%%%%%
%%%%%%%%%%%%%%%%%%%%%%%%%%%%%%%%%%%%%%%%%%%%%%%%%%%%%%%%%%%%%%%%%%%%%%%%%%%%%%%%%%%%%%%%%%%%%%%%%%%
%%%%%%%%%%%%%%%%%%%%%%%%%%%%%%%%%%%%%%%%%%%%%%%%%%%%%%%%%%%%%%%%%%%%%%%%%%%%%%%%%%%%%%%%%%%%%%%%%%%
%%%%%%%%%%%%%%%%%%%%%%%%%%%%%%%%%%%%%%%%%%%%%%%%%%%%%%%%%%%%%%%%%%%%%%%%%%%%%%%%%%%%%%%%%%%%%%%%%%%
%%%%%%%%%%%%%%%%%%%%%%%%%%%%%%%%%%%%%%%%%%%%%%%%%%%%%%%%%%%%%%%%%%%%%%%%%%%%%%%%%%%%%%%%%%%%%%%%%%%
%%%%%%%%%%%%%%%%%%%%%%%%%%%%%%%%%%%%%%%%%%%%%%%%%%%%%%%%%%%%%%%%%%%%%%%%%%%%%%%%%%%%%%%%%%%%%%%%%%%
%%%%%%%%%%%%%%%%%%%%%%%%%%%%%%%%%%%%%%%%%%%%%%%%%%%%%%%%%%%%%%%%%%%%%%%%%%%%%%%%%%%%%%%%%%%%%%%%%%%

\backmatter

\rehead{\headingsf\scshape\headmark}
\lohead{\headingsf\scshape\headmark}

% \addcontentsline{toc}{chapter}{Bibliographie}
% \nocite{*}
\printbibliography[maxnames=11]

\cleardoublepage
\setcounter{tocdepth}{3}
\tableofcontents


% \cleardoublepage

%%%%%%%%%%%%%%%%%%%%%%%%%%%%%%%%%%%%%%%%%%%%%%%%%%%%%%%%%%%%%%%%%%%%%%%%%%%%%%%%%%%%%%%%%%%%%%%%%%%
%%%%%%%%%%%%%%%%%%%%%%%%%%%%%%%%%%%%%%%%%%%%%%%%%%%%%%%%%%%%%%%%%%%%%%%%%%%%%%%%%%%%%%%%%%%%%%%%%%%
%%%%%%%%%%%%%%%%%%%%%%%%%%%%%%%%%%%%%%%%%%%%%%%%%%%%%%%%%%%%%%%%%%%%%%%%%%%%%%%%%%%%%%%%%%%%%%%%%%%
% \chapter[Supports, outils et espaces --- Les mutations des opérations lectoriales]{Supports, outils et espaces\\\scriptsize{Les mutations des opérations lectoriales}}\label{a:hist}
% \input{CONTENU/A2.HIST/H0}
% \input{CONTENU/A2.HIST/H1}
% \input{CONTENU/A2.HIST/H2}
% \input{CONTENU/A2.HIST/H3}
% \input{CONTENU/A2.HIST/H4}
% \input{CONTENU/A2.HIST/H5}

% \cleardoublepage

%%%%%%%%%%%%%%%%%%%%%%%%%%%%%%%%%%%%%%%%%%%%%%%%%%%%%%%%%%%%%%%%%%%%%%%%%%%%%%%%%%%%%%%%%%%%%%%%%%%
%%%%%%%%%%%%%%%%%%%%%%%%%%%%%%%%%%%%%%%%%%%%%%%%%%%%%%%%%%%%%%%%%%%%%%%%%%%%%%%%%%%%%%%%%%%%%%%%%%%
%%%%%%%%%%%%%%%%%%%%%%%%%%%%%%%%%%%%%%%%%%%%%%%%%%%%%%%%%%%%%%%%%%%%%%%%%%%%%%%%%%%%%%%%%%%%%%%%%%%
% \chapter{Fragments pertinents issus des entretiens avec des lecteurs savants}\label{a:preint}
% \KOMAoptions{twoside=no}

% \pagestyle{empty}

% \input{PERRON/GARDE}

% \newpage

% ~

% \cleardoublepage

% \pagestyle{empty}

% \input{PERRON/REMERCIEMENTS}

% \cleardoublepage

% \KOMAoptions{twoside=yes}

\frontmatter

\pagestyle{scrheadings}

\shorttableofcontents{Sommaire}{2}

\cleardoublepage


%%%%%%%%%%%%%%%%%%%%%%%%%%%%%%%%%%%%%%%%%%%%%%%%%%%%%%%%%%%%%%%%%%%%%%%%%%%%%%%%%%%%%%%%%%%%%%%%%%%
%%%%%%%%%%%%%%%%%%%%%%%%%%%%%%%%%%%%%%%%%%%%%%%%%%%%%%%%%%%%%%%%%%%%%%%%%%%%%%%%%%%%%%%%%%%%%%%%%%%
\mainmatter

% \pagestyle{empty}

% ~

% \bigskip

% \vspace{11em}

% \bigskip

% \epigraphii{La totalité est la non vérité.}{Adorno, \ita{Minima Moralia}}

% \cleardoublepage
\addcontentsline{toc}{part}{État de l'Art}
\pagestyle{empty}

~
\bigskip

\vspace{11em}

\bigskip

\epigraphii{Se demander si un ordinateur peut penser ... est aussi intéressant que de se demander si un sous-marin peut nager.}{ Edsger Wybe Dijkstra, \ita{The threats to computing science}}

\pagestyle{scrheadings}

%%%%%%%%%%%%%%%%%%%%%%%%%%%%%%%%%%%%%%%%%%%%%%%%%%%%%%%%%%%%%%%%%%%%%%%%%%%%%%%%%%%%%%%%%%%%%%%%%%%
%%%%%%%%%%%%%%%%%%%%%%%%%%%%%%%%%%%%%%%%%%%%%%%%%%%%%%%%%%%%%%%%%%%%%%%%%%%%%%%%%%%%%%%%%%%%%%%%%%%
\part*{Exposition}
\addcontentsline{toc}{part}{Exposition}

%%%%%%%%%%%%%%%%%%%%%%%%%%%%%%%%%%%%%%%%%%%%%%%%%%%%%%%%%%%%%%%%%%%%%%%%%%%%%%%%%%%%%%%%%%%%%%%%%%%
%%%%%%%%%%%%%%%%%%%%%%%%%%%%%%%%%%%%%%%%%%%%%%%%%%%%%%%%%%%%%%%%%%%%%%%%%%%%%%%%%%%%%%%%%%%%%%%%%%%
% \section*{Préambule (n)}
% \addcontentsline{toc}{section}{Préambule}



%%%%%%%%%%%%%%%%%%%%%%%%%%%%%%%%%%%%%%%%%%%%%%%%%%%%%%%%%%%%%%%%%%%%%%%%%%%%%%%%%%%%%%%%%%%%%%%%%%%
%%%%%%%%%%%%%%%%%%%%%%%%%%%%%%%%%%%%%%%%%%%%%%%%%%%%%%%%%%%%%%%%%%%%%%%%%%%%%%%%%%%%%%%%%%%%%%%%%%%
\chapter{Introduction}\label{chap:intro}
% \epigraphii{On peut s'étonner que les actes spontanés par lesquels l'homme a mis en forme sa vie, se sédimentent au dehors et y mènent l'existence anonyme des choses. La civilisation à laquelle je participe existe pour moi avec évidence dans les ustensiles qu'elle se donne.}{Merleau-Ponty\\\hfill\ita{Phénoménologie de la perception}}

%%%%%%%%%%%%%%%%%%%%%%%%%%%%%%%%%%%%%%%%%%%%%%%%%%%%%%%%%%%%%%%%%%%%%%%%%%%%%%%%%%%%%%%%%%%%%%%%%%%
% \section{Contexte}\label{chap:contexte}
Le travail de thèse dont nous rendons compte dans ce mémoire s'est déroulé dans le cadre du projet MediaMap\footnote{Voir \url{http://www.mediamapproject.org/}} soutenu par le cluster européen Eureka Celtic et la Direction Générale de la Compétitivité, de l'Industrie et
des Services (DGCIS) du ministère de l'Economie, des finances et de l'industrie.
La participation à ce projet de recherche et développement nous a imprégnés de connaissances sur la production audiovisuelle.
%, les besoins qui émergent des tendances actuelles et les problèmes rencontrés pour les combler. 



%Nous nous intéresserons ensuite (\g{Chapitre \ref{chap:problo}}) aux 


%%%%%%%%%%%%%%%%%%%%%%%%%%%%%%%%%%%%%%%%%%%%%%%%%%%%%%%%%%%%%%%%%%%%%%%%%%%%%%%%%%%%%%%%%%%%%%%%%%%
\section{L'impact du numérique sur l'audiovisuel}\label{sec:motiv}
\e{
La révolution numérique initiée depuis une trentaine d'années en conjonction avec la révolution électronique et informatique, s'est progressivement imposée à tous types d'information et de contenus. 
%jusqu'à devenir prépondérante et indispensable dans un monde informatisé.
Ces mouvements d'informatisation des pratiques et de numérisation de l'information impactent les organisations et les métiers en redistribuant les tâches entre humains et machines. 
Dans le cadre de cette thèse, nous nous pencherons sur le cas de l'audiovisuel et de la numérisation de ce contenus et des informations associées.}

%L'audiovisuel représente à plus d'un titre un objet singulier.
%l'exploitation des contenus numériques est en passe de s'étendre à toutes les étapes du cycle de vie d'un objet audiovisuel. % ? pourquoi exploitation ?
Après une informatisation des étapes de postproduction (logiciels de montage, effet spéciaux etc.) puis des équipements de captation (caméra, micro etc.) et de la distribution du côté des diffuseurs, nous avons connu une véritable explosion d'appareils destinés au grand public (lecteur multimédia portable, appareil photo, téléphone portable, dictaphone etc.).


Plusieurs grands chantiers s'ouvrent désormais dans ce mouvement d'informatisation des étapes de la chaîne de production :
\begin{liste}
	\item L'archivage des contenus de manière à les faire rentrer dans l'histoire malgré la dégradation inexorable des supports. 
	Il ne s'agit pas simplement de leur permettre de survivre jusqu'à la prochaine génération d'appareils électroniques, mais également de garantir sa \ciel{lisibilité technique et culturelle}, (\cite[p.~12]{Bachimont2000}).

	\item La préproduction des contenus qui est presque inexistante et qui permettrait de récolter des informations sur les contenus avant même leur fabrication. 
	L'enjeu plus général est d'initier l'indexation des contenus au moment du \e{Scripting} et de la continuer tout au long de la chaîne, chaque étape pouvant rajouter des informations supplémentaires ou bien réévaluer les anciennes.
\end{liste}

À côté de la numérisation et de l'informatisation, il ne faudrait pas oublier le développement considérable des réseaux de télécommunications qui a grandement favorisé les échanges de contenus de manière illégale ou légale, de pairs à pairs (\e{Peer to Peer}), entre diffuseur et spectateurs (\e{Business to Consumers}) ou entre acteurs professionnels de la chaîne (\e{Business to Business}). 
Ainsi, numérisation et informatisation sont dorénavant implictement associées aux facilités de transfert de ces réseaux. 

Sans tenter de tirer toutes les conséquences de ces révolutions technologiques (électronique et informatisation, numérisation, mise en réseaux) sur les usages des spectateurs et des acteurs de la chaîne de production audiovisuelle, on retiendra trois constats important qui charpentent notre réflexion : 
\begin{liste}
	\item \eg{il existe un nombre croissant de contenus audiovisuels en circulation.}
	L'offre et la demande augmentent de même que les capacités de transfert des réseaux et la généralisation des appareils électroniques de captation et de visionnage.

	\item \eg{les pratiques en lien avec les contenus audiovisuels se développent et s'individualisent.}
	La consommation de ces contenus se joue de plus en plus à un niveau individuel depuis l'apparition d'appareils personnels de communication et de visionnage. 
	De même, la production de contenus est facilité par l'informatisation et les progrès des appareils électroniques de captation. 
	La création de contenus n'est donc plus l'apanage de grandes équipes spécialisées et lourdement équipées.
	
	\item \eg{la numérisation complique le maintien de l'unicité des contenus audiovisuels}. 
	%Le principe même du numérique (\ciel{ça a été manipulé} [Bachimont2010]) repose sur la représentation de l'information et le calcul. 
	L'environnement numérique, par rapport à l'analogique, est plus propice à la copie, le transfert, la fragmentation et la manipulation des contenus ramenés invariablement à une donnée muette quelle que soit sa nature (texte, son, image animée ou non etc.) ou sa signification.
	Ainsi coupé des sens et du sens, les contenus semblent perdre leur identité dans le \gui{monomédia} numérique (\cite[p.~13]{Bachimont2000}) dans lequel on a peine à les gérer pour ce qu'ils représentent. 	

	%plus manipulables au sens où chaque visionnage construit une forme perceptive du contenu 

	%est une reconstruction de leur forme perceptible par des appareils de . Il y a donc par définition une certaine instabilité 
	%mobile, versatile, altérable, mouvant, variable, instable
\end{liste} 

%Le développement de l'électronique a ainsi favorisé une  le nombre de contenus audiovisuels, à favoriser leur circulation
%On constate ainsi une augmentation des sources de contenus ainsi qu'une intensification de leur circulation qui ont des conséquences sur les usages des spectateurs et les acteurs de la chaîne de production.
Nous allons maintenant préciser cet argumentaire et présenter les enjeux posés par la numérisation, l'informatisation et le développement de l'électronique.




%%%%%%%%%%%%%%%%%%%%%%%%%%%%%%%%%%%%%%%%%%%%%%%
\subsection{La numérisation des contenus audiovisuels}\label{sec:num}
Le contenu audiovisuel est avant tout un objet temporel, c'est-à-dire un flux d'images et de sons qui s'écoule en un temps donné. 
Un des enjeux liés à l'audiovisuel constitue alors de trouver une technologie d'enregistrement et de manipulation de ces flux.

\paragraph{L'enregistrement analogique du flux}
Les techniques d'enregistrement analogique reposent sur la conversion continue du flux original en un signal analogue dont les variations s'effectuent sur une échelle physique différente, par exemple une fréquence sonore transformée en une tension électrique. 
L'enregistrement s'effectue sur différents types de supports, cassettes magnétiques, films argentiques etc. 
La caractéristique commune de ces supports est que l'accès au contenu enregistré ne peut se faire que de manière linéaire ou séquentielle, c'est-à-dire qu'il faut avancer ou rembobiner la bande jusqu'au point de départ avant de pouvoir commencer la lecture. 
Ainsi, chaque opération de manipulation de ces contenus compte toujours un temps pas forcément négligeable consacré à l'alignement avec le point de départ désiré. 
Pour bien s'en rendre compte, il suffit de se souvenir du temps qu'il fallait pour rembobiner la cassette de votre film préféré, puis du temps passé en avance rapide pour passer les inévitables publicités précédant le film.

\paragraph{Numérisation et délinéarisation de l'accés}
Avec le numérique, cette expérience autrefois familière a complètement disparu et se voit remplacée par un accès immédiat à n'importe quel moment du contenu audio-visuel. 
La numérisation des contenus consiste à discrétiser le flux en un ensemble de valeurs que l'on convertit ensuite en flux binaire. Ce flux est ensuite enregistré sur des supports de mémoire magnétiques (disquettes 3'1/2, disques durs etc.) optiques (CD, DVD etc.) ou électronique (mémoire vive dite RAM, mémoire flash etc.). 
Seules ces dernières garantissent un accès arbitraire ou délinéarisé aux données stockées en mémoire (n'importe quelle donnée, à n'importe quel moment). 
Les supports magnétiques et optiques proposent un accés séquentiel comme l'analogique, mais plus rapide et surtout que l'on peut coupler avec les mémoires électroniques afin de se rapprocher de leurs performances.

\paragraph{Fragmentation}
Par ailleurs, au-delà des avantages liées aux temps d'accès au contenu, le numérique facilite également sa fragmentation et sa manipulation.
Contrairement à l'analogique, le numérique permet de représenter de manière arbitraire tout type d'information puis d'effectuer des calculs sur ces représentations. 
Ainsi, on peut associer au contenu audiovisuel, d'autres contenus de différentes natures pour les enrichir et faciliter son exploitation ultérieure.

\paragraph{Discrétisation}
La discrétisation du flux audiovisuel quant-à-elle remet en cause la temporalité du contenu et permet de ce fait une fragmentation plus aisée et plus fine. 
Lorsque l'analogique effectue une transformation continue et analogue, le numérique définit une fréquence d'échantillonnage et quantifie les valeurs de la source sur une échelle arbitraire finie. 
C'est donc véritablement la fréquence d'échantillonnage qui constitue la première unité de réprésentation de l'information au-dessus du bit.
C'est donc en décidant du nombre de pixels pour représenter une image, ou aux nombres de valeurs par secondes prises pour représenter un son que l'on décide d'une première échelle de fragmentation. 
% Premier niveau de manip technique, ensuite il y a des USI
Toute autre fragmentation à l'échelle supérieure est potentiellement (re)constructible par calcul, pour autant qu'on possède une méthode opérationnalisable. 
%De même que chaque élément de cette fragmentation, chaque unité, devient adressable et donc manipulable par calcul. 
%\cite{Bachimont2000} parle ainsi d'utm, usi ...
% COUPURE DU SENS ?

% \g{== Révision à faire, introduire les UTM et USI ==}
\paragraph{Numériser, c'est informatiser le métier}
Parmi toutes les fragmentations possibles, il convient alors de déterminer leur pertinence par rapport aux types de calculs que l'on souhaite réaliser à chaque étape de la chaîne de production. 
Il s'agit ici de contrôler les calculs à effectuer en suivant les règles du métier qui en disposera.
Chaque étape ayant ses objectifs propres, les calculs effectués varient et s'opèrent à différents niveaux de fragmentation. 
Or la numérisation des contenus et de l'information s'effectue toujours en vue de leur manipulation par des programmes informatiques.
Ainsi, la numérisation entraîne toujours une informatisation qui impacte le métier dans son organisation et ses pratiques parce qu'il transforme les possibilités techniques qui le concerne.
% L'enjeu est donc double, d'une part identifier les niveaux de fragmentation pertinents pour chaque étape de la chaîne de production, et d'autre part de se donner les moyens de reconstruire la cohérence . 

% numériser => (a) fragmenter + (b) mettre en réseau => (a) besoin de conserver la cohérence de l'ensemble + (b) besoin d'autonomiser pour une future situation d'usage





%%%%%%%%%%%%%%%%%%%%%%%%%%%%%%%%%%%%%%%%%%%%%%%
\subsection{Le développement de l'électronique et des réseaux de télécommunications}\label{sec:electro}
L’explosion des appareils multimédia et des possibilités de transférer des contenus par les réseaux de télécommunication a promu de nouvelles pratiques de consommation et d'échanges des contenus tant chez les professionnels de l'audiovisuel que dans le grand public.
% 
Du côté des professionnels, on voit ainsi l’émergence de systèmes de production qui utilisent le réseau pour faire transiter les contenus entre les systèmes d'information de leurs différents départements. 
Ces systèmes reprennent les principes d'architecture multi-tiers utilisés sur le Web avec les contenus représentés par des fichiers ou des flux binaires, d'où leur appelation de \e{file-based production system}.
% 
Ce genre de système favorise également l’échange de fichiers entre organisations, puisque l’architecture permet d'exposer les données stockées de la même manière sur un réseau interne (intranet) ou externe (extranet, Internet).
% 
Du côté du grand public, les appareils portables acquièrent de plus en plus de connectivité avec leur environnement et les réseaux. 
De simples lecteurs à brancher en USB, les appareils sont passés au stade communicant avec la 3G, le Wifi ou le Bluetooth. 
La fonction d'échange se banalise et inversement, les appareils communicant comme les téléphones deviennent eux-même des stations multimédia à part entière, musique, photo, vidéo, courriel, sms etc.\\


Un autre facteur à noter, est que ces appareils portables sont personnels, c’est-à-dire qu’ils sont majoritairement utilisés par un seul individu contrairement à l’usage du téléviseur qui était et reste encore largement un objet collectif. 
Une autre distinction fondamentale se situe dans le fait que ce sont des appareils informatiques qui fonctionnent entièrement dans le numérique. 
Les possibilités d’interaction en font non plus de simples terminaux de lecture, mais potentiellement de véritables instruments de création et de communication. En effet, l'insertion de données et la capture de contenus (photos, sons, textos etc.) est prévu et le couplage avec des plates-formes de publication (réseaux sociaux, dépôts de contenus, CMS etc.) est de mieux en mieux réalisé.

De ce fait, en plus des capacités toujours plus importante de consommation de contenus, ces appareils favorisent eux aussi leur circulation ou la circulation d’information annexes. 
Le partage d’opinions s’est considérablement développé avec la vague d'applications Web dites sociales qui permettaient aux utilisateurs de créer du contenu sans avoir à maîtriser les arcanes techniques du Web. 
Cet ajout d’opinion, même s’il est parfois réduit au minimum à une marque d'appréciation, implique ainsi l’utilisateur dans le processus de diffusion du contenu en le portant à l'attention d'autres utilisateurs (ses contacts dans le cas des réseaux sociaux, ses lecteurs dans le cas d'un weblog ou autres CMS, un ensemble d'utilisateurs anonymes dans le cas de services sociaux tels que Delicious\footnote{Delicious : \url{http://delicious.com/} est une plate-forme de sauvegarde, d'indexation par mot-clé et de partage de marques-pages. Pour l'utilisateur il s'agit soit de sauvegarder et d'indexer ses marques-pages, soit de découvrir les marques-pages correspondant à tel ou tel mot(s)-clé(s) déjà sauvegardées par la communauté. Les grandes tendances d'indexation sont ainsi accessibles à tous, tout en permettant à chacun de développer son système d'indexation personnelle -- adaptation française du mot anglais \e{folksonomy}. Il est également possible de partager directement ses trouvailles avec d'autres utilisateurs par un système d'abonnement et de notification.} ou Digg \footnote{Digg : \url{http://digg.com/} est un site de marque-page social qui fonctione sur le principe du vote. Un utilisateur peut proposer une page qui est alors soumise aux votes des autres utilisateurs. Suivant le succès de la page, celle-ci sera mise en avant sur la page principale de Digg, ou bien mise de côté avec le reste des pages moins populaires, et finira par être supprimée.}).\\


Les appareils numériques multimédia possèdent également des capteurs de plus en plus performant et de moins en moins coûteux qui permettent au grand public de découvrir de nouvelles activités de création (photographie et retouche d’image, tournage et montage vidéo, prise de son et mixage audio etc.).
Cet abaissement du coût d’entrée dans la production a favorisé l'émergence d’une production amateur hétéroclite qui va du passant prenant une photo d’un évènement se déroulant devant ses yeux jusqu’à l’amateur qui pratique par amour mais avec l'exigence d’un professionnel. 
Cette production amateur rentre alors en concurrence avec la production professionnelle, voire la remplace dans certains cas (lorsqu’un passant est le seul témoin d’un évènement inattendu par exemple). 
La concurrence est d'autant plus forte depuis l’apparition de plates-formes de partages qui facilitent la distribution des contenus. 
De manière générale, l’opposition classique entre producteurs et consommateurs se brouille et les professionnels cherchent de plus en plus à mettre à contribution les amateurs dans leurs processus. 
On se dirige ainsi vers un modèle où professionnels et amateurs contribuent à divers degrés et divers moments au cycle de vie des contenus.


Avec l'informatisation des appareils multimédia s’est introduit la possibilité de personnaliser la communication avec l'utilisateur, d'y ajouter de l'interactivité et de connecter les utilisateurs entre eux. 
Ces nouvelles possibilités transforment les attendus et les pratiques du grand public. 
De ce fait, cela impacte les rapports avec les professionnels qui tendent à vouloir intégrer les contributions externes à leurs propres productions. 
Ainsi, il ne s’agit plus simplement de produire des contenus qui s'adressent indistinctement aux masses, mais de trouver des moyens de personnaliser son offre, de recommander des contenus, de faciliter la récupération de contenus, de diversifier les occasions de consommer ou de contribuer.








%%%%%%%%%%%%%%%%%%%%%%%%%%%%%%%%%%%%%%%%%%%%%%%
\section{Le projet MediaMap}\label{sec:mm}
% %%%%%%%%%%%%%%%%%%%%%%%%%
\paragraph{Objectifs}
Le projet MediaMap vise à développer des modèles et des applications pour promouvoir la production audiovisuelle collaborative articulant contenus professionnels et amateurs. 
En particulier, l'ambition est d'intégrer les amateurs et leurs contenus dans la chaîne de production professionnelle en améliorant d'une part la qualité technique et éditoriale des contenus fabriqués, et d'autre part en facilitant la collaboration et l'intercompréhension entre les différents acteurs de la chaîne.

La piste de travail retenue a été de construire des ontologies capables de représenter et décrire les contenus au fur et à mesure de leur processus de production. 
Ces informations serviraient de base de connaissances pour de nouvelles applications s'intégrant dès le début à la chaîne de production audiovisuelle, c'est-à-dire dès la conception du contenu. 
Les partenaires du projet ont ainsi développé des applications d'organisation du processus de production, de description du contenu, d'assistance au tournage ainsi qu'un moteur de recherche utilisant ces ontologies comme modèle d'information de référence.

%%%%%%%%%%%%%%%%%%%%%%%%%
\paragraph{Composition}
Le projet MediaMap a rassemblé une dizaine d'entreprises ainsi que deux équipes de recherche de l'Université de Technologie de Compiègne :
\begin{liste}
	\item l'équipe de recherche \pc{Information Connaissance Interaction} (ICI) chargée de la partie modélisation qui a abouti à la construction d'ontologies.

	\item l'équipe de recherche \pc{Automatique, Systèmes Embarqués, Robotique} (ASER) chargée de la conception d 'algorithmes d'analyse d'images et de vidéos.
\end{liste}


Parmi les entreprises du consortium, on compte les deux grandes chaînes de télévision publiques belges ainsi que de nombreuses PME belge ou française qui apportent leurs expertises dans différents domaines :
\begin{liste}
	\item \pc{BelgaVox} qui gère un des plus grands stocks d'archives audiovisuelles belges et produit des documentaires.

	\item \pc{Exalead} qui est un éditeur de solution de recherche pour les entreprises.

	\item \pc{Kane Consulting} qui propose des analyses du marché et des usages aux acteurs de la production audiovisuelle.

	\item \pc{Memnon} qui est spécialisé dans la numérisation, la documentation et l'archivage de contenus audio et vidéo.

	\item \pc{Perfect Memory} qui s'est créé pendant le projet afin d'accompagner les solutions du projet sur le marché grand public et prospecte également le marché professionnel.

	\item \pc{Skema} qui développe des applications de production de contenu audiovisuel amateur pour mobiles et caméras.

	\item \pc{Solution 2.0} qui est une agence de conception et de réalisation de plate-forme Web.

	\item la \pc{Radio-Télévision Belge de la communauté Française} (RTBF).

	\item \pc{Vitec Multimédia} qui développe et manufacture du matériel vidéo numérique.

	\item la \pc{Vlaamse Radio- en Televisieomroep} (VRT).
\end{liste}









%%%%%%%%%%%%%%%%%%%%%%%%%%%%%%%%%%%%%%%%%%%%%%%
\section*{Organisation du mémoire}\label{sec:plan}
\addcontentsline{toc}{section}{Organisation du mémoire}

\paragraph{Exposition}
\e{
La première partie de ce mémoire a pour objectif de présenter les problèmes qui se posent à la production audiovisuelle depuis son avancée vers la numérisation. 
Elle nous permet également de préciser la manière dont nous posons le problème, à la fois en terme métiers et avec nos lunettes de scientifiques.}

Le chapitre \g{\ref{chap:intro}. Introduction} nous sert à rappeler le contexte technologique général qui s'impose au monde de l'audiovisuel. 
En effet, la production audiovisuelle se dirige progressivement vers une numérisation et une mise en réseau de ses produits ainsi qu'une informatisation de ses pratiques qui n'est pas sans conséquences. 

Le chapitre \g{\ref{chap:problo}. Problématisation} nous permet de préciser comment ces tendances impactent le monde de l'audiovisuel.
Nous nous appuyerons sur l'étude du fonctionnement de la chaîne de production audiovisuelle classique (\ref{sec:prod}) pour dresser un bilan des attentes des professionnels vis-à-vis du numérique (\ref{sec:besoins}).
% En particulier, 
Nous précisons alors la manière dont nous posons le problème sur le plan métier, ce qui nous amènera à définir le problème scientifique. 
Sur le plan métier, le défi posé par le numérique consiste à passer d'une vision à l'échelle du document à une vision à l'échelle du fragment. 
En effet, le numérique favorise la fragmentation et la circulation des contenus qu'il s'agit alors de rendre autonome pour en permettre l'exploitation (\ref{sec:pmetiers}). 
Sur le plan scientifique, nous posons le problème en terme de modélisation des objets audiovisuels et des connaissances associées afin de construire une compréhension commune et dynamique pour tous les acteurs impliqués dans la production audiovisuelle (\ref{sec:scien}).
% Nous concluons en expliquant comment nous mobilisons diverses disciplines scientifiques pour construire une réponse aux problèmes posés (\ref{sec:posd}).
% proposent d'une part des outils et des méthodes de modélisation, et d'autre part proposent des modélisations de l'audiovisuel (\ref{sec:posd}).


\paragraph{État de l'art}
\e{
La deuxième partie de ce mémoire vise à étudier des outils, méthodes et langages de modélisation. Elle permet également d'étudier les modélisations existantes des objets audiovisuels, mais égalements des connaissances métiers qui y sont associées afin de faciliter leur exploitation.}

Le chapitre \g{\ref{chap:omod}. Outils de modélisation} commence par clarifier les besoins de modélisations à partir d'un scénario de production collaborative impliquant des acteurs professionnels et amateurs (\ref{sec:cdcf}).
Ces besoins ne se situent pas seulement au niveau de la modélisation conceptuelle, mais également sur le plan des jargons utilisées pour présenter ces concepts à des contributeurs de la production.
Ainsi, nous examinons les définitions des concepts de \gui{systèmes d'organisation de connaissances} (SOC) et en particulier les relations qu'entretiennent \gui{terminologie} et d'\gui{ontologie} (\ref{sec:defs}).
Nous étudions ensuite les langages de structuration et de représentation des connaissances qui permettent de modéliser ces deux types de SOC (\ref{sec:mods}).

Le chapitre \g{\ref{chap:mav}. Modélisations de l'audiovisuel} a pour objectif de mettre en rapport les représentations des professionnels de la production avec diverses communautés scientifiques, en vue de clarifier la définition d'un objet audiovisuel (\ref{sec:dav}) et de sa réutilisation (\ref{sec:gest}).
Cette étude s'appuie sur la poursuite du scénario d'usage du chapitre précédent, et met en exergue la nécessité de fragmenter la modélisation des objets audiovisuels pour favoriser leur réutilisation (\ref{sec:cdc-av}).
Nous étudions ensuite les solutions utilisées dans l'industrie pour gérer la circulation des programmes (\ref{sec:wrapper}) et les décrire (\ref{sec:desc}).
Cette dernière partie analyse les méthodes de fabrication de ces description ainsi que leur nature, puis les modélisations développées.
Une attention particulière est portée à de MPEG-7 (\ref{sec:mpeg7}), qui tient lieu de référence à de nombreux travaux de formalisation sous la forme d'ontologie (\ref{sec:mpeg7etc}).
Nous présentons également des approches de description plus proche de la perspective de la production audiovisuelle (\ref{sec:insitu}).
% à revoir


\paragraph{Contribution}
\e{La troisième partie de ce mémoire présente notre contribution conceptuelle et informatique aux problèmes de modélisations que nous avons soulevés.
Nous détaillons nos choix de représentation pour opérationnaliser notre contribution en une ontologie informatique.}

Le chapitre \g{\ref{chap:mod}. Approche et modélisation} revient sur les langages et les modélisations étudiés dans le chapitre précédent et introduit les principes de notre approche (\ref{sec:principes}).
Notre positionnement au sein de la chaîne de production audiovisuelle, nous permet de modéliser le déroulement de la chaîne, la définition d'une structure documentaire première, puis les fragments audiovisuels construits ainsi que les connaissances qui s'y rapportent.
Nous détaillons la modélisation conceptuelle en parties, chacune correspondant à un besoin fonctionnel, en expliquant leur mise en relation par des exemples (\ref{sec:concept}).

Le chapitre \g{\ref{chap:op}. Mise en oeuvre} présente la représentation informatique de notre conceptualisation.
Nous argumentons d'abord nos choix de langage et montrons comment nous les utilisons (\ref{sec:ln}). 
Nous détaillons ensuite la structuration de notre ontologie et son articulation avec des thésaurus et des bases de faits (\ref{sec:op}).

\paragraph{Discussion}
\e{La dernière partie de ce mémoire montre comment notre contribution est utilisé dans le cadre du projet MediaMap et ouvre la discussion sur ce travail de thèse.}

Le chapitre \g{\ref{chap:app}. Applications et expérimentations} introduit les diverses applications qui ont été développé (\ref{sec:app}) par nos partenaires et les expérimentations qu'elles ont permis de mener (\ref{sec:xp}).
Il s'agit d'éclairer l'appropriation de notre travail dans le cadre de scénarios de production audiovisuelle collaborative.
En particulier, nous expliquons quelle partie de l'ontologie est mobilisée par les applications pour construire ou bien intégrer des connaissances sur la production, ses contributeurs et ses produits.

La \g{Conclusion} remet en perspective notre contribution et les applications développées par rapport aux problèmes métiers et scientifiques posés.
Nous ouvrons également la discussion sur la poursuite de nos recherches et l'avenir des applications du projet MediaMap.

\chapter{Problématisation}\label{chap:problo}
\minitoc
\section{La production audiovisuelle}\label{sec:metier}
% faire le point sur ce que le numérique pourrait apporté à la production audiovisuelle
\e{
L'objectif de cette première section est de donner des éléments de compréhension du métier de la production audiovisuelle.
Dans un premier temps, nous rappelons comment s'organise classiquement la fabrication des objets audiovisuels (\ref{sec:prod}).
Ensuite, nous détaillons les notions et des mots utilisées par les professionnels pour parler de l'objet audiovisuel à construire (\ref{sec:docvoc}).
Enfin, à partir de ces éléments nous précisons les besoins que rencontrent ces professionnels avec l'émergence du numérique et de la mise en réseau (\ref{sec:besoins}). 
}


%%%%%%%%%%%%%%%%%%%%%%%%%%%%%%%%%%%%%%%%%%%%%%%
\subsection{Déroulement de la chaîne de production}\label{sec:prod}
% \addcontentsline{toc}{subsection}{La production audiovisuelle}
La création de documents audiovisuels est une entreprise collective qui suit généralement ce qu'on appelle la chaîne de production audiovisuelle. 
Organisée de manière linéaire, cette chaîne peut se décomposer en 4 grandes étapes -- voir Figure \ref{img:intro:chaine}.

\begin{liste} 
	\item \g{Préproduction} : Cette première étape consiste à construire une ébauche du futur document audiovisuel de manière à prévoir les moyens à engager pour le réaliser.

	\item \g{Production} : Cette étape vise à tourner plusieurs prises pour chaque partie du document et à commencer à faire le tri entre elles.

	\item \g{Postproduction} : L'objectif est d'assembler les prises et de les retoucher de manière à former un document cohérent et adapté à une audience et un mode de distribution.

	\item \g{Exploitation} : Une fois le document achevé, on valorise sa construction par une distribution auprès d'une audience ainsi qu'un archivage qui permettra de le réutiliser ultérieurement.
\end{liste}

\begin{figure}[ht!]
\centering
\includegraphics[width=\textwidth]{images/Workflow-Thesis-v0.png}
\caption{La chaîne de production audiovisuelle classique}
\label{img:intro:chaine}
\end{figure}


%%%%%%%%%%%%%%%%%%%%%%%%%
\subsubsection*{Préproduction}\label{sec:preprod}
L'objectif de cette étape est de construire une ébauche du futur document audiovisuel, de manière à prévoir les moyens à engager pour le réaliser. 
On distingue deux phases de préparation, l'écriture ou \e{Scripting} et le \e{Planning} ou planification.

À partir d'une idée, l'écriture du document se déroule en plusieurs étapes où l'on fixe progressivement le message à faire passer ainsi que sa forme audiovisuelle. 
Une fiction par exemple s'écrit à partir d'un résumé de l'histoire, puis on développe les scènes, les personnages, les lieux, les dialogues, etc. jusqu'à arriver à la manière dont cette histoire sera racontée à l'écran. 
On fixe ainsi le type de plan à filmer, les mouvements de caméra, le type de lumière, etc. 
Dans certaines grosses productions, on aura même recours au dessin pour aider à la représentation visuelle.


Toutes ces informations sur le contenu et la forme du futur document audiovisuel servent à estimer le temps nécessaire et les moyens à engager. 
L'estimation du coût d'une production est un élément essentiel pour la production audiovisuelle. 
En effet, dès le départ le producteur doit estimer la rentabilité du futur document afin de voir quels moyens il peut engager. 
Il s'agit bel et bien d'investissement, parfois très lourd, surtout en comparaison avec d'autres industries culturelles – musique, littérature, radio, presse – à l'exception récente du jeu vidéo. 
Contrainte budgétaire et forme esthétique sont donc en négociation dans cette première étape.


%%%%%%%%%%%%%%%%%%%%%%%%%
\subsubsection*{Production}
Une fois l'ébauche et les moyens déterminés, la production vise à réaliser chaque partie du document une à une puis à les assembler en un montage cohérent.

La phase de \e{Fabrication} consiste à capter du réel que l'on a mis en scène. La captation d'un évènement est réalisée grâce à des appareils d'enregistrement (caméra, microphones, etc.). 
La mise en scène du réel se construit à partir d'un ensemble de techniques, d'équipements et d'accessoires (lumière, costume, décors, maquillage, etc.) qui permettent d'obtenir l'image et le son souhaités. 
La technique est ainsi mobilisée dans un objectif esthétique. 
Dans le cas d'une fiction, la fabrication du contenu se fait en fonction du planning de mobilisation des personnes et des équipements plutôt que suivant l'ordre chronologique de l'histoire. 
On regroupe ainsi le tournage des scènes dans tel lieu ou avec tel équipement afin de réduire les coûts. Au final, la fabrication
produit des séquences vidéo et audio qu'il s'agit ensuite de redécouper pour mieux les assembler.

La phase de \e{Derushing} consiste à examiner les séquences réalisées pendant la fabrication et à les trier en vue de faciliter le montage. 
Par exemple, le tournage d'une fiction produit des séquences vidéo comprenant plusieurs prises d'une même scène et souvent des séquences captées par des caméras ayant des angles de prise de vue différents. 
Il faut donc redécouper les séquences en prises puis regrouper les prises d'une même scène. 
Le montage se fera d'autant plus facilement qu'on saura également identifier la qualité et les avantages de chaque prise d'une même scène.


%%%%%%%%%%%%%%%%%%%%%%%%%
\subsubsection*{Postproduction}
Lorsque le contenu audiovisuel est fabriqué et trié, il reste à le structurer en un document et à le conditionner en fonction de sa future exploitation.

La phase de \e{Montage} consiste à agencer des petites séquences de vidéo et d'audio pour construire la structure du document audiovisuel.
L'agencement des plans, leur durée et la transition entre ces plans constituent les ressorts esthétiques propres à l'audiovisuel. 
Ils participent à la transmission du message en ceci qu'ils servent de raccord entre les plans, comme le souligne le réalisateur \pc{Sergueï Mikhaïlovitch Eisenstein}\footnote{L'origine de cette fameuse citation est assez obscur, on la retrouve dans de nombreux documents, dont cet article --\cite{Montage et Réalisme}-- datant des années 60 et extrait de la revue québecoise \gui{Séquence : La revue du cinéma}.} :

\begin{cico}
Le montage est l 'art d'exprimer ou de signifier par le rapport de deux plans juxtaposés de telle sorte que cette juxtaposition fasse naître l'idée ou exprime quelque chose qui n'est contenu dans aucun des deux plans pris séparément. L'ensemble est supérieur à la somme des parties.
\end{cico}

Après le montage, une phase de \e{Finition} est nécessaire pour intégrer d'autres ressources au document en fonction de sa distribution. 
Pour une diffusion antenne d'un reportage, on ajoute le logo de la chaîne de télévision, un jingle, le nom des intervenants ou des titres, etc. 
Une diffusion sur DVD nécessitera l'ajout des conditions
légales d'usages, l'intégration d'un menu de navigation, etc. 
La distribution détermine également un format d'encapsulation (.avi, .mkv, .ogg etc.) et un encodage du contenu audiovisuel (MPEG-2, H.264, MPEG-4, theora vorbis etc.). 
Le résultat de cette phase de post-production est de fournir des documents prêts à l'usage et dans certains cas plusieurs variantes pour chacun des modes de distribution envisagés.


%%%%%%%%%%%%%%%%%%%%%%%%%
\subsubsection*{Exploitation}
Une fois le document achevé, on valorise sa construction par une distribution auprès d'une audience ainsi qu'un archivage qui permettra de le réutiliser ultérieurement.

La phase de \e{Distribution} consiste à rendre le document matériellement accessible à une audience. 
Il s'agit d'un transfert qui peut faire l'objet d'une transaction commerciale ou s'appuyer sur d'autres types de modèles économiques (publicité entre autres). 
La nature du transfert varie et porte à la fois sur les modalités d'accès au contenu et les droits d'usages.

La phase d'\e{Archivage} consiste le plus souvent à stocker le contenu diffusé afin de pouvoir le réutiliser tel quel plus tard, soit en le rediffusant, soit en le vendant à un autre diffuseur. 
C'est aussi généralement la phase où l'on construit, récupère et attache des descriptions du contenu au document audiovisuel. 
En effet, l'archivage n'a de sens que s'il permet de retrouver, voire redécouvrir, les documents archivés.
\ciel{Au plan économique, un film est un bien informationnel, d'expérience, caractérisé par une très forte densité d'informations. Son exploitation s'organise autour de versions différentes, distribuées sur des marchés distincts par des acteurs spécialisés.} (\cite{Blanc2006}).%Gilles Le Blanc - Innovations numériques, distribution et différenciation  : le cas de la projection numérique dans le cinéma.




%<TODO
%TODO>
\subsection{Documents et vocabulaire de la production (n)}\label{sec:docvoc}
\e{
Dans cette section, nous présentons des définitions utilisées dans le milieu professionnel et tirées du \gui{Dictionnaire technique du Cinéma (\cite{Pinel2008})} afin de présenter les principaux documents et le vocabulaire utilisé pendant la phase de préproduction. 
Il s'agit ainsi de mettre en exergue la manière dont se construit une description textuelle de l'objet audiovisuel en devenir ainsi que le vocabulaire utilisé pour faire cette description.}
% Des éléments qui nous serviront à mieux cerner les problèmes qui se posent à la production audiovisuelle.

\paragraph{La notion de plan}
La notion de plan peut se présenter de diverses manières suivant le point de vue adopté. 
Du point de vue technique, il s'agit d'une série d'images (ou photogrammes) qui sont enregistré par un appareil de capatation (une caméra) au cours d'une même prise de vue : \ciel{série de photogrammes enregistrés au cours d'une même prise} (\cite{Pinel2008}).
Il s'agit donc de l'enregistrement qui est effectué entre le moment où l'on presse sur le bouton pour lancer et celui où l'on arrête l'enregistrement.
Cette définition, certes robuste, ne permet pas pour autant de caractériser les plans ou de les comparer entre eux. 
Ainsi, on s'intéresse au point de vue de l'écriture filmique et de la réalisation qui considère non seulement l'action d'enregistrement, mais également la manière dont il est effectué : 

\ciel{
Fragment de temps et d'espace enregistré d'un seul tenant, selon un point de vue déterminé, et donnant à la projection le sentiment de la continuité d'une même \e{image en mouvement}.} (\cite{Pinel2008})

Cette définition complète la précédente en considérant le rapport au sujet (le point de vue) et la temporalité du plan et son rapport à un ensemble d'autres plans (la continuité). 
Elle permet d'envisager les plans comme des éléments de base que l'on assemblera ensuite pour construire un objet audiovisuel : 

\ciel{
La notion de plan est apparue [\dots] lorsqu'on a abandonné le point de vue unique du tableau pour envisager le sujet sous différents angles et à différentes distances et lorsque, par la grâce du montage, on a mis en relation ces plans entre eux.
[\dots]
Si le photogramme représente l'unité technique de la prise de vues, la scène et la séquence les unités narratives de l'oeuvre cinématographique, le plan est la cellule fondamentale de l'écriture du film, de sa préparation jusqu'à la copie standard.} (\cite{Pinel2008})

Dans cette citation, on voit aussi émerger l'idée qu'il existe différents niveaux d'analyse dans l'objet audiovisuel : 
\begin{liste}
	\item un \g{niveau technique} avec le photogramme mais également le pixel dans le numérique.
	\item un \g{niveau narratif} avec la scène (unité de temps et de lieu dans l'histoire) et la séquence (suite de scènes constituant une action dramatique autonome ou distincte).
	\item \g{un niveau dont le plan est l'unité de base qui sert tout au long de la chaîne de production}. 
	On parle d'unité, car il s'agit du résultat de base d'une prise de vue, c'est-à-dire de la fabrication (production) de l'objet audiovisuel. 
	La pré-production, étape de préparation du tournage, utilise donc naturellement cette unité. 
	De même, le montage consiste à organiser ces plans pour former un ensemble cohérent, quitte à les ajuster (raccourcir, allonger, modification du cadrage etc.). 
	Ainsi, il semble que cette unité servent non seulement d'unité de travail de référence pour la chaîne de production\footnote{La production sonore d'un objet audiovisuel ne s'organise pas forcément de la même manière que celle de la production de l'image. Néanmoins, par la force des choses, la construction de l'image prime bien souvent sur celle du son et son unité de base sert donc de référence même pour la production sonore.}, mais également du premier niveau signifiant propre à l'audiovisuel (l'image seul pouvant être rattaché à la photographie).
\end{liste}

Ces distinctions nous permettent de définir le plan selon les caractéristiques suivantes : 
\begin{liste}
	\item \g{l'échelle relative du cadre par rapport au(x) sujet(s)} (personnages, objets etc.).
	C'est ce qui permet de définir un ensemble de \gui{valeurs de plan}, gros plan, plan américain etc. que nous définirons par la suite.
	
	\item \g{l'angle de la prise de vue} (plongée, contre-plongée, cadre incliné etc.)

	\item \g{le mouvement de la caméra} et d'autres paramètres de son objectif (panoramique, travelling, rotation, zoom, focale, focus etc.)

	\item \g{l'articulation des plans entre eux}, d'une part en terme de durée, mais aussi en terme de transition et d'impression de continuité entre les plans. 
	Par exemple, il existe des règles de cadrage et de montage pour aider les spectateurs à situer les personnes sur un plateau\footnote{La règle des 180° oblige ainsi à maintenir les mêmes relations est-à-gauche/droite-de entre les personnes, de manière à ce que le spectateur puisse se souvenir des positions des interlocuteurs sur un plateau. En inversant ces relations topographiques, on donne l'impression au spectateurs que les personnes ont échangé leurs places alors que c'est juste la caméra qui a changé de point de vue. Il s'agit donc d'une règle très importante pour assurer la continuité et la compréhension du spectateur.}.
\end{liste}


\paragraph{Valeurs de plan utilisées dans un script}
La valeur de plan est un des éléments le plus utilisé pour distinguer les plans entre eux, notamment au moment de l'écriture. 
Plutôt que de préciser les paramètres optiques de la caméra (considérés comme des détails très difficile à préciser à l'avance), le réalisateur préfère parler d'un type de plan pour donner une idée générale de l'image à obtenir. 
Le Tableau \ref{tab:vplans} présente les principales valeurs de plans utilisées par les professionnels, avec leur abréviations et leur(s) dénomination(s) anglaise(s) tandis que la Figure \ref{img:intro:plans} en propose une illustration.

\begin{table}[ht!]
   \begin{center}
		\begin{tabularx}{\textwidth}{|p{100pt}|X|p{100pt}|}
		   \hline
	\pc{Dénomination française} & \pc{Défintion} & \pc{Dénomination anglaise} \\ \hline\hline
 	\g{très gros plan (t.g.p.)} & plan cadrant une partie du visage, un détail du corps (un oeil, une bouche, un doigt etc.) ou le détail d'un objet. & extreme close-up, e.c.u. ; big close-up, b.c.u.\\ \hline

 	\g{gros plan (g.p.)} & plan isolant un visage, généralement cadré à la hauteur du noeud de cravate, ou un autre détail du corps (plan de détail ; insert), voire \e{tout ou partie d'un petit objet}. & close-up, c.u.\\ \hline

	\g{plan rapproché} & plan cadrant le(s) personnage(s) au niveau de la taille (plan rapproché taille, p.r.t.) ou de la poitrine (plan rapproché poitrine, p.r.p.). & medium close-up, m.c.u.\\ \hline
	
	\g{plan ceinture} & plan coupant les personnages au niveau de la ceinture & belt shot\\ \hline 

	\g{plan américain} (p.a.) & plan coupant les personnages à mi-cuisse & american shot ; medium close-shot, m.c.s.\\ \hline
	
	\g{plan moyen (p.m.) ou plan en pied} & plan présentant le(s) personnage(s) en pied. Il existe également des variations de ce plan qui sont nommées \e{serré} (aussi nommé plan américain large) ou \e{large} et qui font varier légèrement le cadrage. & medium shot, middle-shot, mid-shot, m.s. ; full shot, f.s.\\ \hline
	

	\g{plan de demi-ensemble (p.d.e., 1/2e.)} & plan mettant en place les personnages dans leur milieu en cadrant une bonne partie du décor & medium-long shot, m.l.s.\\ \hline

	\g{plan d'ensemble (p.e.)} & plan cadrant l'ensemble du décor construit & long shot, l.s.\\ \hline
	
	\g{plan de grand ensemble (p.g.e.)} & plan cadrant l'ensemble du décor construit de grande envergure. & very long shot, v.l.s.\\ \hline

	\g{plan général (p.g.)} & plan couvrant un vaste ensemble qui situe le décor construit dans son cadre : le décor dans le décor. & master shot ; extreme long shot, e.l.s\\ \hline 
		\end{tabularx}
		\caption{Valeurs de plans : du plus précis au plus général \label{tab:vplans}}
   \end{center}
\end{table}

% \begin{liste}
% 	\item \g{très gros plan} (t.g.p.) : \ciel{plan cadrant une partie du visage, un détail du corps (un oeil, une bouche, un doigt etc.) ou le détail d'un objet}. [extreme close-up, e.c.u. ; big close-up, b.c.u.]

% 	\item \g{gros plan} (g.p.) : \ciel{plan isolant un visage, généralement cadré à la hauteur du noeud de cravate, ou un autre détail du corps} (plan de détail ; insert), voire \ciel{tout ou partie d'un petit objet}. [close-up, c.u.]

% 	\item \g{plan rapproché} : \ciel{plan cadrant le(s) personnage(s) au niveau de la taille (plan rapproché taille, p.r.t.) ou de la poitrine (plan rapproché poitrine, p.r.p.).} [medium close-up, m.c.u.]
	
% 	\item \g{plan ceinture} : plan coupant les personnages au niveau de la ceinture [belt shot] 

% 	\item \g{plan américain} (p.a.) : \ciel{plan coupant les personnages à mi-cuisse} [american shot ; medium close-shot, m.c.s.]
	
% 	\item \g{plan moyen} (p.m.) ou \g{plan en pied} : \ciel{plan présentant le(s) personnage(s) en pied.} [medium shot, middle-shot, mid-shot, m.s. ; full shot, f.s.]
% 	Il existe également des variations de ce plan qui sont nommées \ciel{serré} (aussi nommé plan américain large) ou \ciel{large} et qui font varier légèrement le cadrage.

% 	\item \g{plan de demi-ensemble} (p.d.e., 1/2e.): \ciel{plan mettant en place les personnages dans leur milieu en cadrant une bonne partie du décor}. [medium-long shot, m.l.s.]

% 	\item \g{plan d'ensemble} (p.e.) : \ciel{plan cadrant l'ensemble du décor construit}. [long shot, l.s.]
	
% 	\item \g{plan de grand ensemble} (p.g.e.) : \ciel{plan cadrant l'ensemble du décor construit de grande envergure}. [very long shot, v.l.s.]

% 	\item \g{plan général} (p.g.) : \ciel{plan couvrant un vaste ensemble qui situe le décor construit dans son cadre : le décor dans le décor}. [master shot ; extreme long shot, e.l.s] 
% \end{liste}
%TODO:source

\begin{figure}[ht!]
\centering
\includegraphics[width=0.4\textwidth]{./images/ValeurPlan-v1.png}
\caption{Différentes valeurs de plan pour le cadrage d'un personnage à l'écran}
\label{img:intro:plans}
\end{figure}


\paragraph{Quelques documents de (pré)production}
%TODO:description + source
La pré-production se repose sur différents types de documents qui permettent de faire émerger progressivement la structure narrative ou documentaire, le découpage en plans et tous les détails de réalisation nécessaire à une bonne préparation du tournage. 
On notera que chacun de ces documents constitue un jalon dans la préparation du projet et que le script, résultat final de cette écriture, constitue une sorte de cahier des charges de l'objet audiovisuel à fabriquer.
\begin{liste}
	\item \g{sujet} : \ciel{matière première du film enrichie et développée lors de la préparation puis mise en forme au cours de la réalisation et du montage.} 
	
	\item \g{synopsis} : \ciel{exposé sommaire en quelques lignes, voire en quelques pages, du contenu dramatique ou documentaire d'un film}. 
	À noter que ce document est également utilisé plus tard dans la chaîne de production, notamment pour être transmis aux journalistes ou aux archivistes.
	De plus, il constitue la première mise en forme narrative du contenu du film, à la différence du sujet qui ne se constitue que de quelques idées directrices. 

	\item en cas d'adaptation d'une oeuvre littéraire en un objet audiovisuel, on développe un \g{traitement} : 
	\ciel{Travail littéraire préparatoire effectué à partir d'une oeuvre pré-existante ou d'une oeuvre originale pour assurer sa transmutation en termes cinématographiques.}

	\item lorsqu'on développe un objet audiovisuel original, à défaut de traitement on peut parler de \g{scénario} :
	\ciel{description de l'action d'un film épousant la forme \e{littéraire} du récit, rendant compte des articulations narratives et comportant une ébauche des dialogues, quelquefois la description plus précise de certaines scènes-clefs.}

	\item lorsque le besoin de préciser encore la construction de la narration, les auteurs peuvent construire une \g{continuité (dialoguée)} : \ciel{étape de la préparation écrite du film qui permet d'enrichir le traitement en développant chronologiquement les fragments d'action, en mettant au point le détail de chaque scène et en précisant le dialogue.}

	\item \g{plan de tournage} : \ciel{ultime travail de préparation effectué par le réalisateur avant le tournage. Il consiste à fragmenter la continuité en unités cinématographiques de temps et d'espace : les plans.}

	\item \g{script} : \ciel{dernière mouture du scénario, guide complet du tournage}.La Figure \ref{img:intro:script} présente un exemple de script extrait d'un film récent écrit et réalisé par Quentin Tarantino.
\end{liste}

\begin{figure}[ht!]
\centering
\includegraphics[width=0.7\textwidth]{images/ScriptExample-v1.png}
\caption{Extrait du script de Kill Bill écrit et réalisé par Quentin Tarantino}
\label{img:intro:script} 
\end{figure}

% des réécritures de documents et des mises à jours qui pourraient être réalisées par des machines, des informations qui pourraient être transmises automatiquement à travers un réseau numérique d'information


% un vocabulaire bien défini qui fait l'objet de nombreux dictionnaires, donc prêt à être formalisé


% des équipes qui sont réduites lors de tournage en extérieur




%%%%%%%%%%%%%%%%%%%%%%%%%%%%%%%%%%%%%%%%%%%%%%%%%%%%%%%%%%%%%%%%%%%%%%%%%%%%%%%%%%%%%%%%%%%%%%%%%%%
\subsection{Besoins Métiers (m)}\label{sec:besoins}
\e{
Face à une numérisation qui fragmente les contenus, une mise en réseau qui facilite la fabrication amateur, intensifie la circulation de ces fragments (\ref{sec:motiv}), la production audiovisuelle rencontre de nouveaux défis qui remettent en jeu son organisation et sa manière de se représenter le monde. 
Ainsi d'une part la fragmentation ne doit pas mettre en péril la cohérence de l'ensemble et d'autre part, la circulation des contenus ne doit pas compromettre l'exploitation future du tout ou des parties. 
L'ouverture d'une chaîne de production à des acteurs tiers implique de clarifier les attendus de chacun. 
Lorsqu'il s'agit de faire fabriquer ou de récupérer du contenu, il devient nécessaire pour le client de décrire la commande de contenu au fournisseur, de même que le fournisseur doit décrire à son client le contenu livré pour faciliter son exploitation. 
Ainsi, on souhaite adjoindre aux contenus des descriptions qui permettent de faciliter leur recherche et leur manipulation.
}


Pour les professionnels de la production audiovisuelle, le défi porte à la fois sur l'organisation de leur chaîne de production et sur la gestion de leurs produits :
\begin{liste}
	\item[(1)] \g{comment transformer la chaîne de production afin de l'ajuster à la diversification des formes de fabrication et de distribution des contenus, mais aussi aux changements dans les pratiques de consommation des audiences ?}

	\item[(2)] \g{comment passer d'une gestion de fichiers à une gestion de contenus audiovisuels considérés comme des objets numériques fragmentés dont il s'agit de garantir l'autonomie dès leur conception et jusque dans leurs différents cadres d'exploitation ?}
\end{liste}

% Problèmes métiers ? Tant que ça ne devient pas une solution, mais des tendances à prendre en compte 

On peut ensuite détailler ces défis en objectifs plus précis :
\begin{liste}
	\item[(1a)] \e{accorder contribution amateur et production professionnelle pour fabriquer ou valoriser du contenu.}

	L'amélioration croissante des capteurs des appareils multimédia ajoutée aux capacités de communication offrent au grand public de plus en plus de manières de participer aux processus de fabrication ou de diffusion des contenus.
	Les possibilités accrues de participation au processus médiatique (participation à l'émission, envoi de contenu, propagation via ses contacts, commentaires etc.) valorisent le spectateur, le contenu et la plate-forme de diffusion (comme d'une certaine manière peut le faire le bouche à oreille).

	De manière générale, l'intégration de contenus externes dans une chaîne de production professionnelle ne s'envisage  qu'à partir d'un certain niveau de qualité du contenu livré.  
	Paradoxalement, dans certains cas les signes d'une production amateur (tremblements, caméra à l'épaule etc.) peuvent être revendiquées comme des marque de style qui suggère une collaboration avec le public ou une proximité avec une réalité éloignée des images diffusées par les médias.
	Ainsi, les professionnels souhaitent encadrer plus ou moins fortement la production amateur par des indications, recommendations, obligations.\\


	\item[(1b)] \e{créer de nouvelles étapes dans la chaîne visant à réutiliser les contenus existants et les adpater à de nouveaux modes de consommation.}

	L'augmentation de l'offre de contenus accessibles aux spectateurs (chaînes, enregistrements, balladodiffusion, vidéo à la demande etc.) se traduit par une mise en concurrence accrue des contenus diffusés par les professionnels.
	Le contrôle de l'offre n'étant plus atteignable, il faut adopter de nouvelles stratégies de valorisation des contenus produits ou diffusés pour maintenir leur visibilité et leur rentabilité. 
	Une autre approche consiste à fournir un service de recommandation aux spectateurs et ainsi rentrer dans une démarche de fidélisation. 

	Par ailleurs, l'augmentation des terminaux de lecture multimédia et leur portabilité offrent de plus en plus d'occasions aux spectateurs de consommer des contenus. 
	Par exemple, les situations de mobilités peuvent impliquer des capacités de transfert diminués, un écran plus petit, des temps de disponibilités plus courts etc.
	Il semble alors que la production doivent évoluer pour fournir de nouveaux formats ou des formes retravaillées de contenus existants.

	Dans tous les cas, cela implique de se consacrer à des tâches d'éditorialisation des contenus pour répondre aux exigences et aux attentes de ces nouveaux modes de consommation.\\
% \end{liste}


% \begin{liste}
	\item[(2a)] \e{gérer l'intégration de contenu externes, les variations d'un même contenu pour les rattacher à un même objet numérique.} 
	% représentation

	L'utilisation de contenus provenant de sources externes de même que la production de multiples variations d'un même contenu augmente le nombre de ressources à gérer. 
	De plus, les relations entre ces différentes ressources nécessitent d'être clarifiées et explicitées dans le système de gestion. 
	% chaîne éditoriale ? 
	
	Il ne s'agit plus simplement de gérer des fichiers mais un ensemble de fichiers et de données qui constituent un ensemble cohérent et fragmenté que l'on nomme un objet numérique. 
	Cet objet doit intégrer à la fois les diverses sources qui le composent mais aussi des variations correspondants aux exploitations visées, des descriptions et tout ce qui permet de garantir son autonomie. 
	Il doit également s'agir d'un objet \e{métier} car son statut, son organisation, sa sémantique correspondent à la vision d'un métier, à la manière dont il pense le monde. 

	% Ainsi, on ne souhaite plus gérer des fichiers mais des objets numériques qui doivent acquérir un statut, une sémantique correspondant à la manière dont les métiers de la production les considèrent.\\
	% qui possède une valeur et une sémantique propre à un contexte d'usage. 	


	\item[(2b)] \e{associer des descriptions aux contenus pour faciliter leur exploitation dans un environnement numérique.}
	% description

	L'augmentation des contenus audiovisuels en circulation, la diversification de leurs modes d'exploitation compliquent la gestion des contenus.  
	Afin de favoriser la réutilisation de ces contenus, il faut pouvoir leur attacher des informations pertinentes pour les professionnels qui les manipulent. 
	La description du contenu peut varier suivant les besoins de chaque métier impliqué dans la chaîne de production. 
	Les opérations n'étant pas les mêmes, les descriptions de ces opérations varient donc également et sont nécessaires pour faciliter la réutilisation du contenu. 
	%de manière à faciliter modalités d'exploitation envisagées  

	Lorsque la réutilisation et production s'entremêlent, il est également nécessaire de construire les descriptions en même temps que le contenu. 
	De cette manière on récupère ou on réévalue l'information à mesure de l'avancée dans la chaîne. 

	Cela nécessite d'informatiser l'étape de pré-production de la chaîne et de modéliser les informations utilisées par les professionnels. 
	%commencer plus tôt, avoir plusieurs niveaux/types de description, raccrocher les bons éléments à la représentation de l'objet numérique
	
	%et embarquent des descriptions explicitant leurs modalités d'exploitation.

\end{liste}

\e{
En guise de synthèse, nous pouvons dire qu'il s'agit de constituer des objets audiovisuels autonomes dans les chaînes de production audiovisuelle. 
Nous précisons ce caractère autonome, car ces objets seront porteurs de leur propre description et associés à des connaissances sur l'organisation de la chaîne de production dans lesquels ils évoluent.
Ainsi, ces objets audiovisuels pourront être (ré)introduits à n'importe quelle étape d'une chaîne de production et fourniront aux contributeurs concernés des informations propres à faciliter leur (ré)utilisation.
}
	% \item \eg{valoriser et éditorialiser les contenus existants pour les rendres plus visibles, plus attrayants auprès des audiences ciblées.}
	% L'augmentation de l'offre de contenus accessibles aux spectateurs (chaînes, enregistrements, balladodiffusion, vidéo à la demande etc.) se traduit par une mise en concurrence accrue des contenus diffusés par les professionnels.

	% Le contrôle de l'offre n'étant plus atteignable, il faut adopter de nouvelles stratégies de valorisation des contenus produits ou diffusés pour maintenir leur visibilité et leur rentabilité. 
	% Une autre approche consiste à fournir un service de recommandation aux spectateurs et ainsi rentrer dans une démarche de fidélisation. 
	% Cela implique de se consacrer à des tâches d'éditorialisation des contenus pour des audiences plus ciblées.
	
	% \item \eg{produire de nouvelles formes de contenus ou adapter les rmes existantes pour satisfaires aux nouveaux modes de distribution/consommation.}
	% L'augmentation des terminaux de lecture multimédia et leur portabilité offrent de plus en plus d'occasions aux spectateurs de consommer des contenus. Par exemple, les situations de mobilités peuvent impliquer des capacités de transfert diminués, un écran plus petit, des temps de disponibilités plus courts etc.
	% Il semble alors s'ouvrir une place pour de nouveaux formats ou des formes retravaillés de contenus existants. 
	% Il s'agit de faire de la production multi-support et d'adapter les contenus en fonction des conditions de distribution et de l'audience visé (réutilisation).
	
	% \item \eg{articuler la contribution amateur avec la chaîne de production professionnelle.}
	% L'amélioration croissante des capteurs des appareils multimédia ajoutée aux capacités de communication offrent au grand public de plus en plus de manières de participer aux processus de fabrication ou de diffusion des contenus.
	% Les possibilités accrues de participation au processus médiatique (participation à l'émission, envoi de contenu, propagation via ses contacts, commentaires etc.) valorisent le spectateur, le contenu et la plate-forme de diffusion.

	% Cependant, il faut être capable d'intégrer ces contributions externes au sein de la production professionnelle en les encadrant plus ou moins fortement, par des indications, recommendations ou des contraintes.

	% \item \eg{gérer les objets audiovisuels dès le début et tout au long de leur cycle de vie.}
	% %
	
	% Une circulation plus importante des contenus implique de trouver un moyen de gérer non plus des fichiers mais des objets numériques qui unifient plusieurs variations d'un même contenu et embarquent des descriptions explicitant leurs modalités d'exploitation.



% numériser => (a) fragmenter + (b) mettre en réseau => (a) besoin de conserver la cohérence de l'ensemble + (b) besoin d'autonomiser pour une future situation d'usage








%%%%%%%%%%%%%%%%%%%%%%%%%%%%%%%%%%%%%%%%%%%%%%%%%%%%%%%%%%%%%%%%%%%%%%%%%%%%%%%%%%%%%%%%%%%%%%%%%%%
\newpage
\section{Problèmes}\label{sec:prob}



%%%%%%%%%%%%%%%%%%%%%%%%%%%%%%%%%%%%%%%%%%%%%%%%%%%%%%%%%%%%%%%%%%%%%%%%%%%%%%%%%%%%%%%%%%%%%%%%%%%
\subsection{Problèmes métiers (m)}\label{sec:pmetiers}
% Nos champs d'applications sont : 
% Prenant acte des besoins de la production audiovisuelle nous distinguons deux problèmes métiers (1) identifier le(s) niveau(x) de fragmentation et (2) le(s) type(s) de description susceptibles de favoriser la fabrication mixte, la circulation et la réutilisation des contenus.
% Ces problèmes remettent en cause à la fois la représentation classique des contenus et leur description. 
% de manière à faciliter (1) la production mixte amateur-professionnel et (2) leur réutilisation dans de nouveaux contextes d'exploitation.
% l'informatisation du début de la chaîne de production, préproduction 
% l'inscription formelle de l'écriture audiovisuelle ?

\e{
L'objectif central qui se pose à la production audiovisuelle est de constituer des objets audiovisuels autonomes et donc (ré)utilisable à n'importe quel étape de la chaîne. 
Or, dans la chaîne de production classique les programmes n'émergent qu'à la fin de la chaîne et sont gérés d'une pièce. 
Il n'y a pas forcément de place pour les éléments de contenu intermédiaires, et ce sont justement à la modélisation de ces fragments que l'on s'attaque. 
Ces fragments doivent devenir des éléments documentaires qui possèdent leur unité propre de même que les objets finis que sont les programmes. 
Une des difficultés réside dans cette articulation entre des fragments et le tout ou l'ensemble que constitue les programmes. 
Par ailleurs, il existe des problèmes sous-jacents à cette fragmentation documentaire : 
}
\begin{liste}
	\item \e{identifier quels niveaux de fragments peuvent prétendre à ce genre de transformation}.
	Si le numérique permet de fragmenter à l'envie, il faut cependant prendre en compte les pratiques du métier pour identifier les niveaux de fragmentation pertinents ou déjà utilisé dans le métier mais non modélisé.
	Par exemple, la prise de vue est le résultat d'une activité de tournage, pour autant elle ne constitue pas un objet éditorial comme peut l'être une interview.
	De plus, il s'agit également de déterminer comment manipuler ces fragments comme des objets à part entière sans compromettre l'articulation de l'ensemble. 
	Par exemple, une interview peut s'intégrer dans un journal télévisé ou bien un reportage dans des versions plus ou moins courtes. 
	Pour autant, il s'agit du fruit d'une même activité, simplement le montage, et donc le résultat, est différent suivant le programme dans lequel l'interview s'insère.\\
	
	\item \e{identifier quelles informations et connaissances doivent être rattacher à ces fragments pour les rendre autonome}.
	D'une manière similaire à la démarche pour les objets entiers, certaines connaissances, notamment relatives au contexte de production, doivent être attachées au fragment pour garantir sa réutilisation et sa cohérence. 
	En reprenant l'exemple de la prise de vue, on peut lui associer le bout de script qui a prescrit ce qu'elle devait montrer. 
	Si l'on pousse encore cette logique, il faut également incorporer le document qui définit le programme pour lequel on a tourné cette prise de vue, la personne qui l'a effectué, les équipements utilisés etc. 
	Ainsi, on étend la modélisation de l'objet audiovisuel à son contexte de production et tout ce qui renforce les possibilités de recherche, de manipulation, de gestion et de transformation de ces objets. 
	De plus, s'ajoute à cela la question de la collecte de ces informations. 
	En effet, la saisie de ces informations au sein d'un système d'information et leur utilisation par les acteurs de la chaîne n'est pas une simple formalité.
	Ce problème pousse également dans le sens d'une modélisation plus contextuelle, de manière à proposer un environnement de travail adapté et utilisable aux contributeurs de la chaîne.\\
	
	% \item \e{}.
\end{liste}


% Notre problème se situe dans le croisement de la représentation des contenus et la représentation des activités humaines qui construisent, manipulent, éditent, transforment, publient et documentent ces contenus. 
% Cet angle de recherche nous amène donc à considérer non pas le contenu audiovisuel dans son ensemble, une fois terminé et validé, mais la construction de tout ces fragments qui le composent. 
% Il nous faut aussi considérer, non pas seulement le contenu audiovisuel, mais aussi d'autres informations, d'autres documents qui constituent son contexte de production, dans un sens très général. 
% Chaque prise de vue constitue donc un objet à représenter en tant que tel, tout autant que le bout de script qui a prescrit ce que cette prise de vue devait montrer. 
% Si l'on pousse encore cette logique, on peut alors représenter de même le document qui définit le programme pour lequel on a tourné cette prise de vue, la personne qui l'a effectué, les équipements utilisés etc. 
% Ainsi, on étend la représentation du contenu à son contexte de production entendu comme toutes les informations qui renforceront les possibilités de recherche, de manipulation, de gestion et de transformation de ces contenus. 

%%%%%%%%%%%%%%%%%%%%%%%%%%%%%%%%%%%%%%%%%%%%%%%
\subsection{Problèmes scientifiques (m)}\label{sec:scien}
Au fur et à mesure que la circulation des contenus s'intensifie, il y a un besoin grandissant de faciliter l'échange d'information tant à la fois sur le plan informatique, que sur le plan humain. 
De plus, l'ouverture de la chaîne de production à de nouveaux contributeurs (amateurs et professionnels) ne fait qu'accentuer la disparité des connaissances et des systèmes utilisés. 
Afin de construire une compréhension commune à tous les contributeurs au cycle de vie, on fabrique un modèle conceptuel capable d'intégrer et de mettre en relation leurs connaissances. 
Il s'agit là d'un apport par rapport à la situation existante où le consensus n'existait pas, ou alors de manière éphémère, locale au sein d'une équipe.
L'objectif est de fluidifier les échanges d'information et de contenus en formalisant les connaissances utilisées pour :
\begin{liste}
% modéliser tous les objets de la chaîne de production audiovisuelle
	\item[(A)] \g{modéliser les objets construits au fil de la chaîne de production audiovisuelle}.
	\item[(B)] \g{modéliser les connaissances sur ces objets} (descriptions, contexte de production, contribution au cycle de vie).
	% \item[(B)] décrire les objets audiovisuels
	% \item[(C)] représenter la contribution de chacun des acteurs au cycle de vie des objets audiovisuels
\end{liste}

\e{
Ainsi, notre problème de recherche général s'articule autour de la modélisation des connaissances et des informations que les contributeurs construisent, utilisent, échangent au cours du cycle de vie des objets audiovisuels. 
Cette modélisation constitue une première étape dans la mise en place d'un système d'information servant à mieux gérer les objets audiovisuels et médier la communication entre systèmes informatiques tout autant qu'entre contributeurs humains.
Après avoir numérisé les contenus audiovisuels, on souhaite transformer chaque élément les composant en objet documentaire et documenter leur cycle de vie.\\}



\g{(A)} Le problème est d'organiser la gestion des objets audiovisuels en proposant une modélisation capable de faciliter leur identification, leur manipulation et leur réutilisation tout au long de leur cycle de vie. 
En particulier,	l'objet audiovisuel professionnel est produit de manière collective, chaque contributeur apportant un élément à l'ensemble. 
Ces contributions doivent donc pouvoir être identifiées comme appartenant à un ensemble, de même que chaque élément doit pouvoir être considéré pour soi afin d'être intégré dans un autre ensemble (réutilisation).

Pour cela on adopte une représentation des différents niveaux d'abstraction des objets audiovisuels numériques de façon à rétablir les liens entre les différentes versions ou copies d'un même contenu, quelque soit la nature des variations entre elles (encodage, format d'encapsulation, montage, finition, langue etc.).
La distinction entre différents niveaux de modélisation (technique, esthétique, éditorial etc.) doit permettre de construire une représentation dynamique de l'objet audiovisuel qui suit l'avancement du processus de production.\\
% La distinction entre différents niveaux de modélisation (technique, esthétique, éditorial etc.) doit permettre de construire une représentation de l'objet audiovisuel au fur et à mesure de l'avancement du processus de production.\\


\g{(B)} Le problème est d'attacher plusieurs types de connaissances aux objets audiovisuels de manière à les rendre autonomes dans leur circulation et leur réutilisation. 
Afin de faciliter l'échange d'information et la réutilisation des objets audiovisuels entre différents contextes, il faut modéliser des connaissances sur ces objets qui sont parfois déjà existantes mais non formalisées, ou bien qu'il faut rendre compréhensibles.
En effet, l'échange d'information dans la production audiovisuelle est primordial et s'effectue entre métiers et organisations différents, voire avec des amateurs. 
En particulier, on souhaite s'appuyer sur le vocabulaire de l'écriture filmique utilisé dans des documents de préproduction pour spécifier les résultats attendus de la production.
L'information contenue dans ces documents est importante mais repose sur des conventions plus ou moins tacites qu'il faut expliciter pour les professionnels, expliquer pour les amateurs.
La formalisation des ces éléments devra donc pouvoir être lu et modifié tout au long du cycle de vie des objets audiovisuels par tout types de contributeurs.

% échange d'info, adaptation par l'explicitation du vocabulaire et la contribution au cycle de vie
La formalisation de l'écriture filmique permettra d'adapter la présentation de l'information en fonction des connaissances, de l'implication du contributeur dans la chaîne (rôle, tâche, niveau de compétences etc.), de son référentiel professionnel ou linguistique. 
Il s'agit alors d'établir des correspondances entre les connaissances connues par le lecteur d'une information et celles utilisées par la personne qui l'a exprimé.
Ainsi, d'une part on explicite l'expression de l'information, ce qui permet l'adpatation, et facilite son interprétation ultérieure.
Le processus prend tout son sens lorsqu'il s'agit de traduire un concept de la réalisation audiovisuelle pour guider un amateur dans son tournage.
% Par exemple, l'action écrite par l'auteur et transformé en scène par le réalisateur, doit ensuite être tourné un caméraman, des acteurs etc. 


% réutilisation, attachement des connaissances pertinentes pour manipuler ou exploiter chaque fragment ou l'objet en entier
De plus, les descriptions utilisées, en plus de permettre de spécifier le résultat attendu de la production, doivent permettre de faciliter la recherche, la manipulation et la réutilisation des objets audiovisuels.
Pour cela, il faut articuler ces connaissances aux objets et aux fragments qui les composent. 
Le vocabulaire de l'écriture filmique sera également précieux, puisqu'il nous donne une unité de base, le plan, ainsi que ces caractéristiques qui permettent de le distinguer des autres. 
La recherche dans des dépôts de contenus se ferra ainsi de manière similaire à la commande de contenu à d'autres contributeurs, qu'ils soient professionnels ou amateurs.
./



% Il s'agit donc de définir un modèle de description susceptible d'être utilisé par les contributeurs professionnels ou amateurs, à toutes les étapes de la chaîne de production. 


% Afin de décrire les contenus, on souhaite s'appuyer sur le vocabulaire de l'écriture audiovisuelle utilisé dans les différentes étapes de la chaîne et notamment dès la préproduction. 
% Cette écriture repose sur un vocabulaire des techniques de réalisation audiovisuelle (prise de vue, transition, composition de l'image etc.) renvoyant à des effets largement connus dans le milieu de l'audiovisuel et chez les cinéphiles. 
% L'écriture est utilisée avant la fabrication du contenu pour la spécifier, puis pendant la fabrication pour enregistrer les différences. 
% Une formalisation de ce vocabulaire permettrait de construire une description textuelle d'un contenu à partir d'une description objective de la réalisation (réglages des appareils, position des acteurs etc.).\\



% \g{(C)} Le problème est de représenter et faciliter l'échange d'information entre des contributeurs hétérogènes dans leurs connaissances et leur implication dans la chaîne de production.
% Une première diffculté réside dans l'articulation entre les connaissances des contributeurs qui expriment l'information et ceux qui l'interprèteront.
% La seconde diffculté consiste dans l'articulation des représentations du cycle de vie, de l'objet audiovisuel et des descriptions qui leurs sont associées. 
% En effet, chaque contributeur peut participer à la constitution d'informations associées au contenu en cours de sa production.
% Ces informations varient en fonction de l'implication du contributeur dans la chaîne (rôle, tâche, niveau de compétences etc.). 
% De plus, les informations construites à un moment sont susceptibles d'être utilisées plus tard dans la chaîne, par un contributeur ne partageant pas forcément les mêmes connaissances ou le même référentiel professionnel ou linguistique.

% On cherche alors à réaliser une adaptation de la forme d'expression de ces informations afin de faciliter le déroulement du processus de production. 
% Dans un premier temps, on explicite les connaissances utilisées par un premier utilisateur pour exprimer une information. 
% Ensuite, on établit une correspondance avec les connaissances connues d'un autre utilisateur et on adapte au besoin la forme de d'expression de cette information pour faciliter son interprétation. 
% L'adpatation qui en résulte prend tout son sens lorsqu'il s'agit de traduire un concept de la réalisation audiovisuelle pour guider un amateur dans son tournage.


%%%%%%%%%%%%%%%%%%%%%%%%%%%%%%%%%%%%%%%%%%%%%%%
\section{Positionnement Disciplinaire (n,i)}\label{sec:posd}
[Ingénierie des connaissances ; Media Asset Management ; Gestion électronique de Documents ; Ingénierie documentaire]
% ingénierie des connaissances (représentation des connaissances) ingénierie des inscriptions numériques de connaissances, dont les documents
% ingénierie documentaire (modélisation des documents propres à la production audiovisuelle)
% indexation et gestion des connaissances (description des contenus audiovisuels)
% la ged s'occupe de la gestion de documents, nous proposons de gérer des fragments de documents, de gérer leur construction en plusieurs étapes, par plusieurs acteurs et dans le cadre de différentes missions.

% Pour définir cet ensemble d'informations qui forment le contexte de production, nous nous sommes appuyés sur les partenaires du projet MediaMap. 



% des réécritures de documents et des mises à jours qui pourraient être réalisées par des machines, des informations qui pourraient être transmises automatiquement à travers un réseau numérique d'information
% un vocabulaire bien défini qui fait l'objet de nombreux dictionnaires, donc prêt à être formalisé
% des équipes qui sont réduites lors de tournage en extérieur

% à mettre dans le pos. disciplinaire, comment on aborde les problèmes posées
% [Nous avons ainsi dégagé plusieurs perspectives métiers qui nous ont servi de guide pour identifier les échanges d'informations les plus importants ainsi que le vocabulaire utilisé pour les exprimer. 
% Chacune de ces perspective possède un objectif propre et des spécificités, cependant il apparaît qu'un langage commun est utilisé par tous les acteurs de la production. 
% En se concentrant sur la description d'un contenu existant ou à venir, ce langage permet à ces acteurs de communiquer entre eux. Le réalisateur qui spécifie un attendu dans son script, les caméraman qui réalisent le cadrage, les opérateurs lumières etc. 
% Tous utilisent ce langage pour imaginer le résultat à produire et en déduire les gestes à opérer. 
% Les usages n'étant jamais complètement figé, chaque organisation développe ses propres idiomatismes de langage. 
% Dans ce cas, la collaboration entre organisations impliquent de pouvoir réaliser des ajustements dans l'expression de la description du contenu. 
% De même, la collaboration avec des contributeurs amateurs soulève un problème de compréhension de ce langage (et donc de l'attendu) mais aussi de connaissances des gestes à opérer (pour produire le résultat attendu).
% Ainsi, à mesure que la circulation des contenus s'intensifie, que les besoins de collaboration augmentent, naît un besoin grandissant d'explicitation des échanges d'information afin de dégager une vue d'ensemble de la chaîne de production, de ses acteurs, de leurs interactions, de leurs produits. 

% Notre proposition consiste à modéliser ces éléments et à en informatiser l'accès de manière à fluidifier les échanges de contenus et faciliter la compréhension des informations afférentes.] 


% \cleardoublepage





%%%%%%%%%%%%%%%%%%%%%%%%%%%%%%%%%%%%%%%%%%%%%%%%%%%%%%%%%%%%%%%%%%%%%%%%%%%%%%%%%%%%%%%%%%%%%%%%%%%
%%%%%%%%%%%%%%%%%%%%%%%%%%%%%%%%%%%%%%%%%%%%%%%%%%%%%%%%%%%%%%%%%%%%%%%%%%%%%%%%%%%%%%%%%%%%%%%%%%%
\part*{État de l'Art}
%%%%%%%%%%%%%%%%%%%%%%%%%%%%%%%%%%%%%%%%%%%%%%%%%%%%%%%%%%%%%%%%%%%%%%%%%%%%%%%%%%%%%%%%%%%%%%%%%%%
%%%%%%%%%%%%%%%%%%%%%%%%%%%%%%%%%%%%%%%%%%%%%%%%%%%%%%%%%%%%%%%%%%%%
%%%%%%%%%%%%%%%%%%%%%%%%%%%%%%%%%%%%%%%%%%%%%%%%%%%%%%%%%%%%%%%%%%%%
%%%%%%%%%%%%%%%%%%%%%%%%%%%%%%%%%%%%%%%%%%%%%%%%%%%%%%%%%%%%%%%%%%%%
%%%%%%%%%%%%%%%%%%%%%%%%%%%%%%%%%%%%%%%%%%%%%%%%%%%%%%%%%%%%%%%%%%%%
\chapter{Outils de modélisation}\label{chap:omod}
\minitoc

% pourquoi on parle de SOC ? Parce qu'on souhaite représenter des objets audiovisuels, leur production, leur description et que ceci ne peut se faire sans une représentation du vocabulaire utilisé dans la production audiovisuelle. 

Dans le cadre de la production audiovisuelle collaborative, qui implique à la fois des amateurs et des professionnels, un des enjeux que nous avons noté (\ref{sec:scien}) est de rendre plus compréhensible l'échange d'informations entre contributeurs. 
D'une part, nous avons des practiciens professionnels utilisant un ou plusieurs vocabulaires métiers suffisamment définis pour que l'on puisse en faire des dictionnaires (\cite{Journot2008}; \cite{Pinel2008}). 
Ces personnes font usage de la langue d'une manière précise, régie par des conventions et portée par une conception de la production audiovisuelle et de ses objets, que l'on suppose stabilisée, au moins localement (au sein d'un même pays, d'une même école de pensée, organisation, équipe etc.).
Toutefois, on s'attend à des variations dans les usages de la langue, au même titre que l'on suppose qu'il existe des variations entre pratiques des gens du métiers.
Cependant, il semble qu'il s'agit seulement de variations et qu'il soit possible alors d'identifier les éléments communs et distincts.

D'autre part, nous avons les amateurs, qui n'ont pas le support de ces conventions travaillées au quotidien.
Ils sont, au mieux, des intermittents éclairés qui ont saisi le sens d'éléments de langage propres aux métiers (par des lectures, des rencontres, des formations etc.). 
Il faut donc supposer qu'il y a tout à expliquer à ces amateurs, plutôt que de parier sur leur compréhension innée des métiers de la production audiovisuelle.
En particulier s'il s'agit de demander du contenu à des amateurs sous la forme d'un script de tournage, il faudra alors trouver un moyen d'expliciter cette commande et d'expliquer ou d'assister sa réalisation. 

Ainsi, l'écart entre collaborateurs (qu'ils soient tous professionnels, ou un mélange d'amateurs et de professionnels) peut se situer sur différents niveaux : 
\begin{liste} 
	\item au niveau du vocabulaire métier, comme les mots utilisés dans le script pour désigner tel ou tel type de plan etc. 
	Dans le cas d'amateurs, il faut supposer que ces mots sont inconnus ou méconnus ; dans le cas de professionnels, on préfèrera expliciter le vocabulaire afin d'éviter toute confusion. 
	
	\item au niveau des connaissances et de la manière de conceptualiser le métier, le cycle de production et ses objets. 
	Là encore, un amateur ne connaît pas ou très peu les détails classiques des méthodes de production professionnelle. 
	De plus, la production audiovisuelle étant organisée en projets distincts, les méthodes peuvent fortement varier entre la production d'un documentaire et d'une émission de variétés. 
	Chaque genre, chaque équipe aura donc ses propres objets, ses propres méthodes qu'il faut alors expliciter aux autres professionnels pour s'assurer de leur collaboration. 

	\item sur le plan pratique, il faut également remarquer que les compétences peuvent également varier fortement entre métiers, suivant les genres de production audiovisuelle. 
	Ainsi, en plus d'expliciter les échanges d'informations, il serait également souhaitable de proposer une assistance aux collaborateurs amateurs ou professionnels pour s'assurer que le résultat produit correspond bien à l'attente initiale. 	
\end{liste}


Nous présentons dans une première section un exemple de commande de tournage qui illustre ces différents écarts en impliquant des communautés d'amateurs et de professionnels (\ref{sec:cdcf}). 
Ce scénario d'usage nous permet de préciser les besoins en modélisation exprimés précédemment (\ref{sec:prob}).
Notamment, il apparaît nécessaire de représenter à la fois la ou les conceptualisation(s), le(s) vocabulaire(s) utilisé(s) ainsi que les résultats attendus par les acteurs de la chaîne de production audiovisuelle. 
Ainsi, nous nous intéresserons à l'utilisation de divers Systèmes d'Organisations de Connaissances (SOC, \cite{Zacklad2010}) pour mettre en place un partage d'information normalisée. 
Après un retour sur les définitions principales que nous utiliserons, nous examinerons les langages, modèles et normes existants qui permettent de représenter des SOC (\ref{sec:defs}).
La distinction entre terminologie et ontologie nous permet de détailler le fonctionnement d'une méthode de construction d'ontologie différentielle (\ref{sec:construction}).
Enfin, nous présenterons les langages permettant de représenter ces SOCs (\ref{sec:mods}). 

%Dans une seconde partie, nous examinerons les modèles de l'audiovisuel existants.




%%%%%%%%%%%%%%%%%%%%%%%%%%%%%%%%%%%%%%%%%%%%%%%%%%%%%%%%%%%%%%%%%%%%
%%%%%%%%%%%%%%%%%%%%%%%%%%%%%%%%%%%%%%%%%%%%%%%%%%%%%%%%%%%%%%%%%%%%
\section{Cahier des charges fonctionnel (n)}\label{sec:cdcf}


%%%%%%%%%%%%%%%%%%%%%%%%%%%%%%%%%%%%%%%%%%%%%%%%%%%%%%%%%%%%%%%%%%%%
\subsection{Scénario de commande de tournage}\label{sec:scenar}
Considérons comme cas d'étude une commande de tournage en vue de réaliser des reportages sur des évènements culturels de type concert ou opéra. 
Il met en jeu trois communautés en collaboration :

\begin{itemize}
	\item la RTBF (Radio Télévision Belge Francophone) établit des commandes de contenu dans un jargon métier propre. Son objectif est d'externaliser dès que possible la réalisation de la commande. Cela implique une compréhension commune sur le contenu à réaliser qui passe par un accord sur la manière de décrire la commande. 
	
	\item le contenu commandé est tourné soit par la VRT (Radio-Télévision Flamande) qui utilise un jargon différent de la RTBF, soit des amateurs qui ne connaissent pas les concepts de la réalisation audiovisuelle. Dans le premier cas, la conceptualisation est commune, seuls les termes changent. Dans le second cas, il s'agit d'expliquer et d'illustrer les concepts utilisés.\\
\end{itemize}

Le développement d'une application d'assistant de tournage pour guider les amateurs paraît souhaitable pour amener le contenu filmé à un niveau de qualité exploitable. 
Il faut cependant faire la distinction entre les propositions de dépôt spontané de contenu (comme le pratique une chaîne d'information telle que BFM\footnote{La rubrique témoins BFM permet à un utilisateur de déposer des photos ou vidéos sur le site. 
Après modération, le contenu est diffusé et peut même faire l'objet d'une vente. Voir http$:$//temoins.bfmtv.com/}) et les appels à contribution où le professionnel passe commande auprès d'amateurs en détaillant ses exigences. 

Dans ce dernier cadre, on souhaite fournir un plan de tournage au caméraman afin de guider sa prise de vue. 
Le plan de tournage est construit à partir de recommandations rédigées par un réalisateur (position, cadrage, lumière, etc.) utilisant un vocabulaire métier. 
L'originalité de l'application est d'adapter l'information présentée au caméraman suivant ses capacités (amateur, professionnel) ou son employeur si c'est un professionnel travaillant dans une tierce organisation. 
On suppose ainsi que malgré quelques variations dans le vocabulaire utilisé, les professionnels de l'audiovisuel utilisent les mêmes concepts pour décrire le contenu. 
Par exemple, la notion de cadrage fait appel à des concepts de valeurs de plan indiquant la portion visible d'un personnage à l'écran (voir Figure \ref{img:intro:script}, page \pageref{img:intro:script}).
Un \textit{plan américain} indique ainsi que le personnage principal est cadré de la tête jusqu'au dessus des genoux. 
Le terme est utilisé en Europe en rappel à son emploi caractéristique dans les films américains des années 1910-1940, notamment dans les westerns où il permettait de montrer l'ensemble du pistolet à la ceinture des personnages\footnote{Roger Boussinot, l'Encyclopédie du Cinéma, Bordas.}. 
Ce cadrage est aussi appelé \textit{plan 3/4} et en anglais 3/4 shot, medium-long shot ou american shot pour traduire l'expression popularisée en Europe. 
Si le terme utilisé varie suivant le lieu et la littérature de référence, la définition de ce type de cadrage est sans équivoque. 

%=============
% \begin{figure}[htb]
% \centering
% \includegraphics[width=0.5\textwidth]{./images/ValeurPlan-v1.png}
% \caption{Différentes valeurs de plan pour le cadrage d'un personnage à l'écran}
% \label{fig:cadrage}
% \end{figure}
%=============

Les amateurs quand à eux ignorent ces concepts et n'ont pas été initiés à ces pratiques. Ils ont donc besoin d'explications et d'illustrations pour comprendre les recommandations du réalisateur. 
Dans le cas du cadrage, une illustration graphique est d'autant plus pertinente. 
L'enjeu se situe donc dans la collaboration entre un prescripteur et un opérateur qui doivent s'accorder sur le contenu à produire malgré la différence de vocabulaire. 
% Un exemple des différences de présentation entre amateur et professionnel est illustré figure \ref{fig:prescription}.


% %%=============
% \begin{figure}[htb]
% \centering
% \includegraphics[width=0.3\textwidth]{./images/ShootingRecommandation-v1.png}
% \caption{Exemple de prescription de tournage à destination de professionnels (en haut) ou d'amateurs (en bas)}
% \label{fig:eda:prescription}
% \end{figure}
% %%=============


%%%%%%%%%%%%%%%%%%%%%%%%%%%%%%%%%%%%%%%%%%%%%%%%%%%%%%%%%%%%%%%%%%%%
\subsection{Besoins en modélisation}\label{sec:bm}
La mise en place d'une telle application nécessite de représenter le vocabulaire de la réalisation audiovisuelle dans toutes ses variations possibles et de le documenter suffisamment afin de le rendre compréhensible pour des novices. 
Cet objectif nous amène à considérer la construction d'une ressource termino-ontologique. L'ontologie permet de représenter les concepts partagés par les professionels de la réalisation audiovisuelle et la terminologie permet de capturer les différentes formes d'expression associées à ces concepts. 

La spécificité de notre problématique est de considérer la collaboration de communautés hétérogènes par leur degré de compréhension des concepts ou leur utilisation de la terminologie. 
Ceci nous amène à envisager la terminologie comme un moyen d'associer à des éléments ontologiques (concept, relation, instances) une chaîne lexicale ou des ressources média. 
Chaque chaîne ou ressource s'adresse en particulier à une communauté dont les membres partagent une capacité d'interprétation commune. 
Il n'existe donc plus une terminologie de référence par langue, mais des terminologies pour chaque communauté d'utilisateurs. 
On remarquera que notre acception de la terminologie sert bien à normaliser les pratiques linguistiques entre les membres d'une même organisation. 
En plus de cela, elle permet de fixer la manière de s'adresser à d'autres communautés.

Par ailleurs, les types de réalisations sont divers et nécessitent des concepts spécifiques pour être décrits. 
Une fiction se structure en séquences et en scènes alors que les documentaires ou magazines d'information se composent de sujets. 
La variabilité des types de contenu à filmer implique donc de pouvoir étendre le fond conceptuel initial pour représenter de nouveaux usages. 
De la même manière, la collaboration avec de nouveaux partenaires nécessite de pouvoir ajouter de nouvelles terminologies au fond conceptuel existant. 
Ontologie et terminologie doivent se gérer de manière indépendante. A partir de ces besoins, nous définissons maintenant les exigences en terme de modélisation. 

Nos besoins en modélisation peuvent être exprimés par les assertions suivantes:
\begin{enumerate}
	\item[(\e{A1}]\e{: multi-jargon}) la variabilité des pratiques linguistiques des organisations et des communautés implique d'associer plusieurs termes à un même concept. 
	Il n'y a pas de choix des termes préférés par une communauté mais une \e{correspondance} entre les termes d'une ou plusieurs communautés, quels que soient la langue, le jargon et le code d'écriture utilisé.
	
	\item[(\e{A2}]\e{: documentation}) la variabilité de compréhension des communautés implique d'associer des explications (chaîne lexicale) ou des illustrations (ressource média) aux concepts afin d'en enrichir la documentation. 
	
	\item[(\e{A3}]\e{: gestion, évolution}) la variabilité des cas de collaboration implique de pouvoir étendre la conceptualisation initiale ou la terminologie pour s'adapter à de nouvelles pratiques ou de nouvelles communautés. 
	Cela implique une gestion et une évolution indépendante de l'ontologie et de la terminologie. 
\end{enumerate}


Dans le cas d'une demande de cadrage en plan américain, la demande est d'abord exprimée dans le jargon de la RTBF puis traduite dans le jargon de la VRT (plan américain pour la RTBF, plan 3/4 pour la VRT) [\g{A1 : multi-jargon}]. 
Ensuite, pour les amateurs, la terminologie est enrichie par des illustrations [\g{A2 : documentation}]. Enfin, un nouveau concept de cadrage est ajouté (plan américain large ou plan moyen serré) [\g{A3 : gestion, évolution}] en vue d'une nouvelle coopération avec la VRT. En plus de cela, le problème de la langue (français et flamand) s'ajoute à la question des jargons métiers. 



\section{Les Systèmes d'Organisation de Connaissances (m)}\label{chap:defs}
%Distinguer entre terme et concept ; dictionnaire, thésaurus, ontologies etc.

\e{
L'objectif de cette section est de clarifier ce qui appartient au domaine de la  linguistique et ce qui relève du domaine conceptuel, en vue d'identifier les notions qui nous serviront à spécifier une solution au cahier des charges dressés dans la section précédente. 
La confusion qui nous intéresse concerne principalement la définition des ontologies par rapport à d'autres SOC tels ques les thésaurus, notamment du fait qu'on utilise parfois les mêmes langages pour les représenter.}

La définition des SOC proposée par \cite{Zacklad2010}, étend celle de \cite{Hodge2000} à \ciel{l'ensemble des formes d'écritures codifiées participant à la description documentaire primaire ou secondaire d'une situation}. 
L'ensemble défini par \citeauthor{Hodge2000} comprend ainsi tout type de :
\begin{liste}
	\item \e{liste de termes} (fichiers d'autorités, glossaires, dictionnaires, répertoires géographiques)
	\item \e{schème de classification/catégorisation} (vedettes-matières, taxonomie)
	\item \e{schème qui se structure par le types de relations qui unit ses membres} (thésaurus, réseaux sémantiques, ontologies). 
\end{liste}

À cela \citeauthor{Zacklad2010} souhaite ajouter des modes de description du contenu émergents ou plus faiblement codifiés comme par exemple les folksonomies. 
Les SOC qui nous intéressent en particulier sont les schèmes structurés par types de relations. 
% pourquoi ? 


\subsection{Thésaurus, terminologie, ontologie}\label{sec:tto}
Dans cette section, nous nous reposerons majoritairement sur les définitions de \cite{bachimont:icc}. Concernant le thésaurus, l'auteur écrit :

\g{Thésaurus :} 
\ciel{
Une organisation de libellés linguistiques selon des relations d'hyperonymie et d'hyponimie. 
Les libellés sont également reliés par des relations dites d'association, qui sont de nature quelconque. 
Même si en pratique les libellés d'un thésaurus correspondent souvent à des termes du domaine, ce n'est pas nécessairement systématique.}

Cette définition situe clairement les thésaurus comme faisant partie du cadre de la linguistique. 
Il s'agit d'un ensemble de mots structurés et reliés suivant leur \e{signification}, c'est à dire leur sens normé ou commun à plusieurs contextes d'usage particuliers (à l'inverse du sens, qui lui varie suivant les usages, \cite{Roche2005}). 
\citeauthor{bachimont:icc} finit sa définition en comparant les mots issus d'un thésaurus aux termes. La distinction se joue à deux niveaux, la stabilité d'écriture du terme (niveau linguistique) et le fait qu'un terme renvoit à un concept (niveau conceptuel) : 

\g{Terme :} 
\ciel{
Une unité linguistique dont le signifié est un concept, c'est-à-dire un signifié normé. 
Le terme se manifeste linguistiquement par une stabilité et régularité de sa forme signifiante.
En particulier, un terme possède des contextes d'occurrence réguliers, obéissant à des canevas morpho-syntaxiques typiques. 
La détection de ces canevas est à la base des outils de détection des termes en corpus. 
Un terme peut posséder des variantes terminologiques.
Dans une optique normative, on détermine une forme préférée.}

Ainsi, plus qu'un repérage des mots (signifiant) utilisés dans un domaine donné, la terminologie s'attache à identifier les signifiés correspondants. 
Au-delà des débats sur les méthodes utilisées pour constituer les couples signifiant-signifié\footnote{L'approche \e{sémasiologique} (initiée par \cite{Bourigault1994}) s'appuie sur l'analyse linguistique d'un corpus de textes pour repérer les couplages signifiant-signifié ainsi que l'organisation conceptuelle sous-jacente. L'identification de ces \e{désignations} est ensuite validée par des experts du domaine. 
Dans une optique différente, l'approche \e{onomasiologique} prend comme appui la modélisation conceptuelle pour nommer ensuite les concepts. On parle alors de \e{dénominations} dont l'objectif est de refléter sans ambiguïté la structure conceptuelle dont elles sont issues.
Une critique faite à la première approche par \cite{Roche2006} est que les relations identifiées entre désignations sont purement linguistiques (hyper/hyponymie, méronymie etc.) et ne se rattachent pas à une structure conceptuelle. Ainsi, l'analyse de texte n'est qu'une première étape dans la constitution d'une terminologie, elle permet d'identifier les usages des mots, mais pas de les raccorder à des concepts. De même, la constitution du corpus va également grandement influer sur les résultats de l'analyse et pose alors un problème de réutilisabilité.}, on cherche à repérer la modélisation conceptuelle sous-jacente d'un domaine, de manière à pouvoir adosser chaque terme à un concept.
À noter que l'inverse n'est pas forcément valable, car il existe des concepts qui n'ont pas d'appelation usuelle et que l'on doit alors désigner par une phrase. 
Une autre conséquence de cette définition est que plusieurs mots peuvent être adossés au même concept.
Il devient alors important de pouvoir expliciter cette équivalence et éventuellement de spécifier un signifié préféré pour le terme.

Cette définition du terme préfigure ainsi la relation qu'entretient la terminologie avec l'ontologie pour \cite{bachimont:icc} : 

\g{Terminologie :} 
\ciel{un recensement et une organisation d'unités linguistiques à l'usage stabilisé et attesté, dont le signifié correspond à un concept du domaine.
La terminologie est l'organisation des termes du domaine.
La terminologie est la face linguistique de l'ontologie, qui en est le côté conceptuel. 
Il n'y a pas une stricte correspondance cependant entre ontologie et terminologie : si tout terme doit correspondre à un concept de l'ontologie, tout concept n'a pas forcément d'usage linguistique régulier attesté.}

De son côté, \cite[\S 2.4]{Roche2005} parle de manière similaire \ciel{[d']un système de termes reflétant une modélisation conceptuelle, [...] plus généralement dénommé \e{système notionnel} [qui] trouve sa raison d'être dans la façon dont nous appréhendons les objets du monde.}
\citeauthor{Roche2005} précise que si les systèmes notionnels ne relèvent pas de la linguistique, ils ne dépassent pas forcément le cadre d'une langue, sauf \ciel{pour des communautés de pratique dont les langues d'usage partagent la même conceptualisation du monde.}.
La distinction est ainsi faite entre les mots d'usage (qui peuvent être polysémiques) et les termes dont on spécifié une forme préférée (signifiant) et qu'on adosse à une signification (signifié normé).  

Concernant les particularismes qui peuvent exister dans chaque communauté, \citeauthor{Roche2005} propose de s'éloigner d'une vision purement normalisatrice. 
Ainsi, sur le plan linguistique, il est possible de rattacher les différents mots d'usages et de préciser leur contextes d'utilisation.
Sur le plan conceptuel, l'auteur propose de constituer des \gui{terminologies régionales} que l'on cherchera ensuite à mettre en correspondance. 




\paragraph{Ontologie}
Concernant les ontologies, nous nous limiterons aux définitions proposés dans le cadre de l'ingénierie des connaissances (IC). Les travaux de \cite{Charlet2002} nous rappelle qu'il existe de multiples définitions : 

Pour \cite{Gruber1993} : \ciel{Une ontologie est une spécification explicite d'une conceptualisation.}

 La définition proposée par \cite{Uschold1996} nous permet de préciser de quoi se compose une conceptualisation et en quoi une ontologie la spécifie : 

\ciel{Une ontologie implique ou comprend une certaine vue du monde par rapport à un domaine donné. Cette vue est souvent conçue comme
un ensemble de concepts -- e.g. entités, attributs, processus --, leurs définitions
et leurs interrelations. On appelle cela une conceptualisation. [...]
Une ontologie peut prendre différentes formes mais elle inclura nécessairement
un vocabulaire de termes et une spécification de leur signification. [...]
Une ontologie est une spécification rendant partiellement compte d’une conceptualisation.}

\citeauthor{Charlet2002} en conclut qu'une ontologie est une conceptualisation, c'est-à-dire un ensemble de concepts et de relations dont on cherche à normer la signification. 
Pour faire de la conceptualisation un objet informatique, il faut spécifier une théorie logique dotée d'un vocabulaire (les concepts et les relations), à la manière des travaux de \cite{Guarino1995}.

% Roche puis Babache
Pour \citeauthor{Roche2005}, une ontologie est équivalente au système notionnel des terminologies, d'où la relation forte établie par les chercheurs en IC : 

\ciel{définie pour un objectif donné et un domaine particulier, une ontologie est pour l'ingénierie des connaissances une représentation d'une modélisation d'un domaine partagée par une communauté d'acteurs. Objet informatique défini à l'aide d'un formalisme de représentation, elle se compose principalement d'un ensemble de concepts définis en compréhension, de relations et de propriétés logiques.} (\cite{Roche2005})

\citeauthor{Bachimont2000a} insiste sur le fait qu'on utilise une sémantique donnée (différentielle, référentielle, psychologique, distributionnelle, conceptuelle etc., \cite{bachimont:hdr}) pour établir la signification des concepts de l'ontologie. Chaque sémantique propose un point de vue particulier qui permet de faire correspondre une signification propre à chaque unité d'expression : 

\ciel{une ontologie est la signature fonctionnelle et relationnelle, munie de sa sémantique, d'un langage formel de représentation et manipulation des connaissances.} (\cite{Bachimont2000a})

Les ontologies se construisent ainsi en s'adossant à des théories, et ce sont ces théories qui fixent des principes pour déterminer la signification des unités linguistiques qu'elles emploient et chargent d'un sens bien précis (sémantique). 

\paragraph{Sémantiques et ontologies}
Pour bien cerner les conséquences de cette définition, voici quelques sémantiques décrites par \cite{bachimont:hdr} qui se distinguent dans leur manière d'expliciter la signification d'une unité d'expression : 
\begin{liste}
	\item \e{sémantique différentielle} : la signification d'une unité consiste en l'identité et la différence par rapport aux autres unités linguistiques de la langue. On reste donc dans le cadre de la linguistique. La différenciation des  unités peut se faire par différentes méthodes ; par observation empirique d'un corpus de texte (\e{sémantique distributionnelle}) ; ou bien suivant une modélisation de la signification des concepts du domaine établie par des experts par exemple (\e{sémantique conceptuelle}). 

	\item \e{sémantique référentielle} : la signification d'une unité est l'objet auquel elle fait référence, dans un univers extralinguistique. Ici, on s'attache à une théorie propre à cet univers et qui explicite la définition des objets.
	
	\item \e{sémantique psychologique} : la signification d'une unité est la représentation mentale que l'on s'en fait. Là encore, il s'agit de suivre une théorie, mais dans le champ de la psychologie. 
\end{liste}
Cette liste de sémantiques permet de comprendre la grande variabilité des ontologies qu'il est possible de construire. 


\paragraph{Classification d'ontologie}
Il peut également être utile de définir des propriétés pour distinguer des "genres" d'ontologies, non pas en fonction de la sémantique utilisé, mais en fonction de l'usage que l'on souhaite en faire. 
\cite{Oberle2006} propose une classification qui repose sur trois propriétés : 
\begin{liste}
	\item l'\g{objectif} de l'ontologie (purpose) où l'on distingue entre deux objectifs, servir de référence ou bien être utilisé dans un cas d'application :
	\begin{liste}
		\item l'\e{ontologie de référence} vise à établir un consensus entre des agents (humains, machines) d'une même communauté, ou bien à servir d'explication et de langages communs avec des agents de communautés différentes. 
		\item l'\e{ontologie d'application} qui se limite à un cas d'application et suit des contraintes et simplifications propres.
	\end{liste}
	La différence réside dans l'arbitrage entre l'expressivité de la représentation et sa décidabilité (\cite{Borgo2002}). 
	Typiquement, une référence n'est consulté qu'occasionnellement et se doit d'être la plus exhaustive possible alors qu'une ontologie d'application doit servir à faire des raisonnements à l'exécution. 

	\item l'\g{expressivité} de l'ontologie (expressiveness) où l'on considère un engagement plus ou moins fort sur le formalisme de représentation :
	\begin{liste}
		\item l'\e{ontologie légère} (lightweight) qui peut se limiter à une hiérarchie de concepts bien connus dans une communauté avec quelques relations. 
		L'apport se situe alors dans la structuration des connaissances qui clarifie leur signification plus qu'il ne les établit.
		\item l'\e{ontologie lourde} (heavyweight) qui vise à exclure toute ambiguïté terminologique et conceptuelle.
		Pour cela, la formalisation se veut beaucoup plus contrainte et détaillée afin de forcer une interprétation. 
	\end{liste}

	\item la \g{spécificité} de l'ontologie qui peut se limiter à une domaine ou s'étendre à un ensemble de domaines voire plusieurs champs disciplinaires : 
	\begin{liste}
		\item l'\e{ontologie générique} (generic, upper/top level) contient des concepts utilisés dans de nombreux champs disciplinaires (évènements, processus etc.)
		\item l'\e{ontologie noyau} (core) comporte des concepts qui se situent à la croisée de plusieurs domaines. 
		La distinction avec une ontologie générique se fait car elle comporte des concepts utilisable quelque soit le domaine.
		De même, on distingue les ontologies noyaux des ontologies de domaine car les premières comportent des éléments réutilisables dans des plusieurs domaines proches. 
		\item l'\e{ontologie de domaine} contient des concepts propre à un domaine, et bien souvent des éléments plus génériques extraits de domaine différents.
		Cependant, l'ontologie de domaine présente généralement des éléments plus spécifiques, propres au domaine concerné. 
	\end{liste}
\end{liste}
\subsection{Méthodes de construction d'ontologie}\label{sec:construction}
Nous avons défini dans la section précédente (\ref{sec:tto}) ce qu'est une ontologie, présenté des exemples de sémantiques et montré comment il était possible de classifier ces ontologies suivant leur usage. 
Nous abordons maintenant les méthodes de construction d'ontologies.


\paragraph{Méthode d'\cite{Uschold1996}}
\citeauthor{Uschold1996} ont défini une méthode de construction à partir de leur expérience de développement d'ontologies en entreprises. 
Elle se décompose en quatre étapes :
\begin{listenum}
	\item Une phase de \g{conception} qui vise à identifier le domaine concerné, le but et la portée de l'ontologie.
	\item Une phase de \g{construction} qui se décompose en trois étapes ; définir les concepts clés et les relations entre ces concepts ; expliciter la représentation de la conceptualisation dans un langage formel ; intégrer des connaissances d'autres ontologies.
	\item Une phase d'\g{évaluation} de l'ontologie construite.
	\item Une phase de \g{documentation} qui doit expliciter les décisions effectuées aux étapes précédentes afin de faciliter la réutilisation de l'ontologie.
\end{listenum}


\paragraph{Methontology}
La méthode proposée par le Laboratoire d'Intelligence Artificielle (LAI) de l'université Polytechnique de Madrid a pour particularité d'intégrer le développement de l'ontologie à une méthodologie de gestion de projet \parcite{Fernandez1997, Blazquez1998}. La méthodologie distingue trois types d'activités se déroulant en parallèle, et dont les deux premières servent à soutenir la construction de l'ontologie :
\begin{liste}
	\item Les activités de \g{gestion de projet}, notamment la planification avant-projet puis le \e{contrôle de la qualité} des résultats produits. 
	
	\item Les activités de \g{support} qui concernent l'\e{acquisition} des connaissances du domaine ; l'\e{intégration} de connaissances d'autres ontologies ; la \e{documentation} de l'ontologie et de sa production ; la \e{gestion de version} des résultats produits ; l'\e{évaluation} technique de la construction de l'ontologie ainsi que de sa documentation. 

	\item Les activités de \g{développement technique} qui permettent de construire l'ontologie par étape. 
	Dans un premier temps, la \e{spécification} définit l'objectif de l'ontologie, les applications et les utilisateurs concernés ; puis la \e{conceptualisation} structure les connaissances du domaine, qui sont ensuite formalisées (étape de \e{formalisation}) et enfin représentées dans un langage informatique (étape d'\e{implémentation}). La séquence se poursuit par une étape de \e{maintenance}.
\end{liste}


% Ontospec (Kassel)
% Guarino et Welty
% voir livre BB pour la justification du manque de sémantique

\paragraph{Archonte}
La méthode \pc{Archonte} (\pc{arch}itecture for \pc{ont}ological \pc{e}laborating), proposée par \cite{Bachimont2000a}, met en avant l'importance cruciale donnée au choix d'une sémantique dans la construction d'une ontologie.
La méthode repose sur trois étapes successives qui aboutissent à une ontologie computationnelle, exprimée dans un langage opérationnel de représentation des connaissances.
% , et à partir duquel on peut effectuer des inférences. 
% Précisons maintenant les étapes de cette méthode ainsi que les résultats obtenus :
% \begin{liste}
Le point de départ de la méthodologie est constitué d'expressions linguistiques (signifiant) issues du domaine considéré.
L'intérêt de disposer d'un tel ensemble de traces linguistiques est qu'elles servent à exprimer des concepts (signifiés) ou des connaissances sur le monde. 
Ainsi, on se retrouve avec un corpus de candidats-termes dont la signification peut être source d'ambiguïtés et dont on cherche à clarifier l'interprétation.\\

\g{[1.]} La première étape de cette méthode (aussi appelée \e{normalisation sémantique}) consiste à établir un \g{engagement sémantique}  qui précise la manière de mener l'interprétation des candidats-termes et de construire une première structure de connaissances. 
Pour cela, on fixe d'abord un contexte de référence, la tâche ou le problème qui a poussé à l'élaboration de l'ontologie, qui permet de cadrer l'interprétation des candidats-termes.

Ensuite, pour préciser l'interprétation on s'appuie sur la sémantique différentielle afin d'expliciter les différences et les similarités entre une notion et son voisinage direct (notion parente, notions soeurs) : 

	\ciel{
	La méthodologie que nous proposons ici repose sur l'organisation générale des unités en un réseau d'identités et de différences.
	Ce sont les propriétés structurelles de ce réseau qui permettent de contraindre l'interprétation des unités définies dans le réseau : la position d'une unité dans le réseau prescrit comment la comprendre et lui prescrit une signification qui pourra dès lors lui être associée, quel que soit le contexte où elle se rencontre.} (\cite[p.139]{bachimont:icc})

Cette caractérisation des notions par leur voisinage repose sur quatre relations à expliciter : 
\begin{liste}
	\item la \gui{communauté avec le parent} (\ciel{similarity with parent}) : pourquoi la notion hérite des proprités de son parent.
	\item la \gui{différence avec le parent} (\ciel{difference with parent}) : en quoi la notion est différente de son parent.
	\item la \gui{différence avec les soeurs} (\ciel{difference with siblings}) : en quoi un notion est différente de ses notions soeurs.
	\item la \gui{communauté avec les soeurs} (\ciel{similarity with siblings}) : quelle est la propriété que partagent les notion soeurs -- dont on distingue plusieurs valeurs exclusives, une par soeur.\\ 		
\end{liste}
% \end{liste}

Cette première étape aboutit à la construction d'un \g{arbre ontologique différentiel} qui structure un ensemble de notions de manière hiérarchique et non ambiguië par rapport à un contexte de référence.
Les candidats-termes sont structurés par des prescriptions interprétatives et deviennent ainsi des primitives de modélisation.\\

% \begin{liste}
\g{[2.]} L'\g{engagement ontologique} consiste à munir l'ontologie différentielle d'une sémantique formelle extensionnelle.
Rappelons que cette sémantique définit les concepts par leur extension, c'est-à-dire tous les individus qu'ils désignent parmi un ensemble de référence. 
Il s'agit donc de relier des primitives dotées d'une signification linguistique normalisée à des concepts désignant un ensemble de référents (ou individus).
Pour cela, il faut adjoindre à l'ontologie différentielle un modèle référentiel : 

	\ciel{
	l'ontologie référentielle obéit aux contraintes sémantiques de l'ontologie différentielle : [s]a structure arborescente se retrouve dans l'ontologie référentielle et lui donne son squelette.
	Chaque relation de spécialisation sémantique au niveau différentiel se traduit par une spécialisation d'extension au niveau référentiel.} (\cite[p.148]{bachimont:icc})

Ce changement de sémantique permet d'enrichir l'ontologie de nouveaux concepts et d'en modifier la structuration. 
En effet, on peut désormais avoir recours à des opérations ensemblistes (réunion, intersection, complémentaire) qui composent le sens des concepts et permettent ainsi de définir de nouveaux concepts.
L'ajout de ces \gui{concepts définis} modifie également la structure de l'ontologie, qui passe d'une arborescence à une structure en treillis, c'est-à-dire admettant l'héritage multiple. 
Par exemple, une primitive différentielle de \cd{mandat politique} spécialisée en concepts de \cd{député} et de \cd{maire} ne permet pas de représenter de double mandat.
Par contre, une définition extensionnelle permet de définir le concept de \cd{député-maire} simplement par l'intersection des extensions de ces concepts\footnote{Pour plus de détails sur cet exemple, se reporter à l'exemple donné par \cite[p.149]{bachimont:icc}}.
% \end{liste}

À l'issue de cette étape on obtient donc une \g{ontologie référentielle}, c'est-à-dire un treillis de concepts définis par une sémantique référentielle.\\

% \begin{liste}
\g{[3.]} L'\g{engagement computationnel} vise à doter les concepts de l'ontologie référentielle d'une signification en termes d'opérations informatiques.
Pour cela, il faut d'abord choisir un langage opérationnel de représentation des connaissances qui détermine l'expressivité et les opérations de calculs à disposition pour élaborer une version informatique de l'ontologie.
Nous présentons quelques uns de ces langages dans la section \ref{sec:onto-mc}.
La transposition dans un langage a des conséquences au niveau de l'expressivité et de la décidabilité du modèle.
% \end{liste}

Nous obtenons une \g{ontologie computationelle} qui est une version de l'ontologie référentielle exploitable informatiquement.





\subsection{La validation en Ingénierie des Connaissances}\label{sec:valid-ic}
En suivant l'analyse proposée par \cite{Bachimont2004}, il en découle que  l'Ingénierie des Connaissances (IC) \ciel{exprime les connaissances d'un domaine dans un langage de modélisation et l'opérationnalise en un système}. 
En d'autres termes, la modélisation de l'IC porte sur les concepts utilisés par les membres de ce domaine pour penser et établir des connaissances sur le monde, mais pas directement sur le monde.
Ainsi, les modèles de l'IC n'ont pas pour vocation à \ciel{prédire quoi que ce soit sur le monde ni sur la connaissance}, mais plutôt d'\ciel{instrumenter le travail intellectuel, l'exercice de la pensée, le travail de la connaissance}. 
Dans cette perspective, \citeauthor{Bachimont2004} nous propose de théoriser l'IC comme \ciel{une ingénierie des inscriptions numériques des connaissances qui vise à instrumenter le travail cognitif associé à ces inscriptions}. 
        
Les inscriptions possèdent une double dimension ; \e{matérielle} (et donc manipulable par des techniques de calcul logique) ; \e{sémiotique} (et donc interprétable selon des conventions propres à une situation d'usage).  
En d'autres termes, le système d'IC permet d'agir de manière prédictible sur les inscriptions de connaissances, ces actions produisant de nouvelles inscriptions qui donnent matière à penser à l'utilisateur. 
        
Il y a donc plusieurs éléments à valider en IC, les calculs qui seront faits sur les inscriptions (on teste le comportement du système informatique) puis l'interprétation de ces inscriptions (on évalue le gain apporté par le système et les inscriptions qu'il fournit à l'utilisateur selon une situation d'usage). 
La modélisation prise en charge par l'IC ne porte donc ni sur le monde, ni sur l'activité cognitive et ne peut être validée uniquement par le formalisme de ses inscriptions. 
        
Les inscriptions de connaissances doivent être considérées sous deux angles : d'un point de vue \e{nomographique} (on formalise la manipulation symbolique des inscriptions pour prévoir/définir le comportement du système) et \e{idiographique} (on décrit le sens des manipulations symboliques et des inscriptions produites par rapport aux normes, conventions, concepts du domaine).



% Au final, on construit un objet informatique en suivant une méthode de construction qui nous guide pour spécifier le comportement de l'ontologie. 

% RTO





\subsection*{Discussion sur la méthodologie suivie}
\addcontentsline{toc}{subsection}{Discussion}
% pourquoi on choisit celle de BB, et pourquoi on ne l'utilise pas globalement, mais seulement localement sur des patrons d'utilisation précis.



Notre étude des méthodes de construction d'ontologie ne vise pas à être exhaustive, pour cela nous renvoyons à \cite{Gomez-Perez2004}.
Il faut cependant remarquer que nous avons écarté les approches d'acquisition des connaissances à partir d'un corpus de texte, comme la méthode \g{Terminae} développée par \cite{Aussenac-Gilles2003}.
Dans cette approche, l'analyse linguistique d'un corpus permet de repérer des candidats-termes (\pc{Syntex}), d'effectuer des regroupements de contexte (\pc{Upery}) et d'identifier des relations (\pc{Yakwa}) afin d'accompagner la modélisation conceptuelle.
Or, dans notre contexte de travail, les documents professionnels qui décrivent en détails la production audiovisuelle sont rares (peu d'organisation ont les ressources de les produire) et constituent des ressources de valeur auquelles il est difficile d'avoir accès. 
De plus, la notion de document audiovisuel est largement absente des ouvrages généralistes, ce qui fonde précisément l'intérêt de notre travail de recherche.
\citeauthor{Uschold1996} proposent un point de vue global sur le processus de construction d'ontologie, qui reste un point de référence dans le domaine.
L'approche de \pc{Methontology} clarifie cependant le déroulement de la construction et s'efforce de l'intégrer dans une vision de gestion de projet assez exhaustive. 
Si cette vision est intéressante, il n'est pas forcément possible de l'appliquer telle quelle en pratique, du fait de contraintes spécifiques d'un projet.
Par ailleurs, \pc{Archonte} détaille une méthode de formalisation logique qui permet de positionner la conceptualisation sur le plan sémantique. 
De ce fait, elle précise la manière de formaliser une conceptualisation et peut s'intégrer de manière complémentaire aux autres méthodes exposées.

Dans notre cas, nous avons suivi une méthodologie \e{ad-hoc}, nécessaire pour suivre les contraintes d'un projet de recherche et développement impliquant de multiples partenaires.
Les aspects de gestion de projet (tels que décrits dans \pc{Methontology}) étaient donc largement assujetis à l'avancement du projet MediaMap.
De même, la spécification et l'acquisition des connaissances et la conceptualisation reposent principalement sur un dialogue avec des experts du domaine (dans le cas du projet MediaMap ce fût principalement les membres de la RTBF et de la VRT) dans le cadre de réunions générales et de séminaires plus focalisés.
Les autres aspects de la construction de l'ontologie ont été réalisés par les membres de l'équipe de recherche ICI, en s'inspirant des séquences de développement technique proposées par \pc{Methontology}.

Sur le plan sémantique, l'étendue de notre modélisation nous pousse à introduire des concepts de haut-niveau afin d'articuler diverses connaissances se rapportant à la production audiovisuelle (l'organisation du processus, ses contributeurs, ses résultats, et leur description par un vocabulaire professionnel et compréhensible par des amateurs).
Ces ontologies génériques de haut-niveau proposent des fondements théoriques important et des patrons de conception détaillés (au sens de \cite{Isaac2005}) pour modéliser certaines situations. 
On pense par exemple, au patron \ciel{Description \& Situations} de l'ontologie DOLCE \parcite{Gangemi2005}. 
Pour autant, la modélisation ne doit pas perdre de vue notre perspective applicative, nécessaire à l'adoption et la compréhension du modèle par des utilisateurs du domaine.
Ainsi, \citeauthor{Isaac2005} proposent d'adapter la structure de ces patrons aux besoins descriptifs de l'application.
Cela consiste à simplifier la structure des patrons en créant des \e{raccourcis relationnels}, quitte à faire disparaître les subtilités de représentation.
L'enjeu d'une telle approche est de faire co-exister la forme simplifiée (adaptée aux usages du domaine et aux besoins expressifs de l'application) et la forme de haut-niveau qui la rend réutilisable dans d'autres domaines.
De notre point de vue, notre modélisation se construit à partir de tels patrons de conception, qu'il s'agira ensuite d'articuler par des relations. 
Nous avons appliqué la méthodologie \pc{Archonte} pour formaliser la sémantique de ces patrons, qui sont ensuite regroupés pour former une seule ontologie.
En effet, nous considérons que le point important est de justifier la modélisation de ces parties et de leur mise en relation, plutôt que de la structure de l'ensemble.






% est-ce qu'on a adapté des patrons de conception générique aux notions du domaine, pour simplifier la modélisation (modélisation réduite) ? VOIR \cite{Isaac2005}









\section{Langages de représentations}\label{s:mods}

% \subsection{Langages de balisages }
\subsection{eXtended Markup Language (?)}
The eXtended Markup Language (XML) aims to give a hierarchical structure to  text in a machine-, yet human-readable way. It is widely used to store or exchange information as it also supports Unicode.
XML is formally defined as a Standard Generalized Markup Language's subset (SGML) designed to improve parser efficiency. Work on XML began in 1996 and it became a W3C Recommendation in early 1998.

\paragraph{Mark-up}
This is achieved by adding mark-up elements that are easily noticed as they begin with '<' and end with a '>'. Mark-up elements are used to enclose unicode text, and give thus a mean to identify them and possibly to process it. As its name indicates, it is said extensible because we can define our own mark-up elements and writes a line like that:

\begin{Verbatim}[fontsize=\small,formatcom=\color{black!70}]
opening mark-up element 			enclosing mark-up element
<structural_element>Some unicode text inside</structural_element>
\end{Verbatim}

Attributes can be defined for each mark-up element. 
For instance, the xml:lang attributes indicates the natural language used to write the enclosed text. The « 1812 Overture » full title can be written like that:
\begin{Verbatim}[fontsize=\small,formatcom=\color{black!70}]
<title xml:lang='ru'>Торжественная увертюра 1812 года, Toržestvennaja uvertjura 1812 goda</title>
<title xml:lang='fr'>Ouverture Solennelle, L'Année 1812, Op. 49</title>
\end{Verbatim}

\paragraph{Syntax}
XML does not only enclose text with mark-up elements. It also enables to imbricate mark-up elements in such a way that the elements conforms to a tree structure. 

\begin{Verbatim}[fontsize=\small,formatcom=\color{black!70}]
<element>
	<sub-element>Example of text</sub-element>
</element>
\end{Verbatim}

Other syntaxic rules have been defined to enables conforming parser to process XML file. Any file conspuing to these rules is said to be well-formed.

\paragraph{Schema}
Furthermore, if XML provides us with a syntax we also have the ability to makes purpose-specific XML-based mark-up languages – i.e. define constraints on structure, mark-up elements or even datatyping definition. Indeed, several schema languages exists and are used to encode documents or serialize text data according to a particular schema. XML files complying with a schema – i.e. conforming to the constraints defined in the schema – are said to be valid.

\paragraph{Namespace}
When creating schemas, ambiguity problems usually arise and namespace declaration can take care of that. Indeed, it provides an abstract container for XML elements and attributes and gives to their name a scope. As each namespace is identified by an URI, the ambiguity between identically named elements or attributes from differents namespace can be resolved. 

Therefore, I can declare my own « title » element and simultaniously use the « DC Terms » property title. We show here a complete example with xml heading:
\begin{Verbatim}[fontsize=\small,formatcom=\color{black!70}]
<?xml version="1.0" encoding="UTF-8"?>
<ex:musical_opus xmlns:dc="http://purl.org/dc/elements/1.1/"
    			 xmlns:ex="http://example.org">
	<dc:title>1812 Overture</dc:title>
    <ex:title>Festival Overture, The Year 1812</ex:title>
</ex:musical_opus>
\end{Verbatim}

In this example, dc:title and ex:title denotes two different elements for which we can give informally different meaning – ex:title indicating the complete name, dc:title a short version. 

\subsubsection*{XML Schema}
This W3C recommendation was published in 2001 and is one of several xml schema languages\footnote{We can cite, the old and very simple \gui{Document Type Defintion}(DTD) as well as the major rival of XML Schema, namely \gui{Relax NG}.}. It is often called XSD in reference to its files suffix – '.xsd'.

XSD can define imbrication, quantification and naming rules for xml elements and attributes – in order to enable vocabulary and content model validation. XSD also supports namespace so parts of other schemas can be included or imported. 

\paragraph{Datatypes}
But one of the main and most criticized characteristic of XSD is DataType validation. It can be applied to elements or attributes to constraint their content.  DataType definition must use XSD primitive or derived datatypes – see the scheme for a detailled hierarchy. This dependence upon specific datatypes is the source of many criticism. 

Derived datatypes can be built by restriction – of the permitted values set –, list – declaration of values –, or union – between several types. 
As an complete example, we define a XSD schema describing « MusicalOpus » as a list of XML elements named:
\begin{liste}
	\item Title: title of the musical opus
	\item Extent: length or duration – as a string
	\item Composer: name of the composer
	\item composed: date of composition of the opus – only the year
	\item Performer: name of the performer
	\item performed: date of performance – only the year
	\item conductedby: name of the person who conducted the performer, this attribute is declared optionnal – thanks to the minOccurs attribute
	\item ComposerNationality: picked from a list of value we enumerate in the scheme
\end{liste}

Here is the resulting XSD scheme:
\begin{Verbatim}[fontsize=\small,formatcom=\color{black!70}]
<?xml version="1.0" encoding="utf-8"?> 
<xs:schema elementFormDefault="qualified"   xmlns:xs="http://www.w3.org/2001/XMLSchema"> 
 <xs:element name="MusicalOpus"> 
   <xs:complexType> 
     <xs:sequence> 
       <xs:element name="Title" type="xs:string" /> 
       <xs:element name="Extent" type="xs:string" /> 
       <xs:element name="Composer" type="xs:string" /> 
       <xs:element name="composed" type="xs:gYear" /> 
       <xs:element name="Performer" type="xs:string" /> 
       <xs:element name="performed" type="xs:gYear"/> 
       <xs:element name="conductedBy" type="xs:string" minOccurs="0"/>  
       <xs:element name="ComposerNationality"> 
         <xs:simpleType> 
           <xs:restriction base="xs:string"> 
             <xs:enumeration value="FR" /> 
             <xs:enumeration value="DE" /> 
             <xs:enumeration value="RU" /> 
             <xs:enumeration value="UK" /> 
             <xs:enumeration value="US" /> 
           </xs:restriction> 
         </xs:simpleType> 
       </xs:element> 
     </xs:sequence> 
   </xs:complexType> 
 </xs:element> 
</xs:schema> 
\end{Verbatim}


And we provide a XML file which states that it conforms to the previous XSD scheme through a xsi:noNamespaceSchemaLocation attribute:
\begin{Verbatim}[fontsize=\small,formatcom=\color{black!70}]
<?xml version="1.0" encoding="utf-8"?> 
<MusicalOpus xmlns:xsi="http://www.w3.org/2001/XMLSchema-instance" 
         xsi:noNamespaceSchemaLocation="MusicalOpus.xsd"> 
  <Title>Festival Overture, The Year 1812</Title> 
  <Extent>14:19</Extent> 
  <Composer>Pyotr Ilyich Tchaikovsky</Composer> 
  <composed>1880</composed> 
  <Performer>Minneapolis Symphony Orchestra</Performer> 
  <performed>1954</performed> 
  <conductedBy>Antal Dorati</conductedBy> 
  <conductedBy>Harold Lawrence</conductedBy> 
  <ComposerNationality>RU</ComposerNationality> 
</MusicalOpus> 
\end{Verbatim}






















\subsection{Ontologie et représentation des connaissances}
\subsubsection*{Resource Description Framework}
\addcontentsline{toc}{subsection}{Resource Description Framework}
The Resource  Description Framework (RDF) is an abstract model which is part of the W3C recommendations for the Semantic Web. 
Let's just bring back to mind how \pc{Tim Berners-Lee} defined it to set RDF back into its context of creation: 

\ciel{
The Semantic Web is not a separate Web but an extension of the current one, in which information is given well-defined meaning, better enabling computers and people to work in cooperation.}

Indeed, RDF aims to describe and link resources – and no more web pages – in a simple, all-purpose and machine-readable way. 
The focus on software agents led to choose a formal semantic and provable inference. 
Thus, such descriptions will foremost benefits to software agents which we'll be able to exploit, process and search into this web of linked data. 

\paragraph{Statements / Triples}
The description consists in making statements that describes or models web resources. The statements are formed as subject-predicate-object sentence called triples that can be represented as a graph – one node/vertex for the subject and the object, and a directed and labeled edge for the predicate. 
\begin{Verbatim}[fontsize=\small,formatcom=\color{black!70}]
Subject		Predicate	    Object	
Tchaikovsky ---- is the Composer of ----> 1812 overture
\end{Verbatim}
Thereby, a set of statements constitute a multigraph, that is a graph in which node/vertex can have multiple ingoing or outgoing edges resulting possibly in loops. 
However, unlike an hypertext the rdf multigraph has labelled edges – also called properties -- connecting a resource with another resource or a literal value.

\paragraph{URI, datatype and literals}
RDF is said to have an URI-based vocabulary, meaning that resources and properties and typed literal are identified by URI reference. 
Indeed, unlike plain literals, typed literals are literals combined with a datatype URI.
Datatypes in RDF are compatible with XML Schema datatypes --which can thus be used as there are-- but any datatype definition conforming to RDF constraints may be used.

Let's rewrite our previous example with Tchaikovsky:
\begin{Verbatim}[fontsize=\small,formatcom=\color{black!70}]
Subject:  http://example.org/Tchaikovsky
Property: http://example.org/Composer
Object:   http://example.org/1812_Overture
\end{Verbatim}

And now an example with a gYear XSD datatype:
\begin{Verbatim}[fontsize=\small,formatcom=\color{black!70}]
Subject:  http://example.org/1812_Overture
Property: http://example.org/composed
Object:   '1880' xsd:gYear
\end{Verbatim}
The following examples make use of the namespace ability to ease the reading. We define here the prefix used for our example and for specific rdf elements:
\begin{liste}
	\item \cd{Example prefix: 'ex:'	Example URI: 'http://example.org'}
	\item \cd{RDF prefix: 'rdf:'	RDF URI:'http://www.w3.org/1999/02/22-rdf-syntax-ns\#'}
\end{liste}
Our previous example could then be written like that:
\cd{ex:1812\_Overture  -- ex:composed -->  '1880' xsd:gYe}


\paragraph{Structured value}
When we want to define a resource composed in fact of several other resources or literals, 
RDF makes us declare an intermediate node – only to conform to the RDF syntax. 
This kind of nodes are called blank nodes because they don't really need to be referenced by an URI. 
As they are only a product of the RDF syntax and don't represent anything in particular they can stay anonymous.

Indeed, when we declare the size of a digital file such as an audio file, we may want to specify the unit. 
Thus considering that the following statement is not sufficient:
\cd{ex:audio\_file\_01  ex:size '33,7'}

So we need to define one blank node and use a particular RDF property called value:
\begin{Verbatim}[fontsize=\small,formatcom=\color{black!70}]
ex:audio_file_01 	-- ex:size -->	_blank_node_01
_blank_node_01	-- rdf:value -->	'33,7'
_blank_node_01	-- ex:unit -->	ex:Mb xsd:decimal
\end{Verbatim}

The value property is not the only one to be used in such context. 
The type property states the nature of a blank node. 
An example will be shown in the next paragraph.

\paragraph{Grouping resources}
% 7.2.3.a  containers
RDF provides two different ways to group things. The first one is called Containers comprises three predefined types which points out some members of the group they define:
\begin{liste}
	\item a bag define a non-ordered group that may include duplicate members. 
	\item a seq define an ordered group that may include duplicate members.
	\item a alt define a group of alternatives resources or literals. 
\end{liste}

Members may be resources or literal, their membership is stated by declaring them as list item (li).
We show here example connected to our information sample. First, a bag of \gui{Performer} having played the \gui{1812 overture}, then a seq describing the content of the audio CD. 
Eventually, an alt group of russian, french and english version of \gui{Tchaikovsky}'s full name.

1) Bag of \gui{Performer}
\begin{Verbatim}[fontsize=\small,formatcom=\color{black!70}]
ex:1812_Overture	-- ex:Performer -->	_blank_node_02
_blank_node_02	-- rdf:type -->		rdf:bag
_blank_node_02	-- rdf:li -->		ex:Minneapolis_S_O
_blank_node_02	-- rdf:li -->		ex:St_Petersburg_Ph
\end{Verbatim}
2) Seq of audio tracks from \gui{Tchaikovsky: 1812 Festival Overture; Capriccio Italien; Beethoven:  Wellington's Victory} CD:
\begin{Verbatim}[fontsize=\small,formatcom=\color{black!70}]
ex:1812_Op49_CD	-- ex:TrackList -->	_blank_node_03
_blank_node_03	-- rdf:type -->		rdf:seq
_blank_node_03	-- rdf:li -->		ex:Op49_audio
_blank_node_03	-- rdf:li -->		ex:Op49_commentary
_blank_node_03	-- rdf:li -->		ex:Capriccio_Italien_audio
_blank_node_03	-- rdf:li -->		ex:Wellington_Op91_audio01
_blank_node_03	-- rdf:li -->		ex:Wellington_Op91_audio02
_blank_node_03	-- rdf:li -->		ex:Op91_commentary
\end{Verbatim}
3) Alt names of \gui{Tchaikovsky}
\begin{Verbatim}[fontsize=\small,formatcom=\color{black!70}]
ex:Tchaikovsky		-- ex:Name -->		_blank_node_04
_blank_node_04	-- rdf:type -->		rdf:alt
_blank_node_04	-- rdf:li -->		'Pyotr Ilyich Tchaikovsky' @en
_blank_node_04	-- rdf:li -->		'Piotr Ilitch Tchaïkovski' @fr
_blank_node_04	-- rdf:li -->		'Пётр Ильич Чайкoвский' @ru
\end{Verbatim}

% 7.2.3.b  collection
Unlike containers, collections can make a closed group definition, that is list all members member of the collection. With a container you can not state that there  is no other member than those you give in your definition. 
Moreover, list have predefined properties to identify the first item, the rest of the list and the end of it – a nil property. 

A rewriting of our tracklist example as a list would be:
\begin{Verbatim}[fontsize=\small,formatcom=\color{black!70}]
ex:1812_Op49_CD	-- ex:TrackList -->	_blank_node_03
_blank_node_03	-- rdf:type -->		rdf:list
_blank_node_03	-- rdf:first -->		ex:Op49_audio
_blank_node_03	-- rdf:rest -->		ex:Op49_commentary
_blank_node_03	-- rdf:rest -->		ex:Capriccio_Italien_audio
_blank_node_03	-- rdf:rest -->		ex:Wellington_Op91_audio01
_blank_node_03	-- rdf:rest -->		ex:Wellington_Op91_audio02
_blank_node_03	-- rdf:rest -->		ex:Op91_commentary
_blank_node_03	-- rdf:rest -->		rdf:nil
\end{Verbatim}

\paragraph{Serialization format}
As an abstract model, RDF statements can be serialized or represented in a variety of form. The most widely known is the \gui{RDF XML} but the W3C also introduced the more readable \gui{Notation} (N3) based on tabular spacing. This last form is closely related to the \gui{Turtle} and \gui{N-Triples} formats.





\subsubsection*{RDF Schema}
\addcontentsline{toc}{subsection}{RDF Schema}
RDFS is the result of 6 years of work from the W3C consortium – from the 1998's first version to the 2004's final recommendation. 
It is formally introduced as a vocabulary description language intended to structure RDF resources. 
Indeed, RDFS presents mechanisms for describing classes of resources, associated properties and also the relationships between properties and other resources. 

These mechanisms are in fact itselves classes and properties that enables us to describe vocabulary or basic ontology. 
Like RDF, RDFS follows a minimalistic approach allowing a relatively basic expressiveness compared to the Ontology Web Language (OWL) – which is built upon it.

\paragraph{RDF(S) Class}
Classes are resources – identified by an URI, described by properties – associated with a set of resources called the class extension. 
Resources in the class extension  are called instances – the rdf:type property may be used to state a resource as an instance of a class. 
Note that a class extension can cointain the class itself as instance – this is why rdfs:Class can be defined as a rdfs:Class in the table below.
SubClasses may be defined by the SubClassOf property. In this case, their extension pertains necessarily to any upper class extension. 
If we state « X » to be the class of all « Opus » and « Y » a sub-class of « X » containing all the « MusicalOpus » ; then every instance of « Y » will be an instance of « X ». 

% \paragraph{Class List}
\begin{table}[ht!]
   \begin{center}
		\begin{tabularx}{400pt}{|l|X|}
		   \hline
		Class name & Comment\\ \hline\hline
		rdfs:Resource & The class resource, everything.\\ \hline
		rdfs:Literal & The class of literal values, e.g. textual strings and integers.\\ \hline
		rdf:XMLLiteral & The class of XML literals values.\\ \hline
		rdfs:Class & The class of classes.\\ \hline
		rdf:Property & The class of RDF properties.\\ \hline
		rdfs:Datatype & The class of RDF datatypes.\\ \hline
		rdf:Statement & The class of RDF statements.\\ \hline
		rdf:Bag & The class of unordered containers.\\ \hline
		rdf:Seq & The class of ordered containers.\\ \hline
		rdf:Alt & The class of containers of alternatives.\\ \hline
		rdfs:Container & The class of RDF containers.\\ \hline
		rdfs:ContainerMembershipProperty & The class of container membership properties, rdf:\_1, rdf:\_2, \dots, all of which are sub-properties of 'member'.\\ \hline
		rdf:List & The class of RDF Lists.\\ \hline
		\end{tabularx}
		\caption{Class list \label{tab:rdfs-classes}}
   \end{center}
\end{table}

\paragraph{RDF(S) Properties}
First of all, let's recall that all properties are instances of the rdf:Property class. A property link a pair of resources, one of them as a subject and the other one as the object. 
RDFS introduces two properties to restrict which can of resources may be linked together. rdfs:domain constraints the subject resource to be an instance of a given class whereas rdfs:range constraints the objet resource likewise. 
SubProperty relationships – stated with rdfs:subPropertyOf – induces that all pairs of resource linked are also linked by the upper property. 
Thus, having a « Contributor » property and a « Performer » sub-property we can state that:
\begin{Verbatim}[fontsize=\small,formatcom=\color{black!70}]
ex:1812\_Overture
ex:Performer
ex:Minneapolis\_SO
\end{Verbatim}
and this would entail:
\begin{Verbatim}[fontsize=\small,formatcom=\color{black!70}]
ex:1812\_Overture
ex:Contributor
ex:Minneapolis\_SO
\end{Verbatim}

% 7.3.2.a  Property List
\begin{table}[ht!]
   \begin{center}
		\begin{tabularx}{450pt}{|l|X|c|c|}
		   \hline
rdf:type & The subject is an instance of a class. & rdfs:Resource & rdfs:Class \\ \hline
rdfs:subClassOf & The subject is a subclass of a class. & rdfs:Class & rdfs:Class \\ \hline
rdfs:subPropertyOf & The subject is a subproperty of a property. & rdf:Property &rdf:Property \\ \hline
rdfs:domain & A domain of the subject property. & rdf:Property & rdfs:Class\\ \hline
rdfs:range & A range of the subject property. & rdf:Property & rdfs:Class\\ \hline
rdfs:label & A human-readable name for the subject. & rdfs:Resource & rdfs:Literal \\ \hline
rdfs:comment & A description of the subject resource. & rdfs:Resource & rdfs:Literal\\ \hline
rdfs:member & A member of the subject resource. & rdfs:Resource & rdfs:Resource\\ \hline
rdf:first & The first item in the subject RDF list. & rdf:List & rdfs:Resource\\ \hline
rdf:rest & The rest of the subject RDF list after the first item. & rdf:List & rdf:List\\ \hline
rdfs:seeAlso & Further information about the subject resource. & rdfs:Resource &rdfs:Resource\\ \hline
rdfs:isDefinedBy & The definition of the subject resource. & rdfs:Resource & rdfs:Resource\\ \hline
rdf:value & Idiomatic property used for structured values (see the RDF Primer for an example of its usage). & rdfs:Resource & rdfs:Resource\\ \hline
rdf:subject & The subject of the subject RDF statement. & rdf:Statement & rdfs:Resource\\ \hline
rdf:predicate & The predicate of the subject RDF statement. & rdf:Statement & rdfs:Resource\\ \hline
rdf:object & The object of the subject RDF statement. & rdf:Statement & rdfs:Resource\\ \hline
		\end{tabularx}
		\caption{Class list \label{tab:rdfs-classes}}
   \end{center}
\end{table}

Eventually, observe these four properties that appear more like annotation than property defining resource: \cd{rdfs:label, rdfs:comment, rdfs:seeAlso, rdfs:isDefinedBy}.





\subsubsection*{Ontology Web Language}
\addcontentsline{toc}{subsection}{Ontology Web Language}
The Ontology Web Language (OWL) is a knowledge representation language intended – as its name states – to build ontology in a web environnement. 
OWL relies on RDF and XML syntax and defines its constructs as extension or subset of RDF/RDFS classes and properties. 
Nevertheless, OWL provides in addition comparaison and cardinality constraints on classes or properties. 
These constructs enables to model domain specific ontology while bringing along generic reasoning tool. 

The W3C created a working group in 2001 and documents became recommendations in 2004.
However, OWL was a revision of earlier work called DAML+OIL initiated conjointly by the « Defense Advanced Research Projects Agency » (DARPA) and the European Union's « Information Society Technologies » (IST) project. 

\paragraph{OWL species}
OWL defines in fact three sub-languages with different level of expressiveness and thus computational efficiency:
\begin{liste}
	\item « OWL Lite » is the simplest, less expressive sub-language. It supports 0..1 cardinality constraints and thus are intended for thesauri and taxonomies migration project.

	\item « OWL DL » gives full expressivity while ensuring computational completeness – all entailments are garanteed to be computed – and decidability – all computations will finish in finite times. Full expressivity means it includes all language constructs but to ensure the rest it needs to restrict their use by some conditions. 

	\item « OWL Full » gives full expressivity without conditions. For instance, a major difference with « OWL DL » is that all resources can be considered as individuals – even Class and Property. There are strong equivalence between « OWL Full » and RDF – RDFS. This comes nevertheless with no computational garantees. 
\end{liste}
% Each level is a sub-level from its predecessor, that is every legal ontology or valid conclusion expressed in « OWL Lite » is a legal ontology or respectively a valid conclusion in « OWL DL » and so on between « OWL DL » and « OWL Full ». 

% 7.4.2  Class
% OWL defines its class like RDFS except that only OWL Full can state that a class is an instance of another class. OWL Lite and DL don't allow a class to be considered at the same time as an individual – thus also as a member of a class extension. 

% 7.4.2.a  Class descriptions
% OWL supports six different kind of class descriptions: 
% 1. a simple class declaration – with a URI reference
% 2. an exhaustive enumeration of the class extension
% 3. definition of a class as a subset of another class depending on property restrictions
% 4. an intersection of several classes descriptions
% 5. an union of several classes descriptions
% 6. a complement of a class description

% 1. Class declaration can be done as follow:
% <owl:Class rdf:ID='Composer' />

% 2. Or if we want to define directly its extension like RDFS Containers, we can write: 
% <owl:Class rdf:ID='Nationality'>
% 	<owl:oneOf rdf:parseType='Collection'>
% 		<owl:Thing rdf:about='ex:French' />
% 		<owl:Thing rdf:about='ex:Russian' />
% 		<owl:Thing rdf:about='ex:English' />
% 	</owl:oneOf>
% </owl:Class>

% 3. We may also use more complex property restrictions, including value or  cardinality restrictions. For this purpose we have a set of properties:

% Value restrictions
% owl:allValuesFrom, owl:someValuesFrom  – with the values being either a class or a datatype but OWL Lite only supports class value.

% owl:hasValue – the value has to be either an individual or a data value. This property is not included in OWL Lite. 

% Cardinality constraints
% owl:maxCardinality, owl:minCardinality, owl:cardinality – the value has to be related to a XML Schema datatype.

% 4,5,6. These constructs are similar to AND, OR, NOT operators acting on classes. Only owl:intersectionOf can be used in some way in OWL Lite, whereas owl:unionOf and owl:complementOf are not included.

% 7.4.2.b  Class axioms
% Classes may be described with the previous properties, class axioms are the three properties that have owl:Class for domain and range: 
% owl:subClassOf enables specialisation to be described. The sub-class' extension set is thus stated as a subset of the class extension set. 

% owl:equivalentClass declares class extension equivalence between two classes descriptions.

% owl:disjointWith establishes that two classes extensions of two classes descriptions have no member in common.

% Note that all owl:Class are sub-classes of the owl:Thing superclass. 

% 7.4.3  Property
% OWL defines four kind of Property that must be mutually disjoint when using OWL DL:

% owl:DatatypeProperty links instance with literal values.
% owl:ObjectProperty define relations between instances.
% owl:AnnotationProperty are intended for human reader. In OWL DL, it is impossible to define restriction or sub-property for Annotation Properties and information will not be taken into account by reasoners.
% owl:OntologyProperty are used for importing ontology and make statements about versionning information. In OWL DL, the same constraints hold as those specified for Annotation Properties. 

% 7.4.3.a  Property declaration
% We take as an example the definition of creation relationships on two different level of specialisation. First, a « createdBy » relation between a « Person » and an « Opus ». Second, a music related « composedBy » relation, involving a « Composer » and a « MusicalOpus ».

% Class declaration
% <owl:Class rdf:ID='Person' />
% <owl:Class rdf:ID='Composer'>
% 	<rdfs:subClassOf rdf:resource='Person' />
% </owl:Class>
% <owl:Class rdf:ID='Opus' />
% <owl:Class rdf:ID='MusicalOpus'>
% 	<rdfs:subClassOf rdf:resource='Opus' />
% </ow:Class>

% Property declaration
% <owl:ObjectProperty rdf:ID='createdBy'>
% 	<rdfs:domain rdf:resource='Opus' />
% 	<rdfs:range rdf:resource='Person' />
% </owl:ObjectProperty>
% <owl:ObjectProperty rdf:ID='composedBy'>
% 	<rdfs:domain rdf:resource='MusicalOpus' />
% 	<rdfs:range rdf:resource='Composer' />
% 	<rdfs:subPropertyOf rdf:resource='createdBy />
% </owl:ObjectProperty>

% Note the use of rdfs:domain, rdfs:range, rdfs:subPropertyOf and rdfs:subClassOf properties within OWL statements.

% Now, if we want to take advantage of Property restrictions to define our « Composer » class, we can state for instance that « Composer » individuals are precisely those « Person » individuals who have « composed » at least one « MusicalOpus ». This lead us to write the following declaration: 

% <owl:Class rdf:ID='Composer'>
% 	<rdfs:subClassOf>
% 		<owl:intersectionOf rdf:parseType='Collection'>
% 			<owl:Class rdf:ID='Person' />
% 			<owl:Restriction>
% 				<owl:onProperty rdf:resource='composedBy' />
% 				<owl:minCardinality rdf:datatype='\&xsd;nonNegativeInteger'>1</owl:minCardinality>
% 			</owl:Restriction>
% 		</owl:intersectionOf >
% 	</rdfs:subClassOf>	
% </owl:Class>

% Note that the intersectionOf property requires a list of Classes declarations which can be given either by owl:Class or owl:Restriction – which is defined as a sub-class of owl:Class. 

% 7.4.3.b  Property characteristics
% Properties can also be defined as:
% transitive: P(x,y) and P(y,z) implies P(x,z). For instance, if x is located in y and y is located in z, then x is located in z. 
% symetric: P(x,y) if and only if P(y,x). If x is next to y, then y must be next to x. 
% functionnal: P(x,y) and P(x,z) implies y = z. The property has only one value for each individual it applies to. 
% inverseOf: P1(x,y) if and only if P2(y,x). If x is the « fatherOf » y, then y is the « childOf » x, thus making « fatherOf » the inverseOf « childOf » property. 
% inverseFunctional: P(y,x) and P(z,x) implies y = z.


% 7.4.4  Individual
% Ontology is not only about Classes and Properties, it enables us to state some facts about individuals. In order to describe them, we state them as Class instance, make use of Property or assert facts about their individuality. 

% Let's use our previous ontology description to decribe « Tchaikovsky » and the « 1812 overture ».

% <Person rdf:ID='Tchaikovsky' />
% <MusicalOpus rdf:ID='1812_Overture'>
% 	<composedBy rd:resource='Tchaikovsky' />
% </MusicalOpus>

% From this description, we can entail that:
% « Tchaikovsky » is not only a « Person », it is also a « Composer » because he composed the « 1812 overture ». As he composed it, we can also say that he has « created » it. 

% As OWL is a web language, it has rejected the « unique name » assumption and hence need to deal with identity uniqueness – and thus URI reference. In order to do so, three constructs have been defined:
% owl:sameAs states that two URI reference refer in fact to the same individual, meaning we can merge their definition and the related conclusions. 
% owl:differentFrom declares that two URI reference refer to different individuals. 
% owl:AllDifferent provides an idiom for stating that a list of individuals are all different. 

% Now, to demonstrate the use of sameAs property and to refer to our information sample, we define several « Tchaikovsky » individuals with different spelling. Then, we declare several MusicalOpus, « Cappricio italien » and « Ромео и Джульетта » which is the russian spelling for « Romeo and Juliet ».

% <Person rdf:ID='Tchaïkovski' xml:lang='fr' />
% <Person rdf:ID='Чайкoвский' xml:lang='ru' />

% <MusicalOpus rdf:ID='Capriccio_italien' xml:lang='fr'>
% 	<composedBy rd:resource='Tchaïkovski' />
% </MusicalOpus>
% <MusicalOpus rdf:ID='Ромео и Джульетта' xml:lang='ru'>
% 	<composedBy rd:resource='Чайкoвский' />
% </MusicalOpus>

% If we change the first statements by:
% <Person rdf:ID='Tchaïkovski' xml:lang='fr'>
% 	<owl:sameAs rdf:resource='Tchaikovsky'/>
% </Person>
% <Person rdf:ID='Чайкoвский' xml:lang='ru'>
% 	<owl:sameAs rdf:resource='Tchaikovsky'/>
% </Person>

% We can entail that there are the same person and that he has composed all three « MusicalOpus » described. 
















\subsection{Thésaurus et vocabulaires structurés}
\subsubsection{Besoins en modélisation}
%%%%%%%%%%%%%%%%%%%%%%%%%%%%%%%%%%%%%%%% à revoir
%La mise en place d'une telle application nécessite de représenter le vocabulaire de la réalisation audiovisuelle dans toutes ses variations possibles et de le documenter suffisamment afin de le rendre compréhensible pour des novices. 
Cet objectif nous amène à considérer la construction d'une ressource termino-ontologique.
L'ontologie permet de représenter les concepts partagés par les professionels de la réalisation audiovisuelle et la terminologie permet de capturer les différentes formes d'expression associées à ces concepts. 

La spécificité de notre problématique est de considérer la collaboration de communautés hétérogènes par leur degré de compréhension des concepts ou leur utilisation de la terminologie. 
Ceci nous amène à envisager la terminologie comme un moyen d'associer à des éléments ontologiques (concept, relation, instances) une chaîne lexicale ou des ressources média.
Chaque chaîne ou ressource s'adresse en particulier à une communauté dont les membres partagent une capacité d'interprétation commune. 
Il n'existe donc plus une terminologie de référence par langue, mais des terminologies pour chaque communauté d'utilisateurs. 
On remarquera que notre acception de la terminologie sert bien à normaliser les pratiques linguistiques entre les membres d'une même organisation. 
En plus de cela, elle permet de fixer la manière de s'adresser à d'autres communautés.

Par ailleurs, les types de réalisations sont divers et nécessitent des concepts spécifiques pour être décrits. Une fiction se structure en séquences et en scènes alors que les documentaires ou magazines d'information se composent de sujets. 
La variabilité des types de contenu à filmer implique donc de pouvoir étendre le fond conceptuel initial pour représenter de nouveaux usages. 
De la même manière, la collaboration avec de nouveaux partenaires nécessite de pouvoir ajouter de nouvelles terminologies au fond conceptuel existant. 
Ontologie et terminologie doivent se gérer de manière indépendante. 
À partir de ces besoins, nous définissons maintenant les exigences en terme de modélisation. 

Nos besoins en modélisation peuvent être exprimés par les assertions suivantes:
\begin{enumerate}
	\item[(A1)] la variabilité des pratiques linguistiques des organisations et des communautés implique d'associer plusieurs termes à un même concept. Il n'y a pas de choix des termes préférés par une communauté mais une \textit{correspondance} entre les termes d'une ou plusieurs communautés, quels que soient la langue et le code d'écriture utilisé.
	
	\item[(A2)] la variabilité de compréhension des communautés implique d'associer des explications (chaîne lexicale) ou des illustrations (ressource média) aux concepts afin d'en enrichir la \textit{documentation}. 
	
	\item[(A3)] la variabilité des cas de collaboration implique de pouvoir étendre la conceptualisation initiale ou la terminologie pour s'adapter à de nouvelles pratiques ou de nouvelles communautés. Cela implique une gestion et une \textit{évolution} indépendante de l'ontologie et de la terminologie. 
\end{enumerate}



\subsubsection*{Simple Knowledge Organization System}
\addcontentsline{toc}{subsection}{Simple Knowledge Organization System}
\e{
SKOS est un langage de représentation de vocabulaires structurés et de thésaurus qui repose sur RDF et OWL. 
Son objectif est de représenter tout type de SOC en vue de le publier sur le web de données,  liées et ouvertes (Linked Open Data). 
En témoigne les travaux et méthodes de conversion proposés par \cite{Summers2008} ou \cite{VanAssem2006} et les applications de gestion de thésaurus développés autour de SKOS, comme celle de \cite{Schandl2010}.
On notera que l'objectif de publication semble pousser vers une représentation minimale mais extensible de SOC déjà construits.
}

% \subsubsection*{SKOS}
\paragraph{Concept et Etiquette}
SKOS centre son modèle sur les concepts (\cd{skos:Concept}) et considère les étiquettes comme des propriétés de ces derniers. 
On distingue trois types d'étiquettes: 
\begin{itemize} 
	\item les étiquettes préférées (\cd{skos:prefLabel}) qui sont uniques par langue et servent de référence.
	\item les étiquettes alternatives (\cd{skos:altLabel}) qui servent de synonymes pour l'étiquette de référence. 
	\item les étiquettes cachées (\cd{skos:hiddenLabel}) qui servent à la récupération d'erreurs de frappes les plus courantes. 
\end{itemize}
Chacune de ces propriétés est formalisée comme une instance de \cd{owl:Anno\-tationProperty}, ce qui permet de l'attacher dans les faits à tout type d'éléments ontologiques, et pas seulement à des concepts. 
Les valeurs lexicales portées par ces étiquettes sont formalisées comme des \cd{rdf:PlainLiteral}, ce qui permet de spécifier la langue et l'alphabet utilisés. 
Par exemple, la chaîne "higashi"@ja-Latn correspond à un mot japonais écrit avec l'alphabet latin.

\paragraph{Documentation}
Différentes notes de documentation existent afin de :
\begin{itemize}
	\item définir un concept (\cd{skos:definition}), expliciter son contexte d'usage (\cd{skos:scopeNote}) ou donner des exemples (\cd{skos:example})
	\item spécifier l'historique de sa signification (\cd{skos:historyNote}), les changements effectués (\cd{skos:changeNote}) ou à faire (\cd{skos:editorialNote})
\end{itemize}
L'ensemble de ces notes est défini comme une spécialisation de \cd{skos:note}, formalisé comme une \cd{owl:annotationProperty}. 
De cette manière, les notes peuvent servir à porter de la documentation écrite (comme les étiquettes), pointer vers des ressources RDF ou des documents identifiés par une URI. 
Cela permet ainsi de prévoir l'extension de ces notes à des besoins plus spécifiques.


\paragraph{Groupes et relations entre Concepts}
Les concepts peuvent être regroupés dans des schémas conceptuels (\cd{skos:\-ConceptScheme} et relation \cd{skos:inScheme}) et structurés par différentes relations:
\begin{itemize}
	\item des relations de structuration hiérarchiques (\cd{skos:broader}, \cd{skos:na\-rrower}) ou associatives (\cd{skos:related})
	\item des relations de correspondances entre concepts de schémas différents, soit une relation d'équivalence exacte (\cd{skos:exactMatch}), une équivalence approximative (\cd{skos:closeMatch}), des relations hiérarchiques (\cd{skos:broadMatch}, \cd{skos:narrowMatch}) ou associative \cd{skos:rela\-ted\-Match}).
\end{itemize}


\paragraph{SKOS-XL}
SKOS-XL est une extension de SKOS développée courant 2008 pour proposer une modélisation alternative au vocabulaire de base et favoriser des extensions plus fines. 
Dans SKOS-XL les termes ne sont plus portés par les concepts mais deviennent des éléments à part entière (\cd{skosxl:Label}). 

Les relations d'attachement entre termes et concepts sont analogues aux attributs de SKOS mais les formalisent comme des instances de \cd{owl:objectProperty}. 
Le domaine de ces relations n'est pas défini ce qui permet de les associer à n'importe quelle ressource RDF, et donc en particulier aux concepts SKOS mais également à des ConceptScheme. 
Si cette dernière possibilité permet de créer des groupes de termes, elle introduit une confusion sur la sémantique des ConceptScheme (groupe de concepts, de termes, de triplets ?). 
Les étiquettes portent une seule chaîne lexicale avec les mêmes possibilités que pour SKOS grâce à l'attribut \cd{skos:literalForm}. 
L'indépendance des étiquettes permet également de spécifier des relations entre eux comme la synonymie, la traduction, etc. \cite{Pastor2009a}. 
Cette possibilité est ouverte par la relation générique \cd{skosxl:labelRelation}. 


Dans SKOS, la gestion de plusieurs jargons métiers dans une même conceptualisation [A1] est rendue difficile par la caractérisation simple des termes par la langue. 
Ainsi, même si on accroche plusieurs étiquettes au même concept, on ne sait pas les sélectionner pour les présenter à l'une ou l'autre communauté. 
Cela implique un dédoublement des concepts et donc des schémas conceptuels nécessaires (un par communauté).%, voir figure \ref{fig:skos}. 
%implique une représentation plus fine par rapport à SKOS ce qui 
Avec le découplage terme-concept de SKOS-XL, on peut gérer terme et concept de manière séparés sans pour autant avoir de primitives spécifiques pour regrouper les termes par jargon ou code d'écriture, voir figure \ref{fig:skosxl}. 
Une solution consisterait à regrouper les termes dans des ConceptScheme (SKOS-XL l'autorise).
Les ConceptScheme serviraient alors à la fois à regrouper les concepts (AV-Scheme) et les termes spécifiques aux organisations (RTBF-Scheme, VRT-Scheme). 
Cependant, si cette modélisation permet de gérer vocabulaire métier et conceptualisation de manière séparés c'est au prix d'un flottement sur la sémantique de ConceptScheme. 
Le support d'un nouveau jargon peut donc se faire sans toucher à la conceptualisation [A3] grâce à la permissivité de l'extension SKOS-XL. 
L'extension ou la mise en correspondance de la conceptualisation est facilitée par les relations sémantiques entre concepts.


\subsubsection*{ISO 25964-1}
\addcontentsline{toc}{subsection}{ISO 25964-1}
Cette norme propose une modélisation terme-concept similaire à SKOS-XL mais se concentre sur la représentation des thésaurus. 
Elle se fonde sur des modèles pré-existants, le méta-modèle \cite{Vandenbussche2009} ainsi que la norme BS 8723. 
L'originalité par rapport à SKOS est de considérer la composition de termes ou de concepts et d'enrichir la description des éléments du modèle par des attributs Dublin Core.

Les termes se distinguent entre termes préférés simples ou composés et termes non préférés simples. 
La caractérisation des termes porte également sur l'appartenance à une langue à laquelle s'ajoutent des attributs de dates, une définition ainsi que des notes d'historique et de révision. 
Des relations sémantiques entre termes sont également considérées en particulier l'équivalence composée, la synonymie, l'abréviation, l'acronyme, etc. 
Le thésaurus est considéré comme l'élément central décrit par l'ensemble des quinze attributs originaux du Dublin Core ainsi que des notes d'historique pour la maintenance. 
Sur les questions de documentation et de groupement de concepts, il existe une similarité importante avec les primitives de SKOS (note, groupe de concepts, etc.).\\

%\textbf{Discussion}\\
Les apports de la norme ISO 25964 par rapport à SKOS concernent davantage les pratiques de création et de maintenance de thésaurus que la gestion des jargons métiers [A1] et l'illustration de concepts par des ressources média [A2]. 
Ainsi, les manques par rapport à nos besoins sont similaires. L'attention portée sur les détails de description de chaque élément du modèle est tout à fait convaincante. 
L'écart avec l'aspect épuré et synthétique de SKOS s'explique certainement par l'écart entre leurs objectifs. 
Alors que SKOS vise la publication de tout type de SOC, l'ISO 25946 se concentre sur la construction et l'évolution des seuls thésaurus. 
La norme ne s'est pas encore attelée aux questions d'interopérabilité et de correspondance avec d'autres vocabulaires. Ce travail est en cours et sera dévoilé dans la seconde partie de la norme (ISO 25946-2). 
De notre point de vue, il manque toujours un moyen de regrouper des termes indépendamment des concepts pour ajouter des jargons à une conceptualisation existante [A3].


\subsection*{Bilan}
L'étude des standards et normes de références précédentes ne semble pas apporter de solution complètement satisfaisante pour l'ensemble de nos besoins. 
En effet, les approches restent centrées sur le concept et sa structuration auquel on intègre (SKOS) ou rattache (SKOS-XL, ISO 25946-1) ensuite les termes. 
Ces derniers sont représentés de manière plus ou moins fine (gestion des compositions dans ISO 25946 absente de SKOS). 
Dans tous les cas, la préférence d'un terme s'établit uniquement sur l'appartenance à une langue et non par rapport à une communauté de jargon. % mais sont caractérisés de manière identique dans les deux modèles (appartenance à une langue). 
De ce fait, ces modèles se limitent à représenter un seul jargon de référence par thésaurus et suppose ainsi l'existence d'une communauté homogène dans sa compréhension et dont on cherche à normaliser l'usage linguistique. %Dans notre cas, la collaboration entre communautés hétérogènes dans leur compréhension des concepts et leur utilisation de la langue exige de pouvoir gérer plusieurs jargons. C'est pourquoi 
Nous proposons dans la suite un modèle Multi-Jargon afin d'associer plusieurs jargons métiers et des explications à une conceptualisation commune. %basé sur des concepts originaux 


\subsection*{Discussion}
\addcontentsline{toc}{subsection}{Discussion}


% % \cleardoublepage

% %%%%%%%%%%%%%%%%%%%%%%%%%%%%%%%%%%%%%%%%%%%%%%%%%%%%%%%%%%%%%%%%%%%%%%%%%%%%%%%%%%%%%%%%%%%%%%%%%%%
\chapter{Modélisations de l'audiovisuel (m,i)}\label{chap:mav}
%
% Il existe de nombreux modèles, schémas et standards pour décrire les divers aspects de la chaîne de production audiovisuelle et des objets qui y sont construits. 
% Parmi ces modèles, certains sont issus d'une refléxion générale sur la description des ressources numériques. 
% L'exemple le plus emblématique étant le schéma de métadonnées de la \pc{Dublin Core Metadata Initiative} (\cite{DCMIUsageBoard2010}) qui doit servir à décrire toutes ressources sur le Web.
% De tels modèles ne suffisent pas à décrire les objets audiovisuels, il ont donc fait l'objet de spécialisation, par exemple \cite{Hunter1999}.
% D'autres adoptent une approche générale qui englobe l'ensemble des contenus multimédia.
% Le \pc{Moving Picture Experts Group} (MPEG) est certainement l'organisation qui a le plus porté cette vision avec leurs standards MPEG-7 et MPEG-21.
% Enfin, il y a des modèles développés spécifiquement par des membres de l'industrie pour répondre aux besoins de l'audiovisuel ou de la télévision.
% Il s'agit par exemple d'organisation comme le \pc{TV Anytime Forum}, la \pc{Society of Motion Picture and Television Engineers} (SMPTE) et l'\pc{Union Européenne de Radio-télévision} (UER ou EBU en anglais)

Les problèmes principaux qui se posent à la production audiovisuelle se situent dans la modélisation des objets qu'elle produit et des connaissances qui y sont associées (\ref{sec:pmetiers}). 
Le besoin d'autonomiser tous les objets de la chaîne audiovisuelle, en vue de les réutiliser dans de nouveaux contextes d'exploitations, exige en effet de réévaluer les modélisations sur ces deux points (\ref{sec:scien}) : 
\begin{liste}
	\item[(A)] \g{la modélisation des objets construits au fil de la chaîne de production audiovisuelle}.
	Il s'agit de modéliser non seulement le produit final et ses composants, mais également tous les produits intermédiaires de la chaîne.
	On entend par là tous les fragments de contenu construits ou transformés au cours de la chaîne de production (les prises de vue du tournage, le montage et la séquence monté, le programme prêt à diffuser, la version pour DVD, le résumé pour le journal télévisé etc.).
	Le fait qu'ils participent ou non à la composition d'un produit final n'implique pas qu'ils ne soient pas exploitable dans d'autres contextes.
	De la même manière, les fragments peuvent être transformés à différents niveaux (technique, esthétique, éditorial etc.) pour les adapter à d'autres usages. 

	De ce fait, la condition pour rendre ces fragments autonomes et réutilisables est de les modéliser en tant qu'éléments documentaires à part entière.	
	Mais identifier des fragments documentaires après coup ne suffit pas. 
	Il faut les modéliser dès que possible, afin de rendre compte de leur statut dans le processus de construction. 
	Ce faisant, il devient alors possible de reprendre ce processus, et de l'adapter aux besoins d'un nouveau contexte d'exploitation. 
	La modélisation de l'objet audiovisuel doit donc se faire sur différents niveaux et de manière progressive, en suivant les opérations meneés au cours de la chaîne de production.\\

	% On s'intéresse donc aux types d'objets audiovisuels modélisés, au niveau d'abstraction et de fragmentation proposé pour rendre compte de la composition de ces objets et de leur construction.\\
	% modèle de composition (\cite{Stockinger2007})

	\item[(B)] \g{la modélisation des connaissances construites sur ces objets}.
:

\end{liste}

Sur ce point, il s'agit de clarifier la nature des connaissances que l'on attache aux objets et leur pertinence vis-à-vis des usages que l'on prend en compte. 
Nous considérons trois types de connaissances à associer aux objets : 
\begin{liste}
	\item les connaissances sur les objets ; leur \e{représentation matérielle} (stockage, encodage, format etc.) ; leur \e{contenu} (ce qui est vu ou entendu par le lecteur).

	\item les connaissances liées à la chaîne de production ; la \e{spécification de la forme et du contenu} (que l'on retrouve dans les documents de pré-production) ; le \e{contexte de production} au sens large, incluant les contributeurs et leurs contributions à la chaîne ; le \e{cadre d'exploitation}  qui détaille l'usage de ces objets (type de distribution, droits et propriété intellectuelle, type de réutilisation et transformations opérées pour la réaliser, etc.).

	\item des connaissances issues de l'analyse du contenu des objets audiovisuels. 
	Par exemple, une analyse rhétorique du contenu permettra de mettre à jour la logique argumentative ou discursive (\cite{Gaillard2008}). 
	Ainsi, une multitude d'analyses peuvent être menées, chacune selon une grille d'analyse du contenu propre. 
	On examinera alors si les modélisations permettent d'ajouter de nouvelles échelles de fragmentation et d'y adjoindre des informations.
\end{liste}


\cite{ThiBui2003} propose 4 types de descriptions d'un contenu :
\begin{liste}
	\item \e{description syntaxique d'ordre sensoriel} : il s'agit de caractériser le signal et la manière dont il peut être perçu.
	Pour les aspects visuels, la couleur, la forme, la luminosité, le mouvement, la position etc.
	Pour les aspects sonores, la tonalité, le rythme etc.
   
	\item \e{description syntaxique d'ordre structurel} : le contenu peut être découpé en éléments de base. 
	Pour un contenu audiovisuel, il peut s'agir d'un découpage temporel (image ou segment temporel), d'un découpage visuel (portion de l'image) ou bien encore d'un mélange des deux.
	Le caractère syntaxique s'oppose au caractère sémantique et indique que le découpage est indépendant de sa signification. 
	Le découpage se fait donc de manière arbitraire. 
	Ilne correspond pas forcément à des objets signifiants pour un humain, comme un plan, une scène ou bien une table que l'on verrait à l'écran.
	Il est toujours difficile de trancher le passage d'un élément syntaxique à un objet signifiant car cela dépend de l'application que l'on considère. 

	\item \e{description sémantique d'ordre structurel} : le contenu est décrit par des objets signifiants. 
	Du point de vue temporel, on distinguera les scènes, les chansons des moments parlés, les refrains des couplets etc.
	Du point de vue spatial, il s'agit d'objets comme une chaise, ou bien de regroupements d'objets.

	\item \e{description sémantique des objets du monde narratif} 
\end{liste}

% Nous présenterons d'abord un scénario de réutilisation pour clarifier les usages que nous visons ainsi que les besoins en modélisation (\ref{sec:cdc-av}).
% Notre état de l'art sera nourri par un examen préalable des définitions de l'objet audiovisuel  (section \ref{sec:dav}).


% On peut tenter de distinguer entre différents approches et objets de modélisation : 
% \begin{liste}
% 	\item \e{les modélisations de l'objet audiovisuel} : celles qui traitent de la composition d'un objet audiovisuel fini, qui détaillent la manière de représenter sa structure interne, ou bien la manière dont on l'a groupé avec d'autres objets.
% 	\item 
% \end{liste}
 

% Certains se concentrent sur la description des caractéristiques du signal, des évènements que montrent le contenu ou encore de la manière dont ces contenus sont produits, échangés, adaptés etc.

% L'objectif de ce chapitre est d'examiner les modèles de l'audiovisuel existants en regard de nos  de réutilisation des objets audiovisuels. 
% modéliser les objets de la chaîne de production et les connaissances associées

% les objets de la chaîne de production
% les connaissances associées à ces objets permettant de les rendre autonome dans leur circulation et leur réutilisation.

% Plusieurs sous-problèmes pour réaliser l'autonomie des objets audiovisuels :  
% nous examinerons la manière dont on peut définir un objet, un document, un contenu audiovisuel. 
% comment modéliser ces objets pour gérer leur circulation 
% comment modéliser ces objets pour faciliter leur réutilisation


% d'un exemple de réutilisation d'objets audiovisuels (\ref{sec:cdc-av}). 
% À 
% , qu'il s'agisse de clarifier les notions d'objet ou de document audiovisuel puis d'investir les problèmes de leur gestion et de leur description. 
% Nous poserons d'abord un exemple de réutilisation d'objet audiovisuels comme élément de base de notre réflexion (\ref{sec:ex-reuse}).

% \e{
% Qu'est-ce qu'un objet audiovisuel et particulier comment peut-on aborder le document audiovisuel ? (\ref{sec:dav})
% Comment les produits de la chaîne audiovisuelle sont gérées par les systèmes informatiques, quelles opérations sont menées sur ces objets ? (\ref{sec:gest})
% Comment décrit-on les objets audiovisuels, comment s'organisent la construction ou la récolte de ces informations dans la chaîne de production ? (\ref{sec:desc})}

\section{Cahier des charges fonctionnel}\label{sec:cdc-av}
% \addcontentsline{toc}{section}{Cahier des charges fonctionnel}



\subsection{Scénario de réutilisation multiple}\label{sec:ex-reuse}
Pour bien saisir la finesse des différentes opérations possibles, nous proposons de prendre un exemple.
Imaginons une chaîne de télévision qui souhaite réaliser la captation d'un évènement culturel (par exemple un opéra, une pièce de théâtre, un concert etc.). 
Les producteurs de la chaîne sont intéressés par trois types de contenus qui seront ensuite exploités de quatre manières différentes (voir Figure \ref{img:intro:reuse}):
\begin{listenum}
	\item[a.] la \e{captation de l'évènement} en tant que tel.
	\item[b.] des \e{entrevues avec l'équipe} (metteur en scène, talents sur scène, programmateur etc.). 
	\item[c.] des \e{commentaires du public} avant ou après l'évènement.\\

	\item une partie de tous les types de contenu sera utilisée pour construire un sujet destiné à un \e{journal télévisé}. 
	\item un montage raccourci de l'évènement et des commentaires du public seront utilisés pour produire une \e{bande-annonce diffusée sur le Web}.
	\item un montage de la captation de l'évènement, des bonus comprenant les entrevues avec l'équipe ainsi que la bande-annonce utilisant les commentaires spectateurs seront intégrés dans le \e{DVD}.	 
	\item tout ou partie du contenu filmé pourra être transmis ou vendu à des \e{organisations tierces}. 
\end{listenum}

\begin{figure}[ht!]
\centering
\includegraphics[width=0.7\textwidth]{images/UC-Tahnhauser-v1fr.png}
\caption{Modèle de la production classique comparé avec une production avec réutilisation}
\label{img:intro:reuse}
\end{figure}

Chaque cas de réutilisation tire sa matière première d'à peu près la même base de contenu filmé, mais en tire partie d'une manière propre à chaque forme d'exploitation visée. 
En effet, chaque audience a ses attentes, de même qu'il existe des contraintes techniques spécifiques pour chaque contexte d'exploitation. 
%En effet, il existe des contraintes techniques et des attentes spécifiques à chaque contexte d'exploitation. 

Ces spécificités exigent des variations dans la qualité de l'encodage, le format d'encapsulation utilisé, le montage réalisé, l'habillage du contenu etc. 
Par exemple, les contraintes de diffusion sur le Web implique d'encoder la vidéo dans un format spécifique et de multiples résolutions, généralement plus petites que pour la diffusion télévisée. 
Ensuite, le montage d'une bande-annonce possède un rythme généralement plus rapide que celui des bonus de DVD. 
Finalement, les cas d'exploitation gérés par la chaîne de télévision posséderont un habillage spécifique (logo de la chaîne, message d'annonces etc.) que ne partageront pas forcément les versions vendu à des organisations tierces. 

L'exemple des commentaires du public -- voir la Figure \ref{img:intro:reuse-process} -- permet de montrer à quels moments des transformations doivent être effectuées afin de produire les différentes formes d'exploitation :
\begin{liste} 
	\item[$\bullet$] On considère que deux commentaires de spectateur ont été tournés. 
	\item[$\bullet$] Un des commentaires est intégré au montage du journal télévisé, alors que les deux sont utilisés pour créer la bande-annonce. La bande-annonce est elle-même intégrée au montage du DVD. 
	\item[$\bullet$] Au moment de la finition, l'encodage de la bande-annonce est adapté à la qualité DVD et Web. De même, le journal télévisé est encodé à la fois pour une diffusion en définition standard (SD) et haute-définition (HD).
\end{liste}


\begin{figure}[ht!]
\centering
\includegraphics[width=0.8\textwidth]{images/EX-Content-Production-v7fr.png}
\caption{Étapes et transformations des contenus pour chaque forme d'exploitation des commentaires des spectateurs}
\label{img:intro:reuse-process}
\end{figure}

% Dans cet exemple, on distingue deux types d'opérations effectuées sur le contenu ; 
% la sélection de séquences au moment du montage qui correspond à une décision éditoriale (quel contenu va-t-on présenter à l'audience ?) ; 
% la tranformation de l'enregistrement du contenu qui correspond à des choix techniques (quelle méthode d'enregistrement va-t-on utiliser ?).
% Afin de préciser la nature de ces opérations, nous présentons différentes approches de la réutilisation des contenus.


\subsection{Besoins en modélisation (n,i,t)}
Le scénario d'usage que nous venons de voir présente un exemple de production incluant directement plusieurs cadres d'exploitation pour des contenus produits en collaboration avec deux organisations professionelles et des amateurs. 
Ce scénario permet d'illustrer les échanges 



The details of this use case specifies more than a simple reuse of material. It specifies the kind of processing that support repurposing and which defines thus the modeling requirements for the audiovisual document:
%Such exploitation cases implies various kinds of operations: 
\begin{itemize}
	\item the \textit{reencoding} of edited material to fit the technical parameters proper to each distribution medium/channel (news report distributed by channel broadcasting and internet).
	
	\item the reuse of shooting materials in two distinct editorial structure (\textit{resequencing} of the opinion shot in the website and news report editing).

	\item the reuse of a part of an editorial structure into another editorial structure (\textit{repurposing} of the public comment editing into the DVD bonus editing). 
\end{itemize}


\section{Qu'est-ce qu'un objet audiovisuel ?}\label{sec:dav}


\subsection{Essence, contenu, asset}
\cite{Cox2006} : Essence + Metadata = Content ; \cite{Austerberry2004} Asset = Content + Rights to use it
Définition de Media Asset etc. de \cite{Furht2008}.

\subsection{Le document audiovisuel}
[Quelles sont les manières de représenter les objets/contenus/documents audiovisuels, quelle sont les différences entre ces notions.
\cite{Morizet-mahoudeaux2005a} ;  Voir thèse Charhad 2005. Voir AAF.]

]

\paragraph{Functional Requirements for Bibliographic Records}
FRBRoo est un modèle conceptuel développé par (\cite{Aalberg2008})

Il vise à faciliter l’échange d’information entre les bibliothèques numériques et les musées. 
Il permet de représenter les personnes participant aux différentes étapes de construction d’un objet culturel, depuis l’idée jusqu’à la réalisation matérielle.
Chaque objet culturel possède trois niveaux de modélisation :
\begin{liste}
	\item le niveau des idées ou des oeuvres (\g{Work}) n’ayant pas pris corps dans une matérialité externe à un sujet (par exemple une mélodie ou une histoire qui nous reste dans la tête). 

	\item le niveau des formes d'expression (\g{Expression}) où l'on distingue parmi toutes les formes possibles pour exprimer une idée (une nouvelle écrite, ses traductions, une adaptation de nouvelle en scénario, une lecture de cette nouvelle etc.).
	On se situe à un niveau intermédiaire qui définit des formes abstraites de  réalisation.
	Il faut préciser qu'on parle de forme abstraite dans le sens où il n'existe pas de réalisation concrète, ce qui n'empêche pas de les définir précisement et donc de distinguer de multiples variantes d'expressions :

	\ciel{
	the form of expression is an inherent characteristic of the expression, any change in form (e.g., from alpha-numeric notation to spoken word, a poem created in capitals and rendered in lower case) is a new expression. Similarly, changes in the intellectual conventions or instruments that are employed to express a work (e.g., translation from one language to another) result in the creation of a new expression.} 
	
	\item le niveau des réalisations concrètes comme les porteurs physique d’information (\g{Information Carrier}) portant les expressions (livre, partition, cd-rom etc.). 
	À ce niveau, il faut également distinguer entre l’original (\g{Manifestation Singleton}) et les copies manufacturées (\g{Item}) issues d’un modèle de publication (\g{Manifestation Product Type}). % à rapprocher de la notion de Media Profile dans MPEG-7
\end{liste}




% \paragraph{Sciences de l'information et de la communication}
% [\cite{Leleu-merviela} : le document comportent à la fois des dimensions sémiotiques (signes et sens), techniques (enregistrements, codages et transmission de signaux) et des dimensions médiatiques (socialisation et diffusion).]

% Avec le numérique : niveau des données (enregistrement en binaire, inaccessible et illisible pour l'humain), niveau du texte (une structure organisée de parties informationnelles), niveau de surface (actualisation effective ou affichage au sens large). 

% À la surface : \e{scénique} (manière de transposer des données en une réalité concrète) et \e{scénation} (manière de restituer temporellement à l'utilisateur les fragments d'un document, \ciel{la structure organisée d’événements et/ou d’états avec lesquels l’utilisateur est effectivement mis en interaction.}).

% \ciel{
% Cependant en numérique, les fragments existaient, au moins potentiellement, dans la mémoire de la machine, ce n’est que leur actualisation sur l’écran et la forme qu’elle prend qui se construit dans l’ici et maintenant de l’interaction. 
% Celle-ci est donc nécessairement volatile. De plus, elle change à chaque fois.
% Ainsi c’est l’affichage, [\dots] qui varie, mais non le document lui-même tel qu’il est mémorisé au niveau des données.}

% \ciel{
% conserver, retrouver l’information n’est pas suffisant. 
% Pour qu’elle puisse être utile, il faut qu’elle puisse être exploitée, c’est-à-dire traitée et rapprochée d’autres de façon à produire de l’information nouvelle. 
% Produire du sens n’est, pour l’essentiel, que rapprocher des informations disparates jamais rassemblées auparavant.} (\cite{Balpe1990})

% % Deux pistes proposées par SLM : 
% Il est alors possible de construire des assemblage cohérent de fragments le temps d'une consultation (d'un affichage) par un utilisateur (documents virtuels personnalisables).
% Plus on a de connaissance sur son activité, ses tâches, ses compétences propres, plus il est alors possible de rendre cette assemblage pertinent. 

% Il est aussi possible de mettre à profit la description des documents pour construire des notions de voisinage indépendamment du profilage des utilisateurs. 
% La proximité entre deux documents pourra s'évaluer d'autant de manière qu'il y a de critères descriptifs.
% Ainsi, des informations auparavant éparpillées dans des documents papier différents pourraient être regroupés. 

\newpage
\section{Circulation et réutilisation des objets audiovisuels}\label{sec:gest}

% [Voir MXF, voir AAF ? \cite{Cox2006}
% [Identifiant : hors du cadre de la thèse, dépendant des choix applicatifs des organisations qui utilisent notre modèle. Plusieurs solutions peuvent être implementés via OWL, les URI pouvant être transformé.]

\e{
Si la promesse du numérique de faciliter la manipulation et la circulation des fichiers semble bien s'être réalisée, il n'est pas si évident de l'articuler avec les besoins de la production audiovisuelle (\ref{sec:besoins}).
Ce que l'on nomme la réutilisation des objets audiovisuels recouvre en réalité diverses pratiques et qui repose plus sur la notion d'objet métier ou d'objet numérique que sur la notion informatique de fichier.
Ainsi, la production souhaite récupérer des contenus existants ou produits par d'autres pour les intégrer dans sa propre chaîne de production, ou bien de réutiliser des contenus dans de nouveaux cadres d'exploitations (variation des modes de consommation, de distribution, de public etc.) quelque soit la manière dont l'informatique représente ces objets.}

\e{
Ces opérations qui semblaient a priori plus simple dans un environnement numérique sont en fait plus compliquées qu'il n'y paraît. 
Le numérique impose le calcul et l'explicitation des informations.
Or toutes les informations construites durant la chaîne de production ne sont pas encore intégrées dans les systèmes informatiques actuels.
Lorsque ces informations s'échangent sur papier, à l'oral, par mail ou dans des fichiers non-structurés, le lien avec les objets audiovisuels est alors bien souvent rompu, ce qui entraîne une limitation des traitements réalisables sur ces objets.}

\e{
Dès lors que l'on s'applique à structurer et associer ces informations aux objets audiovisuels, on ouvre la possibilité de récupérer, manipuler, transformer ces objets de nouvelles manières. 
Ainsi augmentés d'un supplément de contexte, les objets gagnent un supplément de manipulabilité susceptible de satisfaire aux besoins de la production audiovisuelle.
Une des solutions développée et utilisée dans l'industrie de la production audiovisuelle est le format conteneur qui encapsule divers types de données en un seul fichier. 
Ainsi, ces formats permettent d'associer de multiples types de fichiers multimédia avec d'autres types d'informations.}

\e{
Cette section a d'abord un souci de clarification des usages et des solutions adoptées. Nous nous intéresserons d'abord aux pratiques de réutilisations (\ref{sec:reuse}), puis nous expliquerons leurs impacts sur la chaîne de production audiovisuelle (\ref{sec:rechaine}). 
Enfin, nous présenterons des formats conteneurs qui assurent le transport des contenus et des informations associées le long de la chaîne de production (\ref{sec:wrapper}).}





%%%%%%%%%%%%%%%%%%%%%%%%%%%%%%%%%%%%%%%%%%%%%%%
\subsection{Caractériser la réutilisation}\label{sec:reuse}
% \subsubsection{Caractérisations de la réutilisation}\label{sec:caracs-reuse}
Nous avons vu grâce à l'exemple de la section \ref{sec:ex-reuse} à quel moment et dans quel type d'opérations la réutilisation pouvait se concrétiser. 
Nous proposons maintenant d'examiner la manière dont différentes communautés scientifiques  abordent la notion de réutilisation. 
Il s'agit de clarifier les hypothèses et les techniques proposées par chacune de ces communautés, et ainsi identifier les éléments pris en compte dans leur représentation du monde.  % Correction ?

\paragraph{Multimédia et Signal}
Prenons d'abord le cas de la communauté multimédia très orientée analyse et traitement du signal. 
Dans ce cadre, les constats mis en avant sont largement les mêmes que ceux que nous avons présentés précédemment (voir section \ref{sec:motiv}, multiplication et diversification des terminaux de lecture et des réseaux de communication, transformation des usages) :
 
\ciel{ 
Hundreds of device profiles are available for accessing online content and more announced everyday. These devices are connected through a wide variety of networks [\dots] As before, the issue of usage scenarios --activity type, user age and gender, time available, and prior knowledge of the subject matter-- continues to exist.} (\cite{Singh2004}).

Un point diffère cependant, le \gui{problème} de la variabilité des usages est considéré comme de même nature que la variabilité des technologies pour transférer et lire le contenu. 
En effet, l'approche de la réutilisation privilégiée par cette communauté consiste en une transformation automatique du contenu en fonction des paramètres d'un scénario de distribution et de lecture : 

\ciel{
Fundamental to this approach is the need to maintain a single copy of the content in its original form and to repurpose the content to fit the desired scenario in real time and in an automated fashion. [\dots] the next step in the repurposing process is to describe the content so that it can be understood and processed to fit delivery requirements --whether they're technical or usage based.} (\cite{Singh2004}).

L'approche automatique est justifiée par la difficulté à maintenir et gérer différentes versions d'un même contenu, en plus d'être coûteux et chronophage.
Ainsi, la décision humaine est simplement reportée au niveau du paramétrage du système de supervisation des opérations techniques.\\


\paragraph{Ingénierie Documentaire}
Dans la communauté de l'ingénierie documentaire, le principe est de pouvoir modéliser distinctement le message que l'auteur souhaite transmettre et la forme dans laquelle ce message se donne à voir par un lecteur. 
Cette tradition, que l'on pourrait faire remonter à la fin des années 60 avec la création du \e{Generalized Markup Language} (\cite{Goldfarb}) ancêtre des SGML, HTML, XML et consorts, repose sur le balisage d'un contenu source. 
Il s'agit alors d'identifier des fragments de contenu ainsi que leur structuration pour mieux les manipuler, quelque soit les opérations effectuées sur ces fragments (transformation, indexation, réécriture etc. \cite[chap.5.2]{Bachimont2004}). 
Les langages de modélisation documentaires tels que \e{Document Type Definition} ou \e{XML Schema} (\cite{Fallside2004}) permettent de contrôler par une grammaire les systèmes de balises construits en vue de formaliser des usages documentaires. 

Nous noterons le développement récent des \gui{chaînes éditoriales}, ces systèmes qui opérationnalisent l'hypothèse de base de l'ingénierie documentaire reformulée par \cite{Crozat2004} de la sorte : \ciel{tout contenu numérique consiste en une ressource qu’un calcul permet de publier dynamiquement sous différentes formes contextualisées}. 

Ces systèmes se concentrent ainsi sur le maintien d'une ressource de base que l'on peut transformer ensuite de diverses manières, soit par une transformation technique que l'on pourra automatisée, soit par une transformation manuelle réglée sur les usages visés (\cite{Crozat2011}) : 
\begin{liste}
	\item le polymorphisme \ciel{consiste en la possibilité technique de disposer d'une source unique de contenu et de la transformer à volonté selon les supports et mises en formes désirés}. Dans ce cas, on établit une séparation entre le fond (la source documentaire) et les formes de publication qui permet de mettre en place une production multi-support.

	\item la réutilisation \ciel{par référence (sans duplication d'information) consiste en la possibilité technique de désassembler et de ré-assembler des fragments de contenu afin de les partager entre plusieurs documents}. Dans ce cas, l'opération repose sur une modélisation séparée du scénario (la structuration) et le contenu.

	\item la ré-éditorialisation est une \ciel{remise en contexte de fragments issus d'un fonds documentaire, par leur ré-agencement au sein d'un nouveau document, leur augmentation par une création de contenus spécifiques et leur publication sur un nouveau support et/ou pour un nouveau public}.

	% \item[T] l'\e{intégration multimédia} est l'exploitation de la propriété héritée du numérique et du codage binaire de permettre l'inscription sur le même support de formes sémiotiques différentes (texte, image, audio, vidéo, ...), afin de composer des contenus multimédia.
\end{liste}

Notons que les chaînes éditoriales s'orientent vers des pratiques de ré-éditorialisation qui sont réalisées manuellement plutôt que de manière automatique.
Les définitions données du polymorphisme et de la réutilisation sont des définitions d'opérations techniques plutôt que des pratiques en tant que telle. 
Ces opérations sont donc permises et prises en charges par les chaînes éditoriales mais ne constituent par leur horizon d'usage.
 % et paramétrées par des règles définies par un utilisateur.
Il semble donc que l'Ingénierie documentaire traditionnelle et le courant lié aux chaînes éditoriales s'intéressent tous deux à des opérations techniques similaires (le polymorphisme et la réutilisation) mais visent des usages distincts qui ne posent pas les mêmes problèmes :
\begin{liste}
	\item D'un côté, il s'agit d'instrumenter d'automatiser des réécritures, entre objets multimédia mais aussi entre documents structurés en XML, base de données etc. 
	Un exemple classique est la création de compte-rendu (ou reporting) qui s'effectue en extrayant des données de diverses sources puis en les intégrant dans de nouveaux documents.

	\item De l'autre on vise à fournir un nouvel environnement de travail aux métiers de l'édition (auteur, éditeur, graphiste etc.) qui permet de passer d'une production artisanale à une production multi-support, réutilisable, ré-éditorialisable. 
	Dans ce cadre, on s'intéresse plus à la création de documents non automatisable telle que les supports pédagogiques par exemple (\cite{Crozat2007}).\\
\end{liste}


\paragraph{Sémiotique Audiovisuelle}
% transition définition Stockinger
Alors que les approches précédentes se concentrent sur des techniques et des outils particuliers, l'approche sémiotique que nous présentons ici propose un point de vue plus général pour définir les différents types de réutilisations existants. 

La sémiotique s'intéresse aux signes pour étudier les activités humaines associées, qu'il s'agisse des producteurs (et de leur intention de communication), des lecteurs (et de leur interprétation des signes produits) ou des relations entre producteur et lecteur (c'est-à-dire des conventions qu'ils partagent). 
Selon \cite{Peirce1978} le \gui{signe} est composé de trois éléments ; le \e{représentamen}, ce qui représente et qu'on pourrait rapprocher de la notion de signifiant chez \cite{DeSaussure1995} ; l'\e{objet}, ce qui est représenté ; l'\e{interprétant} qui produit la relation entre les deux premiers éléments. 

En sémiotique, le signe fait donc toujours l'objet d'une interprétation de la part d'un lecteur qui mobilise un ensemble de conventions pour tenter d'extraire un sens --qui n'est pas forcément l'intention qu'a voulu exprimé l'auteur.
La transmission d'un contenu ne suffit pas en soi à garantir la réussite de la communcation, celle-ci est toujours suceptible d'échouer (soit par un défaut d'expression, un défaut de convention, un défaut d'interprétation). 

Dans ce cadre théorique, la réutilisation de contenu ne se limite pas à une transformation technique (conversion de formats d'encapsulation, de taille d'image, d'encodage) mais se conçoit comme une \gui{adaptation culturelle} \parencite{Stockinger2007} d'une ressource vis-à-vis d'un contexte qui comprend à la fois un usage et une communauté cible. 
Les contenus n'ont donc pas de valeur en soi, mais une valeur d'usage pour une communauté. 
La réutilisation, l'adaptation culturelle ou encore la republication interviennent alors lorsque les contenus sources ne satisfont pas à leur utilisation ou leur communauté de lecteurs future : 

\ciel{
La \e{republication} (en anglais re-authoring ou re-purposing) recouvre un ensemble d'activités visant à réutiliser un corpus de documents numériques (textuels, audiovisuels, visuels, etc.) pour des usages spécifiques auxquels les documents sources, dans leur forme initiale, ne peuvent que partiellement répondre} \parencite{Stockinger2007b}.


On l'aura compris, ce processus englobe des opérations techniques et éditoriales et place les conventions des communautés dans une position centrale. 
Pour ce qui est de caractériser des communautés d'utilisateurs, \pc{Stockinger} se réfère à des sociologues dont \cite{Bourdieu}, et propose différents critères de regroupement :
% la notion d'habitus développée par
\begin{liste}
	\item le temps ou l'espace occupé
	\item les activités et les objectifs recherchés
	\item les attentes et les intérêts
	\item les compétences linguistiques
	\item et de manière générale les connaissances ou les représentations
\end{liste}

Une fois une communauté cible identifiée, il est alors possible ; (1) de définir le type et la forme de contenu qui est pertinent (utilisable, utile, compréhensible, acceptable etc. par ces utilisateurs) ; (2) les outils nécessaires pour effectuer les opérations propres à adapter le contenu aux besoins de la communauté cible :

\ciel{
La republication est donc un processus, parfois très complexe, d'adaptation d'un document ou d'un corpus de documents sources à des usages spécifiques. Ce processus d'adaptation peut concerner tous les plans constitutifs d'un document (Stockinger, 1981, 1999 et 2003), c'est-à-dire aussi bien le plan du contenu que celui de l'expression. Il s'accomplit à travers un ensemble d'activités intellectuelles et de gestes techniques et en référence à des modèles ou genres de publications qui intègrent les contraintes typiques des contextes et des communautés d'usage auxquels un document ou un corpus de documents republiés est destiné} \parencite{Stockinger2007b}.

% opérations : (traitement linguistique, restructuration, rééditorialisation).
La republication repose donc sur une représentation des communautés d'utilisateurs, de leur capacités d'interprétation ainsi que sur une représentation des contenus dont elles disposent habituellement. 
La republication se définit selon \parencite{Stockinger2007} suivant les critères suivants : 
\begin{liste}
	\item \e{les opérations à effectuer} ; sélection, réorganisation, ajout d'explications, ajout d'éléments complémentaires, traduction, mise en lien avec d'autres ressources, modification de la forme d'expression, création de nouveau contenu etc.
	\item \e{le type} (image, texte, objet audiovisuel etc.) \e{et le genre} (journal télévisé, émission etc.) \e{de ressources à traiter}. 
	\item \e{l'objectif de la réutilisation} ; le contexte d'usage, la communauté cible, le genre de la future publication, le format de distribution etc.
	\item \e{les ressources à disposition pour effectuer la republication} ; les personnes, les outils, le budget, les ressources intellectuelles.
\end{liste}
Cette approche générale de la réutilisation n'est pas purement intellectuelle puisqu'elle se concrétise également dans des développements logiciels. 
En effet, un logiciel nommé \gui{Atelier Sémiotique} se développe dans le cadre de l'\gui{Atelier de Sémiotique Audiovisuelle} et par l'intermédiaire de divers projets (Saphir, Logos) et partenaires (INA, ESCoM, MSH de Paris).

\paragraph{Discussion}
% Notons que cette définition, par rapport à celle des approches précédentes, décrit de manière plus globale ce qu'est la réutilisation en citant de nombreux et nouveaux éléments à prendre en compte. 
% L'analyse proposé par la sémiotique audiovisuelle propose une définition plus générale de ce qu'est la réutilisation. 
L'analyse proposée par \pc{Stockinger} propose une définition générale de la réutilisation qui englobe les pratiques présentées précédemment. 
En effet, l'ingénierie documentaire et la communauté multimédia se concentrent sur la construction d'outils pour automatiser certaines transformations ou réécritures de contenu.
En se concentrant sur un éventail de techniques, ces approches se prêtent plus à certains cas d'usages et visent des objectifs différents. %(comme la ré-éditorialisation pour l'ingénierie documentaire ou bien la construction automatique de compte-rendu pour la communauté multimédia).
% parler de C2M qui vise à une chaîne éditoriale multimédia ?  

\begin{figure}[ht!]
\centering
\includegraphics[width=0.75\textwidth]{images/Reuse-v1.png}
\caption{Les différentes pratiques de réutilisations}
\label{img:intro:reuse}
\end{figure}

Nous proposons donc de définir la terminologie suivante pour distinguer entre trois niveaux successifs de réutilisation, chacun visant à créer un nouveau document mais suivant des opérations différentes (voir Figure \ref{img:intro:reuse}). 
La premier critère distinctif est l'automatisation de la transformation, le second critère repose sur la création originale de contenu plutôt que la réorganisation d'un existant : 
\begin{liste}
	\item le \eg{retraitement} (repurposing) qui se caractérise par une automatisation de la transformation opérée sur le contenu (quelque soit son type), c'est-à-dire que cette transformation est effectué par un logiciel lui-même paramétré par un humain. 
	Ces transformations visent à modifier la forme d'expression du document, extraire des fragments de contenu de différentes sources pour les aggréger dans un nouveau document, ou bien encore réorganiser automatiquement la structure du contenu. 
	Le retraitement dépasse le polymorphisme en ce sens où il est possible de gérer de multiple sources de contenus pour construire dynamiquement un nouveau document. 
	Dans le cas de l'audiovisuel, il s'agit des pratiques de réencodage, de changements de format d'encapsulation etc.
	Pour reprendre l'exemple précédent (\ref{sec:ex-reuse}) d'un contenu TV pour une diffusion Web, ou bien encore la création automatique de résumé de rencontres sportives etc. \\

	\item la \eg{rééditorialisation} (reediting) se caractérise par une transformation (manuelle et automatique) de contenus existants. 
	Le document doit s'adapter à un nouveau contexte de lecture (genre éditoriale, public, forme d'expression etc.).
	La transformation du contenu nécessite une compréhension du nouveau contexte de lecture et consiste en des opérations de réorganisation, de mise en relation avec d'autres contenus, de traduction etc. 
	Ces opérations ne se limitent pas à une transformation de la forme d'expression du document (retraitement) mais ne constituent pas une création originale de contenus (réécriture). 
	Simplement, on réutilise divers contenus existants pour créer un nouveau document.  
	La ré-éditorialisation repose donc sur le polymorphisme et la réutilisation au sens de \cite{Crozat2011}.
	Dans le cas de l'audiovisuel, il s'agit typiquement de pratiques de re-montage et de nouvelles sélections de contenu. 
	Pour reprendre l'exemple précédent, il s'agit de monter différemment une séquence initialement prévue pour un journal TV et qui doit s'insérer dans un DVD etc. \\

	 
	\item la \eg{réécriture} (reauthoring) se caractérise par une transformation de contenus existants accompagnée d'une création de contenu original. 
	L'ajout de contenu sert à satisfaire soit aux attentes spécifiques du nouveau public cible, aux contraintes d'un nouveau genre éditorial (commentaires, explications, exemples etc.) soit à la création d'une version augmentée d'un document existant (pas de changement de public cible, mais de nouvelles attentes).
	Dans le cas de l'audiovisuel, il s'agit par exemple de la construction d'un documentaire à partir de vidéo d'archives, la création originale étant le commentaire proposé.\\	 
\end{liste}

Les pratiques de réutilisations sont donc chevillées aux dimensions techniques, éditoriales et sémiotiques du contenu audiovisuel. 
Leur mise en place pose également des problèmes dans l'organisation de la chaîne de production audiovisuelle et son informatisation
% Il faut donc élargir le champ de la modélisation des contenus à une dimension sémiotique et éditoriale et faire le lien avec le déroulement de la production.








%%%%%%%%%%%%%%%%%%%%%%%%%%%%%%%%%%%%%%%%%%%%%%%
\subsection{Évolutions de la chaîne de production}\label{sec:rechaine}
\e{
Les changements introduits par la réutilisation dans la chaîne de production sont donc plus vastes qu'une simple adaptation technique à de nouveaux modes de distribution. %(canal de diffusion + terminal de lecture). 
Il s'agit également de prendre en compte l'audience visée pour affiner encore plus l'adaptation du contenu à ses futurs consomateurs/lecteurs.
L'objectif est de favoriser le développement de variantes d'un même programme soit par la restructuration du contenu (retraitement ou rééditorialisation) ou par l'ajout de contenus (réécriture). 
L'introduction d'acteurs tiers dans une chaîne de production pour fabriquer ou fournir du contenu ne peut se faire sans une plus grande maîtrise des contenus et une meilleure description de ces derniers dès la pré-production.
En effet, le client qui souhaite déléguer la fabrication de contenu à un fournisseur tiers doit d'abord définir ses attentes. 
À l'inverse, si le fournisseur connaît son contenu le client lui a besoin d'un descriptif pour sélectionner les fragments les plus pertinents.
Ainsi, les chaînes de production des clients et des fournisseurs doivent évoluer pour gérer (fournir/acquérir) non pas juste du contenu, mais des descriptions (adjointes ou pas à du contenu) facilitant le travail de leurs partenaires (fabrication ou réutilisation).
}
% Le producteur-diffuseur rentre alors dans une dynamique d'adaptation de ces contenus.

Comme nous l'avons constaté en \ref{sec:electro}, le développement de l'électronique offre de nouvelles opportunités de production mixte, soit avec des amateurs, soit avec d'autres professionnels. 
Cependant, une organisation souhaitant profiter de ces opportunités devra réussir d'abord à encadrer ses partenaires et clarifier avec eux les termes de leurs accords. 
Ce qui auparavant pouvait se résoudre \e{de visu} ou de manière informelle doit maintenant être explicité afin de clarifier la demande, c'est-à-dire le contenu souhaité. 
Qu'il s'agisse de passer commande, ou bien de rechercher dans des bases existantes, cette étape s'apparente à la définition du besoin, à l'écriture d'un cahier des charges, ou dans les termes propres à la production audiovisuelle, au \e{Scripting} défini en \ref{sec:preprod}. 
Maintenant que la fabrication du contenu est déléguée à des tiers, il reste cependant à récupérer le résultat et à vérifier qu'il satisfait à la demande initiale. 
Cette dernière étape nommée généralement \e{Acquisition} constitue un travail à part entière puisqu'il s'agit de \gui{faire rentrer} le contenu dans les \gui{cases} du système d'information et de gestion des contenus. 
En plus des questions de formats informatiques, s'ajoute souvent le problème de la description des contenus et de leur classification en vue de leur utilisation future.  
L'acquisition dépend grandement des conventions établies avec le fabricant/livreur de contenu et impacte directement sur le temps passé à faire le \e{derushing}. 

Côté client, on transforme d'abord la phase de scripting en l'expression d'une \pc{Commande} ou d'une \pc{Requête} ce qui permet de déléguer la fabrication des contenus à un tiers.
Ensuite, on vérifie par \pc{Acquisition} du contenu que le résultat correspond bien à la demande et on procède aux ajustements nécessaires (si besoin) pour satisfaire aux contraintes de notre système (voir Figure \ref{img:intro:evochain}). 

\begin{figure}[ht!]
\centering
\includegraphics[width=\textwidth]{images/Workflow-Thesis-v6.png}
\caption{Ouverture des chaînes de production du client et du fournisseur}
\label{img:intro:evochain}
\end{figure}

Du côté des fournisseurs de contenu, il existe une distinction entre les chaînes du fabricant et du fournisseur du contenu :  

\begin{liste}
	\item \e{déléguer la fabrication à des contributeurs tiers} : l'utilisation du scripting pour définir la \pc{Commande} de contenu attendu semble une solution satisfaisante, à condition que le vocabulaire utilisé soit normé et rattaché à une conceptualisation de manière à éviter la confusion ou les différences d'interprétations. 
    Lorsqu'il s'agit d'amateurs, la situation se complique car on ne peut pas s'appuyer sur une conceptualisation commune de la production audiovisuelle pour clarifier la commande. 
    De plus, le manque d'expérience et l'ignorance des usages du métier impliquent non seulement de documenter les concepts par des mots et des définitions, mais aussi d'expliquer ce qu'il faut faire durant la phase de \pc{Fabrication}. 
    En d'autres termes, en travaillant avec des amateurs, les professionnels ne se retrouvent non pas à écarter la confusion entre des mots se reférant au même concept, mais à expliquer les opérations auxquelles ces concepts font référence. 
    De même pour l'acquisition, s'il s'agit surtout de se mettre d'accord entre professionnels, travailler avec des amateurs semble plus difficile de prime abord. 
    Les notions de formats d'encodage et d'encapsulation sont souvent confuses ou se mélangent, de même que la description des contenus peut s'avérer compliquée à réaliser sans expérience préalable. 
    Tout du moins, il faut remarquer que la description de la demande initiale sert de description a minima du contenu produit, même si les variations ou les écarts ne sont pas forcément indiqués.
    Le cas échéant, une phase d'\pc{Indexation} peut être nécessaire pour décrire le contenu suivant les exigences du client.\\

	\item \e{rechercher des contenus existants depuis les bases professionnelles} : l'utilisation du vocabulaire de l'écriture audiovisuelle pour définir une \pc{Requête} nécessite une indexation utilisant ce même vocabulaire, ou alors une manière de traduire la requête d'un langage à l'autre (par alignement des vocabulaires par exemple). 
	De même, il faut pouvoir s'accorder sur le niveau de fragmentation recherché (programme complet, séquences, scène, frame etc.), le format du contenu, les descriptions ou les métadonnées à fournir etc.
	Ainsi, la \pc{Livraison} de contenu ne consiste pas en un simple transfert de fichier, mais représente le moment où l'on teste l'interopérabilité entre les systèmes et les formats. 
	Cette étape est d'autant plus cruciale qu'elle se répercute directement sur la phase d'acquisition pour le client. 
	Tout ce qui n'a pas pu être résolu à la livraison (côté fournisseur) devra l'être au moment de l'acquisition dans le système (côté client).\\
	% un vocabulaire de requêtes, s'accorder sur les niveaux de fragmentation, le format de livraison, le contenu de la livraison
\end{liste}



Finalement, il nous faut encore éclairer à quels moments dans la chaîne de production les différentes pratiques de réutilisation sont réalisées (voir définition en \ref{sec:reuse}).
De manière générale, on considère que la réutilisation commence à la phase de pré-production, au moment du \pc{Planning} et du \pc{Scripting} où l'on spécifie les nouvelles formes et formats d'exploitation (voir Figure \ref{img:intro:reutilisation}). 
Mais chaque pratique opére à différents étapes de la chaîne :
\begin{liste}
	\item pour le \e{retraitement}, les variations sur la forme d'expression du document se réalisent en phase de \pc{Finition}. 
	C'est à ce moment que l'original et les variantes sont encodés et encapsulés dans les formats correspondants à leur mode de distribution. 
	Lorsqu'il y a manipulation de la structure des contenus, ces opérations (automatisées) se réalisent à la phase de \pc{Montage}.

	\item pour la \e{rééditorialisation}, le travail commence en phase de \pc{Derushing}, par la sélection des séquences de contenu à ajouter ou à retirer du contenu original. 
	La grande différence avec le retraitement, c'est que cette sélection s'effectue manuellement sur les contenus à disposition. 
	Ensuite, on opére un nouveau \pc{Montage} qui peut également impliquer une \pc{Finition} différente.

	\item pour la \e{réécriture}, la grande différence avec les autres pratiques est que l'on ajoute une nouvelle phase de \pc{Fabrication}. 
	Qu'il s'agisse de création originale ou de récupération de contenu chez un fournisseur tiers, la réécriture consiste à utiliser de nouveaux contenus. 
	Ensuite, on sélectionne en \pc{Derushing} les séquences qui permettront de réaliser un nouveau \pc{Montage}. 
	Finalement, plusieurs \pc{Finition} sont à envisager suivant les cas d'exploitation.
\end{liste}
% conséquences de la réutilisation dans la chaîne : à quel étape ça se joue

\begin{figure}[ht!]
\centering
\includegraphics[width=0.8\textwidth]{images/Workflow-Reuse-v1.png}
\caption{Les différentes formes de réutilisation et leur mise en oeuvre dans la chaîne de production audiovisuelle.}
\label{img:intro:reutilisation}
\end{figure}








\subsection{Formats conteneurs pour l'audiovisuel} \label{sec:wrapper}
% intro
Les formats conteneurs sont des formats de fichiers qui encapsulent des contenus de toutes sortes (audio, vidéo, texte, image etc.). 
Un des exemples le plus connu pour la vidéo est le format AVI de Microsoft, souvent confondu avec un format de compression. 
Ces formats ont la particularité d'associer aux contenus audio-visuels des informations annexes, le plus souvent sous la forme de métadonnées.
L'intérêt de ces formats est de constituer un objet numérique qui structure l'enregistrement du contenu et de ses informations annexes et fournit ainsi une manière unique d'y accéder (\cite{Ferreira2010}).
Sans cela, les informations seraient dispersées et le lien avec le contenu devrait se faire via des références. 
Cette force constitue également un inconvénient lorsqu'il n'est pas possible de faire évoluer le modèle d'information ou bien d'en changer en fonction du type de production ou d'exploitation envisagé.

Dans le cas de la production télévisuelle, deux formats liés et complémentaires ont été progressivement adoptés par l'industrie.  
Il s'agit du \pc{Material eXchange Format} (MXF) et de l'\pc{Advanced Authoring Format} (AAF) que nous présentons par la suite. 
Leur particularité est de pouvoir intégrer des schémas de métadonnées propres aux besoins de l'industrie, mais aussi de pouvoir en utiliser d'autres. 
Comme ces formats servent de référence à l'industrie, il est particulièrement intéressant pour nous de comprendre leur modélisation de l'objet audiovisuel, de voir comment ils gèrent les résultats intermédiaires de la chaîne de production ou quels genres d'informations ils embarquent avec le contenu.


\subsubsection{Material eXchange Format}
% qui / quand
MXF est un format conteneur ouvert développé depuis le milieu des années 90 par des membres de l'industrie et standardisé en 2004 par la \pc{Society of Motion Picture and Television Engineers} (SMPTE).
% objectif
Son objectif est de favoriser les échanges de contenus audio-visuels finis en les associant à d'autres données ou métadonnées (\cite{Devlin2002}).
Ces métadonnées sont structurés par le schéma \pc{Descriptive Metadata Scheme-1} (DMS-1, ) que nous présenterons en détails dans la suite de la section. 

% description
Voici d'abord les principales caractéristiques de MXF en tant que format (\cite{Ferreira2010a}) : 
\begin{liste} 
	\item \e{indépendant d'un système propriétaire}. 
	Le standard se veut avant tout un format qui fonctionne sur tout systèmes. 
	Ainsi, il définit une organisation des données bit par bit qui repose notamment sur le système de \pc{Key-Length-Value} (KLV, clé-longueur-valeur). 
	La clé donne un identificateur de l'élément à suivre, la longueur précise la taille de la valeur à suivre et la valeur contient les données de l'élément.
	Comme dans tout format de fichier, on retrouve la définition d'en-tête et de fin de fichier. 
	L'en-tête contient des métadonnées de description du contenu (\pc{Header Metadata}), de sa structuration (\pc{Partition Metadata}) et une table d'association entre un timecode et une position dans le flux binaire du fichier (\pc{Index Table}).

	\item \e{indépendant des méthodes de compression du contenu utilisées}.
	MXF définit une structure d'encapsulation et d'accès au contenu nommé \pc{Essence Container} (EC) qui permet de transporter du contenu sans le transformer ou bien de faire référence à des fichiers externes.  
	MXF possède des transpositions permettant de synchroniser différents flux de contenus, quelque soit leur encodage. 
	Chaque type de contenu est traité à part, ainsi les EC sont composés de \pc{Content Package}, eux-mêmes décomposé en \pc{Picture Item} (piste vidéo), \pc{SoundItem} (piste audio), \pc{Data Item} (télétexte, sous-titre etc.), \pc{Compound Item} (contenu audiovisuel encodé comme un seul contenu) et \pc{System Item} (autres données comme les timecode etc.).
	La Figure \ref{img:mxf-content} montre deux méthodes d'encapsulation de ces essences.


	\item \e{diffusable en flux continu ou bien par fichier}. 
	Suivant la méthode d'organisation de l'EC décrit ci-dessus, le contenu d'un fichier MXF peut être visionné au cours de son transfert (streaming). 
	Cette caractéristique est particulièrement importante dans le cadre de la diffusion télévisuelle et s'applique à tout les types de contenu d'un MXF (audio-visuel, mais aussi sous-titre ou métadonnées etc.). 
	Naturellement, le fichier MXF peut également être transféré en FTP.

	\item \e{une organisation des contenus indépendante de leur visionnage}. 
	En effet, MXF définit à part l'organisation des contenus encapsulés et la manière de les visionner.
	Le \pc{Header Metadata}	d'un fichier MXF contient une partie \pc{File Pacakage} (FP) qui décrit la manière dont les fichiers de contenus sont encapsulés dans le MXF. 
	Cette description détaille les méthodes d'encodage pour toutes les pistes de contenus (Track) du fichier, de même que les timecode originaux.
	Cependant, si le FP décrit les sources d'un fichier MXF, il existe également un \pc{Material Package} (MP) décrivant la manière dont celles-ci doivent être visionnées.
	Il s'agit là de définir un montage simplifié qui explicite quelle partie et dans quel ordre jouer les contenus sources, à la manière d'une \ciel{Edit Decision List}. 

	\item \e{encapsulation conjointe des contenus et des métadonnées}. 
	Comme nous l'avons vu, un fichier MXF contient un \pc{Header Metadata} transportant des métadonnées propres à l'ensemble du fichier ainsi que des métadonnées propres à chaque paquet de données (Package). 
	Ces éléments sont donc intégrés dans la structure du fichier, au même titre que le contenu.
\end{liste}


\begin{figure}[ht!]
\centering
\includegraphics[width=0.9\textwidth]{images/MXF-ContentPackage.png}
\caption{Deux méthodes d'encapsulation des essences en MXF : image par image (haut, pour le streaming), par séquence vidéo (bas).}
\label{img:mxf-content}
\end{figure}

\paragraph{Adoption et usage}
Du fait de sa large adoption par l'industrie de la télévision, il favorise l'interopérabilité entre les systèmes, souvent propriétaires, des producteurs, diffuseurs, chaînes de télévision etc. (\cite{Ferreira2010}, \cite{Devlin2002}). 
Cependant, MXF n'est pas fait pour gérer les résultats intermédiaires de la chaîne de production. 
Il a été spécifiquement conçu pour favoriser la circulation des programmes finis, indépendamment de la manière dont les contenus sont matériellement enregistrés et structurés.
De ce fait, MXF se positionne comme un format utilisé en fin de chaîne de production, à la diffusion des programmes ou bien dans le cas d'échanges entre professionnels.

\subsubsection{Description Metadata Scheme-1}
Le schéma DMS-1 a été standardisé par le SMPTE en 2004 (\cite{Smpte2004}).
Il propose trois schémas de description (\pc{Framework}), chacun proposant une perspective de description particulière : 
\begin{liste}
	\item le \pc{Production Framework} propose une description du fichier MXF en tant que résultat d'une production. 
	Les informations qu'il regroupe s'appliquent donc au fichier en entier (identification, propriété intellectuelle, droits et contrats, projet, format de publication, format d'image, récompense) mais aussi au contenu de l'objet audiovisuel (évènements relatés, période historique, annotation).

	\item le \pc{Clip Framework} aborde la description du point de vue de la création du matériel audio-visuel, c'est-à-dire des séquences de contenu encapsulées dans le MXF. 
	On retrouve des informations liées à la production (projet, droits et contrat) mais surtout des éléments pour décrire les essences (format de l'image, sous-titre, script utilisé, matériel utilisé et paramètrage, opérations de transformation des essences) et une description par plan.

	\item le \pc{Scene Framework} propose une vision éditoriale du contenu en le découpant en scène et plan. 
	Ces éléments sont ensuite décrits en terme d'évènements, en précisant les participants et les lieux où ils se déroulent etc.
\end{liste}

\paragraph{Framework et ensemble de métadonnées}
Ces \pc{Framework} sont composés de petits ensembles de métadonnées, parfois partagés,  que l'on attachent au \pc{Header Metadata} d'un fichier MXF. 
Par exemple, l'ensemble \pc{Titles} est commun au trois \pc{Framework} et se compose des métadonnées suivantes ; \e{Extended Text Language Code} ; \e{Main Title} ; \e{Secondary Title} ; \e{Working Title} ; \e{Original Title} ; \e{Version Title}. 
De nombreux autres ensembles sont partagés, comme les annotations, la description des lieux, des participants, des organisations etc. 
Ces ensembles prennent alors un sens différent en fonction du \pc{Framework} auquel ils sont associés. 
Par exemple, on distingue les participants à la production, à la création d'une séquence, en tant que présentateur ou acteur (respectivement pour le \pc{Production}, \pc{Clip} et \pc{Scene Framework}). 
De même, pour les lieux il peut s'agir d'un lieu où se trouve l'organisme producteur, du lieu de tournage, du lieu où se déroule l'action qui est différent du lieu de tournage dans le cas d'une fiction (respectivement pour le \pc{Production}, \pc{Clip} et \pc{Scene Framework}). 
Ainsi, la distinction entre éléments réels (issu du \pc{Clip Framework}) ou fictifs (issu du \pc{Scene Framework}) n'est pas clairement spécifié. 
De manière générale, il semble regrettable que les mêmes ensembles de métadonnées soient utilisés pour décrire des objets différents.
Cela propage ainsi une certaine confusion sur le plan sémantique. 

\paragraph{Framework et Package MXF}
Comme pour les ensembles, les \pc{Framework} peuvent s'attacher à un ou plusieurs des \pc{Package} de MXF (\pc{File Package}, \pc{Material Package} etc.). 
Par exemple, le \pc{Clip Framework} appliqué au \pc{Material Package} décrit ce qui est nécessaire au visionnage du contenu (format d'image prévu).
S'il s'appliquait au \pc{File Package}, les informations correspondrait aux informations de création (format d'image original).
Là encore, l'objet de la description change légèrement et si les mêmes éléments de description peuvent être utilisé, il semblerait plus clair de préciser la nature de ces informations. 

Nous remarquerons que DMS-1 utilise la notion de plan de deux manière différentes qui peuvent sembler ambigüe. 
Ainsi, le plan est utilisé à la fois dans le \pc{Clip Framework} et le \pc{Scene Framework}. 
D'après les auteurs, cela correspond à la nature duale des plans, à la fois élément factuel et éditorial. 
Ce choix implique alors que chaque plan peut être décrit à deux endroits à la fois, selon deux perspectives différentes (descriptive de ce qui est perçu, ou bien pour nommer le plan par rapport au script par exemple).

\paragraph{Multilingue et thésaurus}
Concernant la gestion des vocabulaires et des langues, DMS-1 prévoit que certains ensembles de métadonnées soient décrites par un code de langue et puissent faire référence à un élément d'un thésaurus.
Cette perspective est particulièrement intéressante vis-à-vis des besoins que nous avons exprimés dans le chapitre précédent (\ref{sec:bm}).
Cependant, le lien avec un thésaurus externe est limité car il ne s'applique qu'aux ensembles et non à chaque métadonnée de l'ensemble.

\paragraph{Travaux liés}
\cite{Marcos2009} ont construit une ontologie basée sur le schéma DMS-1, MPEG-7 et une ontologie de domaine pour construire un système de Media Asset Management. 
Ce système, développé dans le cadre du projet européen RUSHES, a pour objectif d'aggréger des informations récoltées pendant la production par différentes sources, puis de les associer aux objets audiovisuels et proposer des services de recherche d'information et d'accès au contenu (\cite{Gorka2008}). 
Les objets audiovisuels considérés sont les prises de vue brutes, nommées \ciel{rush} dans le milieu audiovisuel.

L'approche consiste à transformer ou relier des informations de bas-niveau en une indexation sémantique à l'aide des ontologies développées.
Les services sémantiques proposés par le système sont les suivants : 
\begin{liste}
	\item \e{transformation de formats des données échangées pendant la production}. 
	Les annotations recueillies à partir des équipements de tournage et après analyse automatique du contenu sont transformés du format DMS-1 en un autre format utilisé par le système d'indexation.
	\item \e{sémantisation de l'analyse automatique du contenu}. 
	Les auteurs prennent l'exemple d'une reconnaissance des visages qui permet d'identifier le nombre de personnes présents dans une séquence.
	Ces personnes peuvent alors être intégrées à une base de connaissances.
	De même, l'analyse permet de détecter les changements de plans.
	\item \e{recherche, découverte, annotation de séquences}. 
	La recherche peut se faire soit à l'aide de mots-clés, soit à l'aide des concepts de l'ontologie, qui permettent alors d'enrichir les requêtes, de proposer des recommandations etc. 
	De plus, l'ontologie peut être également utilisé pour relier les annotations manuelles avec des concepts ou des éléments de la base de connaissances.
\end{liste}

Ces travaux poussent ainsi l'utilisation de DMS-1 en tant que schéma de description des prises de vue, juste après leur production, et non pas seulement des programmes finis, en fin de chaîne. 
Ce changement de granularité montre l'importance du \pc{Clip Framework} et du \pc{Scene Framework} qui permettent d'attacher la description à ces objets intermédiaires de la production.
Ceci est d'autant plus pertinent que DMS-1 prévoit déjà de décrire, en partie, les participants de la chaîne et leurs contributions.

L'originalité de l'approche se situe également dans la transformation des résultats d'une analyse automatique en objet sémantique. 
Les exemples de l'analyse des visages et de la détection de plan sont éclairants mais les auteurs ne fournissent pas d'indication pour généraliser le procédé à d'autres types d'information.



\subsubsection{Advanced Authoring Format}
% \cite{Gilmer2002} 
% \cite{Austerberry2004}
% qui 
AAF est un format conteneur développé principalement par l'\ciel{Advanced Media Workflow Association} (AMWA) en collaboration avec d'autres organismes tels que le SMPTE et l'EBU.
% objectif, portée et usage 
Ses objectifs sont similaires à celui de MXF, à la différence qu'AAF vise à favoriser les échanges de contenus à l'intérieur de la chaîne de production (\cite{Austerberry2004}).
AAF s'occupe plus particulièrement des informations utilisées au moment de la post-production par les applications de montage : 

\ciel{The traditional workflow – based around tape interchange, isolated non-linear editing and authoring tools, and ad-hoc metadata systems – is being recast as a more integrated networked system with a consistent approach to the format and interchange of essence and metadata.} (\cite{Gilmer2002})

% description
Le modèle de MXF présenté précédemment est en réalité une sous-partie de celui d'AAF. 
On retrouve donc les mêmes principes et fonctions, dont les \pc{Packages} qui portent les descriptions et les \pc{Items} qui encapsulent les contenus. 
Parmi les éléments supplémentaires dans AAF qui le destine particulièrement à une utilisation dans la chaîne de production., nous trouvons :
\begin{liste}
	\item le \pc{Physical Source Package} permet de référencer des contenus enregistré sur d'autres mediums que les disques durs (cassette vidéo, bande 35mm etc.).

	\item le \pc{Composition Package} permet de définir la manière dont le contenu doit être visionné en termes d'ordre (comme MXF) mais aussi en terme d'effets, de transition ou de composition des flux de contenu (ce que l'on appelle des EDL complexes).
	
	\item le \pc{Dictionary} qui permet d'intégrer des définitions des métadonnées autres que celles du dictionnaire du SMPTE dans le AAF.
\end{liste}

De plus, AAF se différencie par l'utilisation de la technologie \e{Structured Storage} de Microsoft pour gérer l'organisation des données (plutôt que la méthode du KLV).
De ce fait, AAF ne permet pas la diffusion en continu (streaming) de ses contenus.
Cependant, les deux formats utilisent le même modèle de structuration, ce qui permet aux applications d'effectuer des transformations de l'un à l'autre aisément, et particulièrement du AAF vers le MXF en suivant le déroulement de la chaîne.
Leurs différences les prédisposent néanmoins à des usages complémentaires. 
AAF se positionne comme un format pour la post-production qui conserve toutes les sources et le master alors que MXF, avec son modèle simplifié et ses capacités de diffusion en continu, est particulièrement intéressant pour les échanges de programmes finis.


\paragraph{SMPTE Metadata Dictionary}
Ce dictionnaire est un gigantesque registre de toutes les métadonnées utilisées par l'industrie télévisuelle. 
Régulièrement mis à jour, la dernière version disponible (\cite{SMPTE2010}) comporte 1476 métadonnées distribuées dans 499 catégories et sous-catégories.
La nature des métadonnées est très diverse, puisqu'on trouve des identificateurs, des informations administratives, interprétative, paramétriques, liées au processus etc.
Le dictionnaire donne une identification unique à chaque métadonnée et donne une définition ainsi que le codage utilisé pour la valeur. 
Malgré sa taille imposante, il est utilisé par les membres de l'industrie et notamment dans DMS-1.



\subsubsection*{Discussion}
\addcontentsline{toc}{subsection}{Discussion}
Les formats conteneurs MXF et AAF reposent sur le schéma de description DMS-1 ainsi que des dictionnaires de métadonnées développés par l'industrie.
L'originalité de ces formats est d'associer directement au matériel audiovisuel plusieurs perspectives de modélisation (\pc{Framework}) [\g{B1 : autonomie}].
L'objet audiovisuel est ainsi modélisé en tant que résultat d'une production qu'il faut valoriser commercialement (\pc{Production}) ; 
décomposé en éléments narratif (plan, scène) faisant partie d'un ensemble documentaire (\pc{Scene}) et dont on décrit le contexte historique (\pc{Production}) et les évènements réels (\pc{Clip}) ou fictifs (\pc{Scene}) qui s'y déroulent ; matériel audiovisuel construit pendant la production dont on décrit les caractéristiques techniques (\pc{Clip}).
Cette pluralité des points de vue pourrait permettre de modéliser les produits intermédiaires de la chaîne tout en leur associant des métadonnées. 
Ce remplissage progressif ne peut intervenir qu'après la production du matériel, et encapsulé dans le format AAF. 
C'est donc les informations de la post-production qui y sont capturées, puis transmises sous une forme simplifiée à un MXF qui symbolise le produit final de la chaîne.
Les produits intermédiaires ne sont donc pas représentés pour eux-mêmes, mais en tant que partie du produit final.
L'approche de modélisation est donc intéressante, mais le couplage matériel et métadonnées empêche de fragmenter la modélisation et de la commencer dès le début de la production.

Parmi les descriptions associées au matériel audiovisuel [\g{B2 : réutilisabilité}], on note un certaine confusion sur le plan sémantique. 
Ainsi, des mêmes ensembles de métadonnées sont utilisées pour représenter des éléments fictifs ou rééls, tandis que d'autres peuvent prendre un sens différents suivant le \pc{Package} ou le \pc{Framework} auxquels ils sont associés.
Ainsi, il ne s'agit pas de critiquer la modélisation qui distingue élément réél et fictif, ou bien encore le plan prévu dans le script et le plan tourné, mais bien la représentation confuse qui en est faite dans DMS-1.

Nous en concluons que les formats conteneurs sont adaptés à la circulation de documents audiovisuels finis mais dont la représentation d'une seule pièce et parfois confuse ne couvre pas nos besoins pour la réutilisation de fragments documentaires.

\input{CONTENU/1-analyse/7-description}

% \cleardoublepage




%%%%%%%%%%%%%%%%%%%%%%%%%%%%%%%%%%%%%%%%%%%%%%%%%%%%%%%%%%%%%%%%%%%%%%%%%%%%%%%%%%%%%%%%%%%%%%%%%%%
%%%%%%%%%%%%%%%%%%%%%%%%%%%%%%%%%%%%%%%%%%%%%%%%%%%%%%%%%%%%%%%%%%%%%%%%%%%%%%%%%%%%%%%%%%%%%%%%%%%
\part*{Contribution}
\addcontentsline{toc}{part}{Contribution}

\chapter{Approche et modélisation}\label{chap:mod}
\section{Principes de l'approche}\label{sec:principes}
\section{Modélisation conceptuelle}\label{sec:concept}

\chapter{Mise en oeuvre}\label{chap:op}
\section{Choix d'un langage}\label{sec:ln}
\section{Opérationnalisation}\label{sec:op}


% \cleardoublepage


%%%%%%%%%%%%%%%%%%%%%%%%%%%%%%%%%%%%%%%%%%%%%%%%%%%%%%%%%%%%%%%%%%%%%%%%%%%%%%%%%%%%%%%%%%%%%%%%%%%
%%%%%%%%%%%%%%%%%%%%%%%%%%%%%%%%%%%%%%%%%%%%%%%%%%%%%%%%%%%%%%%%%%%%%%%%%%%%%%%%%%%%%%%%%%%%%%%%%%%
\part*{Discussion}
\addcontentsline{toc}{part}{Discussion}
\chapter{Applications et validation}\label{chap:app}
% Le modèle évolue en fonction des besoins des membres du projet MediaMap. Les développeurs des applications sont susceptible d"étendre ou d'affiner le modèle suivant les besoins qu'ils rencontrent. Il ne s'agit pas encore d'un standard figé mais bien d'un effort en cours et en collaboration avec les utilisateurs. Nous nous efforçons donc d'adopter une modélisation et un vocabulaire proche de leur contexte de travail en vue de privilégier l'adoption du modèle, son utilisation et sa réappropriation. Ensuite, il s'agit de formaliser la modélisation afin de pouvoir l'opérationnaliser et développer des applications qui produisent des descriptions suivant ce modèle ou se nourrissent d'elles pour restituer une perspective métier à ces utilisateurs. 
\section{Applications du projet MediaMap}\label{sec:app}
\section{Expérimentation du projet MediaMap}\label{sec:xp}
\section{Validation}\label{sec:val}

\chapter*{Conclusion}\label{chap:cc}
\addcontentsline{toc}{chapter}{Conclusion}

% \chapter{}\label{c:}
% \section{}\label{s:}

% \cleardoublepage

%%%%%%%%%%%%%%%%%%%%%%%%%%%%%%%%%%%%%%%%%%%%%%%%%%%%%%%%%%%%%%%%%%%%%%%%%%%%%%%%%%%%%%%%%%%%%%%%%%%
%%%%%%%%%%%%%%%%%%%%%%%%%%%%%%%%%%%%%%%%%%%%%%%%%%%%%%%%%%%%%%%%%%%%%%%%%%%%%%%%%%%%%%%%%%%%%%%%%%%
%%%%%%%%%%%%%%%%%%%%%%%%%%%%%%%%%%%%%%%%%%%%%%%%%%%%%%%%%%%%%%%%%%%%%%%%%%%%%%%%%%%%%%%%%%%%%%%%%%%
%%%%%%%%%%%%%%%%%%%%%%%%%%%%%%%%%%%%%%%%%%%%%%%%%%%%%%%%%%%%%%%%%%%%%%%%%%%%%%%%%%%%%%%%%%%%%%%%%%%
%%%%%%%%%%%%%%%%%%%%%%%%%%%%%%%%%%%%%%%%%%%%%%%%%%%%%%%%%%%%%%%%%%%%%%%%%%%%%%%%%%%%%%%%%%%%%%%%%%%
%%%%%%%%%%%%%%%%%%%%%%%%%%%%%%%%%%%%%%%%%%%%%%%%%%%%%%%%%%%%%%%%%%%%%%%%%%%%%%%%%%%%%%%%%%%%%%%%%%%
%%%%%%%%%%%%%%%%%%%%%%%%%%%%%%%%%%%%%%%%%%%%%%%%%%%%%%%%%%%%%%%%%%%%%%%%%%%%%%%%%%%%%%%%%%%%%%%%%%%
%%%%%%%%%%%%%%%%%%%%%%%%%%%%%%%%%%%%%%%%%%%%%%%%%%%%%%%%%%%%%%%%%%%%%%%%%%%%%%%%%%%%%%%%%%%%%%%%%%%
%%%%%%%%%%%%%%%%%%%%%%%%%%%%%%%%%%%%%%%%%%%%%%%%%%%%%%%%%%%%%%%%%%%%%%%%%%%%%%%%%%%%%%%%%%%%%%%%%%%
%%%%%%%%%%%%%%%%%%%%%%%%%%%%%%%%%%%%%%%%%%%%%%%%%%%%%%%%%%%%%%%%%%%%%%%%%%%%%%%%%%%%%%%%%%%%%%%%%%%
%%%%%%%%%%%%%%%%%%%%%%%%%%%%%%%%%%%%%%%%%%%%%%%%%%%%%%%%%%%%%%%%%%%%%%%%%%%%%%%%%%%%%%%%%%%%%%%%%%%
% \appendix
% \part*{Annexes}
% \rehead{\headingsf\scshape Annexe~\thechapter}
% \noappendicestocpagenum
% \addappheadtotoc

% \cleardoublepage

%%%%%%%%%%%%%%%%%%%%%%%%%%%%%%%%%%%%%%%%%%%%%%%%%%%%%%%%%%%%%%%%%%%%%%%%%%%%%%%%%%%%%%%%%%%%%%%%%%%
%%%%%%%%%%%%%%%%%%%%%%%%%%%%%%%%%%%%%%%%%%%%%%%%%%%%%%%%%%%%%%%%%%%%%%%%%%%%%%%%%%%%%%%%%%%%%%%%%%%
%%%%%%%%%%%%%%%%%%%%%%%%%%%%%%%%%%%%%%%%%%%%%%%%%%%%%%%%%%%%%%%%%%%%%%%%%%%%%%%%%%%%%%%%%%%%%%%%%%%
% \chapter{D'un Web à l'autre : les paradigmes de la lecture informatique}\label{a:webs}
% \KOMAoptions{twoside=no}

% \pagestyle{empty}

% \input{PERRON/GARDE}

% \newpage

% ~

% \cleardoublepage

% \pagestyle{empty}

% \input{PERRON/REMERCIEMENTS}

% \cleardoublepage

% \KOMAoptions{twoside=yes}

\frontmatter

\pagestyle{scrheadings}

\shorttableofcontents{Sommaire}{2}

\cleardoublepage


%%%%%%%%%%%%%%%%%%%%%%%%%%%%%%%%%%%%%%%%%%%%%%%%%%%%%%%%%%%%%%%%%%%%%%%%%%%%%%%%%%%%%%%%%%%%%%%%%%%
%%%%%%%%%%%%%%%%%%%%%%%%%%%%%%%%%%%%%%%%%%%%%%%%%%%%%%%%%%%%%%%%%%%%%%%%%%%%%%%%%%%%%%%%%%%%%%%%%%%
\mainmatter

% \pagestyle{empty}

% ~

% \bigskip

% \vspace{11em}

% \bigskip

% \epigraphii{La totalité est la non vérité.}{Adorno, \ita{Minima Moralia}}

% \cleardoublepage
\addcontentsline{toc}{part}{État de l'Art}
\pagestyle{empty}

~
\bigskip

\vspace{11em}

\bigskip

\epigraphii{Se demander si un ordinateur peut penser ... est aussi intéressant que de se demander si un sous-marin peut nager.}{ Edsger Wybe Dijkstra, \ita{The threats to computing science}}

\pagestyle{scrheadings}

%%%%%%%%%%%%%%%%%%%%%%%%%%%%%%%%%%%%%%%%%%%%%%%%%%%%%%%%%%%%%%%%%%%%%%%%%%%%%%%%%%%%%%%%%%%%%%%%%%%
%%%%%%%%%%%%%%%%%%%%%%%%%%%%%%%%%%%%%%%%%%%%%%%%%%%%%%%%%%%%%%%%%%%%%%%%%%%%%%%%%%%%%%%%%%%%%%%%%%%
\part*{Exposition}
\addcontentsline{toc}{part}{Exposition}

\input{CONTENU/0-expo/1-intro}
\input{CONTENU/0-expo/2-motiv}
\input{CONTENU/0-expo/3-prodav}
\input{CONTENU/0-expo/4-besoins}
\input{CONTENU/0-expo/5-problo}

% \cleardoublepage





%%%%%%%%%%%%%%%%%%%%%%%%%%%%%%%%%%%%%%%%%%%%%%%%%%%%%%%%%%%%%%%%%%%%%%%%%%%%%%%%%%%%%%%%%%%%%%%%%%%
%%%%%%%%%%%%%%%%%%%%%%%%%%%%%%%%%%%%%%%%%%%%%%%%%%%%%%%%%%%%%%%%%%%%%%%%%%%%%%%%%%%%%%%%%%%%%%%%%%%
\part*{État de l'Art}
%%%%%%%%%%%%%%%%%%%%%%%%%%%%%%%%%%%%%%%%%%%%%%%%%%%%%%%%%%%%%%%%%%%%%%%%%%%%%%%%%%%%%%%%%%%%%%%%%%%
\input{CONTENU/1-analyse/0-cdc}
\input{CONTENU/1-analyse/1-ontologie}
\input{CONTENU/1-analyse/2-methode}
\input{CONTENU/1-analyse/3-modele}

% % \cleardoublepage

% %%%%%%%%%%%%%%%%%%%%%%%%%%%%%%%%%%%%%%%%%%%%%%%%%%%%%%%%%%%%%%%%%%%%%%%%%%%%%%%%%%%%%%%%%%%%%%%%%%%
\input{CONTENU/1-analyse/4-cdc-av}
\input{CONTENU/1-analyse/5-dav}
\input{CONTENU/1-analyse/6-gestion}
\input{CONTENU/1-analyse/7-description}

% \cleardoublepage




%%%%%%%%%%%%%%%%%%%%%%%%%%%%%%%%%%%%%%%%%%%%%%%%%%%%%%%%%%%%%%%%%%%%%%%%%%%%%%%%%%%%%%%%%%%%%%%%%%%
%%%%%%%%%%%%%%%%%%%%%%%%%%%%%%%%%%%%%%%%%%%%%%%%%%%%%%%%%%%%%%%%%%%%%%%%%%%%%%%%%%%%%%%%%%%%%%%%%%%
\part*{Contribution}
\addcontentsline{toc}{part}{Contribution}

\chapter{Approche et modélisation}\label{chap:mod}
\section{Principes de l'approche}\label{sec:principes}
\section{Modélisation conceptuelle}\label{sec:concept}

\chapter{Mise en oeuvre}\label{chap:op}
\section{Choix d'un langage}\label{sec:ln}
\section{Opérationnalisation}\label{sec:op}


% \cleardoublepage


%%%%%%%%%%%%%%%%%%%%%%%%%%%%%%%%%%%%%%%%%%%%%%%%%%%%%%%%%%%%%%%%%%%%%%%%%%%%%%%%%%%%%%%%%%%%%%%%%%%
%%%%%%%%%%%%%%%%%%%%%%%%%%%%%%%%%%%%%%%%%%%%%%%%%%%%%%%%%%%%%%%%%%%%%%%%%%%%%%%%%%%%%%%%%%%%%%%%%%%
\part*{Discussion}
\addcontentsline{toc}{part}{Discussion}
\chapter{Applications et validation}\label{chap:app}
% Le modèle évolue en fonction des besoins des membres du projet MediaMap. Les développeurs des applications sont susceptible d"étendre ou d'affiner le modèle suivant les besoins qu'ils rencontrent. Il ne s'agit pas encore d'un standard figé mais bien d'un effort en cours et en collaboration avec les utilisateurs. Nous nous efforçons donc d'adopter une modélisation et un vocabulaire proche de leur contexte de travail en vue de privilégier l'adoption du modèle, son utilisation et sa réappropriation. Ensuite, il s'agit de formaliser la modélisation afin de pouvoir l'opérationnaliser et développer des applications qui produisent des descriptions suivant ce modèle ou se nourrissent d'elles pour restituer une perspective métier à ces utilisateurs. 
\section{Applications du projet MediaMap}\label{sec:app}
\section{Expérimentation du projet MediaMap}\label{sec:xp}
\section{Validation}\label{sec:val}

\chapter*{Conclusion}\label{chap:cc}
\addcontentsline{toc}{chapter}{Conclusion}

% \chapter{}\label{c:}
% \section{}\label{s:}

% \cleardoublepage

%%%%%%%%%%%%%%%%%%%%%%%%%%%%%%%%%%%%%%%%%%%%%%%%%%%%%%%%%%%%%%%%%%%%%%%%%%%%%%%%%%%%%%%%%%%%%%%%%%%
%%%%%%%%%%%%%%%%%%%%%%%%%%%%%%%%%%%%%%%%%%%%%%%%%%%%%%%%%%%%%%%%%%%%%%%%%%%%%%%%%%%%%%%%%%%%%%%%%%%
%%%%%%%%%%%%%%%%%%%%%%%%%%%%%%%%%%%%%%%%%%%%%%%%%%%%%%%%%%%%%%%%%%%%%%%%%%%%%%%%%%%%%%%%%%%%%%%%%%%
%%%%%%%%%%%%%%%%%%%%%%%%%%%%%%%%%%%%%%%%%%%%%%%%%%%%%%%%%%%%%%%%%%%%%%%%%%%%%%%%%%%%%%%%%%%%%%%%%%%
%%%%%%%%%%%%%%%%%%%%%%%%%%%%%%%%%%%%%%%%%%%%%%%%%%%%%%%%%%%%%%%%%%%%%%%%%%%%%%%%%%%%%%%%%%%%%%%%%%%
%%%%%%%%%%%%%%%%%%%%%%%%%%%%%%%%%%%%%%%%%%%%%%%%%%%%%%%%%%%%%%%%%%%%%%%%%%%%%%%%%%%%%%%%%%%%%%%%%%%
%%%%%%%%%%%%%%%%%%%%%%%%%%%%%%%%%%%%%%%%%%%%%%%%%%%%%%%%%%%%%%%%%%%%%%%%%%%%%%%%%%%%%%%%%%%%%%%%%%%
%%%%%%%%%%%%%%%%%%%%%%%%%%%%%%%%%%%%%%%%%%%%%%%%%%%%%%%%%%%%%%%%%%%%%%%%%%%%%%%%%%%%%%%%%%%%%%%%%%%
%%%%%%%%%%%%%%%%%%%%%%%%%%%%%%%%%%%%%%%%%%%%%%%%%%%%%%%%%%%%%%%%%%%%%%%%%%%%%%%%%%%%%%%%%%%%%%%%%%%
%%%%%%%%%%%%%%%%%%%%%%%%%%%%%%%%%%%%%%%%%%%%%%%%%%%%%%%%%%%%%%%%%%%%%%%%%%%%%%%%%%%%%%%%%%%%%%%%%%%
%%%%%%%%%%%%%%%%%%%%%%%%%%%%%%%%%%%%%%%%%%%%%%%%%%%%%%%%%%%%%%%%%%%%%%%%%%%%%%%%%%%%%%%%%%%%%%%%%%%
% \appendix
% \part*{Annexes}
% \rehead{\headingsf\scshape Annexe~\thechapter}
% \noappendicestocpagenum
% \addappheadtotoc

% \cleardoublepage

%%%%%%%%%%%%%%%%%%%%%%%%%%%%%%%%%%%%%%%%%%%%%%%%%%%%%%%%%%%%%%%%%%%%%%%%%%%%%%%%%%%%%%%%%%%%%%%%%%%
%%%%%%%%%%%%%%%%%%%%%%%%%%%%%%%%%%%%%%%%%%%%%%%%%%%%%%%%%%%%%%%%%%%%%%%%%%%%%%%%%%%%%%%%%%%%%%%%%%%
%%%%%%%%%%%%%%%%%%%%%%%%%%%%%%%%%%%%%%%%%%%%%%%%%%%%%%%%%%%%%%%%%%%%%%%%%%%%%%%%%%%%%%%%%%%%%%%%%%%
% \chapter{D'un Web à l'autre : les paradigmes de la lecture informatique}\label{a:webs}
% \input{CONTENU/A1.WEBS/0}

% \cleardoublepage

%%%%%%%%%%%%%%%%%%%%%%%%%%%%%%%%%%%%%%%%%%%%%%%%%%%%%%%%%%%%%%%%%%%%%%%%%%%%%%%%%%%%%%%%%%%%%%%%%%%
%%%%%%%%%%%%%%%%%%%%%%%%%%%%%%%%%%%%%%%%%%%%%%%%%%%%%%%%%%%%%%%%%%%%%%%%%%%%%%%%%%%%%%%%%%%%%%%%%%%
%%%%%%%%%%%%%%%%%%%%%%%%%%%%%%%%%%%%%%%%%%%%%%%%%%%%%%%%%%%%%%%%%%%%%%%%%%%%%%%%%%%%%%%%%%%%%%%%%%%
% \chapter[Supports, outils et espaces --- Les mutations des opérations lectoriales]{Supports, outils et espaces\\\scriptsize{Les mutations des opérations lectoriales}}\label{a:hist}
% \input{CONTENU/A2.HIST/H0}
% \input{CONTENU/A2.HIST/H1}
% \input{CONTENU/A2.HIST/H2}
% \input{CONTENU/A2.HIST/H3}
% \input{CONTENU/A2.HIST/H4}
% \input{CONTENU/A2.HIST/H5}

% \cleardoublepage

%%%%%%%%%%%%%%%%%%%%%%%%%%%%%%%%%%%%%%%%%%%%%%%%%%%%%%%%%%%%%%%%%%%%%%%%%%%%%%%%%%%%%%%%%%%%%%%%%%%
%%%%%%%%%%%%%%%%%%%%%%%%%%%%%%%%%%%%%%%%%%%%%%%%%%%%%%%%%%%%%%%%%%%%%%%%%%%%%%%%%%%%%%%%%%%%%%%%%%%
%%%%%%%%%%%%%%%%%%%%%%%%%%%%%%%%%%%%%%%%%%%%%%%%%%%%%%%%%%%%%%%%%%%%%%%%%%%%%%%%%%%%%%%%%%%%%%%%%%%
% \chapter{Fragments pertinents issus des entretiens avec des lecteurs savants}\label{a:preint}
% \input{CONTENU/A3.PREINT.LEFTOVERS/0}

% \cleardoublepage

%%%%%%%%%%%%%%%%%%%%%%%%%%%%%%%%%%%%%%%%%%%%%%%%%%%%%%%%%%%%%%%%%%%%%%%%%%%%%%%%%%%%%%%%%%%%%%%%%%%
%%%%%%%%%%%%%%%%%%%%%%%%%%%%%%%%%%%%%%%%%%%%%%%%%%%%%%%%%%%%%%%%%%%%%%%%%%%%%%%%%%%%%%%%%%%%%%%%%%%
%%%%%%%%%%%%%%%%%%%%%%%%%%%%%%%%%%%%%%%%%%%%%%%%%%%%%%%%%%%%%%%%%%%%%%%%%%%%%%%%%%%%%%%%%%%%%%%%%%%
% \input{CONTENU/A4.SC01/QuestionnaireSC01} % → Les questions

% % → Quelques macros pour la structuration des questions
% \newcommand{\questionnaireSCUN}[1]{\section{#1}}
% \newcommand{\sectionquestionnaire}[1]{\subsection*{#1}}
% \newcommand{\question}[1]{\par\bigskip\textbf{#1}\par}
% \newcommand{\sousquestion}[1]{\par\medskip\textbf{\textcolor[rgb]{0.0,0.0,0.0}{--- \textbf{\textit{#1}}}}\par} % parcol = 0.73
% \newcommand{\reponse}[1]{\ignorespaces#1}
% \newcommand{\poimp}[1]{\bigskip\textsc{\textbf{#1}}\par}

% \chapter[Questionnaires de retour d'utilisation]{Questionnaires de retour d'utilisation\\{\scriptsize Expérience : «commentaire composé multimédia»}}\label{a:sc01}

% \ita{Note 1 : les réponses des étudiants sont rapportées sans intervention de notre part sur l'orthographe ou la syntaxe.}

% \ita{Note 2 : le prototype a été présenté aux étudiants sous le nom de «Verena».}

% \questionnaireSCUN{Étudiant \as}\input{CONTENU/A4.SC01/as}			% Alain SAAS
% \questionnaireSCUN{Étudiant \cl}\input{CONTENU/A4.SC01/cl}			% Cecile LABORDE
% \questionnaireSCUN{Étudiant \dao}\input{CONTENU/A4.SC01/dao}		% Djamila AIT OUADDA
% \questionnaireSCUN{Étudiant \ds}\input{CONTENU/A4.SC01/ds}			% Delphine SZYMCZAK
% \questionnaireSCUN{Étudiant \dd}\input{CONTENU/A4.SC01/dd}			% Dorine DUFOUR
% \questionnaireSCUN{Étudiant \glb}\input{CONTENU/A4.SC01/glb}		% Gabrielle LE BIHAN
% \questionnaireSCUN{Étudiant \rb}\input{CONTENU/A4.SC01/rb}			% Romain BODINIER

% \cleardoublepage

%%%%%%%%%%%%%%%%%%%%%%%%%%%%%%%%%%%%%%%%%%%%%%%%%%%%%%%%%%%%%%%%%%%%%%%%%%%%%%%%%%%%%%%%%%%%%%%%%%%
%%%%%%%%%%%%%%%%%%%%%%%%%%%%%%%%%%%%%%%%%%%%%%%%%%%%%%%%%%%%%%%%%%%%%%%%%%%%%%%%%%%%%%%%%%%%%%%%%%%
%%%%%%%%%%%%%%%%%%%%%%%%%%%%%%%%%%%%%%%%%%%%%%%%%%%%%%%%%%%%%%%%%%%%%%%%%%%%%%%%%%%%%%%%%%%%%%%%%%%
%%%%%%%%%%%%%%%%%%%%%%%%%%%%%%%%%%%%%%%%%%%%%%%%%%%%%%%%%%%%%%%%%%%%%%%%%%%%%%%%%%%%%%%%%%%%%%%%%%%
%%%%%%%%%%%%%%%%%%%%%%%%%%%%%%%%%%%%%%%%%%%%%%%%%%%%%%%%%%%%%%%%%%%%%%%%%%%%%%%%%%%%%%%%%%%%%%%%%%%
%%%%%%%%%%%%%%%%%%%%%%%%%%%%%%%%%%%%%%%%%%%%%%%%%%%%%%%%%%%%%%%%%%%%%%%%%%%%%%%%%%%%%%%%%%%%%%%%%%%
%%%%%%%%%%%%%%%%%%%%%%%%%%%%%%%%%%%%%%%%%%%%%%%%%%%%%%%%%%%%%%%%%%%%%%%%%%%%%%%%%%%%%%%%%%%%%%%%%%%
%%%%%%%%%%%%%%%%%%%%%%%%%%%%%%%%%%%%%%%%%%%%%%%%%%%%%%%%%%%%%%%%%%%%%%%%%%%%%%%%%%%%%%%%%%%%%%%%%%%
%%%%%%%%%%%%%%%%%%%%%%%%%%%%%%%%%%%%%%%%%%%%%%%%%%%%%%%%%%%%%%%%%%%%%%%%%%%%%%%%%%%%%%%%%%%%%%%%%%%
%%%%%%%%%%%%%%%%%%%%%%%%%%%%%%%%%%%%%%%%%%%%%%%%%%%%%%%%%%%%%%%%%%%%%%%%%%%%%%%%%%%%%%%%%%%%%%%%%%%
%%%%%%%%%%%%%%%%%%%%%%%%%%%%%%%%%%%%%%%%%%%%%%%%%%%%%%%%%%%%%%%%%%%%%%%%%%%%%%%%%%%%%%%%%%%%%%%%%%%

\backmatter

\rehead{\headingsf\scshape\headmark}
\lohead{\headingsf\scshape\headmark}

% \addcontentsline{toc}{chapter}{Bibliographie}
% \nocite{*}
\printbibliography[maxnames=11]

\cleardoublepage
\setcounter{tocdepth}{3}
\tableofcontents


% \cleardoublepage

%%%%%%%%%%%%%%%%%%%%%%%%%%%%%%%%%%%%%%%%%%%%%%%%%%%%%%%%%%%%%%%%%%%%%%%%%%%%%%%%%%%%%%%%%%%%%%%%%%%
%%%%%%%%%%%%%%%%%%%%%%%%%%%%%%%%%%%%%%%%%%%%%%%%%%%%%%%%%%%%%%%%%%%%%%%%%%%%%%%%%%%%%%%%%%%%%%%%%%%
%%%%%%%%%%%%%%%%%%%%%%%%%%%%%%%%%%%%%%%%%%%%%%%%%%%%%%%%%%%%%%%%%%%%%%%%%%%%%%%%%%%%%%%%%%%%%%%%%%%
% \chapter[Supports, outils et espaces --- Les mutations des opérations lectoriales]{Supports, outils et espaces\\\scriptsize{Les mutations des opérations lectoriales}}\label{a:hist}
% \input{CONTENU/A2.HIST/H0}
% \input{CONTENU/A2.HIST/H1}
% \input{CONTENU/A2.HIST/H2}
% \input{CONTENU/A2.HIST/H3}
% \input{CONTENU/A2.HIST/H4}
% \input{CONTENU/A2.HIST/H5}

% \cleardoublepage

%%%%%%%%%%%%%%%%%%%%%%%%%%%%%%%%%%%%%%%%%%%%%%%%%%%%%%%%%%%%%%%%%%%%%%%%%%%%%%%%%%%%%%%%%%%%%%%%%%%
%%%%%%%%%%%%%%%%%%%%%%%%%%%%%%%%%%%%%%%%%%%%%%%%%%%%%%%%%%%%%%%%%%%%%%%%%%%%%%%%%%%%%%%%%%%%%%%%%%%
%%%%%%%%%%%%%%%%%%%%%%%%%%%%%%%%%%%%%%%%%%%%%%%%%%%%%%%%%%%%%%%%%%%%%%%%%%%%%%%%%%%%%%%%%%%%%%%%%%%
% \chapter{Fragments pertinents issus des entretiens avec des lecteurs savants}\label{a:preint}
% \KOMAoptions{twoside=no}

% \pagestyle{empty}

% \input{PERRON/GARDE}

% \newpage

% ~

% \cleardoublepage

% \pagestyle{empty}

% \input{PERRON/REMERCIEMENTS}

% \cleardoublepage

% \KOMAoptions{twoside=yes}

\frontmatter

\pagestyle{scrheadings}

\shorttableofcontents{Sommaire}{2}

\cleardoublepage


%%%%%%%%%%%%%%%%%%%%%%%%%%%%%%%%%%%%%%%%%%%%%%%%%%%%%%%%%%%%%%%%%%%%%%%%%%%%%%%%%%%%%%%%%%%%%%%%%%%
%%%%%%%%%%%%%%%%%%%%%%%%%%%%%%%%%%%%%%%%%%%%%%%%%%%%%%%%%%%%%%%%%%%%%%%%%%%%%%%%%%%%%%%%%%%%%%%%%%%
\mainmatter

% \pagestyle{empty}

% ~

% \bigskip

% \vspace{11em}

% \bigskip

% \epigraphii{La totalité est la non vérité.}{Adorno, \ita{Minima Moralia}}

% \cleardoublepage
\addcontentsline{toc}{part}{État de l'Art}
\pagestyle{empty}

~
\bigskip

\vspace{11em}

\bigskip

\epigraphii{Se demander si un ordinateur peut penser ... est aussi intéressant que de se demander si un sous-marin peut nager.}{ Edsger Wybe Dijkstra, \ita{The threats to computing science}}

\pagestyle{scrheadings}

%%%%%%%%%%%%%%%%%%%%%%%%%%%%%%%%%%%%%%%%%%%%%%%%%%%%%%%%%%%%%%%%%%%%%%%%%%%%%%%%%%%%%%%%%%%%%%%%%%%
%%%%%%%%%%%%%%%%%%%%%%%%%%%%%%%%%%%%%%%%%%%%%%%%%%%%%%%%%%%%%%%%%%%%%%%%%%%%%%%%%%%%%%%%%%%%%%%%%%%
\part*{Exposition}
\addcontentsline{toc}{part}{Exposition}

\input{CONTENU/0-expo/1-intro}
\input{CONTENU/0-expo/2-motiv}
\input{CONTENU/0-expo/3-prodav}
\input{CONTENU/0-expo/4-besoins}
\input{CONTENU/0-expo/5-problo}

% \cleardoublepage





%%%%%%%%%%%%%%%%%%%%%%%%%%%%%%%%%%%%%%%%%%%%%%%%%%%%%%%%%%%%%%%%%%%%%%%%%%%%%%%%%%%%%%%%%%%%%%%%%%%
%%%%%%%%%%%%%%%%%%%%%%%%%%%%%%%%%%%%%%%%%%%%%%%%%%%%%%%%%%%%%%%%%%%%%%%%%%%%%%%%%%%%%%%%%%%%%%%%%%%
\part*{État de l'Art}
%%%%%%%%%%%%%%%%%%%%%%%%%%%%%%%%%%%%%%%%%%%%%%%%%%%%%%%%%%%%%%%%%%%%%%%%%%%%%%%%%%%%%%%%%%%%%%%%%%%
\input{CONTENU/1-analyse/0-cdc}
\input{CONTENU/1-analyse/1-ontologie}
\input{CONTENU/1-analyse/2-methode}
\input{CONTENU/1-analyse/3-modele}

% % \cleardoublepage

% %%%%%%%%%%%%%%%%%%%%%%%%%%%%%%%%%%%%%%%%%%%%%%%%%%%%%%%%%%%%%%%%%%%%%%%%%%%%%%%%%%%%%%%%%%%%%%%%%%%
\input{CONTENU/1-analyse/4-cdc-av}
\input{CONTENU/1-analyse/5-dav}
\input{CONTENU/1-analyse/6-gestion}
\input{CONTENU/1-analyse/7-description}

% \cleardoublepage




%%%%%%%%%%%%%%%%%%%%%%%%%%%%%%%%%%%%%%%%%%%%%%%%%%%%%%%%%%%%%%%%%%%%%%%%%%%%%%%%%%%%%%%%%%%%%%%%%%%
%%%%%%%%%%%%%%%%%%%%%%%%%%%%%%%%%%%%%%%%%%%%%%%%%%%%%%%%%%%%%%%%%%%%%%%%%%%%%%%%%%%%%%%%%%%%%%%%%%%
\part*{Contribution}
\addcontentsline{toc}{part}{Contribution}

\chapter{Approche et modélisation}\label{chap:mod}
\section{Principes de l'approche}\label{sec:principes}
\section{Modélisation conceptuelle}\label{sec:concept}

\chapter{Mise en oeuvre}\label{chap:op}
\section{Choix d'un langage}\label{sec:ln}
\section{Opérationnalisation}\label{sec:op}


% \cleardoublepage


%%%%%%%%%%%%%%%%%%%%%%%%%%%%%%%%%%%%%%%%%%%%%%%%%%%%%%%%%%%%%%%%%%%%%%%%%%%%%%%%%%%%%%%%%%%%%%%%%%%
%%%%%%%%%%%%%%%%%%%%%%%%%%%%%%%%%%%%%%%%%%%%%%%%%%%%%%%%%%%%%%%%%%%%%%%%%%%%%%%%%%%%%%%%%%%%%%%%%%%
\part*{Discussion}
\addcontentsline{toc}{part}{Discussion}
\chapter{Applications et validation}\label{chap:app}
% Le modèle évolue en fonction des besoins des membres du projet MediaMap. Les développeurs des applications sont susceptible d"étendre ou d'affiner le modèle suivant les besoins qu'ils rencontrent. Il ne s'agit pas encore d'un standard figé mais bien d'un effort en cours et en collaboration avec les utilisateurs. Nous nous efforçons donc d'adopter une modélisation et un vocabulaire proche de leur contexte de travail en vue de privilégier l'adoption du modèle, son utilisation et sa réappropriation. Ensuite, il s'agit de formaliser la modélisation afin de pouvoir l'opérationnaliser et développer des applications qui produisent des descriptions suivant ce modèle ou se nourrissent d'elles pour restituer une perspective métier à ces utilisateurs. 
\section{Applications du projet MediaMap}\label{sec:app}
\section{Expérimentation du projet MediaMap}\label{sec:xp}
\section{Validation}\label{sec:val}

\chapter*{Conclusion}\label{chap:cc}
\addcontentsline{toc}{chapter}{Conclusion}

% \chapter{}\label{c:}
% \section{}\label{s:}

% \cleardoublepage

%%%%%%%%%%%%%%%%%%%%%%%%%%%%%%%%%%%%%%%%%%%%%%%%%%%%%%%%%%%%%%%%%%%%%%%%%%%%%%%%%%%%%%%%%%%%%%%%%%%
%%%%%%%%%%%%%%%%%%%%%%%%%%%%%%%%%%%%%%%%%%%%%%%%%%%%%%%%%%%%%%%%%%%%%%%%%%%%%%%%%%%%%%%%%%%%%%%%%%%
%%%%%%%%%%%%%%%%%%%%%%%%%%%%%%%%%%%%%%%%%%%%%%%%%%%%%%%%%%%%%%%%%%%%%%%%%%%%%%%%%%%%%%%%%%%%%%%%%%%
%%%%%%%%%%%%%%%%%%%%%%%%%%%%%%%%%%%%%%%%%%%%%%%%%%%%%%%%%%%%%%%%%%%%%%%%%%%%%%%%%%%%%%%%%%%%%%%%%%%
%%%%%%%%%%%%%%%%%%%%%%%%%%%%%%%%%%%%%%%%%%%%%%%%%%%%%%%%%%%%%%%%%%%%%%%%%%%%%%%%%%%%%%%%%%%%%%%%%%%
%%%%%%%%%%%%%%%%%%%%%%%%%%%%%%%%%%%%%%%%%%%%%%%%%%%%%%%%%%%%%%%%%%%%%%%%%%%%%%%%%%%%%%%%%%%%%%%%%%%
%%%%%%%%%%%%%%%%%%%%%%%%%%%%%%%%%%%%%%%%%%%%%%%%%%%%%%%%%%%%%%%%%%%%%%%%%%%%%%%%%%%%%%%%%%%%%%%%%%%
%%%%%%%%%%%%%%%%%%%%%%%%%%%%%%%%%%%%%%%%%%%%%%%%%%%%%%%%%%%%%%%%%%%%%%%%%%%%%%%%%%%%%%%%%%%%%%%%%%%
%%%%%%%%%%%%%%%%%%%%%%%%%%%%%%%%%%%%%%%%%%%%%%%%%%%%%%%%%%%%%%%%%%%%%%%%%%%%%%%%%%%%%%%%%%%%%%%%%%%
%%%%%%%%%%%%%%%%%%%%%%%%%%%%%%%%%%%%%%%%%%%%%%%%%%%%%%%%%%%%%%%%%%%%%%%%%%%%%%%%%%%%%%%%%%%%%%%%%%%
%%%%%%%%%%%%%%%%%%%%%%%%%%%%%%%%%%%%%%%%%%%%%%%%%%%%%%%%%%%%%%%%%%%%%%%%%%%%%%%%%%%%%%%%%%%%%%%%%%%
% \appendix
% \part*{Annexes}
% \rehead{\headingsf\scshape Annexe~\thechapter}
% \noappendicestocpagenum
% \addappheadtotoc

% \cleardoublepage

%%%%%%%%%%%%%%%%%%%%%%%%%%%%%%%%%%%%%%%%%%%%%%%%%%%%%%%%%%%%%%%%%%%%%%%%%%%%%%%%%%%%%%%%%%%%%%%%%%%
%%%%%%%%%%%%%%%%%%%%%%%%%%%%%%%%%%%%%%%%%%%%%%%%%%%%%%%%%%%%%%%%%%%%%%%%%%%%%%%%%%%%%%%%%%%%%%%%%%%
%%%%%%%%%%%%%%%%%%%%%%%%%%%%%%%%%%%%%%%%%%%%%%%%%%%%%%%%%%%%%%%%%%%%%%%%%%%%%%%%%%%%%%%%%%%%%%%%%%%
% \chapter{D'un Web à l'autre : les paradigmes de la lecture informatique}\label{a:webs}
% \input{CONTENU/A1.WEBS/0}

% \cleardoublepage

%%%%%%%%%%%%%%%%%%%%%%%%%%%%%%%%%%%%%%%%%%%%%%%%%%%%%%%%%%%%%%%%%%%%%%%%%%%%%%%%%%%%%%%%%%%%%%%%%%%
%%%%%%%%%%%%%%%%%%%%%%%%%%%%%%%%%%%%%%%%%%%%%%%%%%%%%%%%%%%%%%%%%%%%%%%%%%%%%%%%%%%%%%%%%%%%%%%%%%%
%%%%%%%%%%%%%%%%%%%%%%%%%%%%%%%%%%%%%%%%%%%%%%%%%%%%%%%%%%%%%%%%%%%%%%%%%%%%%%%%%%%%%%%%%%%%%%%%%%%
% \chapter[Supports, outils et espaces --- Les mutations des opérations lectoriales]{Supports, outils et espaces\\\scriptsize{Les mutations des opérations lectoriales}}\label{a:hist}
% \input{CONTENU/A2.HIST/H0}
% \input{CONTENU/A2.HIST/H1}
% \input{CONTENU/A2.HIST/H2}
% \input{CONTENU/A2.HIST/H3}
% \input{CONTENU/A2.HIST/H4}
% \input{CONTENU/A2.HIST/H5}

% \cleardoublepage

%%%%%%%%%%%%%%%%%%%%%%%%%%%%%%%%%%%%%%%%%%%%%%%%%%%%%%%%%%%%%%%%%%%%%%%%%%%%%%%%%%%%%%%%%%%%%%%%%%%
%%%%%%%%%%%%%%%%%%%%%%%%%%%%%%%%%%%%%%%%%%%%%%%%%%%%%%%%%%%%%%%%%%%%%%%%%%%%%%%%%%%%%%%%%%%%%%%%%%%
%%%%%%%%%%%%%%%%%%%%%%%%%%%%%%%%%%%%%%%%%%%%%%%%%%%%%%%%%%%%%%%%%%%%%%%%%%%%%%%%%%%%%%%%%%%%%%%%%%%
% \chapter{Fragments pertinents issus des entretiens avec des lecteurs savants}\label{a:preint}
% \input{CONTENU/A3.PREINT.LEFTOVERS/0}

% \cleardoublepage

%%%%%%%%%%%%%%%%%%%%%%%%%%%%%%%%%%%%%%%%%%%%%%%%%%%%%%%%%%%%%%%%%%%%%%%%%%%%%%%%%%%%%%%%%%%%%%%%%%%
%%%%%%%%%%%%%%%%%%%%%%%%%%%%%%%%%%%%%%%%%%%%%%%%%%%%%%%%%%%%%%%%%%%%%%%%%%%%%%%%%%%%%%%%%%%%%%%%%%%
%%%%%%%%%%%%%%%%%%%%%%%%%%%%%%%%%%%%%%%%%%%%%%%%%%%%%%%%%%%%%%%%%%%%%%%%%%%%%%%%%%%%%%%%%%%%%%%%%%%
% \input{CONTENU/A4.SC01/QuestionnaireSC01} % → Les questions

% % → Quelques macros pour la structuration des questions
% \newcommand{\questionnaireSCUN}[1]{\section{#1}}
% \newcommand{\sectionquestionnaire}[1]{\subsection*{#1}}
% \newcommand{\question}[1]{\par\bigskip\textbf{#1}\par}
% \newcommand{\sousquestion}[1]{\par\medskip\textbf{\textcolor[rgb]{0.0,0.0,0.0}{--- \textbf{\textit{#1}}}}\par} % parcol = 0.73
% \newcommand{\reponse}[1]{\ignorespaces#1}
% \newcommand{\poimp}[1]{\bigskip\textsc{\textbf{#1}}\par}

% \chapter[Questionnaires de retour d'utilisation]{Questionnaires de retour d'utilisation\\{\scriptsize Expérience : «commentaire composé multimédia»}}\label{a:sc01}

% \ita{Note 1 : les réponses des étudiants sont rapportées sans intervention de notre part sur l'orthographe ou la syntaxe.}

% \ita{Note 2 : le prototype a été présenté aux étudiants sous le nom de «Verena».}

% \questionnaireSCUN{Étudiant \as}\input{CONTENU/A4.SC01/as}			% Alain SAAS
% \questionnaireSCUN{Étudiant \cl}\input{CONTENU/A4.SC01/cl}			% Cecile LABORDE
% \questionnaireSCUN{Étudiant \dao}\input{CONTENU/A4.SC01/dao}		% Djamila AIT OUADDA
% \questionnaireSCUN{Étudiant \ds}\input{CONTENU/A4.SC01/ds}			% Delphine SZYMCZAK
% \questionnaireSCUN{Étudiant \dd}\input{CONTENU/A4.SC01/dd}			% Dorine DUFOUR
% \questionnaireSCUN{Étudiant \glb}\input{CONTENU/A4.SC01/glb}		% Gabrielle LE BIHAN
% \questionnaireSCUN{Étudiant \rb}\input{CONTENU/A4.SC01/rb}			% Romain BODINIER

% \cleardoublepage

%%%%%%%%%%%%%%%%%%%%%%%%%%%%%%%%%%%%%%%%%%%%%%%%%%%%%%%%%%%%%%%%%%%%%%%%%%%%%%%%%%%%%%%%%%%%%%%%%%%
%%%%%%%%%%%%%%%%%%%%%%%%%%%%%%%%%%%%%%%%%%%%%%%%%%%%%%%%%%%%%%%%%%%%%%%%%%%%%%%%%%%%%%%%%%%%%%%%%%%
%%%%%%%%%%%%%%%%%%%%%%%%%%%%%%%%%%%%%%%%%%%%%%%%%%%%%%%%%%%%%%%%%%%%%%%%%%%%%%%%%%%%%%%%%%%%%%%%%%%
%%%%%%%%%%%%%%%%%%%%%%%%%%%%%%%%%%%%%%%%%%%%%%%%%%%%%%%%%%%%%%%%%%%%%%%%%%%%%%%%%%%%%%%%%%%%%%%%%%%
%%%%%%%%%%%%%%%%%%%%%%%%%%%%%%%%%%%%%%%%%%%%%%%%%%%%%%%%%%%%%%%%%%%%%%%%%%%%%%%%%%%%%%%%%%%%%%%%%%%
%%%%%%%%%%%%%%%%%%%%%%%%%%%%%%%%%%%%%%%%%%%%%%%%%%%%%%%%%%%%%%%%%%%%%%%%%%%%%%%%%%%%%%%%%%%%%%%%%%%
%%%%%%%%%%%%%%%%%%%%%%%%%%%%%%%%%%%%%%%%%%%%%%%%%%%%%%%%%%%%%%%%%%%%%%%%%%%%%%%%%%%%%%%%%%%%%%%%%%%
%%%%%%%%%%%%%%%%%%%%%%%%%%%%%%%%%%%%%%%%%%%%%%%%%%%%%%%%%%%%%%%%%%%%%%%%%%%%%%%%%%%%%%%%%%%%%%%%%%%
%%%%%%%%%%%%%%%%%%%%%%%%%%%%%%%%%%%%%%%%%%%%%%%%%%%%%%%%%%%%%%%%%%%%%%%%%%%%%%%%%%%%%%%%%%%%%%%%%%%
%%%%%%%%%%%%%%%%%%%%%%%%%%%%%%%%%%%%%%%%%%%%%%%%%%%%%%%%%%%%%%%%%%%%%%%%%%%%%%%%%%%%%%%%%%%%%%%%%%%
%%%%%%%%%%%%%%%%%%%%%%%%%%%%%%%%%%%%%%%%%%%%%%%%%%%%%%%%%%%%%%%%%%%%%%%%%%%%%%%%%%%%%%%%%%%%%%%%%%%

\backmatter

\rehead{\headingsf\scshape\headmark}
\lohead{\headingsf\scshape\headmark}

% \addcontentsline{toc}{chapter}{Bibliographie}
% \nocite{*}
\printbibliography[maxnames=11]

\cleardoublepage
\setcounter{tocdepth}{3}
\tableofcontents


% \cleardoublepage

%%%%%%%%%%%%%%%%%%%%%%%%%%%%%%%%%%%%%%%%%%%%%%%%%%%%%%%%%%%%%%%%%%%%%%%%%%%%%%%%%%%%%%%%%%%%%%%%%%%
%%%%%%%%%%%%%%%%%%%%%%%%%%%%%%%%%%%%%%%%%%%%%%%%%%%%%%%%%%%%%%%%%%%%%%%%%%%%%%%%%%%%%%%%%%%%%%%%%%%
%%%%%%%%%%%%%%%%%%%%%%%%%%%%%%%%%%%%%%%%%%%%%%%%%%%%%%%%%%%%%%%%%%%%%%%%%%%%%%%%%%%%%%%%%%%%%%%%%%%
% \input{CONTENU/A4.SC01/QuestionnaireSC01} % → Les questions

% % → Quelques macros pour la structuration des questions
% \newcommand{\questionnaireSCUN}[1]{\section{#1}}
% \newcommand{\sectionquestionnaire}[1]{\subsection*{#1}}
% \newcommand{\question}[1]{\par\bigskip\textbf{#1}\par}
% \newcommand{\sousquestion}[1]{\par\medskip\textbf{\textcolor[rgb]{0.0,0.0,0.0}{--- \textbf{\textit{#1}}}}\par} % parcol = 0.73
% \newcommand{\reponse}[1]{\ignorespaces#1}
% \newcommand{\poimp}[1]{\bigskip\textsc{\textbf{#1}}\par}

% \chapter[Questionnaires de retour d'utilisation]{Questionnaires de retour d'utilisation\\{\scriptsize Expérience : «commentaire composé multimédia»}}\label{a:sc01}

% \ita{Note 1 : les réponses des étudiants sont rapportées sans intervention de notre part sur l'orthographe ou la syntaxe.}

% \ita{Note 2 : le prototype a été présenté aux étudiants sous le nom de «Verena».}

% \questionnaireSCUN{Étudiant \as}\input{CONTENU/A4.SC01/as}			% Alain SAAS
% \questionnaireSCUN{Étudiant \cl}\input{CONTENU/A4.SC01/cl}			% Cecile LABORDE
% \questionnaireSCUN{Étudiant \dao}\input{CONTENU/A4.SC01/dao}		% Djamila AIT OUADDA
% \questionnaireSCUN{Étudiant \ds}\input{CONTENU/A4.SC01/ds}			% Delphine SZYMCZAK
% \questionnaireSCUN{Étudiant \dd}\input{CONTENU/A4.SC01/dd}			% Dorine DUFOUR
% \questionnaireSCUN{Étudiant \glb}\input{CONTENU/A4.SC01/glb}		% Gabrielle LE BIHAN
% \questionnaireSCUN{Étudiant \rb}\input{CONTENU/A4.SC01/rb}			% Romain BODINIER

% \cleardoublepage

%%%%%%%%%%%%%%%%%%%%%%%%%%%%%%%%%%%%%%%%%%%%%%%%%%%%%%%%%%%%%%%%%%%%%%%%%%%%%%%%%%%%%%%%%%%%%%%%%%%
%%%%%%%%%%%%%%%%%%%%%%%%%%%%%%%%%%%%%%%%%%%%%%%%%%%%%%%%%%%%%%%%%%%%%%%%%%%%%%%%%%%%%%%%%%%%%%%%%%%
%%%%%%%%%%%%%%%%%%%%%%%%%%%%%%%%%%%%%%%%%%%%%%%%%%%%%%%%%%%%%%%%%%%%%%%%%%%%%%%%%%%%%%%%%%%%%%%%%%%
%%%%%%%%%%%%%%%%%%%%%%%%%%%%%%%%%%%%%%%%%%%%%%%%%%%%%%%%%%%%%%%%%%%%%%%%%%%%%%%%%%%%%%%%%%%%%%%%%%%
%%%%%%%%%%%%%%%%%%%%%%%%%%%%%%%%%%%%%%%%%%%%%%%%%%%%%%%%%%%%%%%%%%%%%%%%%%%%%%%%%%%%%%%%%%%%%%%%%%%
%%%%%%%%%%%%%%%%%%%%%%%%%%%%%%%%%%%%%%%%%%%%%%%%%%%%%%%%%%%%%%%%%%%%%%%%%%%%%%%%%%%%%%%%%%%%%%%%%%%
%%%%%%%%%%%%%%%%%%%%%%%%%%%%%%%%%%%%%%%%%%%%%%%%%%%%%%%%%%%%%%%%%%%%%%%%%%%%%%%%%%%%%%%%%%%%%%%%%%%
%%%%%%%%%%%%%%%%%%%%%%%%%%%%%%%%%%%%%%%%%%%%%%%%%%%%%%%%%%%%%%%%%%%%%%%%%%%%%%%%%%%%%%%%%%%%%%%%%%%
%%%%%%%%%%%%%%%%%%%%%%%%%%%%%%%%%%%%%%%%%%%%%%%%%%%%%%%%%%%%%%%%%%%%%%%%%%%%%%%%%%%%%%%%%%%%%%%%%%%
%%%%%%%%%%%%%%%%%%%%%%%%%%%%%%%%%%%%%%%%%%%%%%%%%%%%%%%%%%%%%%%%%%%%%%%%%%%%%%%%%%%%%%%%%%%%%%%%%%%
%%%%%%%%%%%%%%%%%%%%%%%%%%%%%%%%%%%%%%%%%%%%%%%%%%%%%%%%%%%%%%%%%%%%%%%%%%%%%%%%%%%%%%%%%%%%%%%%%%%

\backmatter

\rehead{\headingsf\scshape\headmark}
\lohead{\headingsf\scshape\headmark}

% \addcontentsline{toc}{chapter}{Bibliographie}
% \nocite{*}
\printbibliography[maxnames=11]

\cleardoublepage
\setcounter{tocdepth}{3}
\tableofcontents


% \cleardoublepage

%%%%%%%%%%%%%%%%%%%%%%%%%%%%%%%%%%%%%%%%%%%%%%%%%%%%%%%%%%%%%%%%%%%%%%%%%%%%%%%%%%%%%%%%%%%%%%%%%%%
%%%%%%%%%%%%%%%%%%%%%%%%%%%%%%%%%%%%%%%%%%%%%%%%%%%%%%%%%%%%%%%%%%%%%%%%%%%%%%%%%%%%%%%%%%%%%%%%%%%
%%%%%%%%%%%%%%%%%%%%%%%%%%%%%%%%%%%%%%%%%%%%%%%%%%%%%%%%%%%%%%%%%%%%%%%%%%%%%%%%%%%%%%%%%%%%%%%%%%%
% \input{CONTENU/A4.SC01/QuestionnaireSC01} % → Les questions

% % → Quelques macros pour la structuration des questions
% \newcommand{\questionnaireSCUN}[1]{\section{#1}}
% \newcommand{\sectionquestionnaire}[1]{\subsection*{#1}}
% \newcommand{\question}[1]{\par\bigskip\textbf{#1}\par}
% \newcommand{\sousquestion}[1]{\par\medskip\textbf{\textcolor[rgb]{0.0,0.0,0.0}{--- \textbf{\textit{#1}}}}\par} % parcol = 0.73
% \newcommand{\reponse}[1]{\ignorespaces#1}
% \newcommand{\poimp}[1]{\bigskip\textsc{\textbf{#1}}\par}

% \chapter[Questionnaires de retour d'utilisation]{Questionnaires de retour d'utilisation\\{\scriptsize Expérience : «commentaire composé multimédia»}}\label{a:sc01}

% \ita{Note 1 : les réponses des étudiants sont rapportées sans intervention de notre part sur l'orthographe ou la syntaxe.}

% \ita{Note 2 : le prototype a été présenté aux étudiants sous le nom de «Verena».}

% \questionnaireSCUN{Étudiant \as}\input{CONTENU/A4.SC01/as}			% Alain SAAS
% \questionnaireSCUN{Étudiant \cl}\input{CONTENU/A4.SC01/cl}			% Cecile LABORDE
% \questionnaireSCUN{Étudiant \dao}\input{CONTENU/A4.SC01/dao}		% Djamila AIT OUADDA
% \questionnaireSCUN{Étudiant \ds}\input{CONTENU/A4.SC01/ds}			% Delphine SZYMCZAK
% \questionnaireSCUN{Étudiant \dd}\input{CONTENU/A4.SC01/dd}			% Dorine DUFOUR
% \questionnaireSCUN{Étudiant \glb}\input{CONTENU/A4.SC01/glb}		% Gabrielle LE BIHAN
% \questionnaireSCUN{Étudiant \rb}\input{CONTENU/A4.SC01/rb}			% Romain BODINIER

% \cleardoublepage

%%%%%%%%%%%%%%%%%%%%%%%%%%%%%%%%%%%%%%%%%%%%%%%%%%%%%%%%%%%%%%%%%%%%%%%%%%%%%%%%%%%%%%%%%%%%%%%%%%%
%%%%%%%%%%%%%%%%%%%%%%%%%%%%%%%%%%%%%%%%%%%%%%%%%%%%%%%%%%%%%%%%%%%%%%%%%%%%%%%%%%%%%%%%%%%%%%%%%%%
%%%%%%%%%%%%%%%%%%%%%%%%%%%%%%%%%%%%%%%%%%%%%%%%%%%%%%%%%%%%%%%%%%%%%%%%%%%%%%%%%%%%%%%%%%%%%%%%%%%
%%%%%%%%%%%%%%%%%%%%%%%%%%%%%%%%%%%%%%%%%%%%%%%%%%%%%%%%%%%%%%%%%%%%%%%%%%%%%%%%%%%%%%%%%%%%%%%%%%%
%%%%%%%%%%%%%%%%%%%%%%%%%%%%%%%%%%%%%%%%%%%%%%%%%%%%%%%%%%%%%%%%%%%%%%%%%%%%%%%%%%%%%%%%%%%%%%%%%%%
%%%%%%%%%%%%%%%%%%%%%%%%%%%%%%%%%%%%%%%%%%%%%%%%%%%%%%%%%%%%%%%%%%%%%%%%%%%%%%%%%%%%%%%%%%%%%%%%%%%
%%%%%%%%%%%%%%%%%%%%%%%%%%%%%%%%%%%%%%%%%%%%%%%%%%%%%%%%%%%%%%%%%%%%%%%%%%%%%%%%%%%%%%%%%%%%%%%%%%%
%%%%%%%%%%%%%%%%%%%%%%%%%%%%%%%%%%%%%%%%%%%%%%%%%%%%%%%%%%%%%%%%%%%%%%%%%%%%%%%%%%%%%%%%%%%%%%%%%%%
%%%%%%%%%%%%%%%%%%%%%%%%%%%%%%%%%%%%%%%%%%%%%%%%%%%%%%%%%%%%%%%%%%%%%%%%%%%%%%%%%%%%%%%%%%%%%%%%%%%
%%%%%%%%%%%%%%%%%%%%%%%%%%%%%%%%%%%%%%%%%%%%%%%%%%%%%%%%%%%%%%%%%%%%%%%%%%%%%%%%%%%%%%%%%%%%%%%%%%%
%%%%%%%%%%%%%%%%%%%%%%%%%%%%%%%%%%%%%%%%%%%%%%%%%%%%%%%%%%%%%%%%%%%%%%%%%%%%%%%%%%%%%%%%%%%%%%%%%%%

\backmatter

\rehead{\headingsf\scshape\headmark}
\lohead{\headingsf\scshape\headmark}

% \addcontentsline{toc}{chapter}{Bibliographie}
% \nocite{*}
\printbibliography[maxnames=11]

\cleardoublepage
\setcounter{tocdepth}{3}
\tableofcontents


\end{document}
