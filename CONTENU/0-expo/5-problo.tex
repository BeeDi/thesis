
%%%%%%%%%%%%%%%%%%%%%%%%%%%%%%%%%%%%%%%%%%%%%%%%%%%%%%%%%%%%%%%%%%%%%%%%%%%%%%%%%%%%%%%%%%%%%%%%%%%
% \newpage
\section{Problèmes}\label{sec:prob}



%%%%%%%%%%%%%%%%%%%%%%%%%%%%%%%%%%%%%%%%%%%%%%%%%%%%%%%%%%%%%%%%%%%%%%%%%%%%%%%%%%%%%%%%%%%%%%%%%%%
\subsection{Problèmes métiers}\label{sec:pmetiers}
% Nos champs d'applications sont : 
% Prenant acte des besoins de la production audiovisuelle nous distinguons deux problèmes métiers (1) identifier le(s) niveau(x) de fragmentation et (2) le(s) type(s) de description susceptibles de favoriser la fabrication mixte, la circulation et la réutilisation des contenus.
% Ces problèmes remettent en cause à la fois la représentation classique des contenus et leur description. 
% de manière à faciliter (1) la production mixte amateur-professionnel et (2) leur réutilisation dans de nouveaux contextes d'exploitation.
% l'informatisation du début de la chaîne de production, préproduction 
% l'inscription formelle de l'écriture audiovisuelle ?

\e{
L'objectif central qui se pose à la production audiovisuelle est de constituer des objets audiovisuels autonomes et donc (ré)utilisable à n'importe quel étape de la chaîne. 
Or, dans la chaîne de production classique les programmes n'émergent qu'à la fin de la chaîne et sont gérés d'une pièce. 
Il n'y a pas forcément de place pour les éléments de contenu intermédiaires, et c'est justement à la modélisation de ces fragments que l'on s'attaque. 
Ces fragments doivent devenir des éléments documentaires qui possèdent leur unité propre de même que les objets finis que sont les programmes. 
Une des difficultés réside dans cette articulation entre des fragments et le tout ou l'ensemble que constitue les programmes. 
Par ailleurs, il existe des problèmes sous-jacents à cette fragmentation documentaire : 
}
\begin{liste}
	\item \e{identifier quels niveaux de fragments peuvent prétendre à ce genre de transformation}.
	Si le numérique permet de fragmenter à l'envie, il faut cependant prendre en compte les pratiques du métier pour identifier les niveaux de fragmentation pertinents ou déjà utilisés dans le métier mais non modélisés.
	Par exemple, la prise de vue est le résultat d'une activité de tournage, pour autant elle ne constitue pas un objet éditorial comme peut l'être une interview.
	De plus, il s'agit également de déterminer comment manipuler ces fragments comme des objets à part entière sans compromettre l'articulation de l'ensemble. 
	Par exemple, une interview peut s'intégrer dans un journal télévisé ou bien un reportage dans des versions plus ou moins courtes. 
	Pour autant, il s'agit du fruit d'une même activité, simplement le montage, et donc le résultat, est différent suivant le programme dans lequel l'interview s'insère.\\
	
	\item \e{identifier quelles informations et connaissances doivent être rattachées à ces fragments pour les rendre autonome}.
	D'une manière similaire à la démarche pour les objets entiers, certaines connaissances, notamment relatives au contexte de production, doivent être attachées au fragment pour garantir sa réutilisation et sa cohérence. 
	En reprenant l'exemple de la prise de vue, on peut lui associer le bout de script qui a prescrit ce qu'elle devait montrer. 
	Si l'on pousse encore cette logique, il faut également incorporer le document qui définit le programme pour lequel on a tourné cette prise de vue, la personne qui l'a effectuée, les équipements utilisés etc. 
	Ainsi, on étend la modélisation de l'objet audiovisuel à son contexte de production et tout ce qui renforce les possibilités de recherche, de manipulation, de gestion et de transformation de ces objets. 
	De plus, s'ajoute à cela la question de la collecte de ces informations. 
	En effet, la saisie de ces informations au sein d'un système d'information et leur utilisation par les acteurs de la chaîne n'est pas une simple formalité.
	Ce problème pousse également dans le sens d'une modélisation plus contextuelle, de manière à proposer un environnement de travail adapté et utilisable aux contributeurs de la chaîne.\\
	
	% \item \e{}.
\end{liste}


% Notre problème se situe dans le croisement de la représentation des contenus et la représentation des activités humaines qui construisent, manipulent, éditent, transforment, publient et documentent ces contenus. 
% Cet angle de recherche nous amène donc à considérer non pas le contenu audiovisuel dans son ensemble, une fois terminé et validé, mais la construction de tout ces fragments qui le composent. 
% Il nous faut aussi considérer, non pas seulement le contenu audiovisuel, mais aussi d'autres informations, d'autres documents qui constituent son contexte de production, dans un sens très général. 
% Chaque prise de vue constitue donc un objet à représenter en tant que tel, tout autant que le bout de script qui a prescrit ce que cette prise de vue devait montrer. 
% Si l'on pousse encore cette logique, on peut alors représenter de même le document qui définit le programme pour lequel on a tourné cette prise de vue, la personne qui l'a effectué, les équipements utilisés etc. 
% Ainsi, on étend la représentation du contenu à son contexte de production entendu comme toutes les informations qui renforceront les possibilités de recherche, de manipulation, de gestion et de transformation de ces contenus. 

%%%%%%%%%%%%%%%%%%%%%%%%%%%%%%%%%%%%%%%%%%%%%%%
\subsection{Problèmes scientifiques}\label{sec:scien}
Au fur et à mesure que la circulation des contenus s'intensifie, il y a un besoin grandissant de faciliter l'échange d'information tant à la fois sur le plan informatique, que sur le plan humain. 
De plus, l'ouverture de la chaîne de production à de nouveaux contributeurs (amateurs et professionnels) ne fait qu'accentuer la disparité des connaissances et des systèmes utilisés. 
Afin de construire une compréhension commune à tous les contributeurs au cycle de vie, on fabrique un modèle conceptuel capable d'intégrer et de mettre en relation leurs connaissances. 
Il s'agit là d'un apport par rapport à la situation existante où le consensus n'existait pas, ou alors de manière éphémère, locale au sein d'une équipe.
L'objectif est de fluidifier les échanges d'information et de contenus en formalisant les connaissances utilisées pour :
\begin{liste}
% modéliser tous les objets de la chaîne de production audiovisuelle
	\item[(A)] \g{modéliser les objets construits au fil de la chaîne de production audiovisuelle}.
	\item[(B)] \g{modéliser les connaissances sur ces objets} (descriptions, contexte de production, contribution au cycle de vie).
	% \item[(B)] décrire les objets audiovisuels
	% \item[(C)] représenter la contribution de chacun des acteurs au cycle de vie des objets audiovisuels
\end{liste}

\e{
Ainsi, notre problème de recherche général s'articule autour de la modélisation des connaissances et des informations que les contributeurs construisent, utilisent, échangent au cours du cycle de vie des objets audiovisuels. 
Cette modélisation constitue une première étape dans la mise en place d'un système d'information servant à mieux gérer les objets audiovisuels et médier la communication entre systèmes informatiques tout autant qu'entre contributeurs humains.
Après avoir numérisé les contenus audiovisuels, on souhaite transformer chaque élément les composant en objet documentaire et documenter leur cycle de vie.\\}



\g{(A)} Le problème est d'organiser la gestion des objets audiovisuels en proposant une modélisation capable de faciliter leur identification, leur manipulation et leur réutilisation tout au long de leur cycle de vie. 
En particulier,	l'objet audiovisuel professionnel est produit de manière collective, chaque contributeur apportant un élément à l'ensemble. 
Ces contributions doivent donc pouvoir être identifiées comme appartenant à un ensemble, de même que chaque élément doit pouvoir être considéré pour soi afin d'être intégré dans un autre ensemble (réutilisation).

Pour cela on adopte une représentation des différents niveaux d'abstraction des objets audiovisuels numériques de façon à rétablir les liens entre les différentes versions ou copies d'un même contenu, quelque soit la nature des variations entre elles (encodage, format d'encapsulation, montage, finition, langue etc.).
La distinction entre différents niveaux de modélisation (technique, esthétique, éditorial etc.) doit permettre de construire une représentation dynamique de l'objet audiovisuel qui suit l'avancement du processus de production.\\
% La distinction entre différents niveaux de modélisation (technique, esthétique, éditorial etc.) doit permettre de construire une représentation de l'objet audiovisuel au fur et à mesure de l'avancement du processus de production.\\


\g{(B)} Le problème est d'attacher plusieurs types de connaissances aux objets audiovisuels de manière à les rendre autonomes dans leur circulation et leur réutilisation. 
Afin de faciliter l'échange d'information et la réutilisation des objets audiovisuels entre différents contextes, il faut modéliser des connaissances sur ces objets qui sont parfois déjà existantes mais non formalisées, ou bien qu'il faut rendre compréhensibles.
En effet, l'échange d'informations dans la production audiovisuelle est primordial et s'effectue entre métiers ou organisations différent(e)s, voire avec des amateurs. 
En particulier, on souhaite s'appuyer sur le vocabulaire de l'écriture filmique utilisé dans des documents de préproduction pour spécifier les résultats attendus de la production.
L'information contenue dans ces documents est importante mais repose sur des conventions plus ou moins tacites qu'il faut expliciter pour les professionnels, expliquer pour les amateurs.
La formalisation de ces éléments devra donc pouvoir être lu et modifié tout au long du cycle de vie des objets audiovisuels par tout types de contributeurs.

% échange d'info, adaptation par l'explicitation du vocabulaire et la contribution au cycle de vie
La formalisation de l'écriture filmique permettra d'adapter la présentation de l'information en fonction des connaissances, de l'implication du contributeur dans la chaîne (rôle, tâche, niveau de compétences etc.), de son référentiel professionnel ou linguistique. 
Il s'agit alors d'établir des correspondances entre les connaissances connues par le lecteur d'une information et celles utilisées par la personne qui l'a exprimée.
Ainsi, d'une part on explicite l'expression de l'information, ce qui permet l'adpatation, et facilite son interprétation ultérieure.
Le processus prend tout son sens lorsqu'il s'agit de traduire un concept de la réalisation audiovisuelle pour guider un amateur dans son tournage.
% Par exemple, l'action écrite par l'auteur et transformé en scène par le réalisateur, doit ensuite être tourné un caméraman, des acteurs etc. 


% réutilisation, attachement des connaissances pertinentes pour manipuler ou exploiter chaque fragment ou l'objet en entier
De plus, les descriptions utilisées, en plus de permettre de spécifier le résultat attendu de la production, doivent permettre de faciliter la recherche, la manipulation et la réutilisation des objets audiovisuels.
Pour cela, il faut articuler ces connaissances aux objets et aux fragments qui les composent. 
Le vocabulaire de l'écriture filmique sera également précieux, puisqu'il nous donne une unité de base, le plan, ainsi que ses caractéristiques qui permettent de le distinguer des autres. 
La recherche dans des dépôts de contenus se fera ainsi de manière similaire à la commande de contenu à d'autres contributeurs, qu'ils soient professionnels ou amateurs.



% Il s'agit donc de définir un modèle de description susceptible d'être utilisé par les contributeurs professionnels ou amateurs, à toutes les étapes de la chaîne de production. 


% Afin de décrire les contenus, on souhaite s'appuyer sur le vocabulaire de l'écriture audiovisuelle utilisé dans les différentes étapes de la chaîne et notamment dès la préproduction. 
% Cette écriture repose sur un vocabulaire des techniques de réalisation audiovisuelle (prise de vue, transition, composition de l'image etc.) renvoyant à des effets largement connus dans le milieu de l'audiovisuel et chez les cinéphiles. 
% L'écriture est utilisée avant la fabrication du contenu pour la spécifier, puis pendant la fabrication pour enregistrer les différences. 
% Une formalisation de ce vocabulaire permettrait de construire une description textuelle d'un contenu à partir d'une description objective de la réalisation (réglages des appareils, position des acteurs etc.).\\



% \g{(C)} Le problème est de représenter et faciliter l'échange d'information entre des contributeurs hétérogènes dans leurs connaissances et leur implication dans la chaîne de production.
% Une première diffculté réside dans l'articulation entre les connaissances des contributeurs qui expriment l'information et ceux qui l'interprèteront.
% La seconde diffculté consiste dans l'articulation des représentations du cycle de vie, de l'objet audiovisuel et des descriptions qui leurs sont associées. 
% En effet, chaque contributeur peut participer à la constitution d'informations associées au contenu en cours de sa production.
% Ces informations varient en fonction de l'implication du contributeur dans la chaîne (rôle, tâche, niveau de compétences etc.). 
% De plus, les informations construites à un moment sont susceptibles d'être utilisées plus tard dans la chaîne, par un contributeur ne partageant pas forcément les mêmes connaissances ou le même référentiel professionnel ou linguistique.

% On cherche alors à réaliser une adaptation de la forme d'expression de ces informations afin de faciliter le déroulement du processus de production. 
% Dans un premier temps, on explicite les connaissances utilisées par un premier utilisateur pour exprimer une information. 
% Ensuite, on établit une correspondance avec les connaissances connues d'un autre utilisateur et on adapte au besoin la forme de d'expression de cette information pour faciliter son interprétation. 
% L'adpatation qui en résulte prend tout son sens lorsqu'il s'agit de traduire un concept de la réalisation audiovisuelle pour guider un amateur dans son tournage.


%%%%%%%%%%%%%%%%%%%%%%%%%%%%%%%%%%%%%%%%%%%%%%%
% \section{Positionnement Disciplinaire (n,i)}\label{sec:posd}
% [Ingénierie des connaissances ; Media Asset Management ; Gestion électronique de Documents ; Ingénierie documentaire]
% ingénierie des connaissances (représentation des connaissances) ingénierie des inscriptions numériques de connaissances, dont les documents
% ingénierie documentaire (modélisation des documents propres à la production audiovisuelle)
% indexation et gestion des connaissances (description des contenus audiovisuels)
% la ged s'occupe de la gestion de documents, nous proposons de gérer des fragments de documents, de gérer leur construction en plusieurs étapes, par plusieurs acteurs et dans le cadre de différentes missions.

% Pour définir cet ensemble d'informations qui forment le contexte de production, nous nous sommes appuyés sur les partenaires du projet MediaMap. 



% des réécritures de documents et des mises à jours qui pourraient être réalisées par des machines, des informations qui pourraient être transmises automatiquement à travers un réseau numérique d'information
% un vocabulaire bien défini qui fait l'objet de nombreux dictionnaires, donc prêt à être formalisé
% des équipes qui sont réduites lors de tournage en extérieur

% à mettre dans le pos. disciplinaire, comment on aborde les problèmes posées
% [Nous avons ainsi dégagé plusieurs perspectives métiers qui nous ont servi de guide pour identifier les échanges d'informations les plus importants ainsi que le vocabulaire utilisé pour les exprimer. 
% Chacune de ces perspective possède un objectif propre et des spécificités, cependant il apparaît qu'un langage commun est utilisé par tous les acteurs de la production. 
% En se concentrant sur la description d'un contenu existant ou à venir, ce langage permet à ces acteurs de communiquer entre eux. Le réalisateur qui spécifie un attendu dans son script, les caméraman qui réalisent le cadrage, les opérateurs lumières etc. 
% Tous utilisent ce langage pour imaginer le résultat à produire et en déduire les gestes à opérer. 
% Les usages n'étant jamais complètement figé, chaque organisation développe ses propres idiomatismes de langage. 
% Dans ce cas, la collaboration entre organisations impliquent de pouvoir réaliser des ajustements dans l'expression de la description du contenu. 
% De même, la collaboration avec des contributeurs amateurs soulève un problème de compréhension de ce langage (et donc de l'attendu) mais aussi de connaissances des gestes à opérer (pour produire le résultat attendu).
% Ainsi, à mesure que la circulation des contenus s'intensifie, que les besoins de collaboration augmentent, naît un besoin grandissant d'explicitation des échanges d'information afin de dégager une vue d'ensemble de la chaîne de production, de ses acteurs, de leurs interactions, de leurs produits. 

% Notre proposition consiste à modéliser ces éléments et à en informatiser l'accès de manière à fluidifier les échanges de contenus et faciliter la compréhension des informations afférentes.] 
