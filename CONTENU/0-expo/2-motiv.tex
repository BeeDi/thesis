

%%%%%%%%%%%%%%%%%%%%%%%%%%%%%%%%%%%%%%%%%%%%%%%
\subsection{Le développement de l'électronique et des réseaux de télécommunications}\label{sec:electro}

L’explosion des appareils multimédia et des possibilités de transférer des contenus par les réseaux de télécommunication a promu de nouvelles pratiques de consommation et d'échanges des contenus tant chez les professionnels de l'audiovisuel que dans le grand public.

Du côté des professionnels, on voit ainsi l’émergence de systèmes de production qui utilisent le réseau pour faire transiter les contenus entre les systèmes d'information de leurs différents départements. 
Ces systèmes reprennent les principes d'architecture multi-tiers utilisés sur le Web avec les contenus représentés par des fichiers ou des flux binaires, d'où leur appelation de \e{file-based production system}.

Ce genre de système favorise également l’échange de fichiers entre organisations, puisque l’architecture permet d'exposer les données stockées de la même manière sur un réseau interne (intranet) ou externe (extranet, Internet).

Du côté du grand public, les appareils portables acquièrent de plus en plus de connectivité avec leur environnement et les réseaux. 
De simples lecteurs à brancher en USB, les appareils sont passés au stade communicant avec la 3G, le Wifi ou le Bluetooth. 
La fonction d'échange se banalise et inversement, les appareils communicant comme les téléphones deviennent eux-même des stations multimédia à part entière, musique, photo, vidéo, courriel, sms etc.\\


Un autre facteur à noter, est que ces appareils portables sont personnels, c’est-à-dire qu’ils sont majoritairement utilisés par un seul individu contrairement à l’usage du téléviseur qui était et reste encore largement un objet collectif. 
Une autre distinction fondamentale se situe dans le fait que ce sont des appareils informatiques qui fonctionnent entièrement dans le numérique. 
Les possibilités d’interaction en font non plus de simples terminaux de lecture, mais potentiellement de véritables instruments de création et de communication. En effet, l'insertion de données et la capture de contenus (photos, sons, textos etc.) est prévu et le couplage avec des plates-formes de publication (réseaux sociaux, dépôts de contenus, CMS etc.) est de mieux en mieux réalisé.

De ce fait, en plus des capacités toujours plus importante de consommation de contenus, ces appareils favorisent eux aussi leur circulation ou la circulation d’information annexes. 
Le partage d’opinions s’est considérablement développé avec la vague d'applications Web dites sociales qui permettaient aux utilisateurs de créer du contenu sans avoir à maîtriser les arcanes techniques du Web. 
Cet ajout d’opinion, même s’il est parfois réduit au minimum à une marque d'appréciation, implique ainsi l’utilisateur dans le processus de diffusion du contenu en le portant à l'attention d'autres utilisateurs (ses contacts dans le cas des réseaux sociaux, ses lecteurs dans le cas d'un weblog ou autres CMS, un ensemble d'utilisateurs anonymes dans le cas de services sociaux tels que Delicious\footnote{Delicious : \url{http://delicious.com/} est une plate-forme de sauvegarde, d'indexation par mot-clé et de partage de marques-pages. Pour l'utilisateur il s'agit soit de sauvegarder et d'indexer ses marques-pages, soit de découvrir les marques-pages correspondant à tel ou tel mot(s)-clé(s) déjà sauvegardées par la communauté. Les grandes tendances d'indexation sont ainsi accessibles à tous, tout en permettant à chacun de développer son système d'indexation personnelle -- adaptation française du mot anglais \e{folksonomy}. Il est également possible de partager directement ses trouvailles avec d'autres utilisateurs par un système d'abonnement et de notification.} ou Digg \footnote{Digg : \url{http://digg.com/} est un site de marque-page social qui fonctione sur le principe du vote. Un utilisateur peut proposer une page qui est alors soumise aux votes des autres utilisateurs. Suivant le succès de la page, celle-ci sera mise en avant sur la page principale de Digg, ou bien mise de côté avec le reste des pages moins populaires, et finira par être supprimée.}).\\


Les appareils numériques multimédia possèdent également des capteurs de plus en plus performant et de moins en moins coûteux qui permettent au grand public de découvrir de nouvelles activités de création (photographie et retouche d’image, tournage et montage vidéo, prise de son et mixage audio etc.).
Cet abaissement du coût d’entrée dans la production a favorisé l'émergence d’une production amateur hétéroclite qui va du passant prenant une photo d’un évènement se déroulant devant ses yeux jusqu’à l’amateur qui pratique par amour mais avec l'exigence d’un professionnel. 
Cette production amateur rentre alors en concurrence avec la production professionnelle, voire la remplace dans certains cas (lorsqu’un passant est le seul témoin d’un évènement inattendu par exemple). 
La concurrence est d'autant plus forte depuis l’apparition de plates-formes de partages qui facilitent la distribution des contenus. 
De manière générale, l’opposition classique entre producteurs et consommateurs se brouille et les professionnels cherchent de plus en plus à mettre à contribution les amateurs dans leurs processus. 
On se dirige ainsi vers un modèle où professionnels et amateurs contribuent à divers degrés et divers moments au cycle de vie des contenus.\\


Avec l'informatisation des appareils multimédia s’est introduit la possibilité de personnaliser la communication avec l'utilisateur, d'y ajouter de l'interactivité et de connecter les utilisateurs entre eux. 
Ces nouvelles possibilités transforment les attendus et les pratiques du grand public. 
De ce fait, cela impacte les rapports avec les professionnels qui tendent à vouloir intégrer les contributions externes à leurs propres productions. 
Ainsi, il ne s’agit plus simplement de produire des contenus qui s'adressent indistinctement aux masses, mais de trouver des moyens de personnaliser son offre, de recommander des contenus, de faciliter la récupération de contenus, de diversifier les occasions de consommer ou de contribuer.








%%%%%%%%%%%%%%%%%%%%%%%%%%%%%%%%%%%%%%%%%%%%%%%
\section{Le projet MediaMap}\label{sec:mm}
% %%%%%%%%%%%%%%%%%%%%%%%%%
\paragraph{Objectifs}
Le projet MediaMap vise à développer des modèles et des applications pour promouvoir la production audiovisuelle collaborative articulant contenus professionnels et amateurs. 
En particulier, l'ambition est d'intégrer les amateurs et leurs contenus dans la chaîne de production professionnelle en améliorant d'une part la qualité technique et éditoriale des contenus fabriqués, et d'autre part en facilitant la collaboration et l'intercompréhension entre les différents acteurs de la chaîne.

La piste de travail retenue a été de construire des ontologies capables de représenter et décrire les contenus au fur et à mesure de leur processus de production. 
Ces informations serviraient de base de connaissances pour de nouvelles applications s'intégrant dès le début à la chaîne de production audiovisuelle, c'est-à-dire dès la conception du contenu. 
Les partenaires du projet ont ainsi développé des applications d'organisation du processus de production, de description du contenu, d'assistance au tournage ainsi qu'un moteur de recherche utilisant ces ontologies comme modèle d'information de référence.

%%%%%%%%%%%%%%%%%%%%%%%%%
\paragraph{Composition}
Le projet MediaMap a rassemblé une dizaine d'entreprises ainsi que deux équipes de recherche de l'Université de Technologie de Compiègne :
\begin{liste}
	\item l'équipe de recherche \pc{Information Connaissance Interaction} (ICI) chargée de la partie modélisation qui a abouti à la construction d'ontologies.

	\item l'équipe de recherche \pc{Automatique, Systèmes Embarqués, Robotique} (ASER) chargée de la conception d 'algorithmes d'analyse d'images et de vidéos.
\end{liste}


Parmi les entreprises du consortium, on compte les deux grandes chaînes de télévision publiques belges ainsi que de nombreuses PME belge ou française qui apportent leurs expertises dans différents domaines :
\begin{liste}
	\item \pc{BelgaVox} qui gère un des plus grands stocks d'archives audiovisuelles belges et produit des documentaires.

	\item \pc{Exalead} qui est un éditeur de solution de recherche pour les entreprises.

	\item \pc{Kane Consulting} qui propose des analyses du marché et des usages aux acteurs de la production audiovisuelle.

	\item \pc{Memnon} qui est spécialisé dans la numérisation, la documentation et l'archivage de contenus audio et vidéo.

	\item \pc{Perfect Memory} qui s'est créé pendant le projet afin d'accompagner les solutions du projet sur le marché grand public et prospecte également le marché professionnel.

	\item \pc{Skema} qui développe des applications de production de contenu audiovisuel amateur pour mobiles et caméras.

	\item \pc{Solution 2.0} qui est une agence de conception et de réalisation de plate-forme Web.

	\item la \pc{Radio-Télévision Belge de la communauté Française} (RTBF).

	\item \pc{Vitec Multimédia} qui développe et manufacture du matériel vidéo numérique.

	\item la \pc{Vlaamse Radio- en Televisieomroep} (VRT).
\end{liste}









%%%%%%%%%%%%%%%%%%%%%%%%%%%%%%%%%%%%%%%%%%%%%%%
\section*{Organisation du mémoire}\label{sec:plan}
\addcontentsline{toc}{section}{Organisation du mémoire}

\paragraph{Exposition}
\e{
La première partie de ce mémoire a pour objectif de présenter les problèmes qui se posent à la production audiovisuelle depuis son avancée vers la numérisation. 
Elle nous permet également de préciser la manière dont nous posons le problème, à la fois en terme métiers et avec nos lunettes de scientifiques.}

Le chapitre \g{\ref{chap:intro}. Introduction} nous sert à rappeler le contexte technologique général qui s'impose au monde de l'audiovisuel. 
En effet, la production audiovisuelle se dirige progressivement vers une numérisation et une mise en réseau de ses produits ainsi qu'une informatisation de ses pratiques qui n'est pas sans conséquences. 

Le chapitre \g{\ref{chap:problo}. Problématisation} nous permet de préciser comment ces tendances impactent le monde de l'audiovisuel.
Nous nous appuyerons sur l'étude du fonctionnement de la chaîne de production audiovisuelle classique (\ref{sec:prod}) pour dresser un bilan des attentes des professionnels vis-à-vis du numérique (\ref{sec:besoins}).
% En particulier, 
Nous précisons alors la manière dont nous posons le problème sur le plan métier, ce qui nous amènera à définir le problème scientifique. 
Sur le plan métier, le défi posé par le numérique consiste à passer d'une vision à l'échelle du document à une vision à l'échelle du fragment. 
En effet, le numérique favorise la fragmentation et la circulation des contenus qu'il s'agit alors de rendre autonome pour en permettre l'exploitation (\ref{sec:pmetiers}). 
Sur le plan scientifique, nous posons le problème en terme de modélisation des objets audiovisuels et des connaissances associées afin de construire une compréhension commune et dynamique pour tous les acteurs impliqués dans la production audiovisuelle (\ref{sec:scien}).
% Nous concluons en expliquant comment nous mobilisons diverses disciplines scientifiques pour construire une réponse aux problèmes posés (\ref{sec:posd}).
% proposent d'une part des outils et des méthodes de modélisation, et d'autre part proposent des modélisations de l'audiovisuel (\ref{sec:posd}).


\paragraph{État de l'art}
\e{
La deuxième partie de ce mémoire vise à étudier des outils, méthodes et langages de modélisation. Elle permet également d'étudier les modélisations existantes des objets audiovisuels, mais égalements des connaissances métiers qui y sont associées afin de faciliter leur exploitation.}

Le chapitre \g{\ref{chap:omod}. Outils de modélisation} commence par clarifier les besoins de modélisations à partir d'un scénario de production collaborative impliquant des acteurs professionnels et amateurs (\ref{sec:cdcf}).
Ces besoins ne se situent pas seulement au niveau de la modélisation conceptuelle, mais également sur le plan des jargons utilisées pour présenter ces concepts à des contributeurs de la production.
Ainsi, nous examinons les définitions des concepts de \gui{systèmes d'organisation de connaissances} (SOC) et en particulier les relations qu'entretiennent \gui{terminologie} et d'\gui{ontologie} (\ref{sec:defs}).
Nous étudions ensuite les langages de structuration et de représentation des connaissances qui permettent de modéliser ces deux types de SOC (\ref{sec:mods}).

Le chapitre \g{\ref{chap:mav}. Modélisations de l'audiovisuel} a pour objectif de mettre en rapport les représentations des professionnels de la production avec diverses communautés scientifiques, en vue de clarifier la définition d'un objet audiovisuel (\ref{sec:dav}) et de sa réutilisation (\ref{sec:gest}).
Cette étude s'appuie sur la poursuite du scénario d'usage du chapitre précédent, et met en exergue la nécessité de fragmenter la modélisation des objets audiovisuels pour favoriser leur réutilisation (\ref{sec:cdc-av}).
Nous étudions ensuite les solutions utilisées dans l'industrie pour gérer la circulation des programmes (\ref{sec:wrapper}) et les décrire (\ref{sec:desc}).
Cette dernière partie analyse les méthodes de fabrication de ces description ainsi que leur nature, puis les modélisations développées.
Une attention particulière est portée à de MPEG-7 (\ref{sec:mpeg7}), qui tient lieu de référence à de nombreux travaux de formalisation sous la forme d'ontologie (\ref{sec:mpeg7etc}).
Nous présentons également des approches de description plus proche de la perspective de la production audiovisuelle (\ref{sec:plan}).


\paragraph{Contribution}
\e{La troisième partie de ce mémoire présente notre contribution conceptuelle et informatique aux problèmes de modélisations que nous avons soulevées.
Nous détaillons nos choix de représentation pour opérationnaliser notre contribution en une ontologie informatique.}

Le chapitre \g{\ref{chap:mod}. Approche et modélisation} revient sur les langages et les modélisations étudiés dans le chapitre précédent et introduit les principes de notre approche (\ref{sec:principes}).
Notre positionnement au sein de la chaîne de production audiovisuelle, nous permet de modéliser le déroulement de la chaîne, la définition d'une structure documentaire première, puis les fragments audiovisuels construits ainsi que les connaissances qui s'y rapportent.
Nous détaillons la modélisation conceptuelle en parties, chacune correspondant à un besoin fonctionnel, en expliquant leur mise en relation par des exemples (\ref{sec:concept}).

Le chapitre \g{\ref{chap:op}. Mise en oeuvre} présente la représentation informatique de notre conceptualisation.
Nous argumentons d'abord nos choix de langage et montrons comment nous les utilisons (\ref{sec:ln}). 
Nous détaillons ensuite la structuration de notre ontologie et son articulation avec des thésaurus et des bases de faits (\ref{sec:op}).

\paragraph{Discussion}
\e{La dernière partie de ce mémoire montre comment notre contribution est utilisé dans le cadre du projet MediaMap et ouvre la discussion sur ce travail de thèse.}

Le chapitre \g{\ref{chap:app}. Applications et expérimentations} introduit les diverses applications qui ont été développé (\ref{sec:app}) par nos partenaires et les expérimentations qu'elles ont permis de mener (\ref{sec:xp}).
Il s'agit d'éclairer l'appropriation de notre travail dans le cadre de scénarios de production audiovisuelle collaborative.
En particulier, nous expliquons quelle partie de l'ontologie est mobilisée par les applications pour construire ou bien intégrer des connaissances sur la production, ses contributeurs et ses produits.

La \g{Conclusion} remet en perspective notre contribution et les applications développées par rapport aux problèmes métiers et scientifiques posés.
Nous ouvrons également la discussion sur la poursuite de nos recherches et l'avenir des applications du projet MediaMap.