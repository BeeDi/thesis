%%%%%%%%%%%%%%%%%%%%%%%%%%%%%%%%%%%%%%%%%%%%%%%%%%%%%%%%%%%%%%%%%%%%%%%%%%%%%%%%%%%%%%%%%%%%%%%%%%%
%%%%%%%%%%%%%%%%%%%%%%%%%%%%%%%%%%%%%%%%%%%%%%%%%%%%%%%%%%%%%%%%%%%%%%%%%%%%%%%%%%%%%%%%%%%%%%%%%%%
% \section*{Préambule (n)}
% \addcontentsline{toc}{section}{Préambule}



%%%%%%%%%%%%%%%%%%%%%%%%%%%%%%%%%%%%%%%%%%%%%%%%%%%%%%%%%%%%%%%%%%%%%%%%%%%%%%%%%%%%%%%%%%%%%%%%%%%
%%%%%%%%%%%%%%%%%%%%%%%%%%%%%%%%%%%%%%%%%%%%%%%%%%%%%%%%%%%%%%%%%%%%%%%%%%%%%%%%%%%%%%%%%%%%%%%%%%%
\chapter{Introduction}\label{chap:intro}
\minitoc
% \epigraphii{On peut s'étonner que les actes spontanés par lesquels l'homme a mis en forme sa vie, se sédimentent au dehors et y mènent l'existence anonyme des choses. La civilisation à laquelle je participe existe pour moi avec évidence dans les ustensiles qu'elle se donne.}{Merleau-Ponty\\\hfill\ita{Phénoménologie de la perception}}

%%%%%%%%%%%%%%%%%%%%%%%%%%%%%%%%%%%%%%%%%%%%%%%%%%%%%%%%%%%%%%%%%%%%%%%%%%%%%%%%%%%%%%%%%%%%%%%%%%%
% \section{Contexte}\label{chap:contexte}
Le travail de thèse dont nous rendons compte dans ce mémoire s'est déroulé dans le cadre du projet MediaMap\footnote{Voir \url{http://www.mediamapproject.org/}} soutenu par le cluster européen Eureka Celtic et la Direction Générale de la Compétitivité, de l'Industrie et
des Services (DGCIS) du ministère de l'Economie, des finances et de l'industrie.
La participation à ce projet de recherche et développement nous a imprégnés de connaissances sur la production audiovisuelle.
%, les besoins qui émergent des tendances actuelles et les problèmes rencontrés pour les combler. 



%Nous nous intéresserons ensuite (\g{Chapitre \ref{chap:problo}}) aux 


%%%%%%%%%%%%%%%%%%%%%%%%%%%%%%%%%%%%%%%%%%%%%%%%%%%%%%%%%%%%%%%%%%%%%%%%%%%%%%%%%%%%%%%%%%%%%%%%%%%
\section{L'impact du numérique sur l'audiovisuel}\label{sec:motiv}
\e{
La révolution numérique initiée depuis une trentaine d'années en conjonction avec la révolution électronique et informatique, s'est progressivement imposée à tous types d'information et de contenus. 
%jusqu'à devenir prépondérante et indispensable dans un monde informatisé.
Ces mouvements d'informatisation des pratiques et de numérisation de l'information impactent les organisations et les métiers en redistribuant les tâches entre humains et machines. 
Dans le cadre de cette thèse, nous nous pencherons sur le cas de l'audiovisuel et de la numérisation de ce contenu et des informations associées.}

%L'audiovisuel représente à plus d'un titre un objet singulier.
%l'exploitation des contenus numériques est en passe de s'étendre à toutes les étapes du cycle de vie d'un objet audiovisuel. % ? pourquoi exploitation ?
Après une informatisation des étapes de postproduction (logiciels de montage, effets spéciaux etc.) puis des équipements de captation (caméra, micro etc.) et de la distribution du côté des diffuseurs, nous avons connu une véritable explosion d'appareils destinés au grand public (lecteur multimédia portable, appareil photo, téléphone portable, dictaphone etc.).


Plusieurs grands chantiers s'ouvrent désormais dans ce mouvement d'informatisation des étapes de la chaîne de production :
\begin{liste}
	\item L'archivage des contenus de manière à les faire entrer dans l'histoire malgré la dégradation inexorable des supports. 
	Il ne s'agit pas simplement de leur permettre de survivre jusqu'à la prochaine génération d'appareils électroniques, mais également de garantir leur \ciel{lisibilité technique et culturelle}, (\cite[p.~12]{Bachimont2000}).

	\item La préproduction des contenus qui est presque inexistante et qui permettrait de récolter des informations sur les contenus avant même leur fabrication. 
	L'enjeu plus général est d'initier l'indexation des contenus au moment du \e{Scripting} et de la continuer tout au long de la chaîne, chaque étape pouvant rajouter des informations ou bien réévaluer les anciennes.
\end{liste}

À côté de la numérisation et de l'informatisation, il ne faudrait pas oublier le développement considérable des réseaux de télécommunications qui a grandement favorisé les échanges de contenus de manière illégale ou légale, de pairs à pairs (\e{Peer to Peer}), entre diffuseurs et spectateurs (\e{Business to Consumers}) ou entre acteurs professionnels de la chaîne (\e{Business to Business}). 
Ainsi, numérisation et informatisation sont dorénavant implictement associées aux facilités de transfert de ces réseaux. 

Sans tenter de tirer toutes les conséquences de ces révolutions technologiques (électronique et informatisation, numérisation, mise en réseaux) sur les usages des spectateurs et des acteurs de la chaîne de production audiovisuelle, on retiendra trois constats important qui charpentent notre réflexion : 
\begin{liste}
	\item \eg{il existe un nombre croissant de contenus audiovisuels en circulation.}
	L'offre et la demande augmentent de même que les capacités de transfert des réseaux et la généralisation des appareils électroniques de captation et de visionnage.

	\item \eg{les pratiques en lien avec les contenus audiovisuels se développent et s'individualisent.}
	La consommation de ces contenus se joue de plus en plus à un niveau individuel depuis l'apparition d'appareils personnels de communication et de visionnage. 
	De même, la production de contenus est facilité par l'informatisation et les progrès des appareils électroniques de captation. 
	La création de contenus n'est donc plus l'apanage de grandes équipes spécialisées et lourdement équipées.
	
	\item \eg{la numérisation complique le maintien de l'unicité des contenus audiovisuels}. 
	%Le principe même du numérique (\ciel{ça a été manipulé} [Bachimont2010]) repose sur la représentation de l'information et le calcul. 
	L'environnement numérique, par rapport à l'analogique, est plus propice à la copie, le transfert, la fragmentation et la manipulation des contenus ramenés invariablement à une donnée muette quelle que soit sa nature (texte, son, image animée ou non etc.) ou sa signification.
	Ainsi coupé des sens et du sens, les contenus semblent perdre leur identité dans le \gui{monomédia} numérique (\cite[p.~13]{Bachimont2000}) dans lequel on a peine à les gérer pour ce qu'ils représentent. 	

	%plus manipulables au sens où chaque visionnage construit une forme perceptive du contenu 

	%est une reconstruction de leur forme perceptible par des appareils de . Il y a donc par définition une certaine instabilité 
	%mobile, versatile, altérable, mouvant, variable, instable
\end{liste} 

%Le développement de l'électronique a ainsi favorisé une  le nombre de contenus audiovisuels, à favoriser leur circulation
%On constate ainsi une augmentation des sources de contenus ainsi qu'une intensification de leur circulation qui ont des conséquences sur les usages des spectateurs et les acteurs de la chaîne de production.
Nous allons maintenant préciser cet argumentaire et présenter les enjeux posés par la numérisation, l'informatisation et le développement de l'électronique.




%%%%%%%%%%%%%%%%%%%%%%%%%%%%%%%%%%%%%%%%%%%%%%%
\subsection{La numérisation des contenus audiovisuels}\label{sec:num}
Le contenu audiovisuel est avant tout un objet temporel, c'est-à-dire un flux d'images et de sons qui s'écoule en un temps donné. 
Un des enjeux liés à l'audiovisuel constitue alors à trouver une technologie d'enregistrement et de manipulation de ces flux.

\paragraph{L'enregistrement analogique du flux}
Les techniques d'enregistrement analogique reposent sur la conversion continue du flux original en un signal analogue dont les variations s'effectuent sur une échelle physique différente, par exemple une fréquence sonore transformée en une tension électrique. 
L'enregistrement s'effectue sur différents types de supports, cassettes magnétiques, films argentiques etc. 
La caractéristique commune de ces supports est que l'accès au contenu enregistré ne peut se faire que de manière linéaire ou séquentielle, c'est-à-dire qu'il faut avancer ou rembobiner la bande jusqu'au point de départ avant de pouvoir commencer la lecture. 
Ainsi, chaque opération de manipulation de ces contenus compte toujours un temps pas forcément négligeable consacré à l'alignement avec le point de départ désiré. 
Pour bien s'en rendre compte, il suffit de se souvenir du temps qu'il fallait pour rembobiner la cassette de votre film préféré, puis du temps passé en avance rapide pour passer les inévitables publicités précédant le film.

\paragraph{Numérisation et délinéarisation de l'accés}
Avec le numérique, cette expérience autrefois familière a complètement disparu et se voit remplacée par un accès immédiat à n'importe quel moment du contenu audio-visuel. 
La numérisation des contenus consiste à discrétiser le flux en un ensemble de valeurs que l'on convertit ensuite en flux binaire. Ce flux est ensuite enregistré sur des supports de mémoire magnétiques (disquettes 3'1/2, disques durs etc.) optiques (CD, DVD etc.) ou électronique (mémoire vive dite RAM, mémoire flash etc.). 
Seules ces dernières garantissent un accès arbitraire ou délinéarisé aux données stockées en mémoire (n'importe quelle donnée, à n'importe quel moment). 
Les supports magnétiques et optiques proposent un accés séquentiel comme l'analogique, mais plus rapide et surtout que l'on peut coupler avec les mémoires électroniques afin de se rapprocher de leurs performances.

\paragraph{Fragmentation}
Par ailleurs, au-delà des avantages liées aux temps d'accès au contenu, le numérique facilite également sa fragmentation et sa manipulation.
Contrairement à l'analogique, le numérique permet de représenter de manière arbitraire tout type d'information puis d'effectuer des calculs sur ces représentations. 
Ainsi, on peut associer au contenu audiovisuel d'autres contenus de différentes natures pour les enrichir et faciliter son exploitation ultérieure.

\paragraph{Discrétisation}
La discrétisation du flux audiovisuel quant-à-elle remet en cause la temporalité du contenu et permet de ce fait une fragmentation plus aisée et plus fine. 
Lorsque l'analogique effectue une transformation continue et analogue, le numérique définit une fréquence d'échantillonnage et quantifie les valeurs de la source sur une échelle arbitraire finie. 
C'est donc véritablement la fréquence d'échantillonnage qui constitue la première unité de réprésentation de l'information au-dessus du bit.
C'est donc en décidant du nombre de pixels pour représenter une image, ou aux nombres de valeurs par secondes prises pour représenter un son que l'on décide d'une première échelle de fragmentation. 
% Premier niveau de manip technique, ensuite il y a des USI
Toute autre fragmentation à l'échelle supérieure est potentiellement (re)constructible par calcul, pour autant qu'on possède une méthode opérationnalisable. 
%De même que chaque élément de cette fragmentation, chaque unité, devient adressable et donc manipulable par calcul. 
%\cite{Bachimont2000} parle ainsi d'utm, usi ...
% COUPURE DU SENS ?

% \g{== Révision à faire, introduire les UTM et USI ==}
\paragraph{Numériser, c'est informatiser le métier}
Parmi toutes les fragmentations possibles, il convient alors de déterminer leur pertinence par rapport aux types de calculs que l'on souhaite réaliser à chaque étape de la chaîne de production. 
Il s'agit ici de contrôler les calculs à effectuer en suivant les règles du métier qui en disposera.
Chaque étape ayant ses objectifs propres, les calculs effectués varient et s'opèrent à différents niveaux de fragmentation. 
Or la numérisation des contenus et de l'information s'effectue toujours en vue de leur manipulation par des programmes informatiques.
Ainsi, la numérisation entraîne toujours une informatisation qui impacte le métier dans son organisation et ses pratiques parce qu'il transforme les possibilités techniques qui le concerne.
% L'enjeu est donc double, d'une part identifier les niveaux de fragmentation pertinents pour chaque étape de la chaîne de production, et d'autre part de se donner les moyens de reconstruire la cohérence . 

% numériser => (a) fragmenter + (b) mettre en réseau => (a) besoin de conserver la cohérence de l'ensemble + (b) besoin d'autonomiser pour une future situation d'usage


