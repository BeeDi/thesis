%====================================================
%====================================================
%====================================================
\section{Décrire les objets audiovisuels (m)}\label{s:desc}
\e{
Avant de commencer notre état de l'art sur la description des contenus audiovisuels, nous souhaitons préciser pourquoi et de quelle manière les sciences informatiques abordent la construction de ces descriptions.}

\paragraph{Pourquoi ?}
Tout d'abord parce que de lui-même le contenu audiovisuel ne se laisse pas saisir dans son entièreté, il n'est pas autant à la disposition de son lecteur que ne l'est un texte. 
Il s'agit d'un objet temporel, qui se déploie progressivement au fur et à mesure de sa lecture. 
On ne peut donc l'appréhender qu'en prenant le temps de le voir, instant après instant. 
À l'inverse, le texte est déployé dans l'espace ce qui lui permet d'être observé d'un point de vue synoptique. 
Ensuite, l'informatique manipule  bien plus facilement le texte que les contenus audiovisuels, surtout du point de vue de l'indexation. 
L'indexation textuelle utilise des mots-clés extraits directement du texte, mots-clés que l'on a appris à utiliser également pour effectuer une recherche d'information. 
L'équivalent pour le contenu audiovisuel n'existe pas, même si l'on voit émerger des techniques pour repérer les similarités entre images.

\paragraph{De quelle manière ?}
Il faut distinguer deux grandes familles d'approches de description des contenus audiovisuels, les approchsees automatisées qui procédent par analyse du signal audio-visuel pour décrire le contenu et les approches manuelles, où un humain éventuellement assisté par un système informatique construit une description du contenu. 
Les descriptions produites par ces approches sont également très variées, et l'on peut très bien imaginer toutes sortes de discours ou de données à apposer au contenu audiovisuel. 
Nous verrons quelles types de description chacune de ces approches construit sur les contenus, et comment ces descriptions sont exploitées, pour quel type d'usage, à l'intention de systèmes informatiques ou d'être humains.

% collecte de MD tout au long de la chaîne : \cite{Rayers2002}

%=============================
%=============================
\subsection{Descriptions humaines}
Script, Storyboard, Analyse sémiotique (\cite{Martin2005}, \cite{ThiBui2003}) etc.
Réappropriations dans le champ informatique : \cite{Chakravarthy2009b} ; \cite{Chakravarthy2009c}

%=============================
%=============================
\subsection{Modèles et standards de description}
News-ML (\cite{Nack2004}) etc.

%=============
\subsubsection{MPEG-7 : Multimedia Description Scheme}
% \addcontentsline{toc}{subsubsection}{MPEG-7 : Multimedia Description Scheme}
\cite{Hunter2001} ; \cite{Troncy2007} ; \cite{Nack2005a} ; \cite{Dasiopoulou2009} ; \cite{Garcia2005} ; 


%=============
\subsubsection{Core Ontology for MultiMedia}
% \addcontentsline{toc}{subsubsection}{Core Ontology for MultiMedia}
\cite{Arndt2009} ; \cite{Arndt2007} ; \cite{Staab2008} 

%=============
\subsubsection{MPEG-21}
% \addcontentsline{toc}{subsubsection}{MPEG-21}
\cite{Burnett2003} ; \cite{Garcia2010} ? 

%=============
\subsubsection{TV Anytime}
% \addcontentsline{toc}{subsubsection}{TV Anytime}
\cite{Evain2000} ; \cite{Tsinaraki2004} ; \cite{Tsinaraki2005}

%=============
\subsubsection{Material eXchange Format : Description Metadata Scheme-1}
% \addcontentsline{toc}{subsubsection}{Material eXchange Format : Description Metadata Scheme-1}
\cite{Marcos2009}


\subsubsection{OntoMedia}
\cite{Lee2012a}

%=============================
%=============================
\subsection{Systèmes informatiques}
\cite{Tsinaraki2005} ; \cite{Tsinaraki2004} ; \cite{Dasiopoulou2009} ;


%=============
\subsection{Discussions}
% alignement : 
\cite{Burger2011}
