%====================================================
%====================================================
%====================================================
\newpage
\section{Décrire les objets audiovisuels (m)}\label{sec:desc}
\e{
Avant de commencer notre état de l'art sur la description des contenus audiovisuels, nous souhaitons préciser pourquoi et de quelle manière sont abordés la construction et la collecte de ces descriptions.
}
% Nous verrons quelles types de description chacune de ces approches construit sur les contenus, et comment ces descriptions sont exploitées, pour quel type d'usage, à l'intention de systèmes informatiques ou d'être humains.

\subsection{Nature et fabrication des descriptions}
% \subsection{Quelles descriptions ?}

\subsubsection{Nature des descriptions}
\paragraph{L'intérêt des descriptions textuelles}
Tout d'abord, nous souhaitons argumenter sur les avantages d'une description textuelle de contenus audio-visuels. 
En effet, pourquoi ne pas se contenter de ce que les images ont à nous offrir pour construire des index et rechercher ces images ? 
La question est d'autant plus pertinente que depuis presque une vingtaine d'années se développent des techniques évaluant la similarité entre images, ce qui permet d'utiliser une image (\gui{Query by Example}, initié par \cite{Flickner1995}, \cite{Pentland1996}) ou bien une représentation graphique (\gui{Query by Canvas}, initié par \cite{Flickner1995}, \cite{Bach1996}) comme élément de base pour une requête.
Ces approches sont donc tout à fait pertinentes dans le cadre de la recherche d'images extraites d'objets audiovisuels. 
Cependant, ces techniques reposent sur une caractérisation visuelle de l'image qui ne suffit pas à rendre compte de la richesse d'information que l'on voudrait associer aux objets audiovisuels.
En effet, de lui-même le contenu audiovisuel ne se laisse pas saisir dans son entièreté, ni ne se laisse manipuler aussi aisément que nous arrivons à le faire avec du texte. 
L'objet audiovisuel est avant tout temporel, il se déploie progressivement au fur et à mesure de sa lecture. 
On ne peut donc l'appréhender qu'en prenant le temps de le voir, instant après instant. 
À l'inverse, le texte est déployé dans l'espace ce qui lui permet d'être observé d'un point de vue synoptique. 
Ensuite, l'informatique manipule  bien plus aisément le texte que les contenus audiovisuels, surtout du point de vue de l'indexation. 
L'indexation textuelle utilise des mots-clés extraits directement du texte, mots-clés que l'on a appris à utiliser également pour effectuer une recherche d'information. 
De plus, les mots peuvent servir à décrire de multiple aspects de l'objet audiovisuel, qu'il s'agisse de nommer des éléments de l'image, de décrire sa composition, le contexte de sa production, son statut légal etc.
La limite d'une telle approche ne semble ainsi pas provenir du langage utilisé pour exprimer ces mots, mais plutôt de la manière de produire ces libellés. 
Nous écartons donc de notre état de l'art les recherches par images pour nous concentrer sur l'étude des modèles de description permettant de prendre en compte plusieurs aspects de l'objet audiovisuel. 


\paragraph{La diversité des descriptions}
La description d'un objet dépend fortement du point de vue de la personne qui la produit. 
En particulier, chaque corps de métier de la chaîne de production peut avoir une perspective particulière sur l'objet audiovisuel, même s'ils partagent des éléments communs pour discuter de sa production. 
Comme ces informations peuvent servir à de nombreux usages, il est intéressant de proposer des catégories permettant de les distinguer. 
Ainsi, \cite[\S 2:Types of Metadata]{Austerberry2004} distingue en particulier trois types d'informations : 
\begin{liste}
	\item \ciel{descriptive information} (informations descriptives) : qui sont relatives à la description du contenu, comme le sujet ou le genre de l'objet audiovisuel. 
	Par exemple, on peut associer des noms propres, des lieux, des évènements etc. à l'objet en fonction de ce dont traitera le contenu audiovisuel. 
	Un vocabulaire contrôlé peut servir à indiquer le "genre" de l'objet, c'est-à-dire les règles de genre qu'il suit au niveau de la structure, du ton etc.
	De même, des résumés textuels peuvent être inclus, à la manière de ceux prévus pour les guides TV. 

	\item \ciel{administrative information} (informations administratives) : qui relèvent de l'aspect métier et légal de la production de l'objet audiovisuel. 
	On cherchera à lister les contributeurs à la production de l'objet ainsi que les détenteurs de droits d'utilisation sur l'oeuvre. 
	Ces informations peuvent servir à retrouver des objets audiovisuels, notamment dans le cas de professionnels.
	Il peut également s'agir d'informations confidentielles comme les contrats de la distribution indiquant les droits d'utilisation de l'oeuvre, ou bien les contrats liées à la production. 
	Ces informations ne décrivent donc pas le contenu des objets audiovisuels, mais servent à faciliter leur gestion et leur exploitation future.

	\item \ciel{preservation information} (informations de conservation) : qui visent à faciliter la conservation et la réutilisation des objets et des fragments audiovisuels. 
	Il s'agit de pouvoir conserver des informations afin de faciliter la réutilisation de prises de vue dans de nouveaux contextes, de nouveaux montages ou bien de version spéciales suite à un changement de support de distribution. 
	De plus, cela peut aussi concerner le suivi des conditions de stockage des objets audiovisuels (lieux, support, état etc.) ou bien la réalisation d'opération de restauration. 
	Il s'agit donc d'informations qui permettent d'identifier et de pister non seulement les objets audiovisuels finis, mais également tous les fragments qui ont été crées à l'occasion et toutes les versions qui ont été construites par la suite.
\end{liste}


Nous pouvons ajouter à ces catégories, d'autres types d'informations proposées par \cite{Rayers2002} : 
\begin{liste} 
	\item \ciel{compositional information} (informations de composition) : qui identifient la liste des fragments composant un objet audiovisuel et la manière dont ils doivent être montés pour reconstruire l'objet. 
	Dans le milieu de l'audiovisuel, ces informations sont représentés sous la forme de ce qu'on appelle une \ciel{Edit Decision List} (EDL).
	Cette liste peut être produite soit sous forme papier, soit par un logiciel de montage en utilisant une des dizaines de formats existants, formats souvent propriétaires et liés au logiciel de montage. 
	Ces informations servent à décrire la structure des objets audiovisuels et peuvent servir à leur gestion.
	Elles sont à rapprocher des \ciel{preservation information} dans le sens où elles permettent de garder trace du montage. 


	\item \ciel{technical or essential information} (informations techniques) : qui correspondent à une description techniques des caractéristiques du signal audio-visuel. 
	Ces informations sont généralement présentes dans les formats d'encapsulation de contenu numérique afin de permettre son décodage.
	Ces informations peuvent également servir à différencier plusieurs versions d'un objet audiovisuel qui ne varieraient que par leur encodage. 
	Il s'agirait alors de rapprocher ces versions avec leur utilisation futures (utilisation comme proxy, version destinée à un support particulier, à un canal de distribution etc.).
\end{liste}


Notons enfin, qu'il existe également toutes sortes de descriptions produites par des intervenants extérieurs à la chaîne de production (les spectateurs, les critiques, les sémiologues etc.), qui sont autant d'analyse pouvant se représenter sous la forme d'une structuration documentaire (voir \ref{sec:uc-sd}).
% pertinent pour notre problème.



\subsubsection{Fabriquer les descriptions, \e{in situ} ou \e{a posteriori} ?}\label{sec:codesc}
Nous distinguons deux méthodes de fabrication des descriptions suivant le moment où l'information est construite.
Ainsi, la fabrication que nous appellons \e{in situ} correspond à la collecte d'informations intervenant dès la pré-production d'un contenu, alors que l'approche dite \e{a posteriori} intervient en fin de chaîne, une fois l'objet audiovisuel constitué.

% \subsubsection{La collecte durant la production}
\paragraph{Fabrication \e{in situ}, la collecte durant la production}
Dans la chaîne de production audiovisuelle, chaque communauté de contributeurs a un intérêt particulier et souvent différent sur les informations à partager sur l'objet. 
Les informations que l'un jugera critique (sur quelle cassettes peut-on trouver la prise de vue n°354) peuvent être jugées insignifiantes pour d'autres (un journaliste se contentera de savoir que la prise de vue existe). 
Il existe donc autant de descriptions possibles d'un objet, que de points de vue sur cet objet. 
De plus, une description n'est jamais utile qu'en fonction d'un usage. 
Savoir où se trouve la prise de vue n'a d'intérêt que pour la personne qui devra en réaliser le montage, ou bien s'assurer de la préservation de la cassette une fois utilisée. 
Les professionnels de la production sont donc confrontés à de multiples informations, pas forcément pertinentes par rapport à la tâche qu'ils doivent réaliser mais qui peuvent s'avérer utiles plus loin dans la chaîne.  

\citeauthor{Rayers2002} et \cite{Austerberry2004} présentent en détails la manière dont les informations sont produites dans les processus de production télévisuels. 
Leur expertise\footnote{Ces auteurs ont travaillé, entre autres, pour les réseaux de télévision public anglais (BBC), néerlandais (NOS), une bouquet privé américain (HBO) et en collaboration avec la principale association de professionnels du milieu de la télévision (SMPTE).} indique que ces informations sont collectées, manipulées,  réutilisées, diffusées à différentes étapes de la chaîne de production. 
Pourtant, la mise en commun de ces informations restent un défi et il semble fréquent que ces informations doivent être retrouvées à la main, voire reconstruite à défaut d'en avoir connaissance :

\ciel{
	However, the point in the process where a piece of metadata first becomes available is not necessarily a point where it is needed for that stage in the process. Worse, frequently two or more stages in the workflow might need the same metadata, but not the intervening processes.} (\cite[p.22, Metadata in the Workflow]{Austerberry2004})

De même, il semble également que la structuration de ces informations se fassent le plus tard possible, c'est-à-dire au moment de l'archivage, à partir des mêmes informations utilisées pour construire les programmes TV\footnote{En plus, des citations que nous avons trouvé dans des livres d'experts, nos partenaires du milieu de l'audiovisuel dans le projet MediaMap nous ont également confirmé ce problème. Il s'agissait d'ailleurs d'un des problèmes centraux auquel le projet MediaMap s'est attaqué.}.
L'accès aux informations constituées pendant le déroulement de la chaîne est donc très difficile une fois la production finie.
Ainsi, les auteurs proposent de favoriser une collecte des informations au fur et à mesure de la chaîne afin d'éviter toute perte de temps mais aussi de favoriser le partage et la qualité des informations : 

\ciel{
	Clearly, it makes sens to capture metadata at the earliest stage possible as a program is made, and to either pass it through the chain or hold it in a common repository. This way, those stages that need the metadata can access it easily and do not need to look for it, reacquire it, or, worse, reinvent it. Sadly, this has \e{not} been the traditional way of doing things.} (\cite[p.23, Metadata in the Workflow]{Austerberry2004})

\ciel{
	At each step in the production process we can collect, and possibly re-use metadata. [\dots] Each re-use point is a saving as we have removed a data re- entry, probably reduced errors and given producers more information to help their task.} (\cite{Rayers2002})

% collecte de MD tout au long de la chaîne : \cite{Rayers2002}



\paragraph{Fabrication \e{a posteriori}}
Les approches classiques de construction de descriptions des contenus audiovisuels se font \e{a posteriori} de la production de l'objet audiovisuel. 
Elles interviennent en fin de chaîne, après la diffusion, voire en-dehors lorsqu'un organisme différent du producteur se charge de l'archivage (c'est le cas de l'INA avec l'audiovisuel public en France). 
Il faut distinguer deux genres d'approches pour construire des descriptions de contenus audiovisuels : 
\begin{liste}
	\item les approches automatisées qui procédent par analyse du signal audio-visuel pour décrire le contenu.
	Par exemple, un programme qui réalise une transcription des paroles prononcées dans une émission, ou bien un système de détection de changement de plans. 

	\item les approches manuelles où un humain, éventuellement assisté par un système informatique, produit une interprétation du contenu avant de le décrire. 
	Il peut s'agir par exemple d'un documentaliste qui classifie et résume le contenu d'une émission dans un système d'archivage.
\end{liste}

\paragraph{L'analyse automatique du signal}
\cite{Staab2008} propose d'examiner les nombreux travaux menés suivant l'approche automatisée. 
De manière générale, ces travaux analysent le signal audio-vidéo pour en extraire par calcul un ensemble de composantes (feature).
Par exemple, le signal vidéo peut être décomposé en termes de couleur, de texture, de forme, de mouvement etc. 
Ces composantes servent ensuite de données d'entrée à des classifieurs qui cherchent à identifier des objets pour en fournir un libellé. 
De multiple sortes d'analyses peuvent être menées automatiquement à partir du signal ou bien de documents annexes comme le rapportent \cite[\S 8 : Cataloguing and indexing]{Austerberry2004} : 
\begin{liste}
	\item \e{analyse de texte} : lorsque des éléments textuels sont fournis avec le matériel audiovisuel, il est alors possible de les analyser pour en extraire des métadonnées. 
	Il peut s'agir par exemple d'un fichier de sous-titre, d'une retranscription textuelle destiné aux malentendants, d'un résumé du programme etc. 
	Mais cela pourrait également s'appliquer directement à des documents de production.

	\item \e{analyse audio} : la première distinction que l'on cherche à effectuer est de savoir si la séquence étudiée contient de la musique, un discours ou bien simplement s'il s'agit de silence.
	Lorsqu'il y a discours ou dialogue, on peut ensuite tenter de faire de la reconnaissance de paroles (pour obtenur une retranscription) et une reconnaissance des intervenants (combien de personnes parlent dans cette séquence, qui dit quoi ?) voire une identification des orateurs à partir de leur voix. 
	Pour la musique, on pourrait chercher de même à identifier l'interprétation et récuperer les métadonnées correspondantes. 

	\item \e{analyse vidéo} : l'analyse vidéo vise principalement à reconnaître des objets, des visages ou des éléments de textes dans les images. 
	La reconnaissance de visages permet d'identifier le nombre de personnes présentes à l'image voire de les identifier à partir d'une base de données. 
	La reconnaissance de caractères permet par exemple de produire une retranscription des titres d'un journal télévisé ou tout autre texte à l'écran juste avec le signal vidéo. 
	De nombreuses autres types d'analyses peuvent être menées, la reconnaissance d'objets ou bien encore la détection des mouvements de caméra. 
\end{liste}


% description humaines (sémiotique des contenus par exemple)


\subsubsection{À propos du fossé sémantique}
% décrire le signal par des descripteurs analytiques/objectifs
Comme nous avons vu, les approches d'analyse du signal audio ou vidéo visent donc à reconnaître puis à nommer ou identifier un certain types d'objets présent dans un objet audiovisuel. 
Cependant, \cite{hare:semantic-gap} pointent la limite de la description par libellé qui ne propose pas d'interprétation aussi complète que celle effectuée par un humain. 
Il propose l'exemple d'une image montrant une manifestation où l'on voit des étudiants et des policiers. 
Si l'analyse permettra d'indiquer le nombre de gens, de reconnaître des bâtiments ou des véhicules, voire le mot police, le fait de nommer ces éléments ne permet pas de conclure qu'il s'agit d'une manifestation. 
Cet écart entre les résultats de l'analyse automatique du signal et l'interprétation humaine se nomme ainsi le \gui{fossé sémantique} (semantic gap). 


\citeauthor{hare:semantic-gap} caractérisent alors ce fossé en deux sous-problèmes distincts : 

\ciel{
It may be instructive to see the gap in two major sections, the gap between the descriptors and object labels and the gap between the labelled objects and the full semantics.}

En d'autres termes, reconnaître et nommer à partir du signal n'est que la première difficulté à surmonter, il s'agit là d'un problème d'ordre calculatoire qui fait l'objet d'un très grand nombre de travaux.
Le second problème consiste à passer d'une description par libellés à une description proche de l'interprétation humaine, ou tout du moins un certain type d'interprétation humaine. 
On retrouve cette caractérisation du fossé sémantique en deux étapes différenciées dans la définition proposée par \cite{Smeulders2000} et reprise dans l'\e{Encyclopedia of Multimedia} (\cite{Furht2008}) :  %!!ref

\ciel{
	The semantic gap is due to two inherent problems. 
	One problem is that the extraction of complete semantics from image data is extremely hard as it demands general object recognition and understanding. [\dots]
	The other problem causing the semantic gap is the complexity, ambiguity and subjectivity in user interpretation.}

Si les libellés ne suffisent pas à décrire les objets audiovisuels pour des agents humains, \citeauthor{hare:semantic-gap} proposent de s'intéresser aux besoins de ces agents et notamment aux requêtes qu'ils souhaitent poser et aux raisonnements que le système d'indexation doit pouvoir effectuer. 
L'exemple pris est celui d'une requête d'une photo d'un frigo des années 50. 
Dans ce cas, la requête mentionne un nom de marque qui ne correspond pas au nom commun pour ce type d'appareil (réfrigérateur). 
Ainsi, l'attente implicite serait de pouvoir effectuer des rapprochements sémantiques entre libellés de la requête et libellés utilisés dans l'indexation. 
Ce qui suppose l'utilisation de vocabulaires structurés à la base de l'indexation et un formalisme permettant d'effectuer des raisonnements. 

Notons enfin que l'approche \e{in situ} consiste à modéliser les connaissances contenues dans les documents de la chaîne de production audiovisuelle.
L'originalité de cette approche est d'attaquer le fossé sémantique par le haut, c'est-à-dire à partir des connaissances humaines, qu'il faut ensuite représenter et associer aux séquences audio-visuelles.
C'est justement ce renversement du problème qui ouvre la possibilité de collecter les informations fabriquées durant la production.

\e{
Dans notre cas, nous nous intéressons à la manière dont les professionnels de la production recherchent des séquences vidéos dans le cadre de pratiques de réutilisation des objets audiovisuels (décrites précédemment en section \ref{sec:reuse}). 
Or, il s'agit maintenant d'examiner les moyens de décrire ces objets audiovisuels en regard de ces pratiques de réutilisation. 
Ainsi, nous nous attacherons dans cette section à présenter les standards, modèles et vocabulaires qui contribuent à décrire les objets audiovisuels. 
Nous détaillerons quelles sont les informations fournies par ces modèles et quel genre de manipulation des objets audiovisuels elles permettent d'effectuer.
Nous distinguons parmi les modèles qui suivent une approche de fabrication des descriptions \e{a posteriori} (\ref{sec:post}) et ceux qui suivent une approche dite \e{in situ} de modélisation des connaissances des professionnels (\ref{sec:plan}).}
