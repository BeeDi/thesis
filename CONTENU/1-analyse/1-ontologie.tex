

\section{Les Systèmes d'Organisation de Connaissances (m)}\label{chap:defs}
%Distinguer entre terme et concept ; dictionnaire, thésaurus, ontologies etc.

\e{
L'objectif de cette section est de clarifier ce qui appartient au domaine de la  linguistique et ce qui relève du domaine conceptuel, en vue d'identifier les notions qui nous serviront à spécifier une solution au cahier des charges dressés dans la section précédente. 
La confusion qui nous intéresse concerne principalement la définition des ontologies par rapport à d'autres SOC tels ques les thésaurus, notamment du fait qu'on utilise parfois les mêmes langages pour les représenter.}

La définition des SOC proposée par \cite{Zacklad2010}, étend celle de \cite{Hodge2000} à \ciel{l'ensemble des formes d'écritures codifiées participant à la description documentaire primaire ou secondaire d'une situation}. 
L'ensemble défini par \citeauthor{Hodge2000} comprend ainsi tout type de :
\begin{liste}
	\item \e{liste de termes} (fichiers d'autorités, glossaires, dictionnaires, répertoires géographiques)
	\item \e{schème de classification/catégorisation} (vedettes-matières, taxonomie)
	\item \e{schème qui se structure par le types de relations qui unit ses membres} (thésaurus, réseaux sémantiques, ontologies). 
\end{liste}

À cela \citeauthor{Zacklad2010} souhaite ajouter des modes de description du contenu émergents ou plus faiblement codifiés comme par exemple les folksonomies. 
Les SOC qui nous intéressent en particulier sont les schèmes structurés par types de relations. 
% pourquoi ? 


\subsection{Thésaurus, terminologie, ontologie}\label{sec:tto}
Dans cette section, nous nous reposerons majoritairement sur les définitions de \cite{bachimont:icc}. Concernant le thésaurus, l'auteur écrit :

\g{Thésaurus :} 
\ciel{
Une organisation de libellés linguistiques selon des relations d'hyperonymie et d'hyponimie. 
Les libellés sont également reliés par des relations dites d'association, qui sont de nature quelconque. 
Même si en pratique les libellés d'un thésaurus correspondent souvent à des termes du domaine, ce n'est pas nécessairement systématique.}

Cette définition situe clairement les thésaurus comme faisant partie du cadre de la linguistique. 
Il s'agit d'un ensemble de mots structurés et reliés suivant leur \e{signification}, c'est à dire leur sens normé ou commun à plusieurs contextes d'usage particuliers (à l'inverse du sens, qui lui varie suivant les usages, \cite{Roche2005}). 
\citeauthor{bachimont:icc} finit sa définition en comparant les mots issus d'un thésaurus aux termes. La distinction se joue à deux niveaux, la stabilité d'écriture du terme (niveau linguistique) et le fait qu'un terme renvoit à un concept (niveau conceptuel) : 

\g{Terme :} 
\ciel{
Une unité linguistique dont le signifié est un concept, c'est-à-dire un signifié normé. 
Le terme se manifeste linguistiquement par une stabilité et régularité de sa forme signifiante.
En particulier, un terme possède des contextes d'occurrence réguliers, obéissant à des canevas morpho-syntaxiques typiques. 
La détection de ces canevas est à la base des outils de détection des termes en corpus. 
Un terme peut posséder des variantes terminologiques.
Dans une optique normative, on détermine une forme préférée.}

Ainsi, plus qu'un repérage des mots (signifiant) utilisés dans un domaine donné, la terminologie s'attache à identifier les signifiés correspondants. 
Au-delà des débats sur les méthodes utilisées pour constituer les couples signifiant-signifié\footnote{L'approche \e{sémasiologique} (initiée par \cite{Bourigault1994}) s'appuie sur l'analyse linguistique d'un corpus de textes pour repérer les couplages signifiant-signifié ainsi que l'organisation conceptuelle sous-jacente. L'identification de ces \e{désignations} est ensuite validée par des experts du domaine. 
Dans une optique différente, l'approche \e{onomasiologique} prend comme appui la modélisation conceptuelle pour nommer ensuite les concepts. On parle alors de \e{dénominations} dont l'objectif est de refléter sans ambiguïté la structure conceptuelle dont elles sont issues.
Une critique faite à la première approche par \cite{Roche2006} est que les relations identifiées entre désignations sont purement linguistiques (hyper/hyponymie, méronymie etc.) et ne se rattachent pas à une structure conceptuelle. Ainsi, l'analyse de texte n'est qu'une première étape dans la constitution d'une terminologie, elle permet d'identifier les usages des mots, mais pas de les raccorder à des concepts. De même, la constitution du corpus va également grandement influer sur les résultats de l'analyse et pose alors un problème de réutilisabilité.}, on cherche à repérer la modélisation conceptuelle sous-jacente d'un domaine, de manière à pouvoir adosser chaque terme à un concept.
À noter que l'inverse n'est pas forcément valable, car il existe des concepts qui n'ont pas d'appelation usuelle et que l'on doit alors désigner par une phrase. 
Une autre conséquence de cette définition est que plusieurs mots peuvent être adossés au même concept.
Il devient alors important de pouvoir expliciter cette équivalence et éventuellement de spécifier un signifié préféré pour le terme.

Cette définition du terme préfigure ainsi la relation qu'entretient la terminologie avec l'ontologie pour \cite{bachimont:icc} : 

\g{Terminologie :} 
\ciel{un recensement et une organisation d'unités linguistiques à l'usage stabilisé et attesté, dont le signifié correspond à un concept du domaine.
La terminologie est l'organisation des termes du domaine.
La terminologie est la face linguistique de l'ontologie, qui en est le côté conceptuel. 
Il n'y a pas une stricte correspondance cependant entre ontologie et terminologie : si tout terme doit correspondre à un concept de l'ontologie, tout concept n'a pas forcément d'usage linguistique régulier attesté.}

De son côté, \cite[\S 2.4]{Roche2005} parle de manière similaire \ciel{[d']un système de termes reflétant une modélisation conceptuelle, [...] plus généralement dénommé \e{système notionnel} [qui] trouve sa raison d'être dans la façon dont nous appréhendons les objets du monde.}
\citeauthor{Roche2005} précise que si les systèmes notionnels ne relèvent pas de la linguistique, ils ne dépassent pas forcément le cadre d'une langue, sauf \ciel{pour des communautés de pratique dont les langues d'usage partagent la même conceptualisation du monde.}.
La distinction est ainsi faite entre les mots d'usage (qui peuvent être polysémiques) et les termes dont on spécifié une forme préférée (signifiant) et qu'on adosse à une signification (signifié normé).  

Concernant les particularismes qui peuvent exister dans chaque communauté, \citeauthor{Roche2005} propose de s'éloigner d'une vision purement normalisatrice. 
Ainsi, sur le plan linguistique, il est possible de rattacher les différents mots d'usages et de préciser leur contextes d'utilisation.
Sur le plan conceptuel, l'auteur propose de constituer des \gui{terminologies régionales} que l'on cherchera ensuite à mettre en correspondance. 




\paragraph{Ontologie}
Concernant les ontologies, nous nous limiterons aux définitions proposés dans le cadre de l'ingénierie des connaissances (IC). Les travaux de \cite{Charlet2002} nous rappelle qu'il existe de multiples définitions : 

Pour \cite{Gruber1993} : \ciel{Une ontologie est une spécification explicite d'une conceptualisation.}

 La définition proposée par \cite{Uschold1996} nous permet de préciser de quoi se compose une conceptualisation et en quoi une ontologie la spécifie : 

\ciel{Une ontologie implique ou comprend une certaine vue du monde par rapport à un domaine donné. Cette vue est souvent conçue comme
un ensemble de concepts -- e.g. entités, attributs, processus --, leurs définitions
et leurs interrelations. On appelle cela une conceptualisation. [...]
Une ontologie peut prendre différentes formes mais elle inclura nécessairement
un vocabulaire de termes et une spécification de leur signification. [...]
Une ontologie est une spécification rendant partiellement compte d’une conceptualisation.}

\citeauthor{Charlet2002} en conclut qu'une ontologie est une conceptualisation, c'est-à-dire un ensemble de concepts et de relations dont on cherche à normer la signification. 
Pour faire de la conceptualisation un objet informatique, il faut spécifier une théorie logique dotée d'un vocabulaire (les concepts et les relations), à la manière des travaux de \cite{Guarino1995}.

% Roche puis Babache
Pour \citeauthor{Roche2005}, une ontologie est équivalente au système notionnel des terminologies, d'où la relation forte établie par les chercheurs en IC : 

\ciel{définie pour un objectif donné et un domaine particulier, une ontologie est pour l'ingénierie des connaissances une représentation d'une modélisation d'un domaine partagée par une communauté d'acteurs. Objet informatique défini à l'aide d'un formalisme de représentation, elle se compose principalement d'un ensemble de concepts définis en compréhension, de relations et de propriétés logiques.} (\cite{Roche2005})

\citeauthor{Bachimont2000a} insiste sur le fait qu'on utilise une sémantique donnée (différentielle, référentielle, psychologique, distributionnelle, conceptuelle etc., \cite{bachimont:hdr}) pour établir la signification des concepts de l'ontologie. Chaque sémantique propose un point de vue particulier qui permet de faire correspondre une signification propre à chaque unité d'expression : 

\ciel{une ontologie est la signature fonctionnelle et relationnelle, munie de sa sémantique, d'un langage formel de représentation et manipulation des connaissances.} (\cite{Bachimont2000a})

Les ontologies se construisent ainsi en s'adossant à des théories, et ce sont ces théories qui fixent des principes pour déterminer la signification des unités linguistiques qu'elles emploient et chargent d'un sens bien précis (sémantique). 

\paragraph{Sémantiques et ontologies}
Pour bien cerner les conséquences de cette définition, voici quelques sémantiques décrites par \cite{bachimont:hdr} qui se distinguent dans leur manière d'expliciter la signification d'une unité d'expression : 
\begin{liste}
	\item \e{sémantique différentielle} : la signification d'une unité consiste en l'identité et la différence par rapport aux autres unités linguistiques de la langue. On reste donc dans le cadre de la linguistique. La différenciation des  unités peut se faire par différentes méthodes ; par observation empirique d'un corpus de texte (\e{sémantique distributionnelle}) ; ou bien suivant une modélisation de la signification des concepts du domaine établie par des experts par exemple (\e{sémantique conceptuelle}). 

	\item \e{sémantique référentielle} : la signification d'une unité est l'objet auquel elle fait référence, dans un univers extralinguistique. Ici, on s'attache à une théorie propre à cet univers et qui explicite la définition des objets.
	
	\item \e{sémantique psychologique} : la signification d'une unité est la représentation mentale que l'on s'en fait. Là encore, il s'agit de suivre une théorie, mais dans le champ de la psychologie. 
\end{liste}
Cette liste de sémantiques permet de comprendre la grande variabilité des ontologies qu'il est possible de construire. 


\paragraph{Classification d'ontologie}
Il peut également être utile de définir des propriétés pour distinguer des "genres" d'ontologies, non pas en fonction de la sémantique utilisé, mais en fonction de l'usage que l'on souhaite en faire. 
\cite{Oberle2006} propose une classification qui repose sur trois propriétés : 
\begin{liste}
	\item l'\g{objectif} de l'ontologie (purpose) où l'on distingue entre deux objectifs, servir de référence ou bien être utilisé dans un cas d'application :
	\begin{liste}
		\item l'\e{ontologie de référence} vise à établir un consensus entre des agents (humains, machines) d'une même communauté, ou bien à servir d'explication et de langages communs avec des agents de communautés différentes. 
		\item l'\e{ontologie d'application} qui se limite à un cas d'application et suit des contraintes et simplifications propres.
	\end{liste}
	La différence réside dans l'arbitrage entre l'expressivité de la représentation et sa décidabilité (\cite{Borgo2002}). 
	Typiquement, une référence n'est consulté qu'occasionnellement et se doit d'être la plus exhaustive possible alors qu'une ontologie d'application doit servir à faire des raisonnements à l'exécution. 

	\item l'\g{expressivité} de l'ontologie (expressiveness) où l'on considère un engagement plus ou moins fort sur le formalisme de représentation :
	\begin{liste}
		\item l'\e{ontologie légère} (lightweight) qui peut se limiter à une hiérarchie de concepts bien connus dans une communauté avec quelques relations. 
		L'apport se situe alors dans la structuration des connaissances qui clarifie leur signification plus qu'il ne les établit.
		\item l'\e{ontologie lourde} (heavyweight) qui vise à exclure toute ambiguïté terminologique et conceptuelle.
		Pour cela, la formalisation se veut beaucoup plus contrainte et détaillée afin de forcer une interprétation. 
	\end{liste}

	\item la \g{spécificité} de l'ontologie qui peut se limiter à une domaine ou s'étendre à un ensemble de domaines voire plusieurs champs disciplinaires : 
	\begin{liste}
		\item l'\e{ontologie générique} (generic, upper/top level) contient des concepts utilisés dans de nombreux champs disciplinaires (évènements, processus etc.)
		\item l'\e{ontologie noyau} (core) comporte des concepts qui se situent à la croisée de plusieurs domaines. 
		La distinction avec une ontologie générique se fait car elle comporte des concepts utilisable quelque soit le domaine.
		De même, on distingue les ontologies noyaux des ontologies de domaine car les premières comportent des éléments réutilisables dans des plusieurs domaines proches. 
		\item l'\e{ontologie de domaine} contient des concepts propre à un domaine, et bien souvent des éléments plus génériques extraits de domaine différents.
		Cependant, l'ontologie de domaine présente généralement des éléments plus spécifiques, propres au domaine concerné. 
	\end{liste}
\end{liste}