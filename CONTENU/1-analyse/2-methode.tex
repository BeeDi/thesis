\subsection{Une méthode de construction d'ontologie (n)}\label{sec:construction}
Nous avons défini dans la section précédente (\ref{sec:tto}) ce qu'est une ontologie,  présenté des exemples de sémantiques et montré comment il était possible de classifier ces ontologies suivant leur usage. 
Dans cette section, nous détaillons la méthode de construction d'ontologie \pc{Archonte} (\pc{arch}itecture for \pc{ont}ological \pc{e}laborating) proposé par \cite{Bachimont2000a}.
Cette présentation est aussi l'occasion de préciser l'importance crucuiale du choix d'une sémantique sur la construction d'une ontologie.

La méthode repose sur trois étapes successives qui aboutissent à une ontologie computationnelle, exprimée dans un langage opérationnel de représentation des connaissances et à partir duquel on peut effectuer des inférences. 
% Précisons maintenant les étapes de cette méthode ainsi que les résultats obtenus :
% \begin{liste}
	Le point de départ de la méthodologie est constitué d'expressions linguistiques (signifiant) issues du domaine considéré.
	L'intérêt de disposer d'un tel ensemble de traces linguistiques est qu'elles servent à exprimer des concepts (signifié) ou des connaissances sur le monde. 
	Ainsi, on se retrouve avec un corpus de candidat-termes dont la signification peut être source d'ambiguïtés et dont l'on cherche à clarifier l'interprétation.\\
% \end{liste}
% \begin{liste}
\g{[1.]} La première étape de cette méthode (aussi appelée \e{normalisation sémantique}) consiste à établir un \g{engagement sémantique}  qui précise la manière de mener l'interprétation des candidats-termes et de construire une première structure de connaissances. 
Pour cela, on fixe d'abord un contexte de référence, la tâche ou le problème qui a poussé à l'élaboration de l'ontologie, qui permet de cadrer l'interprétation des candidat-termes.

Ensuite, pour préciser l'interprétation on s'appuie sur la sémantique différentielle afin d'expliciter les différences et les similarités entre une notion et son voisinage direct (notion parente, notion soeurs) : 

	\ciel{
	La méthodologie que nous proposons ici repose sur l'organisation générale des unités en un réseau d'identités et de différences.
	Ce sont les propriétés structurelles de ce réseau qui permettent de contraindre l'interprétation des unités définies dans le réseau : la position d'une unité dans le réseau prescrit comment la comprendre et lui prescrit une signification qui pourra dès lors lui être associée, quel que soit le contexte où elle se rencontre.} (\cite[p.139]{bachimont:icc})

Cette caractérisation des notions par leur voisinage repose sur quatres relations à expliciter : 
\begin{liste}
	\item la \e{communauté avec le père} (similarity with parent) : pourquoi la notion hérite des proprités de son père.
	\item la \e{différence avec le père} (difference with parent) : en quoi la notion est différente de son père.
	\item la \e{différence avec les frères} (difference with siblings) : en quoi un notion est différentes de ses notions soeurs.
	\item la \e{communauté avec les frères} (similarity with siblings) : quelle est la propriété que partage les frères -- dont on distingue plusieurs valeurs exclusives, une par frère.\\ 		
\end{liste}
% \end{liste}

Cette première étape aboutit à la construction d'un \g{arbre ontologique différentiel} qui structure un ensemble de notions de manière hiérarchique et non ambiguië par rapport à un contexte de référence.
Les candidats-termes sont structurés par des prescriptions interprétatives et deviennent ainsi des primitives de modélisation.\\

% \begin{liste}
\g{[2.]} L'\g{engagement ontologique} consiste à munir l'ontologie différentielle d'une sémantique formelle extensionnelle.
Rappelons que cette sémantique définit les concepts par leur extension, c'est à dire tout les individus qu'ils désignent parmi un ensemble de référence. 
Il s'agit donc de relier des primitives dotées d'une signification linguistique normalisée à des concepts désignant un ensemble de référents (ou individus).
Pour cela, il faut adjoindre à l'ontologie différentielle un modèle référentiel : 

	\ciel{
	l'ontologie référentielle obéit aux contraintes sémantiques de l'ontologie différentielle : [s]a structure arborescente se retrouve dans l'ontologie référentielle et lui donne son squelette.
	Chaque relation de spécialisation sémantique au niveau différentiel se traduit par une spécialisation d'extension au niveau référentiel.} (\cite[p.148]{bachimont:icc})

Ce changement de sémantique permet d'enrichir l'ontologie de nouveau concepts et d'en modifier la structuration. 
En effet, on peut désormais avoir recours à des opérations ensemblistes (réunion, intersection, complémentaire) qui composent le sens des concepts et permettent ainsi de définir de nouveaux concepts.
L'ajout de ces \gui{concepts définis} modifie également la structure de l'ontologie, qui passe d'une arborescence à une structure en treillis, c'est-à-dire admettant l'héritage multiple. 
Par exemple, une primitive différentielle de \cd{mandat politique} spécialisée en concepts de \cd{député} et de \cd{maire} ne permet pas de représenter de double mandat.
Par contre, une définition extensionnelle permet de définir le concept de \cd{député-maire} simplement par l'intersection des extensions de ces concepts\footnote{Pour plus de détails sur cet exemple, se reporter à l'exemple donné par \cite[p.149]{bachimont:icc}}.
% \end{liste}

À l'issue de cette étape on obtient donc une \g{ontologie référentielle}, c'est-à-dire un treillis de concepts définis par une sémantique référentielle.\\

% \begin{liste}
\g{[3.]} L'\g{engagement computationnel} vise à doter les concepts de l'ontologie référentielle d'une signification en termes d'opérations informatiques.
Pour cela, il faut d'abord choisir un langage opérationnel de représentation des connaissances qui détermine l'expressivité et les opérations de calculs à disposition pour élaborer une version informatique de l'ontologie.
Nous présentons quelques uns de ces langages dans la section \ref{sec:onto-mc}.
La transposition dans un langage a des conséquences au niveau de l'expressivité et de la décidabilité du modèle.
% \end{liste}

Nous obtenons une \g{ontologie computationelle} qui est une version de l'ontologie référentielle exploitable informatiquement.


% \begin{liste}
% 	\item
% \end{liste}







\subsection{Conséquences pour une Ingénierie des Connaissances (n)}
En suivant l'analyse proposé par \cite{Bachimont2004}, il en découle que  l'Ingénierie des Connaissances (IC) \ciel{exprime les connaissances d'un domaine dans un langage de modélisation et l'opérationnalise en un système}. 
En d'autres termes, la modélisation de l'IC porte sur les concepts utilisés par les gens de ce domaine pour penser et établir des connaissances sur le monde, mais pas directement sur le monde.
Ainsi, les modèles de l'IC n'ont pas pour vocation à \ciel{prédire quoi que ce soit sur le monde ni sur la connaissance}, mais plutôt d'\ciel{instrumenter le travail intellectuel, l'exercice de la pensée, le travail de la connaissance}. 
Dans cette perspective, \citeauthor{Bachimont2004} nous propose de théoriser l'IC comme \ciel{une ingénierie des inscriptions numériques des connaissances qui vise à instrumenter le travail cognitif associé à ces inscriptions}. 
        
Les inscriptions possèdent une double dimension ; \e{matérielle} (et donc manipulable par des techniques de calcul logique) ; \e{sémiotique} (et donc interprétable selon des conventions propres à une situation d'usage).  
En d'autres termes, le systèmes d'IC permet d'agir de manière prédictible sur les inscriptions de connaissances, ces actions produisant de nouvelles inscriptions qui donnent matière à penser à l'utilisateur. 
        
Il y a donc plusieurs éléments à valider en IC, les calculs qui seront faits sur les inscriptions (on teste le comportement du système informatique) puis l'interprétation de ces inscriptions (on évalue le gain apporté par le système et les inscriptions qu'il fournit à l'utilisateur selon une situation d'usage). 
La modélisation prise en charge par l'IC ne porte donc ni sur le monde, ni sur l'activité cognitive et ne peut être validé uniquement par le formalisme de ses inscriptions. 
        
Les inscriptions de connaissances doivent être considérées sous deux angles : d'un point de vue \e{nomographique} (on formalise la manipulation symbolique des inscriptions pour prévoir/définir le comportement du système) et \e{idiographique} (on décrit le sens des manipulations symboliques et des inscriptions produites par rapport aux normes, conventions, concepts du domaine.



% Au final, on construit un objet informatique en suivant une méthode de construction qui nous guide pour spécifier le comportement de l'ontologie. 

% RTO





