\subsection{Méthodes de construction d'ontologie}\label{sec:construction}
Nous avons défini dans la section précédente (\ref{sec:tto}) ce qu'est une ontologie, présenté des exemples de sémantiques et montré comment il était possible de classifier ces ontologies suivant leur usage. 
Nous abordons maintenant les méthodes de construction d'ontologies.


\paragraph{Méthode d'\cite{Uschold1996}}
\citeauthor{Uschold1996} ont défini une méthode de construction à partir de leur expérience de développement d'ontologies en entreprises. 
Elle se décompose en quatre étapes :
\begin{listenum}
	\item une phase de \g{conception} qui vise à identifier le domaine concerné, le but et la porté de l'ontologie.
	\item une phase de \g{construction} qui se décompose en trois étapes ; définir les concepts clés et les relations entre ces concepts ; expliciter la représentation de la conceptualisation dans un langage formel ; intégrer des connaissances d'autres ontologies.
	\item une phase d'\g{évaluation} de l'ontologie construite.
	\item une phase de \g{documentation} qui doit expliciter les décisions effectuées aux étapes précédentes afin de faciliter la réutilisation de l'ontologie.
\end{listenum}


\paragraph{Methontology}
La méthode proposée par le Laboratoire d'Intelligence Artificielle (LAI) de l'université Polytechnique de Madrid a pour particularité d'intégrer le développement de l'ontologie à une méthodologie de gestion de projet \parcite{Fernandez1997, Blazquez1998}. La méthodologie distingue trois types d'activités se déroulant en parallèle, et dont les deux premières servent à soutenir la construction de l'ontologie :
\begin{liste}
	\item les activités de \g{gestion de projet}, notamment la planification avant-projet puis le \e{contrôle de la qualité} des résultats produits. 
	
	\item les activités de \g{support} concernent l'\e{acquisition} des connaissances du domaine ; l'\e{intégration} de connaissances d'autres ontologies ; la \e{documentation} de l'ontologie et de sa production ; la \e{gestion de version} des résultats produits ; l'\e{évaluation} technique de la construction de l'ontologie ainsi que de sa documentation. 

	\item les activités de \g{développement technique} qui permettent de construire l'ontologie par étape. 
	Dans un premier temps, la \e{spécification} définit l'objectif de l'ontologie, les applications et les utilisateurs concernés ; puis la \e{conceptualisation} structure les connaissances du domaine, qui sont ensuite formalisé (étape de \e{formalisation}) et enfin représenté dans un langage informatique (étape d'\e{implémentation}). La séquence se poursuit par une étape de \e{maintenance}.
\end{liste}


% Ontospec (Kassel)
% Guarino et Welty
% voir livre BB pour la justification du manque de sémantique

\paragraph{Archonte}
La méthode \pc{Archonte} (\pc{arch}itecture for \pc{ont}ological \pc{e}laborating), proposée par \cite{Bachimont2000a}, met en avant l'importance cruciale donnée au choix d'une sémantique dans la construction d'une ontologie.
La méthode repose sur trois étapes successives qui aboutissent à une ontologie computationnelle, exprimée dans un langage opérationnel de représentation des connaissances.
% , et à partir duquel on peut effectuer des inférences. 
% Précisons maintenant les étapes de cette méthode ainsi que les résultats obtenus :
% \begin{liste}
Le point de départ de la méthodologie est constitué d'expressions linguistiques (signifiant) issues du domaine considéré.
L'intérêt de disposer d'un tel ensemble de traces linguistiques est qu'elles servent à exprimer des concepts (signifiés) ou des connaissances sur le monde. 
Ainsi, on se retrouve avec un corpus de candidats-termes dont la signification peut être source d'ambiguïtés et dont l'on cherche à clarifier l'interprétation.\\

\g{[1.]} La première étape de cette méthode (aussi appelée \e{normalisation sémantique}) consiste à établir un \g{engagement sémantique}  qui précise la manière de mener l'interprétation des candidats-termes et de construire une première structure de connaissances. 
Pour cela, on fixe d'abord un contexte de référence, la tâche ou le problème qui a poussé à l'élaboration de l'ontologie, qui permet de cadrer l'interprétation des candidats-termes.

Ensuite, pour préciser l'interprétation on s'appuie sur la sémantique différentielle afin d'expliciter les différences et les similarités entre une notion et son voisinage direct (notion parente, notions soeurs) : 

	\ciel{
	La méthodologie que nous proposons ici repose sur l'organisation générale des unités en un réseau d'identités et de différences.
	Ce sont les propriétés structurelles de ce réseau qui permettent de contraindre l'interprétation des unités définies dans le réseau : la position d'une unité dans le réseau prescrit comment la comprendre et lui prescrit une signification qui pourra dès lors lui être associée, quel que soit le contexte où elle se rencontre.} (\cite[p.139]{bachimont:icc})

Cette caractérisation des notions par leur voisinage repose sur quatres relations à expliciter : 
\begin{liste}
	\item la \e{communauté avec le parent} (similarity with parent) : pourquoi la notion hérite des proprités de son parent.
	\item la \e{différence avec le parent} (difference with parent) : en quoi la notion est différente de son parent.
	\item la \e{différence avec les soeurs} (difference with siblings) : en quoi un notion est différentes de ses notions soeurs.
	\item la \e{communauté avec les soeurs} (similarity with siblings) : quelle est la propriété que partage les notion soeurs -- dont on distingue plusieurs valeurs exclusives, une par soeur.\\ 		
\end{liste}
% \end{liste}

Cette première étape aboutit à la construction d'un \g{arbre ontologique différentiel} qui structure un ensemble de notions de manière hiérarchique et non ambiguië par rapport à un contexte de référence.
Les candidats-termes sont structurés par des prescriptions interprétatives et deviennent ainsi des primitives de modélisation.\\

% \begin{liste}
\g{[2.]} L'\g{engagement ontologique} consiste à munir l'ontologie différentielle d'une sémantique formelle extensionnelle.
Rappelons que cette sémantique définit les concepts par leur extension, c'est-à-dire tous les individus qu'ils désignent parmi un ensemble de référence. 
Il s'agit donc de relier des primitives dotées d'une signification linguistique normalisée à des concepts désignant un ensemble de référents (ou individus).
Pour cela, il faut adjoindre à l'ontologie différentielle un modèle référentiel : 

	\ciel{
	l'ontologie référentielle obéit aux contraintes sémantiques de l'ontologie différentielle : [s]a structure arborescente se retrouve dans l'ontologie référentielle et lui donne son squelette.
	Chaque relation de spécialisation sémantique au niveau différentiel se traduit par une spécialisation d'extension au niveau référentiel.} (\cite[p.148]{bachimont:icc})

Ce changement de sémantique permet d'enrichir l'ontologie de nouveaux concepts et d'en modifier la structuration. 
En effet, on peut désormais avoir recours à des opérations ensemblistes (réunion, intersection, complémentaire) qui composent le sens des concepts et permettent ainsi de définir de nouveaux concepts.
L'ajout de ces \gui{concepts définis} modifie également la structure de l'ontologie, qui passe d'une arborescence à une structure en treillis, c'est-à-dire admettant l'héritage multiple. 
Par exemple, une primitive différentielle de \cd{mandat politique} spécialisée en concepts de \cd{député} et de \cd{maire} ne permet pas de représenter de double mandat.
Par contre, une définition extensionnelle permet de définir le concept de \cd{député-maire} simplement par l'intersection des extensions de ces concepts\footnote{Pour plus de détails sur cet exemple, se reporter à l'exemple donné par \cite[p.149]{bachimont:icc}}.
% \end{liste}

À l'issue de cette étape on obtient donc une \g{ontologie référentielle}, c'est-à-dire un treillis de concepts définis par une sémantique référentielle.\\

% \begin{liste}
\g{[3.]} L'\g{engagement computationnel} vise à doter les concepts de l'ontologie référentielle d'une signification en termes d'opérations informatiques.
Pour cela, il faut d'abord choisir un langage opérationnel de représentation des connaissances qui détermine l'expressivité et les opérations de calculs à disposition pour élaborer une version informatique de l'ontologie.
Nous présentons quelques uns de ces langages dans la section \ref{sec:onto-mc}.
La transposition dans un langage a des conséquences au niveau de l'expressivité et de la décidabilité du modèle.
% \end{liste}

Nous obtenons une \g{ontologie computationelle} qui est une version de l'ontologie référentielle exploitable informatiquement.





\subsection{La validation en Ingénierie des Connaissances}\label{sec:valid-ic}
En suivant l'analyse proposée par \cite{Bachimont2004}, il en découle que  l'Ingénierie des Connaissances (IC) \ciel{exprime les connaissances d'un domaine dans un langage de modélisation et l'opérationnalise en un système}. 
En d'autres termes, la modélisation de l'IC porte sur les concepts utilisés par les gens de ce domaine pour penser et établir des connaissances sur le monde, mais pas directement sur le monde.
Ainsi, les modèles de l'IC n'ont pas pour vocation à \ciel{prédire quoi que ce soit sur le monde ni sur la connaissance}, mais plutôt d'\ciel{instrumenter le travail intellectuel, l'exercice de la pensée, le travail de la connaissance}. 
Dans cette perspective, \citeauthor{Bachimont2004} nous propose de théoriser l'IC comme \ciel{une ingénierie des inscriptions numériques des connaissances qui vise à instrumenter le travail cognitif associé à ces inscriptions}. 
        
Les inscriptions possèdent une double dimension ; \e{matérielle} (et donc manipulable par des techniques de calcul logique) ; \e{sémiotique} (et donc interprétable selon des conventions propres à une situation d'usage).  
En d'autres termes, le systèmes d'IC permet d'agir de manière prédictible sur les inscriptions de connaissances, ces actions produisant de nouvelles inscriptions qui donnent matière à penser à l'utilisateur. 
        
Il y a donc plusieurs éléments à valider en IC, les calculs qui seront faits sur les inscriptions (on teste le comportement du système informatique) puis l'interprétation de ces inscriptions (on évalue le gain apporté par le système et les inscriptions qu'il fournit à l'utilisateur selon une situation d'usage). 
La modélisation prise en charge par l'IC ne porte donc ni sur le monde, ni sur l'activité cognitive et ne peut être validée uniquement par le formalisme de ses inscriptions. 
        
Les inscriptions de connaissances doivent être considérées sous deux angles : d'un point de vue \e{nomographique} (on formalise la manipulation symbolique des inscriptions pour prévoir/définir le comportement du système) et \e{idiographique} (on décrit le sens des manipulations symboliques et des inscriptions produites par rapport aux normes, conventions, concepts du domaine).



% Au final, on construit un objet informatique en suivant une méthode de construction qui nous guide pour spécifier le comportement de l'ontologie. 

% RTO





\subsection*{Discussion}
\addcontentsline{toc}{subsection}{Discussion}
% pourquoi on choisit celle de BB, et pourquoi on ne l'utilise pas globalement, mais seulement localement sur des patrons d'utilisation précis.



Notre étude des méthodes de construction d'ontologie ne vise pas à être exhaustive, pour cela nous renvoyons à \cite{Gomez-Perez2004}.
Il faut cependant remarquer que nous avons écarté les approches d'acquisition des connaissances à partir d'un corpus de texte, comme la méthode \g{Terminae} développée par \cite{Aussenac-Gilles2003}.
Dans cette approche, l'analyse linguistique d'un corpus permet de repérer des candidats-termes (\pc{Syntex}), d'effectuer des regroupements de contexte (\pc{Upery}) et d'identifier des relations (\pc{Yakwa}) afin d'accompagner la modélisation conceptuelle.
Or, dans notre contexte de travail, les documents professionnels qui décrivent en détails la production audiovisuelle sont rares (peu d'organisation ont les ressources de les produire) et constituent des ressources de valeur auquelles il est difficile d'avoir accès. 
De plus, la notion de document audiovisuel est largement absente des ouvrages généralistes, ce qui fonde précisément l'intérêt de notre travail de recherche.
\citeauthor{Uschold1996} proposent point de vue global sur le processus de construction d'ontologie, qui reste un point de référence dans le domaine.
L'approche de \pc{Methontology} clarifie cependant le déroulement de la construction et s'efforce de l'intégrer dans une vision de gestion de projet assez exhaustive. 
Si cette vision est intéressante, il n'est pas forcément possible de l'appliquer telle quelle en pratique, du fait de contraintes spécifiques d'un projet.
Par ailleurs, \pc{Archonte} détaille une méthode de formalisation logique qui permet de positionner la conceptualisation sur le plan sémantique. 
De ce fait, elle précise la manière de formaliser une conceptualisation et peut s'intégrer de manière complémentaire aux autres méthodes exposées.

Dans notre cas, nous avons suivi une méthodologie \e{ad-hoc}, nécessaire pour suivre les contraintes d'un projet de recherche et développement impliquant de multiple partenaires.
Les aspects de gestion de projet (tels que décrits dans \pc{Methontology}) étaient donc largement assujetis à l'avancement du projet MediaMap.
De même, la spécification et l'acquisition des connaissances et la conceptualisation reposent principalement sur un dialogue avec des experts du domaine (dans le cas du projet MediaMap ce fût principalement les membres de la RTBF et de la VRT) dans le cadre de réunions générales et de séminaires plus focalisés.
Les autres aspects de la construction de l'ontologie ont été réalisés par les membres de l'équipe de recherche ICI, en s'inspirant des séquences de développement technique proposées par \pc{Methontology}.

Sur le plan sémantique, l'étendue de notre modélisation nous pousse à introduire des concepts de haut-niveau afin d'articuler diverses connaissances se rapportant à la production audiovisuelle (l'organisation du processus, ses contributeurs, ses résultats, et leur description par un vocabulaire professionnel et compréhensible par des amateur).
Ces ontologies génériques de haut-niveau proposent des fondements théoriques important et des patrons de conception détaillés (au sens de \cite{Isaac2005}) pour modéliser certaines situations. 
On pense par exemple, au patron \ciel{Description \& Situations} de l'ontologie DOLCE \parcite{Gangemi2005}. 
Pour autant, la modélisation ne doit pas perdre de vue notre perspective applicative, nécessaire à l'adoption et la compréhension du modèle par des utilisateurs du domaine.
Ainsi, \citeauthor{Isaac2005} proposent d'adapter la structure de ces patrons aux besoins descriptifs de l'application.
Cela consiste à simplifier la structure des patrons en créant des \e{raccourcis relationnels}, quitte à faire disparaître les subtilités de représentation.
L'enjeu d'une telle approche est de faire co-exister la forme simplifiée (adaptée aux usages du domaine et aux besoins expressifs de l'application) et la forme de haut-niveau qui la rend réutilisable dans d'autres domaines.
De notre point de vue, notre modélisation se construit à partir de tels patrons de conception, qu'il s'agira ensuite d'articuler par des relations. 
Nous avons appliqué la méthodologie \pc{Archonte} pour formaliser la sémantique de ces patrons, qui sont ensuite regroupés pour former une seule ontologie.
En effet, nous considérons que le point important est de justifier la modélisation de ces parties et de leur mise en relation, plutôt que de la structure de l'ensemble.






% est-ce qu'on a adapté des patrons de conception générique aux notions du domaine, pour simplifier la modélisation (modélisation réduite) ? VOIR \cite{Isaac2005}








