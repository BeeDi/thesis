\chapter{Approche et modélisation}\label{chap:mod}
\minitoc
\section{Principes de l'approche}\label{sec:principes}
% Notre hypothèse est que le plan constitue l'unité signifiante de base de l'audiovisuel (FSR). C'est donc à partir d'une description de cette unité que toute description devrait commencer.
% L'indexation structurelle peut ensuite prendre place à partir de cette unité. On définit des fragments (objets éditoriaux) de niveaux supérieurs, ainsi qu'une grammaire de composition de ces fragments, ce qui permet de définir des types de documents.

% construction pendant la chaîne de production

% instrumenter l'écriture des prescriptions, structure documentaire molle, a priori qui servira également d'écriture des descriptions
% le plan est l'unité de base d'une structure documentaire qui exprime des connaissances sur la production audiovisuelle 
% ces connaissances sont adaptés à la réutilisation de contenu, puisque c'est une recherche, des opérations de réagencement qui sont fait par des contributeurs de la production, en vue d'une nouvelle production


% En toute modestie, on ne cherche pas à construire ou même à utiliser une ontologie générique, mais plutôt à se concentrer sur la modélisation de notre problème : la réutilisation dans la production audiovisuelle. Ainsi, nous ne reprennons pas directement le travail sur les ontologies génériques, car notre problème n'est pas l'intégration de connaissances d'un domaine, mais l'expression des connaissances propres à la production audiovisuelle.
% Nous nous distinguons ainsi des approches qui formalisent la modélisation de MPEG-7, car nous nous concentrons sur les connaissances fabriquées dès le début de la production. 

